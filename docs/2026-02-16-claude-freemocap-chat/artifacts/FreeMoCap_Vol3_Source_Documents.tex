\documentclass[11pt,letterpaper,twoside]{book}
\input{freemocap_preamble}

\hypersetup{pdftitle={FreeMoCap: Source Documents --- Volume 3}}

\begin{document}

% TITLE PAGE
\begin{titlepage}
\centering
\vspace*{2cm}

{\Huge\bfseries\color{navy} FreeMoCap\par}
\vspace{0.3cm}
{\Large\color{accent}\rule{0.5\textwidth}{2pt}\par}
\vspace{0.5cm}
{\LARGE\bfseries Source Documents\par}
\vspace{0.3cm}
{\large Volume 3 of 3\par}

\vspace{2cm}

{\large Complete transcripts, interviews, and working documents\\
from which Volumes 1 and 2 were compiled.\par}

\vfill

{\large The FreeMoCap Foundation\par}
\vspace{0.3cm}
{\normalsize February 2026\par}

\end{titlepage}

\frontmatter
\tableofcontents

\chapter*{Source Key}
\addcontentsline{toc}{chapter}{Source Key}

The following source markers are used throughout the three-volume set to cite these documents:

\begin{description}
\item[S1] "This is FreeMoCap" --- Introductory Video (March 11, 2022)
\item[S2] Biomechanics On Our Minds (BOOM) Podcast (March 16, 2022)
\item[S3] VFX Futures Podcast Interview (July 13, 2022)
\item[S4] Dynamic Walking 2022 Conference Presentation (June 16, 2022)
\item[S5] HMN2024 --- Introduction to Class Lecture (September 25, 2024)
\item[S6] State of the Skelly Address (2026)
\item[S7] Interview Transcript I --- On First Skeleton and Vision (2026)
\item[S8] Interview Transcript II --- On Strategy and Liberation (2026)
\item[CL] Compiled Learnings: The Soul of FreeMoCap (February 2026)
\item[SF1] The Soul of FreeMoCap --- Initial Compilation (February 2026)
\item[SF2] The Soul of FreeMoCap --- Expanded Edition (February 2026)
\item[WS1] Movement-Building Worksheet (Version 1) (February 2026)
\item[WS2] Movement-Building Worksheet (Version 2) (February 2026)
\item[WS3] Liberation Movement Worksheet (February 2026)
\item[LC] Liberation Movement Compendium (February 2026)

\end{description}

\mainmatter


\chapter{"This is FreeMoCap" --- Introductory Video}
\label{src:S1}

\begin{framed}
\small
\textbf{Source marker:} S1 \\
\textbf{Date:} March 11, 2022 \\
\textbf{Source:} YouTube Video, https://www.youtube.com/watch?v=WW\_WpMcbzns, Duration: 00:03:23
\end{framed}

\small
\# Transcript: 2022-03-11-This is FreeMoCap

\#\# Source Information

- \textbf{Source Type:} YouTube Video
- \textbf{URL:} https://www.youtube.com/watch?v=WW\_WpMcbzns
- \textbf{Video ID:} `WW\_WpMcbzns`

\bigskip\noindent\rule{\textwidth}{0.4pt}\bigskip

\textbf{Total Duration:} 00:03:23



\begin{quote}\itshape
\textbf{Speaker:} Jon Matthis (solo introductory video)
\end{quote}

\bigskip\noindent\rule{\textwidth}{0.4pt}\bigskip

\#\# Full Transcript

\#\#\# Chunk 1 [00:00:00 - 00:03:23]

\textbf{[00:00:00]} Doing your thing. This is up. Give myself that to look at. Then I come back here, I do this charcoal dance. I used to just leave it statically on the screen, but I think this gives it better calibration and moving it up and down I think will give it better lens distortion estimations.

\textbf{[00:00:22]} Okay. And I'm going to clap for the audio sync and then I'm gonna stand in a pose to give a nice sort of clean view for the rigging, rigify, blender rigging thing. Okay. Then I get my head into like the weird energy of like pitching sincerely a thing that you believe and care about to a bunch of cameras alone in an apartment. Okay.

\textbf{[00:00:48]} And then do the Wayne's World countdown. And five, four, three.

\textbf{[00:00:57]} Hello, my name is Jonathan Samir Mathis and this is freemocap. We're building a free open source markerless motion capture system designed to provide research quality recordings of 3D movement to everyone for free. The sheer volume of world shaking technologies and computational techniques that that went into the generation of this animation is staggering. These technologies will shape the future of our society. And yet, even though many of these techniques are already free and open source, there is a massive technical barrier between the average person and the ability to use these tools for their own needs, whatever those may be.

\textbf{[00:01:38]} The Free mocap project will create a system that makes those technologies a available, accessible and transparent to the communities of people who stand to benefit from them. I'm a research professor who studies neuroscience of full body human movement in natural environments. And the goal is to make a tool that serves both my needs as a professional research scientist while remaining intuitively usable to a 13 year old with no technical training and no outside assistance. We, we want indie game designers and animators to use the same tool to add motion capture assets to their zero budget art project that I am using for my federally funded scientific research program. We believe that Free MOCAP can serve to build practical educational bridges between the wide range of communities that might benefit from the ability to accurately record, reconstruct and measure human movement.

\textbf{[00:02:34]} Not just scientists, not just animators and game designers, but athletes, coaches, dancers, performers, anyone who might have a reason to record human movement. And everyone has a reason to record human movement. And we want to help them do it. Thank you.

\textbf{[00:02:59]} Okay. All right. That was good. That was good. That was good.

\textbf{[00:03:02]} That was good. Okay. Hope it calibrates. Okay. The vanilla.

\textbf{[00:03:06]} A pose just for fun. Another clap because I know it's like the time is weird. Another charuko dance. This is not necessary, but it's good. To have a backup in case the first one fails.

\textbf{[00:03:20]} What am I forgetting? I think that's it. God help us all.

\bigskip\noindent\rule{\textwidth}{0.4pt}\bigskip

\normalsize


\chapter{Biomechanics On Our Minds (BOOM) Podcast}
\label{src:S2}

\begin{framed}
\small
\textbf{Source marker:} S2 \\
\textbf{Date:} March 16, 2022 \\
\textbf{Source:} YouTube Video / Podcast, Duration: 00:39:08
\end{framed}

\small
\# Transcript: 2022-03-16-Bonus Episode Your Friendly Neighborhood Markerless Motion Capture with Jon Matthis

\#\# Source Information

- \textbf{Source Type:} YouTube Video
- \textbf{URL:} https://www.youtube.com/watch?v=kw3hYndzzac
- \textbf{Video ID:} `kw3hYndzzac`

\bigskip\noindent\rule{\textwidth}{0.4pt}\bigskip

\textbf{Total Duration:} 00:39:08

\bigskip\noindent\rule{\textwidth}{0.4pt}\bigskip

\#\# Full Transcript



\begin{quote}\itshape
\textbf{Note:} Speaker identifications below are approximate estimates
derived from automated analysis of speech content and may contain errors.
Timestamps for mid-block speaker changes are estimated proportionally.
\end{quote}

\textbf{Speakers:}
- \textbf{Interviewer (Melissa Boswell / Hannah O'Day)}
- \textbf{Jon Matthis}

\bigskip\noindent\rule{\textwidth}{0.4pt}\bigskip


\#\#\# Chunk 1 [00:00:00 - 00:09:59]

\textbf{[00:00:00] Interviewer (Melissa Boswell / Hannah O'Day):} Welcome to Biomechanics on our Minds. My name is Melissa Boswell. And I'm Hannah O'. Day. And we're PhD students at Stanford University.

\textbf{[00:00:07] Interviewer (Melissa Boswell / Hannah O'Day):} This podcast is brought to you by the International Society of Biomechanics. It's time for boom. Welcome to boom, where we have biomechanics on our minds. Boom. Boom.

\textbf{[00:00:20] Interviewer (Melissa Boswell / Hannah O'Day):} Boom. Boom. Boom. Boom. Boom.

\textbf{[00:00:24] Interviewer (Melissa Boswell / Hannah O'Day):} Boom. Boom. Boom.

\textbf{[00:00:29] Interviewer (Melissa Boswell / Hannah O'Day):} Welcome back to boom. Today we're talking with Jonathan Samir Mathis, who is a tenure track professor of human movement neuroscience in Boston, Mass, hailing from my hometown, and he's also the founder of Free mocap Foundation. And we're so excited to have you. Jonathan, thank you so much for joining us.

\textbf{[00:00:45] Jon Matthis:} Lovely to be here.

\textbf{[00:00:46] Interviewer (Melissa Boswell / Hannah O'Day):} So can you tell us when you first knew that you wanted to be a biomechanist?

\textbf{[00:00:50] Jon Matthis:} I mean, I guess I'm a biomechanist now, but my bachelor's degree, my first degree was in philosophy. And I consider myself more of a neuroscientist than a biomechanist. I studied the part of neuroscience where we move our bodies around. It's a bit of a.

\textbf{[00:01:05] Jon Matthis:} The path is a meandering one, but it went from philosophy to kind of like visual neuroscience, which got me into specifically studying the visual control of locomotion. And then I kind of realized at that point that you can't really plenty try, but you really shouldn't study the visual control of movement if you don't know about the actual movement. And if you want to know about the actual movement, probably going to have to talk about physics. So I kind of, you know, backed my way into biomechanics just as sort of the need arose in my early grad school as a philosophy major trying to learn how to do science and kind of realizing it's like, oh, there's this whole area that I need to know about. And because I didn't know anything, it was just as easy to learn half of vision science and half of biomechanics as it was continuing with your normal regular student.

\textbf{[00:01:53] Jon Matthis:} Yeah, I can see that connection there between neuroscience and then studying human movement.

\textbf{[00:01:58] Interviewer (Melissa Boswell / Hannah O'Day):} I'm curious about the connection from philosophy to neuroscience. Well, it wasn't necessarily planned.

\textbf{[00:02:04] Jon Matthis:} I mean, I was studying sort of philosophy of mind, philosophy of science in my undergrad days and always kind of had a fascination, you know, in retrospect, a fascination with kind of like that interface between, like, you know, what am I and what is the world, which kind of winds up being, you know, if you really want to pare it down into like a study Able question that winds up being perceptual motor control, mind body problem stuff. I actually applied to go to grad school for philosophy and by grace of God was not accepted into any of the programs I applied to and then wound up working at an autism research facility for a year and was like, oh wow, data is cool.

\textbf{[00:02:43] Jon Matthis:} You like say things.

\textbf{[00:02:44] Interviewer (Melissa Boswell / Hannah O'Day):} Yeah, I do, you know. Another slew of graduate programs in cognitive science was the tag I was looking for. And I was planning to study language and then accidentally one of the places I applied for didn't have a language person. So I didn't notice that until it's like writing the personal statement and I was looking through the list of like, well what else have they got here?

\textbf{[00:03:08] Jon Matthis:} And there was a guy named Brett Fajin, he studies visual control of locomotion and sounds pretty cool. And that's the only one that I got into. So now I found myself studying visual control of locomotion and what is that, 13 some odd years later, skeletons in my office and robotics and professorship and whatever.

\textbf{[00:03:26] Interviewer (Melissa Boswell / Hannah O'Day):} Like, well it's cool to see you go and be open to all those different parts of your journey.

\textbf{[00:03:31] Jon Matthis:} Like you said, by the grace of God, you weren't in philosophy and you really listened to the different like things that I think and I'm like, I haven't hit the ground yet.

\textbf{[00:03:42] Jon Matthis:} So you know, here's hoping. Well, you listen to all the awesome things. We're looking at your, you know, on video in this video call, your background of tools and skeletons and I see like a draw mannequin over there.

\textbf{[00:03:53] Interviewer (Melissa Boswell / Hannah O'Day):} Can you tell us clear you're a person of many interests but we'd be excited to hear about some of the projects that you're working on right now.

\textbf{[00:04:01] Jon Matthis:} So the Free Mocap was the big one or whatever, but kind of like a traditional line research because the Free Mocap is like effectively a Covid project that's kind of, you know, come to where and the air.

\textbf{[00:04:12] Jon Matthis:} So like I have like a traditional line sort of NIH funded research program on visual control of foot placement during locomotion and that was sort of how do you get the visual information that you need to put your feet where it needs to go. And we have some projects brewing with that sort of fun, like augmented reality, projected ground planes and large scale traditional markered motion capture and integration with eye trackers and laser skeletons, data types, which is what I call this like full body skeleton data with like lasers shooting out of the eyes showing where people are looking. And I have Google, whatever my name is, I Have videos. And that was kind of, that was the traditional line research that I was sort of, you know, fully planning to devote everything to in roughly January of 2020, which is in retrospect a very funny time to do planning. In addition to that line, there is the Free mocap project which is, you know, free open source markerless motion capture system for everyone.

\textbf{[00:05:09] Jon Matthis:} And we're going to be talking about that. I know, but sort of like in that space we're kind of working. There's like a bunch of stuff going on with it. One is like the core software development, like making the thing, making it good, making it better, making it work, validation studies, sort of comparing it to my like you know, gold standard classical motion capture system. Recordings, connections being drawn from that to like the animation community through open source softwares like Blender and sort of making sure that the thing that's being built is useful for both for scientists and clinicians as well as like animators and game designers because they need MOCAP2 and also just like it's now at a place where like it works and you can do cool stuff with it if you know how to use it.

\textbf{[00:05:48] Jon Matthis:} So we're also doing cool stuff with it, which is pretty fun. So actually like using it to do researchy kinds of explorations. That about fills my day. That's it sounds like enough to fill a few days. It's interesting to hear your start to Free Mocap as a Covid project.

\textbf{[00:06:08] Interviewer (Melissa Boswell / Hannah O'Day):} I'm curious if you could share more about what during that time inspired Free mocap and then when did you realize that this was turning into a bigger project than you expected and you wanted to commit more time to it and really see it advance.

\textbf{[00:06:22] Jon Matthis:} I've been using motion capture for research since 2008 when I started with Brett Fajian of RPI who's a wonderful person and a great advisor and he had sort of inherited this motion captures like a Vicon motion capture system which is probably the biomechanics angle is probably from that. So I've been working with mocap for a while and specifically I think like the start of the idea of free mocap came when in 2017 when open pose released their, you know, they did like a. They had like a siggraph paper. What I consider to be like the first like viable markerless mocap thing that I have seen which is, you know, for, you don't know, it's just, you know, traditional motion capture, marker based motion capture.

\textbf{[00:07:05] Jon Matthis:} You put markers on the person, you specialize cameras to track them and then Yada, yada, expensive software. Openpose released this demo video and it was just RAW videos and, and convolutional neural networks, you know, trained on COCO dataset or whatever. And it would draw a skeleton over the person, like, you know, joint centers, connected dots, full body, hands and face of just raw video. And I remember seeing that, thinking, oh, you could build a motion capture system out of that. And then proceeded to like, forget about the idea for about four years.

\textbf{[00:07:38] Jon Matthis:} I mean, I played with it a little bit, but it was like, limited and like, I didn't really have the technical skills or the time to really like focus on it. And then when I got to Boston to start my profess in summer of 2019, which in retrospect is a funny time to start a professorship, I had a whole plan based on the traditional visual control of locomotion thing. And I was waiting for some equipment to come in. I was waiting for an eye tracker to come in. And the student that I had hired to work on the eye tracking part of it needed something to do.

\textbf{[00:08:06] Jon Matthis:} So I was like, hey, why don't you pick up this old open pose motion capture idea, get some pros and work on that. So it's like a stop gap till the eye tracker came in. And that was in January of 2020. And then the funniest thing happened and all of my actual research plans went out the window. I had a whole plan, I hired a bunch of people, we're gonna like build this equipment, do the foot placement thing.

\textbf{[00:08:30] Jon Matthis:} And then all of a sudden the world shut down and I lost access to my lab for an unknown amount of time. So I was thinking, well, I don't know when we're gonna get back. I don't know. And I gotta do something. These students need, you know, they need things to do, I need things to do.

\textbf{[00:08:46] Jon Matthis:} So the what at the time was called the GoPro Motion Capture Aspect of the lab became front and center. That's what we were working on. So that's kind of when it started. And then it has evolved very in a very interesting way since then. Where kind of like as the sort of the first phase of like the GoPro open pose mocap thing was starting to like achieve like viability.

\textbf{[00:09:10] Jon Matthis:} Like we had, we had reconstructed a skeleton. It was a lot of work, but like, it did work. So it was like, okay, this is a viable option. That was sort of roughly around like summer of 2020. So like George Floyd protests and like all sorts of like, you know, world is burning kind of feelings, kind of just like, you know, rising in equities and like, what.

\textbf{[00:09:28] Jon Matthis:} What are we going to do about it? And so sort of we were all this, like, deeply introspective state about, like, wait, like, what am I doing? Am I actually giving more than I'm taking? I don't think I am. So while I was sort of ruminating in those thoughts, I live my life through YouTube tutorials as a philosophy major.

\textbf{[00:09:43] Jon Matthis:} Turn, like, whatever. Science.

\textbf{[00:09:44] Interviewer (Melissa Boswell / Hannah O'Day):} Yeah. Self education tools. Yeah.

\textbf{[00:09:47] Interviewer (Melissa Boswell / Hannah O'Day):} The only real form of education. And I was, like, using.

\textbf{[00:09:47] Jon Matthis:} Looking at Blender, which is like an animation software, and I had been looking for a really long time trying to find tutorials on how to load motion capture data into Blender, and I wasn't.


\bigskip\noindent\rule{\textwidth}{0.4pt}\bigskip


\#\#\# Chunk 2 [00:09:45 - 00:19:44]

\textbf{[00:09:45] Interviewer (Melissa Boswell / Hannah O'Day):} Yeah, self education tools. Yeah, the only real form of education. And I was like using, you know, looking at like Blender, which is like an animation software.

\textbf{[00:09:52] Jon Matthis:} And I had been looking for a really long time trying to find tutorials on how to load motion capture data into Blender. And I wasn't finding it.

\textbf{[00:10:00] Jon Matthis:} And then I eventually found a tutorial series by a guy, Royal skies is his YouTube name. And he had a series about using a Kinect, like a Microsoft Kinect, get motion capture data. And the long and short of that tutorial series is that it was awful. Like it was a terrible experience. It was unreliable.

\textbf{[00:10:18] Jon Matthis:} Like you're using like this old broken software. It's unsupported by Microsoft. There's no way to really link more than one of them together if you don't want to pay someone thousand dollars for their software. And I realized that the reason why there was no available tutorials, free open source software using motion capture is that people don't really have access to motion capture equipment. Cheapest option out there.

\textbf{[00:10:40] Jon Matthis:} Cheapest options are like, you know, minimum, a few hundred if you want to do something like really don't want to use and like 5000 for like an IMU suit or like beyond that, it's like, you know, we're talking like five, six figures worth of equipment to get viable motion capture in that space. I was sort of thinking on the list of gross inequities of the world, like inaccessibility of motion capture is like pretty low down. But you know, it is on the list like. And so it was, it became a thing that I could feel like I could focus on. It's like, okay, well I'm going to provide this as like, you know, until that point I was thinking like, this is a tool that I will use for my research.

\textbf{[00:11:15] Jon Matthis:} And then that was first point of thinking that this is something that actually would be wanted and desired by the world at large and could actually sort of fill a need that existed. And so that was in summer of 2020. And then so we worked sort of in that expectation that we're trying to make, you know, a motion capture software for someone. And then in summer of 2021, I made a Twitter post that, you know, showing me like juggling on a B and getting tracked by Open Pose and Deep Lab Cut, falling back on a history of posting viral videos to the Internet. And it had got a lot of attention, got a lot, and people really liked it.

\textbf{[00:11:51] Jon Matthis:} And so since then, the past like whatever, six months, there's been this kind of growing Realization, you know, for me and people in my lab and whatnot, that, you know, this is something that people actually want. This is something that what we're doing is, you know, on par with what is out there. Like, you know, there's really even like the paid versions of things are like, really not that good, and they're really not doing anything differently than what we're doing. It's just a little more polished. It's kind of been this sort of growing realization that, you know, like, you know, incrementally sort of putting more and more effort towards this thing and sort of incrementally realizing that the potential future of it is bigger and bigger, which is, you know, exciting, intimidating, terrifying, you know, mix of all those sort of fun feelings.

\textbf{[00:12:34] Jon Matthis:} That was also honestly, an awesome journey and story through that experience. It's so cool you, like, mention your philosophy and sort of introspection and like, I love how you reflecting on both your place in the world and how you contribute to it and putting your work in the context of everything that's going on is just awesome. The pandemic did a number on me for sure. This would have always been like a strange time for me, like starting a professorship, moving to a new place. But I don't think that those introspective moments would have had quite the same impact if I had not been a professor, like, if I'd still been a postdoc.

\textbf{[00:13:09] Jon Matthis:} Because when you're in academia, you're on temporary positions for pretty much the entire time, like undergrad, grad school, postdoc. Those are all temporary positions. Then you become a professor and all of a sudden you're part of the team, you're part of the institution. And when you're in academia, you have privilege, but you don't really have much power until you become a professor. And so that was sitting in that sort of feeling of like, oh shit, I have, like, I have power.

\textbf{[00:13:38] Jon Matthis:} I was already kind of feeling weird about that, like walking into it and then all of a sudden it's like everything else starts going down. I was just like, okay. No, but really though, like, what am I going to do? With great power comes great responsibility.

\textbf{[00:13:50] Interviewer (Melissa Boswell / Hannah O'Day):} Like, I love the mcu, Spider man, but like, how are you going to drop Uncle Ben for real?

\textbf{[00:13:55] Interviewer (Melissa Boswell / Hannah O'Day):} I know, every time.

\textbf{[00:13:56] Jon Matthis:} So you talked a lot about the different open source softwares that are part of like Open Pose and Deep Lab Cups that are part of Free mocap. I'm wondering they you Open Pose is very known for being open. And what is the significance to Free mocap and why not open mocap or some other. Yeah, and maybe too I think you've talked a lot about the need for it, but maybe also the like overview of what it is.

\textbf{[00:14:22] Jon Matthis:} Exactly. I just am like wait, do we like really what is like the meat of freemo Cap? Why is it different from other macro less motion capture softwares that are available right now? Well the nice thing about freemo Cap is that it produces its own advertising. So if you just post this along with a video, everyone will know what it is before.

\textbf{[00:14:39] Jon Matthis:} So what it is, it's a free open source markerless motion capture system. So basically it's designed to work with effectively any cameras including like \$20 webcams and GoPros. And we haven't used more expensive cameras yet because that was part of that shift that you know, around that sort of summer 2020. My plan at the time was use GoPros and then when that works, shift up to more expensive research grade cameras. In that sort of shift I was like actually let's see what the cheapest possible system and it turns out that the floor on that is \$20.

\textbf{[00:15:09] Jon Matthis:} Like if you pay less than 20 dol webcam it doesn't work. I don't know why I don't understand cameras but, but it's effectively a way to use, you know, off the shelf hardware and open source software to achieve viable three dimensional. What I'm calling research grade motion capture data or research grade means that I would use it for my research and so I'm a scientist, therefore it counts. So that's the hope is, is basically the tagline is free motion capture for everyone. Like using the tools that are becoming available through the advance of technology and being made available by open source groups like openpose, Deep Lab Cut, anypose, any of these folks and basically trying to put them into a package that people can use.

\textbf{[00:15:52] Jon Matthis:} So back in 2017 I'm already a postdoc here, so I'm not untrained in anything but it was incredibly difficult to get openpose running on my computer. I saw it and I was like I need to use that, I need to figure out how to use it. So I went to their, you know, their GitHub and like I sort of tried to work through their instructions and it was months, it was an effort of months to even get it running. So there's kind of this tension right, because there's all of these people out there, amazing groups, you know, obviously big one Deep Lab Cut is another one that I, you know, sort of modeled My, you know, early stuff after and you know, a lot of groups that are making these open source technologies based off of like these absolute future shaping tech that's coming down the line. You know, machine learning, neural networks, convolutional neural networks, computer vision, what people sort of broadly call AI, but we don't use that term in this field because it's not specific enough.

\textbf{[00:16:45] Jon Matthis:} But there's all this amazing stuff that is available free and open source, but to use it you need to have this incredible. There's a really high tech barrier to being able to use it because people make the thing, they throw it up on GitHub and usually just call it a day. Which I now in retrospect, having worked on this for so long, I now know it's because making something usable by the general population is incredibly difficult. So right now Free mocap is basically just a duct tape job where I'm taking all of these great tools that are online that are available and free open source softwares and trying to duct tape them together into package that is sort of usable and user friendly and is accessible to people who don't have that kind of technical training and ideally could serve as an inroad for people to get that kind of technical training. Because if you're going to use the thing, if you design the tool correctly, using it teaches you how it works.

\textbf{[00:17:38] Jon Matthis:} That's kind of part of the aspirational aspects of the current design process. I like that philosophy of using it teaches you how it works. That's the hope, right? And it's interesting because you know, talking about free software, free open source software or whatever. So in contrast, if you are building a proprietary software, if your entire business model is I'm going to sell you this software that does markerless motion capture or whatever and I'm going to support myself by putting a very high, you know, five figure ish price tag on it.

\textbf{[00:18:08] Jon Matthis:} You actually have a vested interest in ensuring that the people who are using your software never actually understands how it works. All of the core equipment, all the core software, biomechanics is just linear algebra. All these things are just massive GUIs to solve linear algebra equations. And yet they cost, you know, \$5,000 for a sport license or \$23,000 for a license. And you know, all of these things can be done, you know, in Python or in Matlab.

\textbf{[00:18:39] Jon Matthis:} Matlab's also not open source, but Octave. I think that that's one of the great advantages of being a free open source software is that I can commit to the design that whoever uses the software for long enough will learn how it works. You know, I want you to understand every little bit and piece of it. But if I was trying to sell this thing as like a secret black box that does the thing that only I can do for you, I have a vested interest in the opposite. I don't want you to understand.

\textbf{[00:19:04] Jon Matthis:} I want you to know how to use my tool, but I don't want you to know enough to be able to recreate it on your own. Yeah, it's so interesting because I feel like a lot of the marker list motion capture systems, as you're saying, are quite challenging to set up and require an extensive background of coding as well. And it's hard because a lot of these are developed by scientists who do want to help and do want to make an impact and, and would love to see their project move forward into something that's usable. But I think there are some barriers to that in that Sometimes these are PhD projects and then a person graduates and it doesn't continue. Is there funding to support translating this tool into, you know,


\bigskip\noindent\rule{\textwidth}{0.4pt}\bigskip


\#\#\# Chunk 3 [00:19:30 - 00:29:30]

\textbf{[00:19:30] Jon Matthis:} Move forward into something that's usable. But I think there are some barriers to that is in that Sometimes these are PhD projects and then a person graduates and it doesn't continue. Is there funding to support translating this tool into beyond just you publish the paper and then you upload the code to GitHub. But it's like that's not really necessarily. Then someone's going to be able to download the code and put it together.

\textbf{[00:19:56] Jon Matthis:} It's so much more complicated than that. And having some really usable and simple tool to be able to do markerless motion capture, I could see could be really powerful. And I'm seeing that in one of my studies now is we developed a tool to do that just like a web app where people can upload videos and it automatically uses open pose to process it and gets results from that. We've had so many people reach out and ask if they can use it. When you get down to it, we're really just running open, calculating joint angles.

\textbf{[00:20:28] Jon Matthis:} But it's made me realize how hard that is for people and it's hard for me too to figure that out as well. So.

\textbf{[00:20:34] Interviewer (Melissa Boswell / Hannah O'Day):} And also with, you know, where do you process these? You need storage space, you need security. It's like, it's a lot like, how do you manage all of this?

\textbf{[00:20:44] Jon Matthis:} And to me, it seems like kind of an overwhelming. You hit a couple really important things. I think there. One is that I'll talk about the second thing you said first. The structure of academia doesn't support these kinds of activities.
\textbf{[00:20:58] Jon Matthis:} We operate on the basis of students who are supposed to be, you know, they're temporary, they're not allowed to stick around, they're supposed to move on and try to, you know, be another one of you, even though there's not enough spots in the world to do that in papers they get, you know, you publish them and then you walk away. And we really don't get rewarded for things like going back and forth, going back. And like, that's not, you know, the gatekeeping dinosaurs above us on the hierarchy aren't impressed by that because dinosaurs are never impressed by the future dinosaurs. So there's a problem of, like, the activities that are necessary to make things usable are not rewarded. In addition, making things usable is really hard.

\textbf{[00:21:39] Jon Matthis:} It's very difficult to make something that anyone can use. And so there's a term in the software world of academic spaghetti code, which is the way that we tend to code discovery mode thing where like, you start out, like, I don't know if you can make this thing possible and you do the complicated thing and it does the thing and then don't touch it. And that's how we do. And that's the scientific process, that's discovery. But to make the thing actually usable, it's a much longer and more challenging iterative process of, you know, you take it, you know, you thank it for what it taught you, put it in the ground and you start over again.

\textbf{[00:22:15] Interviewer (Melissa Boswell / Hannah O'Day):} You talk to people who work in the tech industry say hey, what is best coding practices? You actually spend the time doing like the soft skills of, I'm doing air quotes for listening of like user interface design, user experience design. Like yes, we in the sciences, we don't respect it because it's not fancy enough and like it's a challenge. So I don't know. The question is how do you do it?

\textbf{[00:22:38] Jon Matthis:} The answer is I don't. I've never known what to do. So it's mostly asking for help.
\textbf{[00:22:42] Jon Matthis:} You know, the only way that I've managed to sort of take my weird long meandering path is to just kind of like look for the people around me that seem like they're doing the things that I want to do and then just copy them. So groups like Deep Lab Cut were really, you know, they're like one of the few people.

\textbf{[00:22:58] Jon Matthis:} The technical capacity of Deep labcat is impressive but I think the reason why they've had the impact that they have is that they've built into year round it. I'll just say that GUI has a special place in my heart but they did the effort to make it accessible to people that can't run code from the command line and can't write their own Python scripts. And so because of that people use it. I saw them kind of trying to build that bridge between like high level machine learning, computer science and like biology, neuroscience kind of people and sort of thinking that I kind of want to start do that, build that same kind of bridge but instead of building it from like scientist to scientist, build it from like scientists to like general population, which is easier to do because I'm not actually making the tools. I don't know how like I don't know how to do like the machine learning, neural network stuff that it underlies these technologies.

\textbf{[00:23:44] Jon Matthis:} Like I kind of know how it works, but I can't make it myself. I know how to use it. So that becomes the whole job is like how do you wrap these things in a package that is actually accessible to people that are not high level computer scientists, kind of the like Netflix of Like just hosting, maybe not Netflix, but it's duct tape. I mean it's basically saying it's like, okay, here's a cool thing and here's a cool thing and I'm going to duct tape it together so that you know, you can use it as a whole as opposed to like, you know, having to use the pieces in their sort of, in the form that they sort of are presented like you. Because people present their stuff as usable as they can get it.

\textbf{[00:24:19] Jon Matthis:} And so then if I can sort of, you know, me and my lab, if we can go in there and figure out how to use it from that perspective and then just like smooth that out, like build a user interface for it and make it so that you don't have to, you can interface with it in a way that we recognize and like GUI's and button clicks and whatnot. That's pretty good. The other question you're asking about like hosting and privacy and longer term stuff, we will when we get there. I have friends who are sort of working with us now who are decade plus experience in like tech startup software development. So they know how that works.

\textbf{[00:24:50] Jon Matthis:} So I'll ask them when it comes up. So you're not creating or training a new model, you're packaging these other softwares in a way that's accessible. How do you imagine as there's multiple different markerless motion capture softwares popping up, they're improving and coming out with new versions, there's going to be a, you know, upkeep needed to keep that working.

\textbf{[00:25:16] Interviewer (Melissa Boswell / Hannah O'Day):} And I'm curious, I guess, for that in general, but then also as a free and open source software, how do you continue that upkeep and making it free to people. So this is very much a long term project for me.

\textbf{[00:25:31] Interviewer (Melissa Boswell / Hannah O'Day):} This is not a hit and run, drop it on GitHub and call it a day thing. I am thinking and planning.

\textbf{[00:25:35] Jon Matthis:} You know, I formed the Free Mocap foundation nonprofit. We're trying to get 501C3 status which is approved by the university and you know, software disclosures and, and all that. But you know, trying to think about the questions of like, how do you maintain long term viability and like how do you keep afloat?

\textbf{[00:25:52] Jon Matthis:} Especially considering what we just said about how like the standard trappings of academic science really don't support this kind of endeavor. One of the questions you asked is easy about the different technologies and the different tools that are coming out. And that's actually always been sort of a core part of the project is like, we want it to Work for all of those. Like we started with Open Post. You've been using media.

\textbf{[00:26:09] Jon Matthis:} We fished a mediapipe because it's a little easier to use. Not quite as accurate, but much faster and easier to use. E Blab gets in there, sort of like have like a train, your own network and there's a bunch of other options that people keep sending to me and I haven't had time to integrate them yet. But you know, the software is designed to sort of really focus on that sort of transition step I mentioned of like, okay, I want to see how your software works and I'm just going to reshape it into the way that it needs to be to work with free mocap thing. So that way, ideally as new things come out, then it's just you write the wrapper, you buy a little interface wrapper that says, okay, I'm going to take, you know, all of them taken video and spit out pixel locations.

\textbf{[00:26:46] Jon Matthis:} Some of them try to guess 3D from one camera, but it's not very good. But they are coming out, they are getting better and like who knows what the next big one is going to be tomorrow. And so the hope is to sort of have something that can interface with all of them and sort of provide that way.

\textbf{[00:27:01] Interviewer (Melissa Boswell / Hannah O'Day):} The other question of as a free software, how do you maintain viability? How do you get funding?

\textbf{[00:27:07] Jon Matthis:} How do you get money? That is a challenging question. And to that I have done the thing that I like to do, which is look around at other people who are doing it and see what they're doing. There are plenty of large scale open source projects, Blender, I've mentioned Python, Linux, you know, whatever, you know, gimp Free, cad, Inkscape, There's a ton of these out there. Alice Vision.
\textbf{[00:27:30] Jon Matthis:} So I basically stalk all of them. If most of them went, you know, numpy, matplotlib, like all these guys, I basically stalk their tax records, stock their websites, allow them to do annual reports. If they're a nonprofit, their taxes are public. Fun fact, any nonprofit, just Google name of the nonprofit IRS 990 and you'll see what their money, how they get their money and where it goes. Also works if you work for a private university.
\textbf{[00:27:54] Jon Matthis:} And it turns out what most of them do is a mishmash of sources. You get some large scale grants, small scale grants to sell your services. You know, maybe you have like maybe someday we start making our own cameras and we sell those. And so it winds up just being kind of like a patchwork network.
\textbf{[00:28:09] Jon Matthis:} And also like I still plan to use this to do my research.

\textbf{[00:28:12] Jon Matthis:} And my research is funded by the nih. So as long as they're like, hey, we like what you're doing. That was part of the kind of the thought process of this. It's like I already live, you know, as a professor, research professor, I'm already expected to get enough funding to support a research program. And as I'm sort of looking at this, it's like, you know, there's costs that are associated with this that are not associated with like standard academic research.

\textbf{[00:28:37] Jon Matthis:} But like I'm already kind of trained on like how to get money. It's just a matter of like kind of shifting the flavor a little bit and sort of trying to continue to build a free open source software. I just want to go back to one thing you said in there about filtering like, or about, you know, all these different people are coming to you and like they're excited about this, which is awesome, and excited to have their software integrated with yours. I guess I'm wondering what is the filtering process like and do you feel that's your responsibility for, you know, what are your thresholds for what's good enough to include?

\textbf{[00:29:06] Interviewer (Melissa Boswell / Hannah O'Day):} And also about, you know, how do you be mindful about being inclusive?

\textbf{[00:29:10] Jon Matthis:} Probably want to incorporate all of these different, you know, pose detectors that can do anything from, or sorry I should say video analyses that can do anything from animals to humans to xyz. Yeah. How are you doing that and what are maybe some of the challenges there too? So there's a lot of challenges. First of all, I just going to also just like for context with.

\bigskip\noindent\rule{\textwidth}{0.4pt}\bigskip


\#\#\# Chunk 4 [00:29:15 - 00:39:08]

\textbf{[00:29:15] Jon Matthis:} Do anything from or sorry I should say video analyses that can do anything from animals to humans to xyz. Yeah. How are you doing that and what are maybe some of the challenges there too? So there's a lot of challenges. First of all I just going to also just like for context of like the dream of freemocat being like a, like an easy to use thing is aspirational.
\textbf{[00:29:35] Jon Matthis:} We are not there yet. We're working in that direction and we'll get there sooner rather than later. But that is kind of the goal. And so in terms of that level of inclusivity that's kind of. It's a continuous process.
\textbf{[00:29:45] Jon Matthis:} That's right. Vetting potential algorithms that get included. There's a really nice sort of filter built in right now which is like I really don't have the time to do that for most. So you know, I think that there's a futures where there's more time and more, more effort, more people working on this. But for now you know we're focusing on just like core capacity.
\textbf{[00:30:04] Jon Matthis:} And for that we're currently using any pose for the camera calibration.
\textbf{[00:30:07] Jon Matthis:} Open pose is supported but we also using MediaPipe now which is a Google project, which is cleaner, easier code and then Deep Lab Cut is sort of under supported at this point because they haven't focused on it recently. But it's kind of, it's a build your own, you can train your own network so you can kind of track whatever you want. So the questions you asked will come up eventually. But for now we're kind of focusing on like the core capacity specifically now is like 3D motion, 3D full body, hands and face of a human.

\textbf{[00:30:33] Jon Matthis:} Accurate enough that like I as a researcher will look at it and think that's something I could do my science with. But you're not wrong. I mean there's a lot of issues of inequity here. Like you know any neural network based is on what it was and if it's a camera then it can see things like your skin tone. So you must be really mindful of these.

\textbf{[00:30:53] Jon Matthis:} One of the nice things about Media Pipe is because it's a large scale Google project, they have you know, a section of the website about like you know, skin tone and things like that. It's a legitimate concern. Good that people are looking into it. There's a documentary called Coded Bias on Netflix which is nice but also like at this point in development like, like I'm not really in a place to be like changing the underlying algorithm. So there's a there's a little bit of trust going on there that the people who are making them are kind of doing their best.

\textbf{[00:31:21] Jon Matthis:} Speaking of trust, I can't say trust and Google in the same word. That's not a good idea. Like, I am absolutely concerned about like the thing. So here's the thing about markerless motion capture. It's going to happen somehow, somewhere.

\textbf{[00:31:37] Jon Matthis:} Like we have the ability to track people accurately through cameras and you better believe we're going to do it as a culture. You know, technology advances autonomously and I absolutely am concerned about what free mocap would look like if it was Facebook mocap. You know, like there are massive privacy concerns there. All I can really hope is that, you know, being making a free open source version of this thing, which is absolutely going to be a technology that is a part of our future, hopefully that will provide some sort of protection from a potential world where that technology developed by one of these mega tech corporations who have proven time and time again that they cannot be trusted. I appreciate your points about the inequities sometimes in machine learning models.

\textbf{[00:32:27] Jon Matthis:} And I've seen that in our study as well. We see women who are wearing long skirts that it's having a hard time with that depending on people's weight, it is less accurate. And yeah, it's frustrating to see too, as a researcher, but then knowing that we've only been training on this certain types of data and now if you are able to create the software or a GUI or a way that we can more easily capture video and do markerless motion capture, then perhaps we can also start collecting more data. And I would hope that maybe those companies would then want to incorporate those and retrain their models and do things that are more inclusive and we continue to improve in that way. So I appreciate the freemo cap will I think contribute to that potentially as well.

\textbf{[00:33:18] Jon Matthis:} I mean, one of the things that I found really is helpful, but the only thing that's really helpful is when you're interacting with a new discipline that you don't know much about, to learn the culture of that, of those researchers, what they value, you know, how they choose their questions, what they consider an explanation, what does their data look like. And so as I interact with the computer science community, the research side of computer science that are actually building these algorithms, the corporations aren't building the algorithm. These are being done by labs and then bought by corporations. That community is driven by benchmarks. Either you do something like an open pose where you come out like, hey, I'm doing something Literally no one else can do.

\textbf{[00:33:53] Jon Matthis:} You know, most people don't do that. So what they do is they, they compete on benchmarks. And so you say, oh Well I got 94, you know, you got 94\% on the Coco data set and I got 95.6. And therefore I like literal like leaderboards and say like, you know, there's in bold and like that's how that community considers progress. So when you have that kind of methodology for, for establishing who works on what, it has these huge blind spots because as a community to computer science, they move really fast and they have no attention span.

\textbf{[00:34:24] Jon Matthis:} So like a lot of them kind of are acting like two dimensional tracking is a solved problem because it's a little like as quote someone, like it's a little too close to the pixels. You know, they're always trying to scale up. Now they want 3D, now they want volumetric tracking, now they want faster, they want us to run on your phone. And they're not trying to make it more accurate, they're not trying to make it more inclusive. They're not trying, you know, because.

\textbf{[00:34:50] Jon Matthis:} And I think the reason why is because they don't have, it's not obvious to them that that's a need. So I'm hope one of them in the myriad of hopes around this project is that if, you know, we can come up with definitive methods to say, hey, look like on these metrics that matter, here's where you all stack up. If there can be a quantitative measurement of those kinds of things, here's hoping that the people who actually have the capacity to drive that technology forward will start working on it or I'll get the funding to hire the students.

\textbf{[00:35:20] Interviewer (Melissa Boswell / Hannah O'Day):} Thank you so much for sharing all that. This has been really interesting.

\textbf{[00:35:25] Jon Matthis:} We're really excited about the future of markerless motion capture and the ideas that you're posing and we hope we do see we're rooting for Free MOCAP in making some these amazing technologies more usable. And having a team behind that is really exciting to hopefully push that forward. Where that Blender, which is a free and open source computer graphics software, has been a big inspiration behind Freemo Cap. And we've heard you bring up Blender a few times as you've been talking about us and the story behind Freemo Cap.

\textbf{[00:35:56] Interviewer (Melissa Boswell / Hannah O'Day):} Can you share a little bit more about that inspiration?

\textbf{[00:36:00] Jon Matthis:} And then this connection between human movement and 3D animation, which is Blender. So Blender has been, you know, one of the people I like to sort of pretend copy. I think that, you know, My interest in them comes from kind of twofold. One is like what Blender is and what it can do.
\textbf{[00:36:15] Jon Matthis:} So basically like if you look at like an animation, you look at like, you know, Toy Story or whatever, you know, 3D animated, soft animated movie, you want to look at.

\textbf{[00:36:25] Jon Matthis:} What you're looking at is artists who have used a GUI to have an incredibly fine grained control over minute aspects of 3D space, which is an incredibly powerful that. You know, I've been trying to do that from like code and sort of raw level linear algebra for years. And it's knowing things at that level is very valuable, but it's limited. There's only so much you can do. So I think that having the tools that are available to animators, available to scientists, will make scientists better scientists and will also connect that artistic side of technology and 3D animation to the sciences and biomechanics that are happening.

\textbf{[00:37:04] Jon Matthis:} Other thing that's inspiring about Blender is that it is a great example of a large scale, free, open source software that has its own legs and is growing in really interesting ways. I look at that and I think of that sort of large scale collaborative endeavor with an actual end goal project that is usable and useful for general population. And I see that as like, that's a better model for science. That is how we should be doing science. I think the open source community is what science and academia pretends to be like.

\textbf{[00:37:38] Jon Matthis:} We are actually very siloed. We don't really work well together. We don't work on large scale collaborative projects. We do papers and piecemeal stuff because we're all fighting for survival in a world. So Blender has been a particular inspiration, both because of the capacity that it affords and also for what it represents.

\textbf{[00:37:55] Jon Matthis:} And again, it's been so nice to hear that echoed throughout this conversation. Like not only the awesome technology and accessibility, you know, things you're doing to make it more accessible, but also the thoughtfulness and intentionality that you have behind it. And I think that is a little bit on that artistic and creative side, like in your philosophy. And like, I don't know, it's just been really awesome to kind of go in and out of all those different perspectives in this conversation. So we've really enjoyed it and enjoyed talking with you.

\textbf{[00:38:24] Jon Matthis:} If people want to learn more about your work or more about Freemo Cap, how can they do that? So we have a lot of ways. Freemocap.org is the website, Twitter is the thing. So twitter.com freemo cap is where I do most of my, like, communication there. There's also Freemo Cap accounts on Twitch.

\textbf{[00:38:42] Jon Matthis:} There's a Discord server that you can get to through. Through the Twitter, the website. I don't know, you find all these things. So the website, we got Twitter, we got Twitch. There's technically a TikTok.

\textbf{[00:38:51] Jon Matthis:} I don't use it very much. There's not Facebook page and Instagram and I think there's a subreddit. If you want to send us those links, we'll add them all in the description so people can find them.

\textbf{[00:39:01] Interviewer (Melissa Boswell / Hannah O'Day):} Well, thank you so much, Jonathan. This is amazing and a great way to start our day.

\textbf{[00:39:06] Interviewer (Melissa Boswell / Hannah O'Day):} Well, thank you so much for the conversation. A lot of fun.


\bigskip\noindent\rule{\textwidth}{0.4pt}\bigskip


\normalsize


\chapter{VFX Futures Podcast Interview}
\label{src:S3}

\begin{framed}
\small
\textbf{Source marker:} S3 \\
\textbf{Date:} July 13, 2022 \\
\textbf{Source:} YouTube Video / Podcast
\end{framed}

\small
\# Transcript: 2022-07-13-The FreeMoCap Project VFX Futures podcast

\#\# Source Information

- \textbf{Source Type:} YouTube Video
- \textbf{URL:} https://www.youtube.com/watch?v=Bqt8ZC5C4h8
- \textbf{Video ID:} `Bqt8ZC5C4h8`

\bigskip\noindent\rule{\textwidth}{0.4pt}\bigskip

\textbf{Total Duration:} 00:34:34

\bigskip\noindent\rule{\textwidth}{0.4pt}\bigskip

\#\# Full Transcript



\begin{quote}\itshape
\textbf{Note:} Speaker identifications below are approximate estimates
derived from automated analysis of speech content and may contain errors.
Timestamps for mid-block speaker changes are estimated proportionally.
\end{quote}

\textbf{Speakers:}
- \textbf{Ian Failes}
- \textbf{Jon Matthis}

\bigskip\noindent\rule{\textwidth}{0.4pt}\bigskip


\#\#\# Chunk 1 [00:00:09 - 00:10:00]

\textbf{[00:00:09] Ian Failes:} Hi everyone. Welcome to this latest episode of VFX Futures. I'm Ian Fails from befores and afters and today we're going to be talking about free open source markerless motion capture with free mocap. To help me do that, I'm joined by John Mathis. Hi John, how are you?

\textbf{[00:00:29] Ian Failes:} Hello, how are you? I'm good. John, whereabouts are you in the world at the moment? So I'm currently in my apartment in Boston, Massachusetts, United States of America, and I'm recording from Sydney. So we're on opposite ends of the world.

\textbf{[00:00:45] Ian Failes:} But you know, this is how we connect these days, Zoom and probably for a long time to come, I think. I suppose we'll see. John, a few weeks ago I noticed a post that you had made on Twitter and it sort of filtered around the world actually on LinkedIn and some other places about free mocap. Now I want to talk exactly about what that actually is later, but what might be interesting is to hear about yourself a little bit, where you come from, what you do, and then we'll sort of tell the story of how FreeMobCap became a thing. So that's great.

\textbf{[00:01:28] Ian Failes:} What do you do?

\textbf{[00:01:29] Jon Matthis:} So I am presently a professor studying human movement neuroscience, entering into my third year. And my, so my background, my bachelor's degree is in philosophy. And then somehow made my way into a neuroscience lab studying locomotion. And especially during my, you know, got my PhD with Brett Fajian from RPI and then in my postdoc I was working with Mary Hayhoe at University Texas at Austin.

\textbf{[00:02:03] Jon Matthis:} And especially during that time I started getting into an area of like sort of building my own methods to study human movement, especially in the natural environment. And my postdoctoral work was people walking outside over rocks wearing eye trackers and wearing kind of a, you know, IMU based motion capture systems sort of analogous to like a Rokoko or Xsense. And so during that time I kind of developed a sort of more technical ability to kind of like hack apart stuff and do some basic computer vision. And that sort of, that research project got me this position here in Boston. And now I am sort of into a new space of studying human movement.

\textbf{[00:02:53] Jon Matthis:} And then sort of the way that the timing of the past, I guess coming up on two years now, was right around the time that I was sort of starting to get my feet under me in Boston. And sort of the lab was built and sort of like about ready to get started. The pandemic hit. And so we all got kind of booted back to our homes. And this Project of sort of like I had been interested in markerless motion capture since, I know, since Open Pose came on the scene in 2017.

\textbf{[00:03:25] Jon Matthis:} And I've always had in the back of my mind like this could be used to just build a motion capture system. Because I've been working with mocap since like 2008 and you know, time was right. So put the time in.

\textbf{[00:03:39] Ian Failes:} Yeah, I guess for people who don't know what. What does the markerless side of motion capture mean?

\textbf{[00:03:46] Jon Matthis:} Because we're very familiar with suits. But. But maybe just explain for people who. Not who are not too sure. Sure.
\textbf{[00:03:52] Jon Matthis:} So you're the most. The image of motion capture that most people have in mind is sort of like the Spandex elastic suits with the reflective dots on it. There's other forms of motion capture, but that's kind of a classic one.
\textbf{[00:04:05] Jon Matthis:} So those dots are markers. They're usually the retro reflective markers that are designed to be picked up by the camera.

\textbf{[00:04:13] Jon Matthis:} So the cameras are usually not like your standard like color. Color video cameras are infrared and the markers and cameras kind of work together to make it kind of easy for the system to reconstruct the three dimensional points. And then the output is a bunch of, you know, 3D dots floating around. You connect the dots and sort of, you know, fit that to your rigs and all that kind of stuff. And that technology has been around, you know, decades, many decades, forever.

\textbf{[00:04:41] Jon Matthis:} Feels like. Yeah, I'm trying to think of like the earliest thing I can think of would be like Nikolai Bernstein, who was a, who was a Russian researcher in like the 30s and 40s who was drawing on top of images. But arguably this goes back to like Muybridge and the picture of the horse running across that was a bar bet by Leland Stanford, which sort of the origin of both video and motion capture. So that's. So then markerless motion capture, in contrast, is the idea of recording the same kind of information of sort of precise recording of a person's movement without the benefit of markers in the image.

\textbf{[00:05:23] Jon Matthis:} So when you put. Because marker based motion capture, the subject has to be wearing special stuff in order to make the reconstruction easier. And markerless systems are basically like far more advanced computer vision these days almost. These days are pretty much all using sort of neural network deep learning based methodologies. And that technology is getting to the point where you can just feed these neural networks RAW video and they will spit out pixel locations.

\textbf{[00:05:57] Jon Matthis:} It's like, oh, here's your eye, here's your shoulder, here's your elbow, and that's Strange magic.

\textbf{[00:06:09] Jon Matthis:} It's the kind of thing like I don't, I kind of understand the underlying technology, like I understand it enough to use it. But it is sort of an emerging area that basically the technology of people in my field don't tend to use the word AI all that often, but the sort of common term would be AI. AI based tracking methods to perform tasks that used to really only be possible with dedicated hardware in terms of like specialized markers and specialized cameras. And so that's markless in a nutshell. So I again, the history of that, I couldn't tell you the full history, but I noticed it in 2017 when Open Pose, which is from some lab who, I can't remember, I don't know the names of the people, but it was from Carnegie Mellon released like a, you know, a demo video of people like dancing and they're having the skeleton drawn on top of them.

\textbf{[00:07:09] Jon Matthis:} And I remember seeing that and thinking, that's the future of motion capture. I need to learn, I need to learn how to use it. And so I went to their GitHub and went through their.

\textbf{[00:07:24] Jon Matthis:} I discovered at that point that these, the people that make these tools are not speaking to the general population, they're speaking to other advanced computer scientists. And so I basically then spent the next three or four years trying to figure out how to interpret the documentation in these GitHub tutorials in order to be able to produce something that looked as cool as what they were showing on their demo vid.

\textbf{[00:07:51] Ian Failes:} Yeah, right.

\textbf{[00:07:52] Jon Matthis:} We will get to what Free mocap ultimately has become or is at the moment, which as I said is a free and open source markerless motion capture system, but importantly uses very cheap hardware to make that happen. We'll get to that.

\textbf{[00:08:08] Ian Failes:} But John, in that three or four year span in which you got excited about the possibilities of markerless mocap, what kind of experiments or things did you do to help yourself learn and find out more about it in the community? Sure. So I guess around at that time still today, there was kind of two sides of the work I was doing.

\textbf{[00:08:33] Jon Matthis:} The primary work I was doing was the NIH funded visual control of locomotion stuff using basically IMU based motion capture and eye tracking to record people's eye movements and foot placement as they're walking over terrain for all these health applications. That's where my funding came from.

\textbf{[00:08:56] Jon Matthis:} That's the papers I was publishing at that time and that's where most of my work went into.

\textbf{[00:09:02] Jon Matthis:} And sort of on the side I was playing around with Open Pose, which was that first software and really struggling with it, because it's really hard to get these things working if you don't know what you're doing. And I guess. So I was playing with open post. I was finally getting it to work. I was sort of learning how the data was being saved.

\textbf{[00:09:24] Jon Matthis:} And I was also making these sort of center of mass, sort of biomechanics analysis gifs that I was posting on Reddit and discovering that if I made one of these things and posted to Reddit, it would get, you know, a couple tens of thousands of points, which is cool.

\textbf{[00:09:40] Ian Failes:} Yeah. But what were they? They were gifts. They were, they were gifts of.

\textbf{[00:09:45] Jon Matthis:} So the, the first one was a, an athlete named Simonster. She's a French parkour guy. And I saw the animation of him doing this really intense handstand where he was.

\bigskip\noindent\rule{\textwidth}{0.4pt}\bigskip


\#\#\# Chunk 2 [00:09:45 - 00:19:45]

\textbf{[00:09:45] Jon Matthis:} So the first one was an athlete named Simonster. He's a French parkour guy. And I saw the animation of him doing this really intense handstand where he was just. He set up his cell phone, he was doing a handstand, going down and pushing himself back up and doing all this amazing stuff.
\textbf{[00:10:06] Jon Matthis:} And I realized watching that gif, that because of my training in biomechanics and robotics and all the sort of education I had, I had a certain visual intuition looking at that animation that's like, I'm looking at this.

\textbf{[00:10:23] Jon Matthis:} I know where his center of mass is. I know center of mass, basically this like the summation of a person's mechanics, you know, sort of center of gravity, center of mass, same thing. And I know that, you know, it has to be within the boundaries defined by his hand because, you know, he can't pull on the ground, he can only push it. So we. So I knew, watching that, like I knew what, what the mechanics of that situation were.

\textbf{[00:10:47] Jon Matthis:} And I realized that I also knew how to draw that intuition on the animation. So I went home and I opened up matlab, which is the only way I knew how to code at the time. Since learn Python and manually frame by frame, drew X's over each joint, hundred, couple hundred frames through, calculated the center of mass, calculated the base of support, which sort of defined by his, like hands. And you know, lo and behold, it popped out exactly the way it had to pop out because of physics. And, you know, took me like two, three hours.

\textbf{[00:11:23] Jon Matthis:} And I woke up the next morning because, you know, you got to post in the morning on the east coast, that's what everybody knows, and posted it and walked to work, which was like a 10 minute walk at the time. And when I got to work, it had like 3,000 points. And I was like, oh, shit, I guess, I guess that's a thing.

\textbf{[00:11:41] Ian Failes:} Yeah, people are interested. Wow.

\textbf{[00:11:43] Ian Failes:} I guess people are into it. Yeah. So then I made a couple. I made, I don't know, maybe half a dozen or more of those over the next couple years, each one increasing effort from the one before it because I can never do the same thing twice. And that every one of those was hand labeled.

\textbf{[00:12:02] Ian Failes:} I would go through every frame and just click, you know, hundreds of times to label the dots and thinking all the while, wouldn't it be great if there was an automated system to do this? And then, you know, that's. I think that's kind of the basis that things kind of grew out of, sort of. I kept making these animations, I kept learning More and more technical skills. And then all of a sudden I had things kind of aligned in the right way and kind of here we are.

\textbf{[00:12:40] Ian Failes:} Right. So again, perhaps for someone who doesn't know how that really works technically, how did you take what your experience with those GIFs and turn it into what effectively is a motion markerless motion capture process?

\textbf{[00:12:59] Jon Matthis:} So the gifts were always kind of, there was always a challenge because there was always a single camera. Because it's just, I just find videos online. So it wound up kind of, there are certain constraints about the kinds of stuff you could do. Like it has to be kind of a straight on shy, you know, it can't be like up and to the right, which is how people like to film stuff. If the person's moving, if the camera's moving, you know, there's challenges there.
\textbf{[00:13:26] Jon Matthis:} But so mostly what I was learning from that was how to work with video, how to work, you know, basically doing like computer vision based programming. The only. And you know, I was doing, working with a single video with hand labeled data points. And so, but, but keep in mind that at this point I was already getting a postdoc. I had already gotten a whole PhD where I was doing research using traditional motion capture, like a Vicon system.
\textbf{[00:14:00] Jon Matthis:} So I had already kind of internalized like the basic geometry of like oh, you have all these cameras and they're calibrated and if you project you can get the 3D stuff.
\textbf{[00:14:10] Jon Matthis:} So I was kind of, it was a lot of kind of percolation happening where I had sort of this long experience working with motion capture, you know, bicom based stuff in my grad school. And then in my postdoc I was working with like imu, like sort of like a suit that you wear as opposed to having cameras and then making these sort of like video based things. I was also working with eye tracking which is all video based. So sort of like the pieces started to come together and then also like another important factor there and the reason why you saw the video was I started getting, I have some experience in like publishing like viral video clips online.

\textbf{[00:14:50] Jon Matthis:} Like what is the Internet? Like how does it like to see stuff like what are the things that people respond to, what are the things people don't respond to? And that is its own weird skill that I wouldn't have expected to pick up. But you know, and then again the pandemic happened and some, you know, the things came together. So all of those factors of like basic understanding of motion capture, basic understanding of human movement, you know, basic computer vision Basic understanding of how to like, market on the Internet kind of came together into this system.

\textbf{[00:15:28] Ian Failes:} And when did it become free mocap? What was your thinking behind making it open source and giving it that name?

\textbf{[00:15:35] Jon Matthis:} So that was an interesting moment in my life because basically, so like I said, I had been sort of thinking about making a system like this for years and just never really having the time to put it together. And when the pandemic happened and my sort of, well laid, my whole research plan basically just collapsed on itself. I had a bunch of students working for me and I was like, all right, well the only thing that we had, and one of the students I had given this task of like, let's use GoPros and we'll just, we'll see if we can get this to work.

\textbf{[00:16:14] Jon Matthis:} And my plan was like, he would work on that while we were waiting for some equipment to come in. And then we sort of focus on like the quote unquote, real work. And then the pandemic happened and all of the plans that I had, every one of them evaporated except for this one. This was the only one that we could actually work on from home.

\textbf{[00:16:34] Jon Matthis:} So we all kind of worked together to make this system using GoPros. And the plan at the time was get it working using GoPros and then steer in the direction of sort of more like fancier, expensive, like research grade cameras.

\textbf{[00:16:54] Jon Matthis:} And right around the time that it started working with GoPros, the George Floyd protest started happening in America. And all of a sudden I was sitting here introspecting pretty hard about all the sort of rights, privileges and responsibilities that I am sitting in now as a professor, as a white man in power in America, with all of this scientific and technical knowledge that the American taxpayer paid for. And I started kind of thinking, well, maybe it's not the right move to steer this in the direction of more expensive cameras. What if I steered in the direction of the cheapest possible cameras? Because also around that time I realized there really aren't a lot of options out there for low cost motion capture.

\textbf{[00:17:47] Jon Matthis:} It's like people are still talking about the Kinect, which is like a dead technology that has been unsupported for like 10 years. And I'm sitting here, you know, I discovered Blender, which sort of really opened my eyes to like the power of the open source community.

\textbf{[00:18:04] Jon Matthis:} And so, you know, so I was sitting there like, okay, I remember I was looking for, like, I was looking for tutorials on how to load motion capture data into Blender, and I wasn't finding them. And then eventually I found there's this guy. Royal Skies is like a YouTube guy who finally made like here's how you use Kinect for Blender. And watching through his videos and especially like his final sort of like wrap up video is when I realized the reason why I have been finding mocap tutorials for Blender is because the people who are using Blender can't afford motion capture in almost any form. And so, okay, so there's this need, there's this gap.

\textbf{[00:18:45] Jon Matthis:} And in terms of like, you know, meaningful, like social problems, inaccessibility of motion capture technology is maybe not at the top of the list of like injustices in the world, but it was one that I could, you know, directly affect from where I was sitting. So it was originally called Open mocap. And then I got a little more. I started reading more into sort of like the Free Software versus Open Software debate and you know, sort of thinking also there was already, there was already a repository on GitHub called Open Mocap. It's like a dead repository from like eight years ago.

\textbf{[00:19:23] Jon Matthis:} But then I was like, you know, I might as well just go pretty aggressive on the open source thing and call it freemo Cap, in my brother's words, if I saw Free mocap and Open mocap next to each other, I'd pick Free mocap first.

\textbf{[00:19:37] Ian Failes:} Well, it kind of works both ways, doesn't it? Because it's free as in open source. But then you're freeing the process up for.


\bigskip\noindent\rule{\textwidth}{0.4pt}\bigskip


\#\#\# Chunk 3 [00:19:30 - 00:29:29]

\textbf{[00:19:30] Jon Matthis:} My brother's words, if I saw free Mocap and Open Mocap next to each other, I'd pick free Mocap first.

\textbf{[00:19:37] Ian Failes:} Well, it kind of works both ways, doesn't it? Because it's free as in open source, but then you're freeing the process up. I'm assuming that's. You enjoyed that. Yeah, yeah, it's pretty on the nose.

\textbf{[00:19:52] Ian Failes:} But John, I think what would be great for readers, listeners, sorry as well, is like if they got their hands on it, what does it provide them? What do you do with free mocap? How do you set something up to shoot a mocap scene? So if you got your hands on it now, you'd have. It'd be a bit of a struggle if we're being honest, but if you got your hands on it in like a couple months to a year, it's basically you set up some cameras.

\textbf{[00:20:20] Jon Matthis:} Right now it's all focused on these like \$20 USB webcams.
\textbf{[00:20:24] Jon Matthis:} Hook them up to your computer, you know, point them at the subject of the scene, you know, push, record, hold up like a calibration board, which is like a checkerboard with some, you know, they're called fiducials, the like trackable patterns in the middle there and then do the movement and then what. Get what comes out of that is a reasonably accurate recording of the. Basically the skeleton of the person in three dimensions. So like, you know, I'll basically think about the three dimensional trajectory of each one of your joints in space gets output from this system.

\textbf{[00:21:09] Jon Matthis:} We haven't really done much by way of like the formatting, but the idea is to make it sort of compatible with your blenders and your Mayas and your Houdinis and whatever else, whatever the software you want with the idea being, you know, for me as a scientist, that's the data that you want to analyze, to look at whatever kind of neuroscience question you might have. But if you are an animator or a video game designer, those are also the motion capture assets you would use for your animation scenes or your video game assets, whatever it is that you're doing with that.

\textbf{[00:21:44] Ian Failes:} Right. And you've set this up. But as you mentioned before, it's.

\textbf{[00:21:48] Jon Matthis:} Unless I've got this wrong, it's leveraging on open pose and possibly other tools as well, isn't it?

\textbf{[00:21:54] Ian Failes:} Yeah.

\textbf{[00:21:55] Jon Matthis:} So Open Pose, MediaPipe and Deep Lab Cut or the. And then any pose is doing the camera calibration. So yeah, it's actually in terms of like, we're actually writing very little, like meaningful sort of heavy lifting code over here, little to none.

\textbf{[00:22:12] Jon Matthis:} It's basically the task that we're doing is the way that I think about it is like we're trying to provide a framework so that other people can take advantage of all these technologies without having to spend four years sitting on their couch banging their head against their laptop, trying to get the dang things working. Because like I was saying, like, there's all of this amazing stuff coming out of the computer science, computer vision community, but you got to think about the people who are making it. They're all living in survival mode too. They don't have the time. It's hard to make code usable by the general population.

\textbf{[00:22:55] Jon Matthis:} I'm mostly just trying to make that connection easier. This is a lot inspired by Deep Lab Cut, who I think have been doing a similar task to, to connect really like high level computational neuroscientists to the same kinds of community for, mostly for like animal tracking, for like neuroscience research. And so I'm kind of pulling a similarish move, but instead of connecting computational neuroscientists to, you know, computer vision researchers, I'm trying to connect general population to those same kind of communities and again, just trying to make it so that the general populace can take advantage of all these really amazing advances that are coming out. Which right now they are free, they are available, but there's this huge technical barrier to being able to use them. So I'm just trying to smooth out that transition a bit.

\textbf{[00:23:51] Jon Matthis:} And because it is leveraging on those other tools already in existence, does that mean, because you mentioned machine learning previously, that it leverages any machine learning implementations that the other tools have?

\textbf{[00:24:05] Ian Failes:} Yeah, oh yeah, it's. It uses them directly. Yeah. So which also means that it inherits their licensing.

\textbf{[00:24:12] Jon Matthis:} So open pose has the most challenging license wise. It's free for commercial stuff, but it's like Deep Lab code, it's open. And MediaPipe is a Google project, which is also open. But that's actually sort of a secondary aspect of this, is considering it as almost an educational tool to say, hey, here's like, so here you are, you're a person, you just want to record some people moving for whatever reason you might have to do that. And everyone has their own reason for wanting to measure people.

\textbf{[00:24:45] Jon Matthis:} So the goal for me as a scientist and as an educator is to make a tool that allows you to, to use these new technologies to do the thing you want to do with as little effort as possible. So that way, but also like expose to you as much of the underlying technology as I can so that you, who previously had no familiarity with this stuff, now have a little bit of familiarity and sort of the idea is that if you use this for long enough, eventually these things will just become familiar and you will sort of pick up some familiarity with this sort of like really important and emerging technology just by way of your own desire to do the things you wanted to do that involve recording humans. So that's something that I, you know, as a, as an educator, that's kind of an important part of this project is making sure that it does do both of those things. And it's also, interestingly, like something that as an open source developer, I am freely available to do because there's a lot of proprietary softwares that do similar stuff, but they always have to reach a point where they have to hide some level of their operation from their users, because their business model is to some extent predicated on their users never understanding how their software actually works. Whereas for me, I want you to understand every piece of it from top to bottom.

\textbf{[00:26:19] Jon Matthis:} That's a fun thing about working in this space. The other cool thing I think for the moment is the Blender ui. And that's, I guess, as you mentioned, part of the open source thing, but also that you felt it was accessible.

\textbf{[00:26:34] Ian Failes:} Are you looking to change that or open that up or allow it to be in other tools as well? Oh, absolutely, absolutely.

\textbf{[00:26:42] Jon Matthis:} Like, Blender is very much a, is, you know, I'm pointing straight at it right now from a sort of development standpoint.
\textbf{[00:26:50] Jon Matthis:} I've been focusing almost all of my efforts recently on just like the core components of it, the core Python bits, just because, like, it has to work and ideally it works in the most generalizable way that you can, which for me means Python. But now that it is starting to work and starting to stabilize, connecting it to Blender is, you know, or basically that community is a very big part of the next sort of phase. And I think the first part of that is just going to be like a file reader, just like reading the data into a Blender scene. But I have like half of a broken Blender add on that will continue to be developed.

\textbf{[00:27:36] Jon Matthis:} Because I think that, you know, I was very like learning, learning about Blender, really. I found, I found it to be very inspiring, just sort of like the story of how that software was developed. And it really, it struck me as just like the right way to do things. Like the open source community is what the scientific community pretends to be. And I think that we should really adopt more of that model and I think that, you know, one of the beauties of the open source community is that if you, if you are working on an open source project that's related to another open source project, you just combine forces.

\textbf{[00:28:11] Jon Matthis:} And so that's now that FreeMokApp is starting to be more of like a stable process, you know, table software that actually works. Connecting it to, you know, something like Blender is a very big part of hopefully next, you know, call it next, let's call it the next year. By this time next year there will be some meaningful blender integration. You can heard it here first. It's very early days.

\textbf{[00:28:37] Ian Failes:} But has, has anyone been experimenting with this and showed you any results?

\textbf{[00:28:45] Jon Matthis:} At this point I don't have any evidence that any other person outside of my lab has successfully gotten it to work, which is nobody's fault but my own because like the documentation isn't really where it needs to be. Like the, you know, the code is still a little spaghetti ish. But you know, I'm moving, I'm hoping to have something like an alpha release in the next couple next one to three months. And the goal of that is this is something that other people could reasonably use right now. It's still like you could use it, but it'd be, it'd be hard.

\textbf{[00:29:22] Jon Matthis:} And this has actually been, you know, with the release, you know, I have been contacted by some more experienced.


\bigskip\noindent\rule{\textwidth}{0.4pt}\bigskip


\#\#\# Chunk 4 [00:29:15 - 00:34:34]

\textbf{[00:29:15] Jon Matthis:} That other people could reasonably use right now. It's still like you could use it, but it'd be hard. And this has actually been with the release. I have been contacted by some more experienced software developers who are like, hey, this is a cool project. Can you please, can we help you can fix your code, please.

\textbf{[00:29:37] Jon Matthis:} And so I've had a couple meetings with some experienced software developers because I learned Python during the pandemic. Like I was a matlab developer before that and a philosophy major before that. So like this is all sort of very like self taught kind of like seat of your pants kind of stuff that you know, luckily is, you know, I have enough experience that it's not complete spaghetti, but it's definitely going to be the kind of thing like get an alpha out so that something can be used and then basically gut the whole thing and rebuild it in a way that can, that will be more, that can grow in the future.

\textbf{[00:30:17] Ian Failes:} Wow. I'm going to put a link in the show notes John.

\textbf{[00:30:21] Jon Matthis:} But what's the best place for people to check out a bit more about Freemocap? Probably freemocap.org I think I just got that domain recently and it redirects to a page about this, other than that, the GitHub. So GitHub.com johnmathis freemo cap, which has links to the Discord server, which is probably going like I don't know, 5, 600 strong at this point. And that's so. Which, you know, so those are, that's probably like, you know, the freemocap.org is probably the first one to go to.

\textbf{[00:31:00] Jon Matthis:} And then the GitHub is where the code lives and then the Discord is where conversations happen. Oh, I also stream on Twitch, Twitch TV, Freemocap. I've been doing that Thursday afternoons, 4 to 6 Eastern US time.

\textbf{[00:31:19] Jon Matthis:} I do actually love that you've sort of taken advantage of the viral or the sharing side of all this. It's nicely unusual and I think that's a fun thing that you're doing. It's been fun. I mean, you know, and it's kind of like, you know, to be perfectly honest, like I feel like if you're sitting here in 2021 looking at the future and you're no. And you don't feel like you're staring on the barrel of a gun, you should probably maybe pay better attention.

\textbf{[00:31:52] Jon Matthis:} So a lot of this kind of like viral sharing stuff is just kind of like you got to spread it out, you got to get it out to the people and you got to Give it as many hands as possible because I, I don't trust the ivory tower that I work in to do the right thing. So this is a lot of, you know, I'm thinking about, you know, I'm pitching, I'm a scientist pitching to the general population. And ideally I'm pitching to people at or below me on the social hierarchy, whatever that means. And often what that means is young people. Like, I feel like if I can get this into the hands of some young indie developers, some interested athletes, performers, dancers, whoever, give them this thing that they want.

\textbf{[00:32:38] Jon Matthis:} Everybody has a reason to want to record a person. And if accidentally, through the execution of their own desires, they learn a little bit about computer vision and AI and technology and science. That's probably a good thing.

\textbf{[00:32:59] Jon Matthis:} That's all I got sitting here as a white man in power in 2021. That's what I got. I like it.

\textbf{[00:33:06] Ian Failes:} Now, John, the most important question is in your demo video, are you the juggler on the balance board? Right.

\textbf{[00:33:16] Jon Matthis:} I am the juggler on the balance boarder. I got a lot better at juggling over the past year and a half.
\textbf{[00:33:25] Jon Matthis:} And yeah, like I said, I've, you know, I've been hanging out with circus performers for, for a number of years now. I am not on. Like, I have friends who are so far beyond my level, it's unbelievable. But I can hang and I can juggle and that's part, you know, like that's, that's how you. What I know about how to go viral is you put some.
\textbf{[00:33:48] Jon Matthis:} The way I, the way I do my thing is I put stuff out there that doesn't over explain itself. It's just kind of eye candy and you stare at it and the longer you look at it, the more you see. And ideally what you see is what you are interested in. There's enough going on there that some part of it is going to attract your attention. And some people are into the tech, some people are into the mocap, some people can't get over the juggling.
\textbf{[00:34:15] Ian Failes:} And I'm here for all of them. Nice.

\textbf{[00:34:17] Jon Matthis:} Well, it's been really fun talking to you about free mocap, John, and hope people can check it out. There'll be plenty of links in the show notes.

\textbf{[00:34:27] Ian Failes:} Thank you so much for sharing that info with me, John.

\textbf{[00:34:31] Ian Failes:} Really appreciate it. Yeah, thank you for your interest. Thanks for asking the questions.


\bigskip\noindent\rule{\textwidth}{0.4pt}\bigskip


\normalsize


\chapter{Dynamic Walking 2022 Conference Presentation}
\label{src:S4}

\begin{framed}
\small
\textbf{Source marker:} S4 \\
\textbf{Date:} June 16, 2022 \\
\textbf{Source:} Conference Presentation
\end{framed}

\small
\# Transcript: DW22\_Mattis, Jonathan S. - June 16th 2022, 12\_33\_15 pm [1\_ucmxydwq]

\#\# Source Information

- \textbf{Source Type:} Local File
- \textbf{File Path:} `C:\textbackslash{}Users\textbackslash{}jonma\textbackslash{}Videos\textbackslash{}DW22\_Mattis, Jonathan S. - June 16th 2022, 12\_33\_15 pm [1\_ucmxydwq].mp4`

\bigskip\noindent\rule{\textwidth}{0.4pt}\bigskip

\textbf{Total Duration:} 00:19:39



\begin{quote}\itshape
\textbf{Speaker:} Jon Matthis (conference presentation)
\end{quote}

\bigskip\noindent\rule{\textwidth}{0.4pt}\bigskip

\#\# Full Transcript

\#\#\# Chunk 1 [00:00:02 - 00:10:00]

\textbf{[00:00:02]} Cool. All right. Hello, everybody. So I started my professorship in the summer of 2019, and let me tell you, things did not go as planned, as you might have expected from the research I've done before. My plan was to show up in Boston and continue the kind of laser skeleton, visual control of foot placement stuff that I've kind of been doing for my whole career.

\textbf{[00:00:31]} And eventually we did in fact, get to that. That's the stuff that Trent was presenting earlier in the week, just sort of walking overseas. And it's going quite well. Trent, by the way, is going on the job market this year. So if you want someone with that kind of skill set, he's somewhere.

\textbf{[00:00:51]} His ultimate dream is to be a research professor at a state school in the Midwest. So. So if you know anybody, he would do well in a sort of psychology, neuroscience type of environment. Anyway, so what became. So the plan was to study laser skeletons, visual control of foot placement, sort of that classic kind of stuff.

\textbf{[00:01:13]} And all that kind of went out the window when like, you know, like, the situation started to unfold. And so what became the free mocap project started as a stopgap. It was just sort of something that we could work on as a group while we were waiting to get access back to the lab. And it quickly kind of took over the lab as sort of started being thing to do, which also happened to coincide sort of right around when we were starting to get like, viability of the project. Also happened to coincide with the start of the George Floyd protests, at which point I think we all kind of started to ask these kind of questions about whether or not our position.

\textbf{[00:01:55]} We were sort of really adequately using our positions of power to do the most good for the most people. And I kind of generally had some beef with my particular institution, but also just the whole institution of higher learning and education in general was starting to kind of concern me. Sort of the idea of this sequestration of science and sort of like gatekeeping of knowledge. Not sort of, not like nefariously. It's just kind of how the structure of things worked.

\textbf{[00:02:23]} And so let's see, what are we doing here?

\textbf{[00:02:29]} And we got them there. Missing one. Missing you.

\textbf{[00:02:38]} Because of that.

\textbf{[00:02:43]} And so we started asking kind of the questions of like, oh, no one should ever do a live demo. Just to be clear, it's a bad idea. But anyone who knows me knows that I really like to make my life as difficult as possible.

\textbf{[00:03:09]} Also good to have backup plans if necessary.

\textbf{[00:03:16]} A not you. Cool.

\textbf{[00:03:23]} Pretty sure the laptop camera's camera one.

\textbf{[00:03:28]} We'll find out shortly I was wrong. It is camera three.

\textbf{[00:03:37]} And that one has its eyes closed.

\textbf{[00:03:42]} All right, gotta restart.

\textbf{[00:03:48]} I've never done this before in my entire life.

\textbf{[00:04:04]} And.

\textbf{[00:04:09]} And so basically I began to ask the questions. Okay, camera zero is rotated.

\textbf{[00:04:17]} That one is upside down. No, that one's good. And that one. Oh, that's camera zero. Can solve that problem mechanically.

\textbf{[00:04:30]} Hello.

\textbf{[00:04:40]} So the idea became to try to figure out, as we're sort of getting things starting to work with the motion capture system, and things started working more or less, the question became, what is the cheapest motion capture system that someone can make that can produce research quality data? Where research quality here is defined as stuff that I personally would use for my research. Camera zero is rotated. There we go. And that's also a little dark.

\textbf{[00:05:18]} There you go. Nope. Numloc7 go a and so with that sort of idea of trying to find a motion capture system that's as cheap as possible, the most free mocap project was born. Because on the list of global inequities, inaccessibility of motion capture systems, it's probably not particularly high, but it is on the list. And so around that sort of moments of sort of introspection and things like that.

\textbf{[00:05:56]} What am I doing there? See, that's too bright.

\textbf{[00:06:04]} I encountered a very powerful concept of lift where you stand. And the idea is that when you encounter a problem in inequity that sort of causes you some sort of moral harm, that you don't just turn around and drop everything that you've done in the hopes of doing sort of like going off to join Greenpeace or something like that. I don't know shit about the climate. I know about how to make laser skeletons. And so the concept of lift where you stand is when you're having a problem, what you do is you look around where you're standing, you find something that looks like a handle, you grab it and you pull as hard as you can in what feels like the right direction.

\textbf{[00:06:41]} Because on the list of global inequities, maybe motion capture is not particularly high, but what is high is inaccessibility of science and technology. There are all sorts of insane DW live no, not plus live nonsense.

\textbf{[00:07:02]} Because as we have all sorts of insane new technologies coming down the pipe, sort of all sorts of markless tracking methods, machine learning, AI. The people who really stand to benefit from these tools often can't access them. They don't have access, they don't understand them. Even the ones that are free and online 8 are the people who try to make them freely accessible. We have about like an 80\% chance to success here.

\textbf{[00:07:31]} So people don't have the technical ability to use them. And even the people who do like people in this room who wants to spend the time sort of wrapping these things together into something that you can actually use to generate real behavior, real usable data. Okay. And so we sort of started doing this thing of trying to try and do the science, make a tool that's usable for me so I can enact my research while also remaining accessible to people without technical training and sort of sharing it out loud, sort of. Not only to the lovely people in this room.

\textbf{[00:08:07]} It's great to see you all again. But also to people outside. And so in the process, as we're sort of chugging along here, if it's going to fail, it's going to fail right here.

\textbf{[00:08:25]} We started building all of this sort of online infrastructure. I think we're good. Ish. We'll see. The skeleton might be a monster, but monsters can be friends.

\textbf{[00:08:37]} So anyone who's been sort of on Twitter has probably seen me sort of posting my ridiculous dancing gifs. This is the Free Mocap website. It kind of currently lives in this state of being as friendly of a GitHub repository as I can make it. But obviously there's room to grow. As you've probably noticed that this is a fully offline process.

\textbf{[00:09:00]} You probably would like for it to be real time. I would like for it to be real time, but I'm not that good of a programmer. But you know who is is my friend, Endurance Adehan. He has 16 years experience as a software architect for tech companies, and he's currently just accepted the position as CTO of the Free mocap Foundation, which is a nonprofit that I made. He gets no money, but he gets a nice title.

\textbf{[00:09:24]} So you can join. You can sort of follow. Here I am. It's on Twitter. I do Twitch streaming every now and again.

\textbf{[00:09:31]} I forgot your real name, but your username is Glitch Robot. Happy to see you in person. Here's the GitHub repository and here are the instructions. They're sort of as simple as I can make it. This is sort of a good time to remind people that my first degree is is a bachelor's degree in philosophy.

\textbf{[00:09:49]} So trust me, I have sympathy for people who have a lack of technical background. In the next coming months, Endurance and I will be restructuring the software.

\bigskip\noindent\rule{\textwidth}{0.4pt}\bigskip

\#\#\# Chunk 2 [00:09:45 - 00:19:39]

\textbf{[00:09:45]} Remind people that my first degree is a bachelor's degree in philosophy. So trust me, I have sympathy for people who have a lack of technical background. In the next coming months, Endurance and I will be restructuring the software into what we're calling the Alpha release. It's currently version like 0, 0.52, 0.1 and above are going to be a completely different story that will have some form of real time reconstruction. And what fun that will be.

\textbf{[00:10:14]} We also have a Discord server. No, you don't. Those are not good friends. Has about 800 people on it. And one of the biggest communities that's shown excitement about this project have been independent video game makers, independent animators, people who basically are making their little projects in their basement and thought that something like motion capture was outside of their realm entirely.

\textbf{[00:10:38]} And I say, well, in fact, with nothing but three garbage webcams, \$10 a pop, you can have your own decent motion capture system. And so here on this sort of free Mocap discussion, this is probably about a half a dozen to a dozen animators that sort of are communicating constantly about how you can reconstruct, how you can basically shape the data from free mocap into the kind of forms that they need to do their fun animation stuff. I'll show you some clips there a little bit later. Here's one. Here's Roald Dahl, who's made sort of a package in R to analyze the data.

\textbf{[00:11:19]} And then here's Manitoba Mike talking about analyzing the data of him and his son doing hockey slap shots. I'll show some of that in a second.

\textbf{[00:11:31]} And so this whole process of sort of trying to share the research as much as possible has been a lot of fun and is starting to now finally get to the point where it's producing data that I can look at and think, you know, maybe I can do something with that. So no promises about this particular one because there is like a really sensitive part in the middle there that may or may not have worked. So this might be a monster, but if we're lucky. Okay, grab in the Y. This is Blender, another free open source software.

\textbf{[00:12:07]} Rotate around X, grab on Z, click on U. Get closer and rotate around. Show me that material. Preview, please. And there you go.

\textbf{[00:12:25]} So that is a \$30 motion capture system recorded in 11 minutes or less.

\textbf{[00:12:39]} Okay, I kind of want to pass out right now, but I think I'm going to continue the talk. I forgot to put my other talk, honestly. I mean, I brought all the cameras, but of course, that's a fun skeleton. But you Know, I promise you laser skeletons. And in fact I have laser skeletons.

\textbf{[00:13:03]} So this is a recording that I got from my living room. It is. This is a skeleton with lasers coming out of its face, built with a bunch of voltages recorded on the back of some silicon wafers and some garbage webcams and, and a pretty nice people labs eye tracker and then just a crap ton of linear algebra. The video's not synced and that doesn't bother me at all.

\textbf{[00:13:28]} So this is the kind of stuff that it's similar to the kind of data that you would want to do, the kind of research that I would normally do. The accuracy is not nearly as good as a traditional motion capture system, but the cost is roughly five orders of magnitude less. So I can accept a little bit of noise in that regard. I forgot to start my timer. How are we?

\textbf{[00:13:50]} What's that? Okay, cool. That is less than I would have hoped. You said. So I'm going to be doing a demo and my students are going to be doing some presentations after this talking about the accuracy.

\textbf{[00:14:03]} But this is the. These are the number of visitors to free mocap by country. In particular, I'm excited about that one, that's Nigeria, because it is the Nigerian film industry and I really want to see some free mocap data in Nigerians films. The GitHub is approaching 2000 stars. This is a small.

\textbf{[00:14:23]} This is a small number of people who have been presenting, making their own free mocap stuff. This is one of my favorites from user Vandalo. I don't know why we're not playing. Oh, because we're not actually presenting. There we go.

\textbf{[00:14:40]} So this is free mocap data with someone who knows how to make animation. And this is from user manitobamike who has been setting it up in his. Set up a motion capture system in his garage and has been recording his son doing stuff. This was his first one. It's okay, it has problems.

\textbf{[00:14:58]} So then on Twitch I did like a review of it and these have been his latter ones. These ones in particular are looking really good. And look at this kid, he's learning motion capture, he's learning computer vision, he's learning machine learning, he's learning kinematics and who knows what else. And this was sent to me out of the blue on Twitter. This is from a high school in Ohio.

\textbf{[00:15:23]} The students found it on Twitter and with no assistance from their teachers, set up a free mocaps area, calibrated it with the Churucco board and then brought all their teachers in and said hey, do A little dance for us.

\textbf{[00:15:40]} And I just think that's the coolest thing ever. That's their principle.

\textbf{[00:15:47]} So who knows, who knows where this will go? We're just getting started. We're just now getting to that point where the core processing pipeline is stable enough that it's probably start. It's worth starting to actually do stuff beyond there. This is a photogrammetry scan of my apartment.

\textbf{[00:16:02]} Don't worry about that.

\textbf{[00:16:06]} And here's my limerick. Today I've presented a roadmap for a project we're calling Free mocap. You can build skeletons as markerless friends. And I'm giving way webcams with gift wraps.

\textbf{[00:16:26]} So these are. This is 45 webcams, which by my math is approximately 15 motion capture systems. I will be giving them away to anyone that wants them and maybe some people that don't. So come find me in the whatever thing later and get a motion capture system.

\textbf{[00:16:55]} What would you like to have?

\textbf{[00:17:00]} So we've been intentionally pushing it as low into the hardware requirements as possible. Chris is going to be presenting on using GoPros, sort of like wide field of view, wide resolution, higher frame rate cameras. There's actually some additional challenges that pop up from that, but ultimately. So you could also incorporate LiDAR, you can incorporate Stereo Camera, RGBD. The calibration should work for those as well.

\textbf{[00:17:27]} But sort of the part of the divergence that happened because I was originally using GoPros and then the plan was once it works on GoPros scale up to your flir, your basils, your sort of higher research grade cameras. And then around the time of the George Float protest, I was like, well, what if we just pushed in the other direction and what if instead of sort of pushing it towards research grade stuff, I push it towards the most garbage cameras you can possibly get on the assumption that if it works for the cheap cameras, it will also work for the good cameras, but not necessarily the other way around. So now that it is stable and there is that sort of minimal cost option out there now is when I'm starting to sort of explore sort of higher grade options. It's hard to get better than GoPros for outdoor stuff, for indoor stuff. I think adding depth cameras, lidar, that kind of thing, higher frame rate cameras will really, I think be very beneficial for getting towards dynamics and things like that.

\textbf{[00:18:25]} If you're recording 240 frames per second, you can sacrifice half of them so that the remaining 120 frames per second will have higher reliability and things like that.

\textbf{[00:18:39]} Do you need a human shaped target. Great question. So, we've been using pre trained models for humans, but we've designed the software to be fairly agnostic to the tracking software. So if you sort of watch the end of Chris's talk, you'll see an implementation of a Deep Lab Cut model. So Deep Lab Cut is sort of like a train your own network usually made for.

\textbf{[00:19:00]} Oh, that's the other kind of. The other part. Just finished processing. And basically, if you have videos of, I don't know, like a robot, and you were to sort of click on the frame to sort of train the network, you basically slot that in and have something that, as long as you have something that can track the person or the object in pixel coordinates, it will feed into the rest of the reconstruction pipeline and everything else should play nicely.

\textbf{[00:19:38]} Yeah. So, next speaker.

\bigskip\noindent\rule{\textwidth}{0.4pt}\bigskip

\normalsize


\chapter{HMN2024 --- Introduction to Class Lecture}
\label{src:S5}

\begin{framed}
\small
\textbf{Source marker:} S5 \\
\textbf{Date:} September 25, 2024 \\
\textbf{Source:} Class Lecture, Northeastern University
\end{framed}

\small
\# Transcript: 2024-09-25-HMN2024 - 00 - Intro to class

\#\# Source Information

- \textbf{Source Type:} YouTube Video
- \textbf{URL:} https://www.youtube.com/watch?v=cB6lWKBlEhE
- \textbf{Video ID:} `cB6lWKBlEhE`

\bigskip\noindent\rule{\textwidth}{0.4pt}\bigskip

\textbf{Total Duration:} 00:57:40



\begin{quote}\itshape
\textbf{Speaker:} Jon Matthis (class lecture)
\end{quote}

\bigskip\noindent\rule{\textwidth}{0.4pt}\bigskip

\#\# Full Transcript

\#\#\# Chunk 1 [00:00:00 - 00:10:00]

\textbf{[00:00:00]} So I'm now starting recording and starting to talk. So let's see, I gave myself notes.

\textbf{[00:00:12]} So this is a topics course, which, as I'm sure you all know, is typically. Topics courses are generally like they're off of the main schedule and there's some sort of other kind of topic that doesn't have like a proper class around it yet. And oftentimes those topics classes are taught to allow a professor, often a research professor like me, to kind of like highlight their own research in a way that sort of doesn't fit within the standard sort of like curricula of the university, sort of the standard STEM classes that you might typically take. And this. So like I said, I am technically a professor of biology, even though I don't really do much biology, because in this school there isn't like a dedicated neuroscience department.

\textbf{[00:01:07]} There's the BNS program, sub major minor. I can never remember what the status is on that, but my research would kind of fit in a number of possible different departments, the most obvious ones being biology, because I study like, you know, the neural aspects of things and think about things such as neurons. Psychology would be, I think, an obvious fit because it's got the sensation and motor control kind of stuff. These days I do a tremendous amount of software development in the sort of methods building and research, sort of research tool development side of life, which is a relatively recent development. I've also worked sort of close to robotics, close to biomechanics, close to physical therapy.

\textbf{[00:01:58]} And the specific little sub domain of research that me as an individual carves out, it just doesn't really fit in any particular department. So part of the main structure of this course is to try to think about what is the, like, what are the inroads to the specific thing that I study, which I will talk about in a second. And the goal is to kind of give that sort of very broad survey of the various subtopics that sort of come together to make the specific domain that I study, which is human movement in natural environments. But more so than that, like my specific way of studying that, because you can study a particular topic in a bunch of different ways. And I have come from such a weird and wild background that it's sort of a interesting question of sort of how would you introduce someone to that field?

\textbf{[00:02:59]} It's not a field, it's just sort of a thing I've learned how to do over the course of a couple decades.

\textbf{[00:03:08]} So there is something that I kind of. It's something I have to say at the beginning of every class that I teach. And I have never found a super great way to say it, but I don't. I would not feel comfortable teaching a class to a room full of undergraduate people without acknowledging the moral elephant in the room, which is that none of this is okay. None of the general structure of academia education is morally defensible at almost any level.

\textbf{[00:03:48]} The amount to which you are being exploited financially to be in this room is absolutely indefensible and unconscionable. And there is truly nothing that I could ever do on this side of the podium that would ever be worth the financial burden that's being put on you by being here. You can do the math. If you want to calculate the number of people times the annual tuition, room and board, the number would come out to be something like 2 to 3 million dollars. If you wanted to be more conservative and only count credit hours, it would be somewhere around a quarter million dollars.

\textbf{[00:04:27]} It's indefensible. Absolutely not. Okay. And I have over the course of my four to five years, let's lose track time. As being a professor here, I have sort of had to find a lot of different ways of trying to handle the term is moral harm, which is the term that refers to the specific kind of sort of psychological trauma that occurs when you are forced to participate in an institution or a system that morally repugnant to you in some way.

\textbf{[00:05:00]} I'm here for a lot of the reasons that you are here, which is because I love learning, I love teaching, I love sort of scientific exploration. And the nature of society as it is is that to do that at the highest level you sort of have to come to for legal reasons, I will say institutions like this one in order to participate in that. And it's the kind of thing that I didn't fully comprehend what that all meant until I became a professor. It's the kind of thing where like you're all students and you experience life as a student does, which is basically as a temporary person in an institution like this. When you have a set amount of time that you'll be here, you'll go through, you'll have the degree and then you'll leave.

\textbf{[00:05:53]} And so everything up to the point of becoming a professor type person has that kind of characteristic. You're an undergraduate, then you're a graduate student, then you're a postdoc. And then if you sort of out compete your peers, which is an insane concept, why are you trying to out compete your peers? You should work with your peers to. Anyways, we'll get on to that.

\textbf{[00:06:14]} But all up until that point there is this sort of temporary nature to it. And I think in retrospect there felt like it was a kind of. It felt more. I felt like I was struggling against the system to try to succeed in that context. And then I happened to become a professor in the summer of 2019.

\textbf{[00:06:37]} So I had about a year of like, boy, I sure did make it. And this is probably fine. And then summer of 2020 rolled around. Covid happened. And with all of the various ways that that affected the world at large and me in particular, I basically just lost all faith in the academic and scientific systems in that summer and sort of had a whole crisis of conscience and tried to.

\textbf{[00:07:05]} Was trying to find ways of basically restructuring my life and my effort and sort of. I've got. I had 35 and change years at that point of not 35 directly, but like close to 20 years of effort at that point to make it to this point. This is a position of non zero social power. And there is kind of this general idea of like don't give up power if you can use it to the benefit of your own moral beliefs or whatever.

\textbf{[00:07:35]} And it was a hard and challenging thing to think about just to like try to sort of come. You know. What oftentimes happens when you work within institutions, which we are all forced to do, is there's a conflation between the values that you hold and the institutions that are sort of the vehicles for those values. And when you lose faith in an institution, which I highly recommend, just never trust institutions, trust people, don't trust institutions. There was a whole thing of trying to figure out what I was actually here for, what my actual goals and morals and values actually are.

\textbf{[00:08:15]} And a lot of things come out of that. Probably stuff that I still should not say on camera. But the general idea was a shift of focus from the doing the traditional modes of research and education, which pretty much to this day still involves do research for publications, apply for grants, produce students that will try to climb the same ladder. You are on understanding entirely that that's to get tenure. You're supposed to produce like you know, more than two people who are also trying to get tenures at other school, which is sort of the definition of a Ponzi scheme.

\textbf{[00:09:05]} And that was sort of the effort. It's oftentimes when you're working in these types of systems that we tend to live in, which are competition based hierarchies predicated on assumptions of false scarcity. The game that you play is sort of you're asked to compete with the people who are sort of at your same level in the hierarchy in order to please the people who are above you in the hierarchy, most of whom have been in that position for longer, sort of a gerontocracy type of thing. And I realized that a lot of the effort that I was putting in, almost like a very big percentage of the effort I was putting into the world, was made in the service of the people who were above me on the hierarchy and trying to do the things that they wanted me to do so that I could outperform my peers in and be allowed to stay in the warm, happy, shining tower.

\bigskip\noindent\rule{\textwidth}{0.4pt}\bigskip

\#\#\# Chunk 2 [00:09:45 - 00:19:45]

\textbf{[00:09:45]} Made in the service of the people who were above me on the hierarchy and trying to do the things that they wanted me to do so that I could outperform my peers and be allowed to stay in the warm, happy, shining tower at the top of the hill that we are all currently in. One of the big changes that I sort of brain shifts that happened was deciding to just stop doing that and to just stop caring or considering what I was being asked to do by people above me in the hierarchy and start asking instead what I could do to support people who were at or below me in that same hierarchy, which is largely y' all types.

\textbf{[00:10:32]} From a practical perspective, what that turned out to be was a. So in that time of sort of doing research, I sort of like, I've got all these papers, we'll talk about them. And with that sort of crisis of conscience, there was this question like, well, I don't. So universities are probably evil, journals are definitely evil. But I do love the research, I do love the teaching.

\textbf{[00:11:00]} So what do you do if you don't want to, you know, do the dumb work of publishing a bunch of papers to try to make the powers that be happy to please the gatekeeping dinosaurs above you? What do you do with your time? And it was around that time, sort of through other modes. I mean that was. I had basically, I have a big Fancy Lab at Richard 453 which will be.

\textbf{[00:11:26]} Y' all will be touring with Aaron, not me next week because I'll be out of town. And when Covid hit, so I showed up, I had all these plans for like the research I was going to do in the big fancy lab. And they were doing the construction and the construction wrapped up around January and I was sort of gearing up to start the research cycle. And then quarantine Hit got sort of sent home and had to sort of for a while just kind of flailed around trying to figure out like, what am I going to do with my time? What are we going to do with our time?

\textbf{[00:12:03]} And at that time, I wound up basically there were a couple things happening in my lab at that point, most of which were related to work in like the big fancy mocap lab. But. And then there was one little side project that involved this computer vision based method for doing motion capture with like cheapo webcams and free software. And so when Covid hit, I basically we shifted effort towards working entirely on that because that was a project that we could work on sort of remotely with ease and sort of kind of figure that out. And so as, so that was like March or whatever.

\textbf{[00:12:46]} And then as we sort of got into the summer and that project started to sort of like pick up some steam and I started losing faith in the system and all that kind of stuff, I found myself kind of coming across software such as Blender, which we're going to talk about. Blender. Blender is a free open source. Sorry, got that sort of stuck in my head. It's a big free open source 3D animation software.

\textbf{[00:13:12]} It's been around for some decades. It's a large scale, wide scale thing and it has been. It is 100\% free, 100\% open source, and built over the course of decades by thousands of people across the world and given away freely as a gift, freely given for anyone to use and benefit from. And I kind of had this moment of realization that that was actually the thing that I wanted to be doing. The open source community is what the scientific community pretends to be like.

\textbf{[00:13:45]} We pretend as scientists that we are doing that, that we're working in this big sort of global collaborative endeavor, but it's not. But practically speaking, you're just not really able to do it. So most of the research that gets done really shouldn't be done. It's not good research. It's just smart research.

\textbf{[00:14:02]} It's strategically the right move to get the publication, to get the grants, to get the job, to get the tenure. And it never often will serve. Like, I'm not saying that all scientists are idiots and that all research is garbage, although the percentage is higher than you wish it was, but the nature of the institutions that sort of house it and the realities of like the hyper competitive structures of society that we've sort of grown for ourselves are that you're not really able to do it in a way that really benefits anybody outside of the club. And, and when you see scientists, sort of traditional scientists doing outreach, it's often this kind of just like side thought, like, I'm going to go to a high school, show them how cool it is to be super highly educated and then just leave and then hope that they're inspired enough to follow in your path out compete their peers, and then maybe they someday can get into that shining tower at the top of the hill.

\textbf{[00:15:00]} It just didn't seem right to me. So long story short, such as it is, all of that kind of culminated into a sort of not culminated, but sort of over a long course of time that became basically. Did I break it? This guy. This is computer.

\textbf{[00:15:24]} There you go. So this isdukefreemocap.org so this is a software that I've been working on with Aaron for about five years. It's called FreeMap. It's a free open source markerless motion capture system. And it's basically been the thing that has soaked up the energy that would have been spent publishing papers and trying to make people happy.

\textbf{[00:15:57]} I instead spent building this tool to do full body motion capture with cheapo garbage webcams and free software. And it's been going quite well and we're going to talk plenty more about that. It's on GitHub, free mocap there there's a Discord server with looks like pushing 2500 members in it got about, I think last I checked approaching 5,000 unique users in like 115 countries, which is all quite good. And the focus of the project and the software is to, instead of taking my particular skill set and expertise which I have sort of accumulated over the course of decades now, using mostly taxpayer funding to pay my salaries and my paychecks. And rather than trying to use that to just publish super esoteric papers that will only be beneficial to other super esoteric people, the goal was to try to package that same, use that expertise to build a tool that sort of meets a general need of society at large and then sort of try to syrup building that tool as both the thing that the world wants, also the tool that I want to do my research and ideally trying to use that as a mechanism for building bridges and sort of pulling people in.

\textbf{[00:17:28]} So a lot of the. I would say at this point still the majority of users of this software are animators, often young 3D animation artists, video game designers, things like that. And also a bunch of students and grad students and biomechanics, clinical, whatever this, that and the other, a lot of dancers and musicians and things like that. And they are using this software for their own purposes. That's sort of part of it.

\textbf{[00:18:00]} I don't want to tell you what you should care about. I'd rather just give you useful tools and let you do what you want to do with it. But because I, with Help Global, we designed this thing based up from my perspective. I am also trying to sort of surreptitiously make this as a tool that is also kind of an educational platform in its own right. Something that will, it doesn't teach you directly, it's not giving you like lessons and quizzes, but the nature of the tool is to try to teach you how it works and to try to kind of basically the idea is to give you kind of a greased rail into the densest part of the forest and then shine enough lights that you can sort of navigate yourself around there.

\textbf{[00:18:46]} A lot of the education, the general structure of education that y' all tend to receive is sort of this like, let's start with the basics. Let's start with the simple, boring stuff that frankly, is not particularly motivating on its own. But because trust me, trust me, it's going to lead somewhere so cool. And if you can only hang on and out compete your peers for long enough, then you'll be able to see all the cool, fun stuff. And there is actually a reaction that I've had more than once when I talked about the software to people, which is, you should be careful not to make it too easy to use, otherwise the students won't understand what's going on in there.

\textbf{[00:19:28]} And it's like insane to me to think that, because I would much. Because the reality is that in the natural world, in the real world, your first reaction upon encountering something is often going to be.

\bigskip\noindent\rule{\textwidth}{0.4pt}\bigskip

\#\#\# Chunk 3 [00:19:30 - 00:29:30]

\textbf{[00:19:30]} Insane to me to think that, because I would much. Because the reality is that in the natural world, in the real world, your first reaction upon encountering something is often going to be confusion. Like when you first come across some strange artifact, that's the least you were ever going to understand it. And you will wind up, if you're lucky, slowly kind of backing your way into a level of understanding. And so rather than trying to basically hold your hands on a very specific path with an intended outcome of the thing that I'm trying to show you, I am instead going to be trying to show you a broad landscape that you are frankly, that is frankly going to be pretty over your head for the most part and then allow you and give you the tools that you need to kind of gravitate towards the parts of that landscape that are the most motivating to you as individuals and then allow you to kind of give you the tools that you need to kind of like back your way into a place of understanding and basically working your way from like the advanced stuff in the center of the woods back to the beginner shit that we really should have started with so that you can basically ground yourself from that position.

\textbf{[00:20:59]} But the reality is that because it's such a broad landscape, you're all going to sort of find different parts of it and you're going to need different parts of that beginner basic stuff and will kind of sort of gravitate in that direction.

\textbf{[00:21:17]} So little bit of. Yeah, let me see where I'm at with this.

\textbf{[00:21:30]} There's more.

\textbf{[00:21:33]} This is still context, Right? So that's the general gist of the life that I live. And there's a little bit of. I mean, the class is obviously overlapping with that general thing because I'm the one standing up here and you guys are over there. Specifically within this class, there is another.

\textbf{[00:21:56]} There's another player, which is that the story I was telling you about sort of like the loss of faith and the finding of the free open source thing and sort of the reshaping around this software and educational blah, blah, blah. All of that is basically started in roughly summer of 2020 and it's kind of been progressing since there. I've taught a couple of classes, sort of win that vibe. I see a couple of somewhere familiar faces, possibly from inquiries in some cases. Has anyone been in a class with me before?

\textbf{[00:22:35]} No. Who are you people?

\textbf{[00:22:42]} Well, yeah, so I've worked on trying to like teaching this type of stuff and trying this like broad landscape thing and trying to incorporate as much of These sort of, like deep tech tools as I could. And then I guess it was, what was it like two years ago now, April, a year and a half ago, there was a certain technological event which occurred and you heard about it. And the, the hype cycle has sort of come and gone to the point where you're probably annoyed about it at this point. But the thing in question is artificial intelligence. My background, my first degree is in philosophy.

\textbf{[00:23:26]} I got a degree in philosophy, focus on philosophy of mind and philosophy of science. There's going to be a lot of that, specifically the philosophy of science in this course.

\textbf{[00:23:35]} And from there I got a PhD in cognitive science, which is effectively AI studies these days. And my particular domain within cognitive science, which is another one of this, like, mishmash, like sort of comedically broad fields, was on the, like, the biological side, like psychology, neuroscience kinds of stuff. But I also did a lot of work with robotics. I did a lot of, of. I learned a lot about AI, machine learning and sort of theories of language and all that stuff.

\textbf{[00:24:07]} And when GPT 3.5 came out, like, I've been following the various language models since, you know, roughly 2008.

\textbf{[00:24:20]} And when 3.5 came out, I saw it and I was like, oh, wow, this is, this is getting really good. This is really a very. They've really been working over there. And then when GPT4 came out, it was just like, okay, there is now a shift has occurred and it is now important to take note of it. So I contacted my.

\textbf{[00:24:44]} Remember very clearly. I was, you know, I can't remember exactly the timing of it, but at some point I have a friend named John Linstadt who will probably come up later, who's a friend from graduate school. And he was much more on that AI side of things. He was a director of the cognitive science program at SUNY Oswego for a while. And I sort of.

\textbf{[00:25:04]} We got onto the Discord chat or whatever and he says, john, we need to discuss the situation. And we both immediately knew that the situation was that AI exists and it is not a.

\textbf{[00:25:23]} It's not like you don't have to worry about it as coming to life and biting you. That's not a thing. And pretty much anything you're going to see from almost any source is going to be missing the point to some degree or another. It's one of the nice things about being an expert is, I guess basically looks like, oh, yeah, I'll have it wrong.

\textbf{[00:25:44]} And it's not endgame. And there's still a lot to happen. But it is a major shift in technology that 100\% is going to reshape our society in a lot of predictable and unpredictable ways over the course of the next five to 10 to 20 years.

\textbf{[00:26:04]} And not necessarily for the better. It is as most technologies, it has come from the techno capitalist corpo hegemony that is the source of so much of the sort of problems and harm in our society. And in much the same way that they have sort of, you know, I'd say caused significant damage to our social infrastructure over the past 20 some odd years, they are going to attempt to continue to do that and they're going to attempt to use the. Anytime there's a new technology that comes around, the techno capitalists, just the capital, the corporal capitalists basically try to rush to the new domain and start building moats around it so that they can sequester it and then use their access to that thing as a way to exploit the world at large. That is the true core of capitalism is to exploit disparities between yourself and others to extract resources from them.

\textbf{[00:27:06]} And so a lot of, and I think at this point a lot of that's probably become somewhat apparent and there's probably at least a percentage of you who are sort of like feel like AI ick because it is so overwhelmingly tied to the corpo techno capitalist infrastructure. But there is another angle to that and I think that it can be very difficult to recognize the. It's very difficult I think, especially when you're coming at something from like a non expert position to discern a separation between the core technology that exists and the sort of implementation of that technology. Like it's hard for us to see things like GPT4 on OpenAI as anything other than just like another arm of Microsoft. It's hard to see whatever the hell it is that Google is trying to do as anything other than just like another way to steal your data and another way to shape the world around sucking \$20 a month out of you for sort of a bunch of basic functionalities that you are going to eventually require for your daily life.

\textbf{[00:28:24]} But I think that it's really important to, I think that this particular technology is different because unlike a lot of other tools and other like so things like computers and phones and things like that, like these are also world shaping technologies and they did shape the world and you were sort of a part of it. And I think a lot of y' all are the both beneficiary and harmed individuals from what happened over the past 20 years when the entire world basically shifted towards. Everything that we do in our daily life is going to be pushed and pulled through. Apps that are made by corporations that do not have your best interest at heart.

\textbf{[00:29:12]} And there was sort of an inherent. That disparity between the ability to make these applications, to build these types of technologies was one that was pretty easy to exploit because, you know, I can't make this. I can't make a chip, I can't. I mean, I guess I can build an app, but it's not going to be the same thing. If you would.

\bigskip\noindent\rule{\textwidth}{0.4pt}\bigskip

\#\#\# Chunk 4 [00:29:15 - 00:39:14]

\textbf{[00:29:15]} Between the ability to make these applications, to build these types of technologies was one that was pretty easy to exploit because, you know, I can't make this. I can't make a chip. I can't. I mean, I guess I can build an app, but it's not going to be the same thing that you would get from a Google or a Facebook. And so they get to use that, and they get to use the fact that this is now an essential technology.

\textbf{[00:29:37]} You know, we find these technologies and we sort of reshape our society to them.

\textbf{[00:29:43]} And there's sort of this limitation there. AI, I think, is different because it is fundamentally. It's like they're trying to build moats around the Mote Blaster 9000. Like, it is a tool that will teach you how it works and it will show you everything there is to know about itself. And they're trying to create, corral it and control it.

\textbf{[00:30:08]} And sort of in the same way that they used that they sort of. That you sort of were used to, and it's kind of just not working in the way that they would expect. We'll talk about that in detail. I don't want to go too deep down this rabbit hole because that is definitely something I like to do. But I'm mostly kind of saying that just to sort of comment on the tech ick that I think is a fair response to the world that we live in.

\textbf{[00:30:39]} Because I do think that sort of AI used properly is something that has the ability to be a transformatively liberative tool and one that has the potential to make every single human person that learns how to use it and learns how to wield it a more powerful and capable version of yourself.

\textbf{[00:31:06]} It's not an oracle that will solve all of your problems, which is how it's often trying to be pushed on you.

\textbf{[00:31:14]} It's not something that's automatically going to make the world better, but it is an unbelievably powerful tool that if you learn how to use it, whatever it is that you are trying to do in your life, if it involves the generation, sort of understanding, manipulation of text on a screen, AI can help with that. It's not automatically. It won't build all your life for you, but it is an unbelievably powerful tool with a very shallow initial learning curve and as far as I can tell, an un unlimited maximum possibility.

\textbf{[00:31:55]} So that will lead us.

\textbf{[00:32:06]} Discord.

\textbf{[00:32:09]} Don't do that.

\textbf{[00:32:15]} So I live a lot of my life on Discord, which I didn't. I appreciate the Software for what it is, but I also kind of hate it for what it is. And when GPT4 came online would have been like, I guess, spring of 2023 or something like that. One of the first things I did is I built a bot out of it and started using it to teach a class. And it was a summer class.

\textbf{[00:32:48]} And then I taught another version of this class last fall and then I taught a class Capstone in the spring. Didn't go so well. And then now this is the fourth iteration of using AI Empowered Discord Bot as a way to try to teach and provide yaw types with a. The ability to sort of explore a topic that is sort of broad and vast and complex.

\textbf{[00:33:22]} So we're going to talk about details of that sort of ad nauseam at some point. But the basic idea is that this. So here's the server, which is sort of structured roughly around the topic of the class. And here is this bot named Skellybot that lives in the server. So the, this is Skelly, it's like the logo of mascot of Free Mocap.

\textbf{[00:33:49]} And so Free Mocap, the software has a bunch of little small parts and they're all named like Skelly something. So like the camera part is called skellycam, the bot part is called Skelly Bot. If you're curious about it, you can click on Skelly Bot and then this GitHub link here will take you to the source code for it. And I think the most important part of this is this phrase right here.

\textbf{[00:34:13]} This bot helped me build this bot. I could not have made this thing without this thing.

\textbf{[00:34:23]} The skills and sort of like technical aspects of building a thing like this. I wasn't trained in computer science. I managed to hack and slash my way through grad school and sort of like building the. In order to like. And I was able to build the basics of the FreeMap software because that is my domain of expertise.

\textbf{[00:34:42]} This is not. I have under, I mean I have like an understanding, theoretical understanding of AI, but building a tool like this was beyond me. But the thing about these AIs, you just say, hey, could you help me with this thing? And it's like, yeah, absolutely, no problem, 100\%. And it will give you an answer which is mostly right often, not always, but often.

\textbf{[00:35:04]} And if you learn how to navigate that and use that and sort of like surf the various ways that it is right and wrong and learn the kind of things that it like, oh, this is the problem that AI is going to. It's not going to be able to do that. Oh, it's absolutely going to be able to do that. If you can learn how to navigate that space and use these tools, it will power level you through spaces that you could have hacked and flashed your way through over time, that you could have sat down with a real human primate expert who would explain it to you piece by piece. Or you can just learn how to ask the machine and the machine will sort of help you do these things.

\textbf{[00:35:45]} How exactly that happens is complex. There's not like a quick answer. It's like most complex tools. The learning process is a little bit of instruction with a lot of just like try it and see how it goes. And part of the structure of this course is to sort of give you the opportunity to do that within the landscape of the neural control of real world human movement.

\textbf{[00:36:10]} Doopa doopa doo. So specifically now coming to like AI within an educational context.

\textbf{[00:36:24]} If you. So when I am up here as like a career research scientist with some decades of experience in this very sort of rather niched out field, like I'm not. So like a lot of times if you're like a standard specialist, you're like in the niche of a niche of a niche of a niche of a field. So you're sort of like a very, very specific person. My particular version of that is I'm within a niche of a niche of a niche of a niche.

\textbf{[00:36:50]} So it's sort of not that deep down, but it's like it's sort of a breadth versus depth kind of thing, which can be harder to do. It's not necessarily harder to do. It takes the same amount of time. But it can be harder to teach because I can't assume that you all have gone through like if I was like an advanced epigenetics immunology expert, I can say, okay, have you all done like, you know, intro bio, cell bio, like, you know, whatever, da da da da. All the expertise is that you would need to get to this sort of level of specificity and depth with this kind of very broad area.

\textbf{[00:37:29]} I can't assume that it's possible that some of you have some of the pre reqs, but none of you are going to have all of them because I've checked and pretty much one of the, like, there are roughly one person in the world that has all of my same background because I just, I bounce around so much that the specificity that you get to is just kind of, you become a uniqueness when you start overlapping enough skill sets pretty quickly which is just also just generally good career advice. It is way easier to be unique by just combining a bunch of different things and then eventually you sort of see that all the you overlap enough Venn diagrams and you look at sort of who is shared amongst all those Venn diagrams. It's like oh wait this is actually just me. Which is way sort of easier and more fun to do than to try to sort of out compete everybody that sort of is in the same sort of space and domain but you know follow your heart. Oh shit y' all found your way in.

\textbf{[00:38:37]} Good job.

\textbf{[00:38:42]} So the general vibe of AI within this class will be that this so who here has used GPT at all like prompted the a little bit talk to it you say hi it says hi back you ask it a question it says something back the low the base level introduction there is prompting AI large language models in particular.

\bigskip\noindent\rule{\textwidth}{0.4pt}\bigskip

\#\#\# Chunk 5 [00:39:00 - 00:48:59]

\textbf{[00:39:00]} A little bit, talk to it, you say hi, it says hi back. You ask it a question, it says something back.

\textbf{[00:39:08]} The base level introduction there is prompting AI. Large language models in particular are at their core language calculators. In the same way that you would sort of punch a bunch of numbers into a calculator, push go and it gives you another number. Language models do roughly the same thing. You push a bunch of words into their context windows, push go, and it looks at its unbelievable sort of data set and provides a statistically most likely answer that is the expected and desired outcome of the input to the output.

\textbf{[00:39:53]} Yeah, so and I'm going to say this, even though this is a little bit off, off of that topic, but the main thing that happened, I think the main is like there was like an empirical event that happened, like an interesting empirical event that happened arguably with something like GPT4 versus like the earlier incarnations of it, which was when I was in, when I was an undergraduate and studying philosophy of language, there were sort of this, this question of does language require some sort of like native, native genetic, like neurophysiological architecture to be able to happen? Like, is there something special about like the human evolutionary lineage that allows us to do language? Or the question was, is it possible that you could just sort of extract language from a large enough data set and some clever statistics? And when I was in undergrad at like, what was that, 2006 or so, it was kind of like, it's hard to say, who knows. And I remember at the time there was like, you know, like undergraduate, right.

\textbf{[00:41:09]} So there were like, there were some people there who had been like, the professor was sort of more attuned to like the, you know, what was going on in the field. It's like, yeah, those like the, the statistical people are making some interesting progress in that space, but they're still not doing anything that resembles proper language. And what happened for me, as from my sort of perspective with GPT4, was that question was answered very definitively. And the answer was yes, it is in fact possible to extract language from a large enough data set and some clever statistics without ever giving it explicit rules of grammar or explicit concepts of words and anything like that. And I consider that to be an empirical discovery and the sort of, from a psychological, like a psych testing experiment.

\textbf{[00:41:57]} It's sort of like, how do you know that that happened? It's like, well, I'm a human and I talk to it and I'm like, oh yeah, that's interesting. Like I have a conversation with the machine. And it's like, you are definitely. The machine is definitely doing language.

\textbf{[00:42:09]} And that language is derived truly purely from the statistics of the data set, which in this case represents like every stitch of text that was written on the Internet since 1985, and some clever statistics, which is basically, in this case, transformer models, which just search for that if you're interested in it. So, yeah, and so that's a bit of an aside, but I think that that is the thing that, that was the situation that my friend John wanted to talk about. It's like, oh, shit. We just, the world may not really may not realize it. Like, this is not how.

\textbf{[00:42:51]} I do not hear this being discussed in this way in sort of the highest and broad channels. But the question to the answer to that question has been empirically discovered. And yeah, you actually can extract language from statistics. And we also now have. And so another thing that happened is we now have a secondary example, for the first time in history, a secondary example of a thing that can carry out a conversation.

\textbf{[00:43:21]} It's not a good conversation in many cases. It's certainly not a human. It's not really. It's definitely not conscious, but it's a cognitive agent. It.

\textbf{[00:43:32]} It does language, math in its head. And before you hear people saying like, oh, it's not really doing language because it's just like using a statistic to like, to predict the best output. And it's like, yeah, I got some real bad news for you about what you're doing when you're doing language. So I guess that's.

\textbf{[00:43:57]} Yeah, anyway, that's part of the, the landscape of what's going on here. And so there is also a part of, again, like the ethical obligations that I feel like I have as the person standing on the chairs are facing this way, which is that there is this new technology. It has emerged onto the landscape. I don't really see it being understood or utilized at large.

\textbf{[00:44:30]} And I am sort of like fairly uniquely suited to teach it, use it, that type of thing. And so I think that this is one of the most useful things I can do for you, is introduce you to this technology, this tool, show you how it works and teach you sort of how you can utilize it in your own cases. Even if you wind up not using it in the future, it will affect your life from the outside. So. And like this technology, like all technologies, they can and will be used against you.

\textbf{[00:45:00]} So you should understand it to the extent that you can utilize it, to the extent that it helps you become a more effective version of yourself. And my job and hope is to allow you and help you do that.

\textbf{[00:45:19]} So. Right. So the specific way that I want to, but I have used it and will continue to use it, is to prompt the bot, the term is prompting to basically try to get the bot to act like, effectively like a simulation of myself in a simulation of my expertise. Obviously the very best education that you all could get from this class is if I sat down with each one of you individually, asked you what your interests were, helped you work through things, answered your questions, gave you some answers and did that individually for you on call 24 7, every answer. But I unfortunately a limited piece of meat and that is not within my capacity.

\textbf{[00:46:21]} So instead I can make this bot which is absolutely not as good as having a sit down conversation with me, but it's not bad and it will allow you to. Basically there's this question of like how good are these things? And I'm sure you get all sorts of sort of like the responses that I've seen to AI from a lot of the educational landscape is very defensive and it's sort of like don't use it for teaching, don't use it for, don't use it for your. Like don't use it for. It's don't cheat, don't basically don't cheat, don't whatever, write your papers, blah blah blah.

\textbf{[00:46:59]} It's not in this like it gets it wrong. It's got these hallucinations, blah blah blah, like all these things sort of like try to get you not to use it.

\textbf{[00:47:11]} And it's true that it will get most things, it gets things wrong at some level and it has absolutely no idea when it's saying something wrong. There's this thing called the grounding problem which is in philosophy of AI, it's just like a thing, it will tell you the exactly right answer and the exactly wrong answer with the exact same level of confidence in itself. And so there is a certain. That's one of the things you have to learn is when to recognize when it's getting things right and wrong. However, there is another aspect of it which is that its ability to be right about things is pretty much entirely contingent on its data set.

\textbf{[00:48:03]} If the question that you're asking is present in the data set, it will tend to get it right. It's kind of like a Google search. Like you can search for pages that exist. If you search for something that doesn't exist, well, these days it will just get you a bunch of garbage that's close by. And I guess a good analogy would be imagining searching for something that doesn't exist.

\textbf{[00:48:24]} And instead of showing you a bunch of stuff that's not the thing you want, the answer was I'm just going to take the top three things that you don't want and then tell you that this is the answer to your question. So it's not gaslighting you because it truly believes the things that it said, but it is an aspect of the thing that it does not know whether it's right or wrong.

\textbf{[00:48:49]} However, because the nature of this class is a lot of basically mid level expertise in a bunch of different domains.

\bigskip\noindent\rule{\textwidth}{0.4pt}\bigskip

\#\#\# Chunk 6 [00:48:45 - 00:57:40]

\textbf{[00:48:45]} It does not know whether it's right or wrong.

\textbf{[00:48:49]} However, because the nature of this class is a lot of like, basically like mid level expertise in a bunch of different domains, most of the questions that y' all will have about these fields and subfields are things that exist in textbooks. If it's a question, if the answer to the question exists in a textbook somewhere, it tends to nail it. And this is, I'm saying this on the basis of having like taught classes with this. If you ask simple questions about like, you know, what's the oculomotor system? Like what's the cerebellum up to?

\textbf{[00:49:24]} What's superior colliculus? Like how does a muscle work? Like, what's the evolution? What's up with the elbow? What's that deal?

\textbf{[00:49:29]} What's a motor unit? What's a Golgi tendon organ? It nails all those questions because those are all not. If the question has been asked and answered at least 10,000 times since 1985 on the Internet, it will tend to get it right. And so the interesting space that you are now in.

\textbf{[00:49:53]} So for me, because I've been alive for long enough to have expertise, you all don't, not because you're dumb, just because it takes longer than that to develop expertise.

\textbf{[00:50:07]} I've been studying what I study for as long as many of y' all have been alive. And I'm still like, you know, I'm starting to get to the point where it's like, oh, most of what I know now is like, I have a good enough sense of the landscape that I know the difference between the things that I know, the things that we know as a species and the things that we don't quite know as a species. So I'm in a very good position to, to be able to talk to the bot and sort of to push it into areas that I know, like the really fiddly answers to it and I get to see it kind of go wrong and go off the rails and sort of like start making stuff up and start guessing and making frankly the kinds of guesses that I see like non experts make. Like when you, if you're. I don't want to like talk down about non research professors, but like a teaching professor who is not in like the specific domain of research that you're talking about is like not even teaching classes.

\textbf{[00:51:04]} Textbooks. Textbooks are mostly wrong. That's kind of one of the secrets of if it has been around for long enough to be written down in a textbook, it's probably wrong at Some level, it's at the very least a cartoonized version of the real thing, which is always going to be more complex than up there. It's always going to be less stable than you are often led to believe. One of the many bad.

\textbf{[00:51:32]} Yeah. One of the many themes of this class will be acting under uncertainty. And one of the recurrent themes there is that we know a lot. We know a lot more and a lot less than you're often tend to taught. I think that the nature of undergraduate education is like, there's tests, there's quizzes, there's paper, there's graphics, grades.

\textbf{[00:51:52]} There's this sort of, like, certification system that we go through. It's like, yes, you understand organic chemistry. And the nature of that is that things are often presented to you as if they are right or wrong, true or false. Like, what's the answer to this question? Oh, it's B.

\textbf{[00:52:08]} That's not really how the world works. Questions can be like, get into the philosophy of science aspect of that later. But the world's a lot murkier than we would like for it to be. And that is true of science. It's true of every field.

\textbf{[00:52:28]} And I lost my point. But, yeah, anyways, so the idea will be that within this space and that the structure of this class is going to be encouraging you to navigate this area. And then rather. And what AI will allow us to do versus an earlier era of life is it will allow each of you to navigate a much more uniquely. You'll basically all be able to choose your own trajectory through this space.

\textbf{[00:53:04]} Because the thing to be, like, if I'm up here giving a lecture about, you know, biomechanics or neuromechanics or something like that, I have to kind of think about, like, are they going to be able to follow any of this? Is this. Any of this going to. Going to land? And, like, if I can't.

\textbf{[00:53:19]} So I have to, like, limit the density of the things I say to try to make sure that, like, at least a percentage of you can sort of follow the plot.

\textbf{[00:53:29]} But is it 125? Is the class over? Yeah, but one thing I get to do with AI and I'm going to give it a shot at the end. I think we should have time when I start talking about the content of the class.

\textbf{[00:53:47]} But one of the things that this allows me to do is to be much, much, much denser in the information that I present to you, because I'm going to talk about things that are going to be over your head and it's going to come real fast and you're not going to follow all of it, I can guarantee you already. But that's going to be okay because I'm going to record it. And then once I record it, I'm going to scrape the audio, put it into the AI as like an outline of topics, put that in one of the channels, and then one of the assignments will be go ask the bot about whichever. Which part of this was the most interesting to you? Which part of this was the most confusing to you?

\textbf{[00:54:30]} Which part of this do you wish I had said more about? Which part of this did I say something about? But you didn't follow it at all. And so my goal will be to say all the things that need to be said, to say in a way that you can follow, because you're all smart people and your job is going to be to basically stand in front of the fire hose and try not to let your brain turn off entirely and try to just let it wash over you. And to notice when you have something that sort of like wakes up, like when I mention some this or that or some topic over here, some topic over there, to notice what those things are.

\textbf{[00:55:13]} And then in your own time and with your machines or whatever, explore those topics and sort of seek out deeper depth of knowledge about those things. And you can do that 24 hours a day, seven days a week. This thing will answer every question you have at any point forever. It will never get tired. It will never get annoyed.

\textbf{[00:55:35]} There are no dumb questions. It is always down and learning how to navigate that, I think. So basically it will stop the traditional problems of student teaching, which is like, to get. It will break you out of fail states and it will help you jump over hurdles and sort of that space of like, I don't know what to do. I have no.

\textbf{[00:56:02]} I'm so lost. I don't even know which way to go. Just ask the machine and it will give you help.

\textbf{[00:56:09]} Okay, so.

\textbf{[00:56:17]} From Here we have 35 minutes left. Calm down.

\textbf{[00:56:25]} Trying to think of what I want to do specifically.

\textbf{[00:56:33]} I'm going to do this kind of in the.

\textbf{[00:56:46]} Okay, I'm going to try to make sure I want to get you all. I'm choosing between, like the. With the possibility of running out of time, getting you sort of set up with the technical aspects of this versus like dumping a bunch of content on you traditional first day of class stuff. I'm going to focus on the technical setup. And there's a whole rest of the semester left.

\textbf{[00:57:08]} No syllabus. Survives first contact with the classroom. So if the schedule shifts, I don't mind if we're okay with that.

\textbf{[00:57:17]} So, Yeah.

\textbf{[00:57:27]} So something like 20 of you have made it here already.

\textbf{[00:57:36]} So, to start, let's start with the canvas course.

\bigskip\noindent\rule{\textwidth}{0.4pt}\bigskip

\normalsize


\chapter{State of the Skelly Address}
\label{src:S6}

\begin{framed}
\small
\textbf{Source marker:} S6 \\
\textbf{Date:} 2026 \\
\textbf{Source:} Community Address / Presentation
\end{framed}

\small
\# Transcript: 2026-0\textasciitilde{}2

\#\# Source Information

- \textbf{Source Type:} Local File
- \textbf{File Path:} `C:\textbackslash{}Users\textbackslash{}jonma\textbackslash{}SYNCTH\textasciitilde{}1\textbackslash{}JON-AL\textasciitilde{}1\textbackslash{}videos\textbackslash{}VIDEO\_\textasciitilde{}1\textbackslash{}2026-0\textasciitilde{}1\textbackslash{}2026-0\textasciitilde{}2.MP4`

\bigskip\noindent\rule{\textwidth}{0.4pt}\bigskip

\textbf{Total Duration:} 01:17:55



\begin{quote}\itshape
\textbf{Speaker:} Jon Matthis (community address)
\end{quote}

\bigskip\noindent\rule{\textwidth}{0.4pt}\bigskip

\#\# Full Transcript

\#\#\# Chunk 1 [00:00:00 - 00:09:59]

\textbf{[00:00:00]} All right, I'm gonna go ahead and start. It's funny, I referenced the purple monkey recording here, which is the most viewed video on the Primo Cap sort of pantheon. And it starts with a similar kind of like lament about the. The parasocial weirdness of like trying to like pull up the energy of talking to a group when you're actually sitting in an empty room talking to a webcam. So it feels appropriate that we shall be sort of doing similar things now.

\textbf{[00:00:31]} All right, and three, two, one, go. What's up? Okay. Hello everybody and welcome to the 2026 State of the Skelly address. My name is Jonathan Samir Mathis.

\textbf{[00:00:40]} I am the creator, president and CEO of the Frimo Cap foundation, which is a 501c3 IRS certified public charity dedicated to the development and dissemination of scientific and educational, free and open source software, among other things. It is 2026, which by some math that we're going to sort of show in a second here means that freemo cap is approximately 5 years old. Depending on how you start counting, could be six, I guess, if you really wanted to be it that way. But I think FreemoCap as like a, like a thing specifically is about five years old, which is quite a long time. The nature of these types of project is that they get bigger as they grow, the surface area increases, the complexity increases, and the need to sort of manage the project in different ways changes as well.

\textbf{[00:01:32]} So this presentation is both to give a little bit of context for people who kind of are like floating around, sort of wondering about how things are going with the project. People might sort of wander in and out from the street and people who have been around for a while who kind of want to have a sense of, of where we currently are and where we are going in the future, mostly focusing on the next year, but we can go, keep going from there. So we are entering or we're somewhere around five years old. And it's been a long and fun journey since then. I'm going to go through in the sort of traditional three way, three part.

\textbf{[00:02:09]} I'm going to talk about a quick abbreviated history of freemocapt. Not because it's super important, but because it's sort of like, I think it's just good to tell the story every now and again. So that way the general history in sort of context is sort of known, a big part of the transitional phase. And I'm sort of hoping for free mocap over the next five years is one that really, you know, continues to build the community around it and kind of like de centers me as a specific person. But for now at least, freemocap is really very, very tied up in the situation in this rough vicinity of the world.

\textbf{[00:02:44]} So having a little sense of the background in the current context will be helpful, I think, for people who are following along. Then we'll talk about the current state numbers and financials and sort of clients, people who work, we're working with, and then talk, hopefully probably more of the time. A lot of the time will be spent talking about the future plans specifically around the transition to v2, the version 2 of the software which we are currently transitioning to. Talk about some work I've been doing building something like curriculum around freemo Cap and Shop and some other stuff and yeah, just general projects that we have floating right now. So part one, we're just going to do a quick abbreviated timeline of free mocap.

\textbf{[00:03:25]} I would say that it roughly started around 2017 when open pose happened for the first time. Really should have the GIF show up here of them using it. But after, I would say that was the first time I saw a viable markerless motion capture system that was capable of doing the 2D skeleton tracking. That is the sort of the core technology that makes freemocap work as a markerless motion capture system. So from there, after that I was, you know, I was hired at Northeastern in 2019 and had one year of sort of standard human professorship for 2020.

\textbf{[00:03:58]} Hit Covid hit and March of 2020, we all went into quarantine. And then through various ways and means, around August of 2020, I lost all faith in the scientific, educational and academic process. And. And then sort of roughly around that time of transitioning from somewhere around 2020 to somewhere around 2021, I reacquainted myself and realigned myself from standard academic to more free open source zealot, largely inspired by Blender and sort of understanding that this statement that the open source community is what the scientific community pretends to be and they have a cat walking around. And also in the standard cat way of things, I have no idea how to do that with my computer, but somehow she always finds it with her feet.

\textbf{[00:04:49]} Okay, yeah. So going through a quick sort of history of the actual technicals of the project, this is. I went back and found the first public post which was 2021, January 2021, sort of. This was open pose, four cameras. And I think, God, who even knows?

\textbf{[00:05:07]} I doubt this code is. I doubt I could even find anymore. I'm not sure if it was on GitHub. Oh, did we just lose. Sorry, more cat crawling around.

\textbf{[00:05:17]} Problems. Pulled out the camera. Okay, we're back. So 2021, this was the first public post on Twitter and then this one, 2021. So what is that about six months later?

\textbf{[00:05:29]} This was now what I like to call the Meowmaline video because the song that is playing in the background here, there's a song playing in the background which is Meowmaline, which is written by the inimitable Neon Xdeath in the server. So I made this video using. I guess you would call that the pre alpha version of. It doesn't even have a version on here. They have a date.

\textbf{[00:05:52]} I didn't put a date on it. Rookies. Rookie mistakes. And yeah, posted it to Twitter, got a lot of attention and I think that was probably the. The first big boost of sort of people giving a.

\textbf{[00:06:04]} About these silly skeletons. I guess they already had the logo at that point. That was designed by the also inimitable Taylor Davies, who. I don't know why people are always trying to imitate folks. Can you make that go away?

\textbf{[00:06:19]} What came next? Oh, wait. Oh, that's not. That's not the Purple Monkey video at all. That's annoying.

\textbf{[00:06:25]} One second, let me find. That's the month. What? Anyway, so this video came out also around three years ago using. Yeah, this was part of a application for Epic Megagrant which we did wind up getting eventually.

\textbf{[00:06:38]} And it has a sort of brief manifesto of the reasons for and you know, sort of like motivation behind Freemo Cap. And it's sort of still front and center on the website and still gets a lot of views. Still gets more views than anything else combined. Well, that's not really true. Well no, it is true actually.

\textbf{[00:06:57]} And I should reprocess it with the new version. I still have the data, it would work. I would be able to reprocess it with a new version with a new Skelly. But I do kind of like the Purple Monkey just because this was first getting started around the blender. So that's pretty.

\textbf{[00:07:12]} That's the one. I also did post this rough cut tutorial with the invaluable help of brother Paul. There was also a cat who was being rambunctious in the video in a similar sort of sense of this was rough cut because I was going to edit it down and then that was like whatever three years ago. It's fine. Yeah.

\textbf{[00:07:32]} And then September 2023 was the release of version one. At that point I think there was no fanfare. It was more of a quiet, quiet affair of just pushing it. I think it was like it had like, we had a bunch of versions that had RC on the end. And then at one point I was just like, okay, I think 1.1 was the first one without the RC.

\textbf{[00:07:51]} So it was a little bit, what's the word? Anticlimactic. Because it was just sort of. Yeah, we skipped 1.0 and actually went straight to 1.1. But that's this one and that is the current architecture that we are still using.

\textbf{[00:08:03]} It has improved a lot in many, many ways, I think special shout out to Philip specifically for doing a lot of the plumbing necessary to keep the dang thing up and running and fixing the bugs and making it workable and also just basically dealing with all of my code slop in there. There was a lot being learned here. And ultimately the way that these kind of projects work, it is typically the decisions you make first that burn you in the end. And that's kind of the situation with 1.0 here, even as we are now onto what's the next slide look like? 1.7 now with many, many, many improvements in the background now mostly at the data level, some at the UX UI level, some at the base functionality level.

\textbf{[00:08:46]} I think it's 1.7 that adds to ground plane calibration, which is a big technical leap. And yeah, just a thousand tiny tweaks to make things good. I'm not going to talk too much about the details here about a lot of the parts because that those things kind of deserve their own video. But anytime I'm showing Blender in the add on, I have to give one of you know, probably several dozen appropriate shout outs to old AJC for his hit again, inimitable third time inevitable work in the Freemo cap sort of community space. And he made also a lot of the work in the Blender add on.

\textbf{[00:09:22]} Yeah, so that was so November 23rd, that was about two years ago and then a long slow inching up from there. So around that, around that time I was trying to look if, in my, in the records of the social medias, if there's any specific moment that I could cite as the moment where I decided that we needed to slow the growth curve a bit and I couldn't quite find anything. So it's somewhere in that range of like 23 to 25 that I was. The community was growing, but it was growing kind of fast. And because there was, because it's a complicated software with many just.

\bigskip\noindent\rule{\textwidth}{0.4pt}\bigskip

\#\#\# Chunk 2 [00:09:45 - 00:19:45]

\textbf{[00:09:45]} Curve a bit and I couldn't quite find anything. So it's somewhere in that range of like 23 to 25 that I was. The community was growing, but it was growing kind of fast. And because there was. Because it's a complicated software with many just.

\textbf{[00:10:00]} It's just, it's. Even if it was a perfect software, it would be hard to use and it's not a perfect software. So like I was just concerned about the rate at which people were coming in. Yeah. So sometime around the 23, 24 range the project started to grow and it was.

\textbf{[00:10:19]} Yeah. And it started to get a little. I was concerned because I was feeling. I think that might have been around the time that I read this book for the first time or I haven't read it multiple times, but it. I read it once and it was very formative.

\textbf{[00:10:34]} Nadia Egbal Working in Public Making and Maintenance of Open Source Software so the book is kind of a like an anthropological survey of open source software developers. And one of the main conclusions is that the. The primary resource to be maintained and protected in an open source environment is maintainer energy because maintainer burnout, which maintainer is the person who maintains the code for those that don't know. And maintainer burnout is the number one cause of, of death for open source software projects. And I was concerned that as people started, as the project started to grow and the surface area started to grow, we didn't really have the infrastructure to receive the people that were coming in because of kind of the inherent contradictions associated with the core mission of freemocap which is to make a software which is simultaneously a high level, high performance research software, but also something that is usable by 13 year olds and people without technical background that's inherently in conflict with itself.

\textbf{[00:11:38]} And so that started to. Started to feel that and so basically just stop posting, stop streaming. Just the stopping of streaming was less of the control thing. It's just I. Streaming is just, you know, somewhat exhausting for me, but I suspect it was something around here when you can kind of. And it worked.

\textbf{[00:11:57]} It worked actually quite well. So we stopped doing posting. So the. At that point the scale was such that there was enough organic growth that people were kind of making their way in. I was sort of, you know, getting into a conference every now and again and then it's really focused the energy on.

\textbf{[00:12:11]} And then we also have really good SEO because it's people. A lot of people search for free motion capture it turns out. And so when they do, they find us and then you know, a lot of the effort of developing the project is making it sort of accessible to more people and then more people show up. So we focus a lot of energy on the discord community community. And having, you know, the, the, the weekly community calls have been really nice sort of anchor points and once again shouting out the specifically ajc, but also just all the people in the server who kind of just like picked up, picked up the shovel and started to help.

\textbf{[00:12:48]} Mostly in terms of just helping the people who showed up, you know, get the thing installed, you know, oh, hey, my can't tap my cameras. Oh my calibration didn't work. You know, all the things that realistically could have been solved by like ux, better ui, better documentation, you know, all that kind of stuff and all the things that we are developing and have continued to develop, but because they didn't exist at that time, a lot of that sort of work was sort of pulled by people in the community itself again, specifically Andres. But a lot of people are sort of in there talking and. Yeah, and it's really, it's been really sort of gratifying to watch that happen and sort of like, you know, appreciate the kind of the space that that buys, you know, like just sort of holding, holding the thing up a little bit just every time you respond to a message.

\textbf{[00:13:36]} And that really gave, you know, me and the core devs like the time that was needed to actually write the code. And so now here we are in. And now I think we're, we're kind of ready, we're kind of ready for, you know, the things to become. I think that, you know, like version two is a full refactor of version one. I think it will be generally more amenable to accepting outside help.

\textbf{[00:13:56]} And I think we're sort of ready to start, start up again, which is good because I guess the dates cut off here. But posted this with Paul the other day and it, you know, went out to the various socials who apparently are still aware that I exist to some extent because that, when, that, that's when this happened. So these are the stars, these are the basically like, you know, points on GitHub people that like the repository. This was the meowling post. This was sort of the long, slow, healthy, steady growth there.

\textbf{[00:14:26]} And then I, you know, we make one more post and then this happens and it's like, see, this is why, this is why you don't want to wake up. And then there's another zero on your user base. That's not, that's not comfortable. Yeah, this is and I think a nice representation of like about the current state of the data. This is nice, very clean data.

\textbf{[00:14:44]} Someone on a, you know, perfect treadmill, perfect lighting and yeah, it looks, looking pretty dang good. So that's a bit of the history, that's kind of how we got here. Now let's look at the current state of the thing, state of the project. Again. This, this particular presentation will not be focused too much on the technical side of freemo Cap.

\textbf{[00:15:02]} This is that like org level kind of thing. So by the numbers GitHub stars we have 4533 that you know, sort of slow growth over time there. The discord server has 3,000 plus people in it. Global users have over 10,000 and 140 people reached which we will sort of take this moment to take a brief aside field trip into freemocap.orgdata.HTML sort of up here on the panel. This panel is going to change how it looks in a second but you'll, you'll figure it out.

\textbf{[00:15:32]} And so this is, this is the growth curve over the, these are all the people. So these are the pings from the front of the GUI versus you know, uncheck if you don't want to send anonymous pings or whatever. And so these are all the pings that have come in. Actually these are not the pings, these are the unique IPs, these are the, the people who, which is our proxy for like individual people. And this is.

\textbf{[00:15:56]} I stole this plot from climate change representation. But it's, I think it shows here. And yeah, so if we look. So this is the fun animation and this is the more static chart. We got users everywhere, all over the place.

\textbf{[00:16:11]} We have received 42,000 total pings from that thing which if you kind of assume that like most people, if you see the box that says share anonymous data, most people say no so. Or I think I'm guessing it's probably about half that might be optimistic, pessimistic, hard say. And yes, the. So yeah, so these are underestimates. And then the users sort of estimated users are sort of the unique IPs which is of course, you know, not accurate for a number of reasons.

\textbf{[00:16:41]} 140 countries represented by, from those pings, including you know, the. So I made this blob after I scheduled this talk and when I saw this I'm like, wow, maybe I shouldn't have scheduled it in the evening on the east coast because this is the middle of the night for where apparently our biggest core of people is which is Central Europe specifically in the kind of Germany, France, England era area which is where a lot of like research labs are. We got a lot of people over. A lot of people in India, a lot of people in China. This is their total uses over time.

\textbf{[00:17:14]} These are new users per day and this is pings per day. I could be flipped. So rate of growth is growing. And then this is some analysis of like do people come back? Because a lot, you know, almost half of these people are single shot like bouncers.

\textbf{[00:17:29]} Like they, they see it, they turn it on, they can't figure it out and they leave. So this number I, I think is kind of like a key with like a performance number I really want to look at is like, you know, how many pings do we get? But how many times do people come back? Like we want people to be sticking around and learning how stuff works. Which is, you know, a design question that we'll get to.

\textbf{[00:17:49]} Yeah, yeah. So US is the. Is the biggest. China's number two. Germany, India, France, Australia, United Kingdom.

\textbf{[00:17:58]} There are approximately 49,000 free mocap users for 100,000 square kilometers of land in Monaco. So in case you were wondering about that. Anyways, I had some fun making these data blops. But yeah, so, so it. So the growth rate I would say is nice and healthy and I think pretty neat that it is continuing to grow organically just through, you know, word of mouth and SEO.

\textbf{[00:18:22]} So you can kind of see, I think like there are probably some moments around here where I was like oh, maybe we should slow down and then. But generally speaking the pattern here shows that like we are growing and the rate of growth is increasing. So feels pretty healthy and it does look like nuclear war. But you'll be okay. Skelly will save you.

\textbf{[00:18:42]} Yeah. To do where I was I. Nope, not that I lost the presentation again. Okay. So yeah, so that's so doing pretty okay there. So now let's keep.

\textbf{[00:18:52]} So now the financials. I'm not going to go into too much detail here. There this I should maybe, I don't know, I didn't really plan out too much of this. So in terms of financials and you know, as a 501c3 public charity, our numbers are. They are publicly available through IRS 990s.

\textbf{[00:19:11]} Our 990s have not I think made it to like ProPublica at all yet. But I assume they'll be there by next year. Long story short, we're doing okay. The up until now freemocap has largely and I would say has been, you know, by the numbers, primarily supported by the fact that I am currently employed at a university that shall remain nameless. And because freemocap is a part of my research program, this is kind of, you know, it's overlapping with my responsibilities there enough that I've been able to continue to sort of work on it here.

\textbf{[00:19:43]} However, a big part of.

\bigskip\noindent\rule{\textwidth}{0.4pt}\bigskip

\#\#\# Chunk 3 [00:19:30 - 00:29:30]

\textbf{[00:19:30]} Employed at a university that shall remain nameless. And because freemocap is a part of my research program, this is kind of, you know, it's overlapping with my responsibilities there enough that I've been able to continue to sort of work on it here. However, a big part of the, you know, losing faith in academia thing that I mentioned was part and parcel with like basically making the decisions in service of the things that I think I thought should exist, which is free mocap. While understanding that that was not going to my, my life and academia and are not compatible for moral reasons, I. There's a lot more that I can say about that and I will tell that story in its entirety, but not while I am still employed by said employer.

\textbf{[00:20:16]} And so that moment. And so basically we are now approaching a scary cliff on the road that I chose for myself five or six years ago, which is the end of my official contract. So now the question has. So the question of six years ago was can I set something up that will allow me to continue to live roughly the kind of life I like to live doing research and building tools and teaching and doing mentorship and all that kind of stuff without having to be attached to a degree granting institution that charges \$93,000 a year for, for tuition, room and board for these exactly the undergraduate students that are my responsibility. And so I've had five plus years to plan for it and the time is now coming and again, I think we're doing okay.

\textbf{[00:21:06]} So we'll be able to survive past the, you know, the last paycheck of June 30th. We'll still be able to keep paying Philip, we'll still be able to start paying Aaron and the various ET Als around and I'll be able to continue paying my myself. The question mark I think now is about, you know, we will survive, will we thrive? This is the question. These are the questions of the next 4, 6, 3, 445 months.

\textbf{[00:21:33]} And a lot of it I think is dependent on how some things pitch pan out and also, you know, but ultimately I think we're doing okay. I will promise you a more specific update next year here, but for now I'm going to leave it at that. A big part of. So a big part of the question with something like remote cap is, you know, how do you make money selling a free software? And the answer is it's hard and complicated and probably something we can talk about in various ways.

\textbf{[00:22:00]} But long story short, there's been the financial strategy that I've been approaching is based off of Stocking the records of every open source project that I can think of. And, and it's basically to do something of a patchwork approach. So you have a number of different routes that you take to try to bring in money and they all have different validity at different timescales and they bring in different amounts to different, you know, whatever currently. So eventually the plan will be to be, you know, we're building the infrastructure to start applying for grants which you know, it's a simple, it's we, we run relatively light so a single sort of like solid grant would sort of would serve us for pretty well. And I have some targets for that in mind probably sometime in June and then the NSF Pose Award is something that we're going to be applying for in September.

\textbf{[00:22:51]} But the main sort of short term target that we're looking at bringing a little bit of money is looking at clients. And so right now we have two research partners that are paying us to do dedicated work. Ben Scholl. The name Ben Scholl shall ring loudly in the histories of Freemo Cap for being kind of the first person to, you know, not the first person to pay us anything but the first person to really like, you know, develop like a long term research connection centered around doing motion capture and eye tracking in ferrets. And so this is a video that we put together before sfn.

\textbf{[00:23:26]} This is a ferret wearing a binocular eye tracker. We actually have better stuff kind of floating around. Recently made some advances. The reason why I haven't been able to work on Freemo CAP since SFN Society for Neuroscience conference in November or whatever is because I've been doing ferret stuff. But it's really fun and it's kind of like this is the kind of work that's more akin to like the you know, standards of like high science and you know, academia and like you know, big publications and stuff like that.

\textbf{[00:23:53]} Also not the most generally applicable to the world. I guess not everybody has an animal research lab. But it's, it's been really fun sort of coming back into that space with the skill sets that you know, me and Aaron and Philip et al have accrued building Freemodel Cap and being able to kind of go back into these scientific research questions and just like just do stuff that you just can't do as a science like a standard academic for a lot of different reasons. And I think it's going to go. There's, there's a lot, my scientific career I do not believe is over despite the fact that My official academic employment is coming to an end.

\textbf{[00:24:32]} Free MOCAP is sort of at its core a research institute. So I'm just now a self employed rogue scientist doing eyeball work on tiny animals. So the, the latest addition to the short list of wonderful, wonderful people is Diego Fernandez who works at Cincinnati Children's Hospital Medical College, who we met through his student Burgundy Walters at Society for Neuroscience in San Diego. And they have commissioned us to build a mouse eye tracker. So smaller than a ferret, but we can, I think it's handleable for, you know, again into the technicals.

\textbf{[00:25:12]} But there's, there's a, I will post this, the a link to this sort of information in the description to the video. So you can go to the video. In this video there's a longer description of the methods that are being used. But we have, yes, we have also been commissioned to build a mouse eye tracker for Diego Fernandez specifically to look at pupillometry over sort of extended periods of time. So if you are a person who studies mice and are interested in doing mouse eye tracking, this is a really good time to reach out to us to, to get sort of, you know, early bird special.

\textbf{[00:25:45]} I guess as we are building the thing to begin with, this is a good time to sort of let it be known what your specs are and you can be more, you can support it early on or you can buy it later and it's I guess advantages for both. But you know, reach out to us if you have thoughts more generally, I guess slides are just a little bit out of order. We are now offering paid service contracts. So unofficial logo, unofficial motto is knowledge is free. Labor is unbelievably expensive.

\textbf{[00:26:14]} So if you have funding and are interested in building particular research apparatuses, tracking the eyes of some strange animal, combined motion capture, eye tracking in any context, or just generally setting up freemo CAP as a research tool, let us know. It is not. Yeah, it's. Yeah, reach out to us through this services page on our website and yeah, we can, we can talk. It's, it's not cheap work.

\textbf{[00:26:38]} These are not like. And no offense, but I do not run. I do not want to write a grant with you. If I, if you want to write me into a grant, I'm happy to do that. But it's, you know, at this point in the project I'm not really in a position to like do free work for people.

\textbf{[00:26:53]} So if you currently have funding and are interested in working and building tools, let us know and always happy to talk but you know, gotta be really protective of the actual labor these days. We're also like, I'm not really pushing this quite yet. There's gonna be, I think, more of this towards the like once the V2 kind of comes online, but kind of like pushing for like, you know, bronze, silver, gold sponsorship types of thing using a very similar model to Blender, where you kind of have a website that you can sort of put people's names on, you know, company logos and stuff like that. So if you are a business or working with a business who might want to, who sort of, whose interests are aligned with freemo Caps, I guess and sort of like if, if, if it is in your interest for freemocap to be sort of successful and have specific sort of shapes reach out to us and we can, yeah, corporate sponsorship, the general sort of structure there is like we put your logo in some place of honor and then you get to have kind of conversations with us around, you know, like it's like shift. Like I'm not going to shift the project, but I might shuffle the priority list based off of, you know, what the, what the constraints are.

\textbf{[00:28:01]} And we are, you know, a certified public charity, so we are any. So donations are tax exempt. See. Is there anything else to say here? I guess we're.

\textbf{[00:28:11]} Yeah. So let's move it on. So that's about the current state of things. Let's. Let's talk about the feature again.

\textbf{[00:28:16]} Kind of a strange choice of link. Oh, that's why I lost it. But here. GitHub freemo cat projects 34. This is the Freemo Cap planning operations.

\textbf{[00:28:26]} You can get to it through the. Oh, I see how that works. The Primo Cap foundation repository. But this is a place that you, you know, come check this to see what kind of the, the task list is. We, we keep it relatively up to date.

\textbf{[00:28:39]} It may not capture all the details, but we're sort of getting into that. Okay, so the biggest news of the sort of, the biggest future plans, sort of the current work plan is the transition from the version one of the software which we've been using since 2023 into version 2.0. 2.0 is a complete sort of more or less from scratch refactor of version one. I think there might be some parts that survive, even though I, Even though I pains me to say so, but I think there might be. I think it might be a 100\% reset from scratch refactor.

\textbf{[00:29:14]} But all the main parts are different and sort of generally built in an architecture which is sort of more modern, more robust, more professional and critically more amenable to outside help. It turns out that if you have sloppy code.

\bigskip\noindent\rule{\textwidth}{0.4pt}\bigskip

\#\#\# Chunk 4 [00:29:15 - 00:39:15]

\textbf{[00:29:15]} All the main parts are different and sort of generally built in an architecture which is sort of more modern, more robust, more professional and critically more amenable to outside help. It turns out that if you have sloppy code, it is harder for people to help you out with it. But as the sort of general sophistication of like my skill set and the general sophistication of the software improves, it should become easier to bring more people in. So again, looking at this book. So Nadia Egg Ball differentiates between four types of projects.

\textbf{[00:29:50]} You got your toys with small, small users and small user and small developer base stadium stadiums, which is a lot of users and a small number of developers. Oh, I got these backwards clubs which are more developers than users, which is like Astropie and then Federations, which is sort of like a good mix of developers and users. And right now freemokap is very much a stadium. Small number of users, a very small number of developers and a very large number of users. Pretty large number of users that is.

\textbf{[00:30:27]} It's sort of. It's been somewhat necessary of up until now just because the tool itself is so complex. And I've been really protective of like going in there and like filling around like, like very, very protective of who actually gets to like contribute code to the software, which is sort of good in some ways because it keeps things working and it helps us to avoid tech debt which can you know, really. Tech debt and bloat and like, you know, meandering a focus which can I think really kill a project like this. But ultimately we really want to transition to a world where we can have yeah, a more, more, more people helping out, more people lending hands and yeah, female cap has that again.

\textbf{[00:31:06]} I'm not going to go too much into the technicals here I think, but I think there's. I gave a stream a couple, I guess months ago now about you know, a deep dive into Skelly Cam and I think a new one of those is about red, like percolating and ready for a deep dive into like the V2 architecture. So we'll talk about that when we get to it. But currently version two lives on the development branch of Freemo Cap core, Freemo Cap repository and all the sub skellies. So if you go to the development of FreeMo Cap, you can install that and see what version two is sort of going to look like.

\textbf{[00:31:39]} Currently it looks like this, which the UI is, is again kind of busted, but it is functional and it's kind of a react electron front end with a Python fast API server, back end and then yeah, we'll talk more about that at some future time. I'm going to come back to this UI in a second. So the release plan is going to go through a standard development cycle. So currently we are in a state I call Pre Alpha. Pre Alpha is where it's like it's known to be broken.

\textbf{[00:32:06]} Like it's not like you can use the free. The V2 you can use. It does have real time processing which we have shown works at sfn. It does not have post type post offline. What am I saying?

\textbf{[00:32:21]} Yeah, it's so it's work. We're currently in state where like the best workflow we actually have right now is to use the version 2 of Skellycam to record the videos and then process them with the version one. So getting from the state of like fixing all the parts that we know are broken is what we was how we transitioned from Pre Alpha to Alpha. So Pre Alpha is known busted. Alpha is now possibly working.

\textbf{[00:32:46]} I am not aware of any specific things that are known to be broken within the software. We're also going to be transitioning from this. So this UI is kind of like it's built on good bones, so to speak, but it is not. It's. It's ugly, you can tell.

\textbf{[00:33:02]} And so we're going to be another remote or cell. Another person who deserves some laudations is Puya who has been working on this sort of new UI here. It's got a nice splash screen. These little guys got this cool skelly and some links, you know, and generally just a much better, much nicer looking more modern I guess you'd say soft, just ui. The only issue is that it's not, it's not correct.

\textbf{[00:33:32]} So Puya, Pui was looking at the V1 GUI when he was designing this. So a lot of the workflows are just kind of like not quite right for that. Like the way things are going to be in V2 because it's like things have changed a bit. And so we, we, he and I have talked, we both kind of got busy but basically the plan was to kind of like use this UI to just get it over the hump. So of like, okay, let's just like make sure all the pieces work in the connection of the front end GUI to the back end processor and then sort of in parallel to that we can start taking inspiration from like the workflows that have sort of started to emerge in the version 2 to adapt as UI so that we can then sort of transition over.

\textbf{[00:34:14]} There's also a lot of stuff in like the like redux and state management. Those are just. It's complicated. So, so we will. So, so in the transition from pre alpha to Alpha, whether it's before or during, is when I envision that sort of transition to, to occur.

\textbf{[00:34:33]} Once it's in alpha, the idea is that I would be able to like that's, that's when we say you should use this. It's work, it works. It, it will be better than version one for a lot of reasons, mostly skellycam. A lot of the actual data processing may not change that much at first but, but the Skellycam integration is a big technical improvement and also it will handle like real time processing so you'll be able to have real time reconstruction but you will also still be able to record the videos at full resolution and frame rate so you can post process as well. So yeah, I think if I had planned my life a little better, I'd be giving a little demo of it right now.

\textbf{[00:35:09]} But it's kind of just. I made the data dashboard instead of doing that. So yeah, alpha is possibly working. Let me kick the tires a couple, a little bit and sort of start off inviting people in and then let the world discover that it's actually a busted and broken piece of thing. And then as it gets sort of kicked around in alpha mode for a while, eventually it starts to get to a point where we would move it up to what we would call a beta.

\textbf{[00:35:35]} I think historically we've skipped beta. In some cases we may wind up skipping beta now, but beta is kind of like I am not aware of anything currently broken. It's right, that was alpha. Beta is like, it might actually be like ready to for full. Historically in most contexts alpha is closed and then beta is open.

\textbf{[00:35:55]} But since we are open source kind of they kind of bleed together. But then once we sort of get from there, we'll go to the full release B2 probably we'll do some kind of like you know, end of life, like you know, sort of lts long term support, like support for version one, but probably not. It might just kind of reach a point where it's like it kind of. Here's V1. It is what it is.

\textbf{[00:36:16]} It's probably not getting better. If you need, you can use it until you're ready to transition onto version 2. Anyway that's roughly the plan and yeah, so now let's. There's a couple more little pieces to talk about timeline wise at some Point in past I said that we would have something out by quarter by the end of the first quarter, which would be the end of March. Optimistically I think we should hopefully be living in V2 Alpha mode by the end of March.

\textbf{[00:36:45]} But you know, historically, you know, every estimate should be multiplied by some factor. So we'll see. And then I, I, I don't think it's too optimistic to say that by June or so we'll be in full V2 land. Oh, I also didn't have it on the, on the, the, the slide here, but a huge part of the V2 planning and V2 architecture is single click installer would like the ability to install it, run it and use it without ever touching the command line is something that we don't currently have. We sort of have it with the PI app installers but those are not the most stable for a variety of reasons.

\textbf{[00:37:20]} So version two will have like a proper regular old standalone installer. I figured out how to do code signing so you're your virus checker shouldn't eat it up and that's, that should be a world where we're living, you know, freemocap.org download and it's just like every other software you've ever used in your life. And that is a time I think once that milestone gets hit of you can use it without ever touching a terminal. I think that's when that plus real time plus like the improvements to the, just the animation outputs and then the general like scientific validity of it increasing. I think that once we have the standalone installer kind of like really solidly working, that's when the vibes of the project might change a lot.

\textbf{[00:38:03]} I think that no matter how easy you make it to install something from terminal, the terminal itself is going to knock out at least 90\% of your possible user base. So you know, I think given the bones that we already have with this sort of terminal based kind of janky app, my belief is that once we sort of transition to a more modern looking, you know, sophisticated architecture, that's when we might see another zero show up. Every so often, you know, we, I post one 20 second clip of someone on a treadmill, we, we see a noticeable jump in the, in the, what's the word? The star count. So who knows?

\textbf{[00:38:43]} Who knows? Yeah, something. That's something. Not a lot of detail here the version 2, but I do want to make the official public announcement. The version 2 data model will be wildly different from the version.

\textbf{[00:38:53]} So up until now we've done a fair amount of work to keep things kind of like backwards compatible in, across the different versions and try to do open close type of stuff. So we added a lot of new stuff, but we tried to keep things kind of where they were so that if you're a grad student, anyone who has some animator who has a bunch of.

\bigskip\noindent\rule{\textwidth}{0.4pt}\bigskip

\#\#\# Chunk 5 [00:39:00 - 00:48:59]

\textbf{[00:39:00]} And you know, across the different versions and you know, try to do, you know, open close type of stuff. So like we added a lot of new stuff but we tried to keep things kind of where they were so that if you have, if you're a grad student, anyone who has some animator who has like, you know, a bunch of pipelines built on a particular data model, we don't, we try not to break that. Version two is going to break all of that. We probably build some kind of like adapter thing because the core reality of like, if you like you'll still be able to process and reprocess or old, old stuff because as long as you have synchronized videos and calibration data, we can, we can work with it, we can, we can build from it. But we're, we're definitely stepping back and we're going to like rebuild the data model from the ground up in a sort of.

\textbf{[00:39:42]} It's going to be a lot of tidy CSVs and then a parquet database to sort of handle most of each recording will have its data and mostly tidy formatted CSVs, you know, config yamls and then a parquet database to sort of handle, you know, a lot of the intermediate data just to keep each recording folder kind of cleaner and easier to use. As the new software is more sophisticated, it will also have more sort of, it'll be easier to add more complexity around like user, user input. So we'll in, in the past we've had it set up like you push the button and it goes and it kind of like it, it saves out pretty much the same thing for everybody every time. The new version will have more of a, like, you know, you can check the boxes. Oh, I want, I want this, I don't want that, you know, use you know, output in this format, not that format, you know, so tidy CSVs versus wide format CSVs is a thing that I have been introduced to as a thing to think about.

\textbf{[00:40:39]} And we're going to continue the general strategy of having sensible defaults for everybody. But we're going to start adding more actual like user settings and capacities to like make more choices if you, if you choose to. Oh, I see a goat's eye. Because it's. Yeah, no, because of the.

\textbf{[00:40:54]} Yeah, fair eyeballs are football shaped which does in fact look like goat eyeball, which is like rectangularly shaped. I say looking at the chat from some time ago. Jawas also have weird eyes, don't they? Cool. So, yeah.

\textbf{[00:41:08]} So as we transition as we're as we also, as we're going to be changing up the data model, I'm also going to want to start doing more kind of like public facing, kind of like request for comments. A lot of open source projects use these kind of enhancement protocols. So PEP is the classic one. So Pep 8 you may have heard of Python enhancement proposal I think is what that stands for. And so we're going to start trying to do something around scep so scaling enhancement proposals and the general sort of move there is like you, you, you know, you know, we in the, in the core maintainer team will make some post somewhere publicly in GitHub, sort of findable in various ways that says, hey, here's the plan, request for comments open for this amount of time and then sort of invite people to show up and have opinions about things other people like I'm talking about it now from sort of like a centralized way but you know, eventually people can come in from the outside and say, hey, I have an idea for a thing that we can do that.

\textbf{[00:42:08]} And then other people can post their own Scaling Hampshire proposals. And it's a way to sort of like invite community input without being like, what's the word? Locked down by the need to have like everything be a conversation in the commons, you know. So we post the proposal but we sort of keep it as kind of like a conversational space. We sort of have conversations about it, but ultimately the decision comes down to the core maintainers and ultimately those decisions come down to me.

\textbf{[00:42:36]} Another thing that probably should be written down in, you know, next year I'm gonna have to write a lot about the governance structure of freemo Cap for this NSF pose grant. So I'll get some of that written down, but in passing, freemo Cap is currently in the stage of what's called a bdfl, which stands for Benevolent Dictator for Life, which is a joke that was probably really funny in the 90s. It is the standard governance structure, structure of open source projects. Like someone like there's some person in the center that makes that ultimately, you know, makes the decisions. So Linus Torvald running Linux is the classic and BDFL is the simplest architecture, but it is also the most centralized and sort of, you know, problematic for a lot of a lot of other reasons.

\textbf{[00:43:22]} So I'm kind of proposing architect governance structure called BD Fast Potential, which stands for a benevolent dictator for a set period of time. I don't know exactly what that time would be, but the way I'm kind of thinking about it now is we've been doing this for five years. I think slash hope that in another five years it'll be in a good, it'll be a good time to start planning some sort of like this dissolution of centralized responsibilities. I am hoping that know five years from now, 2031, God bless us all if that. Around that time I'm hoping that the, the general community structure of freemo Cap will be such that like, you know, we can start having a governance structure and a sort of a conversation around the software that is not just like this guy saying stuff.

\textbf{[00:44:14]} Yes it is. These are the challenges. But. And the nice thing about the plan to have, you know, the benevolent dictator for a set period of time is that you're, you know, you're a dictator. So if five years happens and then I'm just like, oh, we're not ready yet, you know, you can't.

\textbf{[00:44:29]} Okay, we're going to push it out, we're going to push it out and if you don't like it, let's put it to a vote. This is my Syrian heritage coming in. It's like I know how to run elections. As a dictator, you just say it because it's, yeah, free Syria. We could free Syria.

\textbf{[00:44:44]} Death to tyrants. But not me just yet. Give me some time. There are models. So what's his name?

\textbf{[00:44:49]} Guido. Guido. The guy, the Python person had a. They. He stepped back.

\textbf{[00:44:54]} Taunt ton Roosenthal. A blender is. He's. There's some complex sort of transitional plan. I think he's, he might be out now, I can't remember, but he had a.

\textbf{[00:45:03]} He made an announcement. It's like I'm, I'm stepping back and there's a whole transition team that went on there. So, you know, we're not there yet. We don't. It's not a problem.

\textbf{[00:45:11]} Yeah. Tan is now officially the advisor, which is kind of, you know, it's like more of a. More of a big picture guy, you know. Like, I just, I. Not right now, but I really want to be, I aspire to be more of a big picture guy.

\textbf{[00:45:25]} And I think that, you know, a big part of Free mocap is, you know, so I have a certain set of skills as a research scientist, you know, whatever we want to say about that. And those skills were unique enough that it got me the fancy science science job and. Yeah. And I could, I could live a whole career being the guy that can do the thing, like the guy that can track the ferrite eyes the best or whatever. But a lot of free mocap.

\textbf{[00:45:49]} A lot of, like, the general philosophy of Free MOCAP is like, I kind of want to disseminate these. This skill set and this knowledge and whatever it is that sort of got me here broader and wider and better than you could from the ivory tower. And the way things are going, I think we're getting there. And so I. Right now, I think a big part of the reason why I still feel like the need to be the central guy is that Free mocap is at the intersection of so many different things that I just don't.

\textbf{[00:46:16]} It's. It's hard to really. It's hard to hand off work because, you know, again, I kind of built my career on living in the Venn diagram of different fields and sort of expertises and so. And Free MOCAP is the thing that I was able to build because of that sort of, like, positioning strategy that I took. And so disseminating those skills is hard and.

\textbf{[00:46:35]} But it's kind of like, it's the work that I'm doing and I've been doing, and I think it's going pretty well. So, again, my hope is that sometime, say in five years from now, because I am tired, there will be enough people around who have been doing it for long enough that like, you know, I could. I can. You know, it's already happening. Like, you know, your Phillips and your errands and your Andres are like, you know, every.

\textbf{[00:46:56]} Every unit of time, they, they. Their skill sets ratchet up. They're taking. They're, you know, making bigger decisions that are. They're running bigger, bigger pieces, bigger chunks.

\textbf{[00:47:07]} And that's going to continue and more people are going to sort of roll in and have fun and learn and get better, and we'll see. You know, you saw the nuclear war map. Like, we're. It's. It's whatever the hell it was that happened in that, you know, manic episode of 2020.

\textbf{[00:47:22]} This appears to be resonating at some level. Speaking of dissemination and education and blah, blah, blah, blah, another big project of the year is the documentation overhaul. And specifically, you know, rewriting a lot of the docs, you know, gearing them, you know, what exists, tuning it to version two and then just expanding the crap out of it. So I have been a professor, I for, you know, six years, whatever, many years now, and I've taught a lot of classes and the classes that I teach are sort of around this stuff. A lot of my classes are, you know, you can watch them on my YouTube channel, you know, if you're into that.

\textbf{[00:47:56]} But now that that sort of official academic ivory tower phase of life is sort of drawing to a close, I am ready to start, you know, thinking about how to sort of, you know, build more actual education and structure into the. Into the Freemo Cap sort of lifestyle or whatever or project like the Free mocap has always been at its core an educational project. Let's see. So wait, what am I trying to show here? Trying to show this.

\textbf{[00:48:22]} So I already have pushed a thing to. There's a repo. GitHub.com freemo cap university it is almost entirely, almost entirely AI generated. At this point, I found myself in a situation where I was like stuck in an airport for quite a long time. So I kind of like vibe coded this whole thing.

\textbf{[00:48:43]} So it's not like it was kind of like very long voice transcriptions type of stuff and like, like materials I already had. So it's, it's really fairly well tuned to like my actual. What it actually needs to be. But just saying don't go if it's AI slop in here. So don't get too.

\bigskip\noindent\rule{\textwidth}{0.4pt}\bigskip

\#\#\# Chunk 6 [00:48:45 - 00:58:45]

\textbf{[00:48:45]} Like very long voice transcriptions type of stuff and like, like materials I already had. So it's, it's really fairly well tuned to like my actual, what it actually needs to be. But just saying don't go if it's AI slop in here. So don't get too excited about it. But the general structure is it going to be.

\textbf{[00:49:04]} Not that what do we look for? This is a good representation. So this as well as a little vibe coded sort of viewer. But the basic idea is going to be to like building a curriculum using both my experience running the project and also my experience as a college professor teaching students and have kind of a module based curriculum with micro certifications associated with it. So the idea would be that you'd have like a beginner class, this is sort of pretty similar to our current documentation and then sort of a more advanced, you know, Freemo CAP usage class, you know, 101, 101, 201 type of thing and then splitting out into kind of different subspecialties.

\textbf{[00:49:42]} So you know, if you want to focus on like you know, the three main sort of sections are the technology meaning mostly like development and you know, the core tools. So you know, skellycam, Skelly Tracker, skellyforge, Skelly Blender, Skelly core kind of so splitting up the curriculum around the technology according to the structure of the actual project and then also subspecialties under like the scientific route which is the basically like the, the scientific applications of FREEMO cap. So using a FREEMO CAP as a tool for studying perception, motor control, biomechanics, motor control, getting towards clinical applications, things like that. And then the third, third, third leg of the tripod I guess is the artistic route. So that's the animation and game designer stuff.

\textbf{[00:50:34]} Realistically this is the part that I personally know the least about. But there's a lot, I'm learning a lot every day and there's a lot of people in the community who would I'm sure be able to help out with that. So this is sort of the idea of I guess official unofficial announcement of the possibility of starting to work on this. The, the sort of like the Freemo Cap University approach which will basically involve slowly over time me adding new, basically fleshing out each little blob here. Each.

\textbf{[00:51:08]} Each.in this graph is intended to be kind of like a little short course and short course meaning like, like you know, like a lecture plus some sort of optional. To have micro certifications you have to have some kind of an Output So we can, we're going to, you know, like record like one of the things might be, you know, instructions on how to like record your own recordings. Then you can upload it to some server. It does some sort of internal checks of, you know, validating that your data is not bonkers. And then if you get it, then you get a little check mark and then you get the micro cert and then that's, you know, basically it's a badge you can put onto various places and you give you a little badge in this, in the Discord server and stuff like that.

\textbf{[00:51:46]} So this is something, it could take a good long time, but I think on the scale of, you know, a year, I would hope to have some, some success in chunking this out. There's also like, once you have a curriculum like this, there's, there's funding that you can get around that like, like this is fundable at some level or at least can be a component of a grant. So we'll see. Oh yeah, another announcement. So I feel like I've been teasing this for a while, but Skelly Bot coming soon to a server near you, optionally, if you're not into it.

\textbf{[00:52:16]} So this is like a project that I've been, I've been working on for years now, but it hasn't really made it much to like the forefront of like the free mocap conversation. Long story short, I have been since I think it was the summer of 22, I can't remember exactly, I've been building and deploying AI based teaching tools in my classrooms. And so I wound up teaching six total semesters. Use. Oh, this busted.

\textbf{[00:52:43]} Using this Skelly Bot repository. Skelly Bot is basically an AI chatbot that lives in Discord. The tech, the tech has shifted. I don't. Discord has never been the right tool for the job there.

\textbf{[00:52:53]} But the general strategies that I have learned around using AI in a teaching context and sort of like, like, like building educational structures around the use of AI that open up the possibility of teaching in ways that were just not possible beforehand. And a big part of it is like using the AI bot as kind of like a simulation of what it would be if you could do what everybody knows is the best, the best way to teach, which is mentor based self learning. Right. Sit down with an expert and they like hold your hand and tell you stuff. That's obviously the best way to learn stuff.

\textbf{[00:53:28]} But you can't really scale that out. So a lot of the teaching structures I did was based around like Me giving some sort of centralized lecture kind of thing, usually in a way that's super dense and super like way over the heads of the undergrads in the room and then using kind of the specifically prompted AI bot to kind of like give them the experience of what it would be like if I were able to sit down and talk with them specifically. Had a lot of really good success with that. And I think that that same kind of model of using the AI is sort of like the buffer layer between the one to many of teaching where there's like one professor like object and many, many student like objects. Having like the AI middle layer there to sort of smooth out that transition is something that I want to incorporate into this world of open source software because I think that the, the one to many problem, boy howdy, it shows up.

\textbf{[00:54:22]} And I think that in the world where we're trying to bring more people into the project who can contribute and sort of like find their space within like the pretty broad technical landscape that we cover, these kind of teaching tools I think can do really well. Um, so we'll see. Right now, trying to get to the point of feeling comfortable enough with the structure to just put it in the server. I think that there's a lot of what we would call RTFM questions that, that show up. No shade, but it happens a lot.

\textbf{[00:54:50]} There's a lot of questions people ask that the bot could answer pretty well. So there's a certain social aspect of like, how do we sort of integrate AI into a community space without it like feeling shoved in in an uncomfortable way and an unhelpful way. So I'm trying to be very careful around that. But it is sort of, you know, again, on the space of a year I would hope to have this shoved in. And minimally it's definitely going to be a part of the educational program here because I'm going to be applying the teaching strategies that I've developed to this free mocap university project.

\textbf{[00:55:23]} Where are we at? Getting close. Skelly. Shop Hoo boy. Talking about money, cash money, guys.

\textbf{[00:55:29]} So this is one of those things. In terms of the patchwork of potential income streams, this is, I'd say, you know, short term small potatoes, long term medium potatoes. The main thing now is that there is now shop.frmocap.org it does exist. You can use it. And specifically right now the main thing is it does.

\textbf{[00:55:50]} It does is that you can. Does it not show the two where isn't. Wait, wait, where's the other one? How do I see the second page. Am I blocking it?

\textbf{[00:56:04]} Okay, I don't know why that isn't showing up there. So this is. You can buy a Truico board.

\textbf{[00:56:15]} It is not, it's not this, we changed to the smaller one. But this, this is an older one. It's this size, so it's 18 by 24. So it's pretty, pretty solid. It's on the sort of corrugated plastic.

\textbf{[00:56:26]} This one just has the skelly head. But the one you order here has the full cheat sheet, which looks really nice blown up big. So it's kind of like you use the front of the board to calibrate it and then the back of the board to sort of teach you what you're actually doing and give you sort of helpful tips and advice. It's currently running through a print on demand service. So it's like expensive and slow and we don't make any money off of it.

\textbf{[00:56:47]} So I don't like that. So it costs \$25. I think shipping winds up being like 10 or more. I don't, I don't know. And we don't, we don't make like.

\textbf{[00:56:55]} We don't make any money off of that. That's like breaking almost even. So I hate that it's expensive and I hate even more that it's expensive and then it's not. Not even getting, we're not even getting any money out of it. But it, but it does exist and that's the important part.

\textbf{[00:57:08]} And there are some, obviously, I think a lot of opportunities for more sophisticated systems. There's like, you know, like, if we just print them in bulk and then send them out ourselves, we can have a much different, you know, like the cost structure changes dramatically there. But then now you have like sort of more labor involved. This may be talking ahead of myself, but I'm sort of expecting to get connected to sort of a local like tech incubator, high school type of thing. And then, you know, people doing work, studying, stuff like that who can actually do the, the little manual labor there.

\textbf{[00:57:37]} So we're going to try to flesh this out. So I think currently this is a pretty good way to get a charcoal board. Like, I think we're competitive with what you would pay like a FedEx or a Kinkos or that type of thing. I have no idea what the landscape would look like internationally, but I'm speaking for America. And the.

\textbf{[00:57:54]} One of the more important pieces of work here is that the shop exists, it's connected to a bank, it's got Shopify, it runs and we're gonna, we're gonna keep working there. There's also again similar story. You can buy a seven dollar sticker but don't think that you're, you know we don't make any money off of this. This is, it's like the print on demand stuff is around it. Yeah, it's fine.

\textbf{[00:58:15]} Another ad has hit the stream, that's okay. But yeah, so we have, we do have this. You can now buy it. It's not it. We're getting, we'll get there.

\textbf{[00:58:24]} We'll get there one thing at a time. Yeah. And then so the other thing that the shop will eventually do is sell camera hardware. Camera hardware, extension cables, try little tripods, camera mouse, stuff like that. That's another thing that is going to take a little bit of time and sort of realistically would benefit from having you know, you know, some kind of captive audience of you know, low level labor.

\bigskip\noindent\rule{\textwidth}{0.4pt}\bigskip

\#\#\# Chunk 7 [00:58:30 - 01:08:30]

\textbf{[00:58:30]} Hardware, camera hardware, extension cables, try little tripods, camera mouse, stuff like that. That's another thing that is going to take a little bit of time and sort of realistically would benefit from having you know, some kind of captive audience of, you know, low level labor. Because I tried to figure out a way to do it through just like drop shipping methods, but ultimately you just kind of don't have enough control over the camera. I wasn't able to quite figure out options that sort of felt good to be able to sell. So similarly for this, I think it would want to be something where we, we buy some stuff in bulk and we kind of package it out on our own and that for that situation, cameras get real cheap when you buy them in bulk.

\textbf{[00:59:12]} So especially these kind of. I don't know if I have any I can reach right now, but like the kinds of cameras we've been using in a lot of my daily life, I bought you know, in a pack of a hundred, pack of 100 for a thousand dollars. Oh, did I lose my video again? One second. Hello.

\textbf{[00:59:30]} Hello. Okay, we're back.

\textbf{[00:59:37]} Hello, we're back. Speaking of cameras. Yeah, so if you buy, if you buy cameras in bulk, they get really, really cheap per camera. And I think that it wouldn't be too hard to sort of find a source like you know, the right kind of factory in China to, to get, you know, large amounts of, of cameras, then sell them and then sell like packages for classroom settings, stuff like that. And I think that people would, people would be into that.

\textbf{[01:00:03]} And then once you have a system like that where you're, you're fulfilling the, the mission and having like the cheapest possible entry point for motion capture, then you can also sell better cameras, higher quality stuff, you know, higher cameras. Cameras go all the way up and all the way down. And a lot of people, there's people out there who would be willing to, who want the research cameras. Like the people who are building research labs are not actually trying to minimize cost. So that's kind of a part of the infrastructure that we're kind of like planning out.

\textbf{[01:00:31]} And yeah, you know, in the short term probably not going to make any like, like re, like actual like meaningful income out of that. But you know, while, while I'm trying to get the cheapest possible option onto the shop, it's much more of a community support thing. Like I'm just trying to find a way that y' all can get the stuff you need to do motion capture like as cheaply as possible. Like I don't want you to have to go to Amazon or you know, I think we could even do better than like AliExpress and stuff like that with the right kind of planning and then just taking the question marks out of it. And then, you know, I don't want to make money off of that layer of the community.

\textbf{[01:01:05]} But you know, once we have that set up, the people who are, who have more money to spend, you know, we can sort of have different arrangements there. We'll see. Where are we now? Ah yes, validation. One of the very exciting moments of freemocap history is coming up, which is Aaron Ol.

\textbf{[01:01:22]} Aaron Cherian is about to defend his dissertation in the next couple of months. Aaron has been working dutifully since day one really, I think doing like the, the really hard work of validating freemo Cap as a, as an actual clinical research tool. These are just kind of some, it's more placeholder than actual anything else. Just like here's here, let me dump all the data down. And so his dissertation is about using FreeMap, validating FreeMap as a clinical research tool.

\textbf{[01:01:53]} And he is soon to be sort of defending that. And then the idea that he's got this. Where'd he go? Did I lose again? Oh, oh, I know here I clicked too many times.

\textbf{[01:02:04]} There it is. So we have this. Oh, is that like he's been working out of this Primo Cap validation repository and that will be both kind of like a source of code and data and visualizations and documentation and then also kind of serve as like a white paper for the validation. So he's going to defend that. We're going to sort of, you know, clean it up for more public facing capacities, publish it in some kind of peer reviewed journal or another.

\textbf{[01:02:29]} And then once that's kind of working, then we're kind of ready for a new plan, a new, a new target which is, this is another kind. This is like a multi year target and that's to get free mocap FDA certified as a clinical research tool. 510k certification they call it currently. So the rules are that Primo Cap cannot be used. We can, we can call ourselves FDA certified as a research, as a research tool simply by putting this statement on here that says that it is not a clinical tool.

\textbf{[01:03:03]} Which is really funny to me. Basically the idea is that you, you just, you say this is certified as a research tool. It is not to be. Yeah, not cleared for clinical use. So not to be used in diagnosis, not to be used in treatment, stuff like that.

\textbf{[01:03:16]} So you can use it for Research publication, like that kind of thing. But you can't use it to like diagnose somebody with an illness or something like that. To get to FDA certification as a, as a clinical tool is a set of somewhat complicated. It's. It's some complicated stuff like it's federal, federal stuff and it involves a lot around the, the code like the software itself being to a certain point, the documentation being to a certain point and showing that it is possible for researchers in other places to, to use the tool and have reliable outputs from it.

\textbf{[01:03:53]} It's you know, FDA sort of like the rule, the ruling of the for stuff like this, for free stuff like female cap is that there are not really risks to the patient for using it, but there are patients because it's complex enough that there's a possibility of clinicians using it and getting like bad data and then having bad sort of decisions based off of that. That's the risk that they're concerned about, which honestly is super valid. So the, the work that Aaron has been doing for his validation for his dissertation is kind of step one towards this sort of longer term journey towards getting this FDA certification. Realistically, it's like that's again, that's the kind of thing you can write grants around. It's probably going to be like a multi year target to, to shoot for.

\textbf{[01:04:39]} But once we do get there then we're now actually like you could put us into hospitals and kind of like I've had this dream for a long time of like the, you know, the \$500 gate lab. Like you usually like you don't get a gate lab in a hospital that doesn't have less than like, you know, five, six figures worth of funding and we could potentially build something that like you could put in, you could put in places that have really no resources and I don't know if you can think of any places that might be in the need of really low cost prosthetics fitting, but I certainly can. And this is something that you know, again you could pursue it in various levels. But you know, since we're talking about org level planning, we do want to target, you know like, like trying, like trying to like getting the federal certification I think is, is a target to be shooting for on the next, you know, n number of years, shall we say getting close. And then finally last but not least, the blender add on.

\textbf{[01:05:33]} The female cat blender add on. Which again Andres has been key and just key and critical for this and just doing really amazing work, you know. Yeah, if you, if you use Remote Cap at all. You've used the Blender add on and if you haven't, then you should because it's really cool. And the, the next sort of goal there is first of all we're going to transition the name.

\textbf{[01:05:55]} It's, it's. We're going to call it Transition, you know, transitioning to Skelly Blender so it fits the model. It's also kind of funny. And then the next target there is that there is a relatively small but sort of important list of things that have to be true that we'd have to hit to be able to be hosted in the official Blender extension. Oh sure.

\textbf{[01:06:16]} Come on man, we're not doing all that. So one of the things that happened with one of the, one of the more recent versions of Blender is that they've put out a unofficial extensions sort of repository where like single click install and I would love for FreeMailCap to be in there. So there's look not a ton but not horrible amount of work to be done to, to be able to be hosted there. And so that's going to be something we're shooting for and I believe that's it. So thank you to everybody who stuck around.

\textbf{[01:06:46]} It's always, you know, fun, kind of fun talking about plans, fun getting into it. I always feel a little just have to like get past the awkward. Like every so often you're like, I'm sitting alone in a room and then it's like, oh wait, no, there's people here. So it's Icon. Good to see you and Studio Golf.

\textbf{[01:07:02]} Good to see the names. Gonna see. Yeah, some of y', all, some of y' all been around for a long time. We've gone through, it's been. The world has, has gone through its paces and we are still here and we will continue to be here.

\textbf{[01:07:15]} Free mocap is a long term project, as you might be able to tell. Let's say, you know it. It was born out of a moral imperative that I sort of found for myself. A, you know, it's labor of love, you know, sort of necessary passion. People talk sometimes about like, people sometimes describe this as like a charitable activity and I guess technically, officially it is a charitable activity, but like it's not like I feel like I have a lot of other options, you know, like not.

\textbf{[01:07:42]} I mean I do have options but like this is, this is kind of like it just feels so much. It's just, it's the thing I should be doing, so I'm doing it and it's working out and it's not a, you know, a lot of people are around doing it. A lot of people helping. Community's growing. It's really, it's really great, really gratifying and exciting.

\textbf{[01:08:00]} And, yeah, it's like, it was like, what am I going to do, charge for it? That's insane. If I was going to charge for it, I would just stay a professor, you know, or go work for Google. When I talk to people about Free mocap, by far, by far, by far the most common thing that people say is, why is it free? You know, why is it free?

\textbf{[01:08:18]} What do you mean? Why aren't you charging for it? That's a thing of value. They say it's a thing of value. Why aren't you charging for it?

\textbf{[01:08:24]} But, you know, ultimately, one of the rules of free mocap is that.

\bigskip\noindent\rule{\textwidth}{0.4pt}\bigskip

\#\#\# Chunk 8 [01:08:15 - 01:17:55]

\textbf{[01:08:15]} Is why is it free? You know, why is it free? What do you mean? Why aren't you charging for it? That's a thing of value.

\textbf{[01:08:20]} They say, it's a thing of value. Why aren't you charging for it? But, you know, ultimately, one of the rules of Free MOCAP is that anything that can be duplicated infinitely is. We don't charge for. So I don't.

\textbf{[01:08:37]} I can't imagine the ethical grounds of, like, just charging for a piece of binary code for charging for a pile of Python, you know, like, you know, by whatever strange quirks and twists of history and my talent and mentorship and training and whatnot that I've managed to receive, I, for whatever reason, you know, found myself with the ability to build a project like this and to teach other people how to build a project like this. And the idea of, like, there was a time when I would look at, like, the paid versions of Freemo Cap, like the move AIs and the Cocos and whatnot, and I was like, well, you know, no harm, no foul. But I radicalized a lot and I, you know, yes, harm, yes foul. You know, like, it's. It's just, I see that as like you're another person, whoever, another person, group of people who also have this, like, this, like, capacity to build this kind of thing.

\textbf{[01:09:32]} And then rather than giving it away, you leverage the. The distinction, that distance between you and the world around you to, like, suck money out of people. And it just. I don't like it and I don't support it. So, yeah, so we will free mocap.

\textbf{[01:09:47]} We seize the means and. And we see where it goes. Thanks. Thanks, Micah. And so then to that question, to the question of, like, you know, why is it free?

\textbf{[01:09:56]} I kind of. I've discovered a response to that. I've discovered a good answer, and that's. You just say it's a moral thing, and people go, oh, oh, I get that. I get morals because it's like, it short circuit the conversation.

\textbf{[01:10:10]} Because it's like everyone, you're living in this world of capitalism and people, I think you literally, you know, they say, oh, that's a thing of value. Why aren't things of value are based and you charge for that. That's how you survive in this horrifically idiotic structure where the base level human interaction is assumed to be competition. And so when you. So if you flip the question and you say, rather than asking, why is it free?

\textbf{[01:10:38]} Just. Just sort of start from the basis that it must be free and then ask the question of how are you going to support yourself? Building the thing that must be free. And. And I think that there's answers to that, and I think that we're finding them.

\textbf{[01:10:52]} And I think that a lot of it comes from, you know, understanding the value of your labor and charging for it. Charging and, you know, charging a lot, because that's what it's worth. I know the value of my work, it's high. And, you know, we'll see. We'll see.

\textbf{[01:11:09]} It seems at this point, you know, I'm not sure what our bus factor is right now, but I think it's pretty okay. I think it would be. It'd be a bummer, but I think we could survive a relatively unfortunate incident with a bus, possibly multiple buses, but I'm not sure about that. At least one bus. And.

\textbf{[01:11:27]} And, yeah, and we'll see. You know, I think. I think Close out, yo. It literally used to be called Open Mocap. There was a period of time where it was called Open Mocap.

\textbf{[01:11:38]} And I think there was some, like. There was some. Some. There was, like, a name conflict. Something else was called Open Mocap.

\textbf{[01:11:44]} And then I transitioned the name and 2. To quote brother Paul, Neo Next Death. I was like, I'm thinking about, like, Open Mocap versus Free Mocap. And. And he said, well, if I saw both of those next to each other, I would definitely go for Free Mocap first.

\textbf{[01:11:59]} And. Yeah, I think. I think that's correct. And it's like, you know, you're absolutely right. Open means something different now than it did five years ago.

\textbf{[01:12:08]} And I think if we. It's like, you know, it's like, if it was Open Mocap, we could charge for it without changing the name. And that feels wrong. With Free Mocap, it would. You have to change the whole name, change it to Fee Mocap.

\textbf{[01:12:22]} Someday. I have a dream about, like, having a version of Free Mocap, which is literally exactly the same, except it's called Fee Mocap. And Skelly's got, like, a monocle and a hat, and it just. You just charge \$20 for it. It's like a way of giving a donation.

\textbf{[01:12:36]} And, like, everything is exactly the same, except that the Skelly Mesh is wearing a top hat with a monocle, and. And the logo says FEMA Cap. That's. We'll get there. Not yet.

\textbf{[01:12:46]} That's going to be. That's. That's down. Down the road a little bit. Okay.

\textbf{[01:12:50]} So I said I was going to talk for 45 minutes to an hour. I've been talking for an hour and a half. This is how I roll. But we'll end it here with this nuclear explosion, this sort of global. Global thermonuclear apocalypse of people building skeletons out of light in their living rooms.

\textbf{[01:13:10]} And, yeah, guess it's like, I see you. I see you there in the Faroe Islands. I see you doing your thing. Who knows? Look at the.

\textbf{[01:13:18]} Look at you in Oslo and Stockholm with your syndromes. Look at you in India. Who's this guy? Who are you? Right?

\textbf{[01:13:27]} Who are these guys? What are they doing? What are they building in there? You know? That's what I want to know.

\textbf{[01:13:32]} Thanks for not unchecking the box, guys. I like to see the dots. It really, truly, truly keeps me going to just kind of like, every so often, log and be like, holy shit. Holy. You guys are using it.

\textbf{[01:13:44]} Maybe someday I can go on, like, a Johnny Appleseed tour and just pick two blobs and then just, like, go there. Yeah. 10K, man. Yeah, I. I'm so glad I set up that ping thing, you know, it was a little bit of. It was.

\textbf{[01:13:58]} It felt weird to do any, like, data collection, but, like, boy, I don't. Man, it's hard to imagine. It's hard to imagine. I. I don't know. Like.

\textbf{[01:14:06]} I don't know. Look at them. Look at them there in Manila. Quazon. K Zone.

\textbf{[01:14:11]} I can't read that word. Yeah. If you make something for a community that doesn't have resources. No, that's not. It's not that.

\textbf{[01:14:20]} It's like, if you make, I don't know, people, like, free stuff. And I just feel like, you know, from a business perspective, like, if your user base is. If your cost is zero, then your user base is theoretically infinite. And in software money world, they. What they talk about, like, what's it called?

\textbf{[01:14:36]} It's like money per user. Like, price per user, resources per user, whatever it is. Dots are light in the darkness. I should do that. That would be nice.

\textbf{[01:14:44]} Oh, can I have that? Do I have that? Hold on. Oh, no. Oh, wait, no, I can still do it.

\textbf{[01:14:47]} There you go. See, that's nice also. Yeah. Some good contour maps, heat maps. I don't know how to make it go away.

\textbf{[01:14:54]} Anyways, I. Oh, that is. I agree. That's nicer. Shows up. Yeah.

\textbf{[01:14:59]} Okay. Oh, wait. Actually. Oh, Anyway. All right.

\textbf{[01:15:03]} Last call for comments then. I am. I'm gonna go pass out. I'm gonna. I feel like I say this every time I do a stream.

\textbf{[01:15:10]} I'm gonna try to stream more because I do, like, it. And it's fun and it's a good way to kind of like, you know, just think. I think with the project where it is now, the communication stage, it. It continues to be really important because I think the. The scale of the project is such that it would be easy from the outside to not realize that there's a lot of action here because it's.

\textbf{[01:15:29]} We move slow because we have a, you know, you know, the effort, low available effort. But I just think it's important to. Yeah. So having streams, having talk, you know, putting. Putting stuff out.

\textbf{[01:15:41]} Generally we get like. It's famous last words. I plan on stream more often as famous last words. And they're the last words I think of every single stream I've ever made. Someday it'll be true.

\textbf{[01:15:51]} Maybe. I guess. I guess that's not a guarantee. But someday it might be true. You never know.

\textbf{[01:15:55]} Okay, let me now let's play the game of let's figure out where to raid y' all to While I'm looking for a raid is a good time to say stuff if you want. But yeah. Thanks for watching. Thanks for. If you're watching this live or if you're watching this not live.

\textbf{[01:16:09]} Thanks for watching. Thanks for your interest. I don't know the science person. Chunky stuff. Yeah, let's kick you over here.

\textbf{[01:16:19]} That seems like a good choice. All right, now I go over. I know how to do this. I know how to do this. Oh yeah, Nikon's talking about fancy equipment.

\textbf{[01:16:28]} It is another thing that I haven't, I didn't mention. But the. As the core software stabilizes, it opens up more room to like have like non default options. So a big part of the work that's happening for V2, which I didn't mention, is like removing the necessity to use MediaPipe for everything and having the more hot swappable models. Like the Ferret stuff was built with deep Lab cut and then Aaron's been working with RTM pose and Vit Pose as.

\textbf{[01:16:57]} As options. So in the similar space of like making more generic make the software becoming more generic and able to handle more options. Weirdo. Fancy cameras is also kind of a part of that conversation. Like I still have all my 3D depth camera cameras and you know, stereo and you know, Kinect kind of cameras that would be so much fun to integrate into the free mocap thing.

\textbf{[01:17:20]} But I have not allowed myself to do it because it's like a sort of a. For, you know, specialized equipment is sort of forbidden. But you know, I think as things get easier to integrate. It'll be more possible to do that kind of thing. Because it's fun.

\textbf{[01:17:33]} Weird. Cameras are fun. Raid. Raid. Okay, go talk to this guy.

\textbf{[01:17:38]} I hope he's nice. And good talking to you guys. Have a nice night. Bye. Thanks.

\textbf{[01:17:44]} And then I think I have to wait 10 seconds. Yeah. Okay. And there it goes. Three, two, one.

\textbf{[01:17:50]} Oh. Now I click the button. Bye. Okay, now I have to figure out. Turn this off.

\bigskip\noindent\rule{\textwidth}{0.4pt}\bigskip

\normalsize


\chapter{Interview Transcript I --- On First Skeleton and Vision}
\label{src:S7}

\begin{framed}
\small
\textbf{Source marker:} S7 \\
\textbf{Date:} 2026 \\
\textbf{Source:} Transcribed interview
\end{framed}

\small
Some other things, I think, to talk about from your questions. You asked, what do I want someone to feel when they successfully capture their first set skeleton That's an interesting question. And I think there's kind of multiple ways to think about that. I think for some people, like for young people or people from the technological outside who don't feel connected to technology, who don't sort of feel themselves as someone who can use technology fluently, who kind of feel like their technology is being pushed on top of them rather than something that they can harness and choose for themselves, whoever that might be. I want them to kind of think of like, oh, I can do this. I can make this happen. I can set it up. And even if I don't understand everything that I'm seeing, there's little glimmers of things that I can start to understand. I can feel the understandings beginning. And I can feel the opportunities to pull threads and take in. And free mocap wants to be designed kind of as a tool that teaches you things about how it works. And I think every individual person is going to be pulled in a different direction and sort of drawn to different parts of the thing. And I think that's something we want to encourage. Outside of that, I think from a broader picture of people outside looking in the project, I think I want people to see free mocap and the free mocap project and the free mocap foundation as an example of like, oh, look. Look at what we can make. Look at what you can build if you just step away from the standard path. Look at what we have the capacity to do. We can build tools that perform in the ways that they should, that just work and don't try to take anything from you and just give to you. It is possible to build and give gifts freely given to the world without extracting resources as a result. And I think that I want to be able to show that that's a world you can live in, that that's a thing that you can do, and that we collectively as a group have the options of building things the way that they should be. And that the world that is is not the world that has to be. And that even though the path is forever, the work is there. The handles are there. The shovels are there. We know how to pull. We know what the right direction is. We know how to pull in that direction. We know how to dig the holes that need to be dug. And even if we don't, the distance just looks insurmountable. There is a way to start. And I think that's something I want to show and communicate.,
  

\normalsize


\chapter{Interview Transcript II --- On Strategy and Liberation}
\label{src:S8}

\begin{framed}
\small
\textbf{Source marker:} S8 \\
\textbf{Date:} 2026 \\
\textbf{Source:} Transcribed interview
\end{framed}

\small
Okay, I'm going to try to get a broader cut across your questions. So breadth over depth, I guess right now. Yeah, in terms of the question like staying in academia, the actual practical strategy that I took was to after deciding that I couldn't stay sort of deciding that in the context of, I'm not going to try to get tenure at the end of my tenure contract, which was, you know, six year contract. There was a part of me that kind of wanted like when you want to like stomp out the door and be in like a huff or whatever, but that's not actually a strategic move because you're leaving a lot of resources on the table and like you don't have anything to step into. So, you know, I decided that, you know, I signed a contract to be there for several years and I did not, I never, you know, tenure is a two way street, you know, and I had already determined that this institution was not worthy of my, you know, it wasn't morally aligned with my principles. But I was still going to stick around and, you know, use the resources and the time that it afforded me to try to figure out like what, what would be better, like to do the real deep introspective, you know, you know, effort and also just like the work of trying to build an alternative system, which is kind of what free mocap became. And so from a purely practical level, it was, you know, the question was, how can I, what, how can I set up my life and what kind of institutions could I build that would be more, more morally aligned? It was sort of like a prefigurative kind of approach of trying to, you know, just sort of like, like what, what if this institution is not what I think it should be, what is the thing that should be like, what is the like, what would work, what would allow me to continue to be the kind of like scientist and researcher and mentor and teacher and all that I wanted to be without having to be attached to an institution that I couldn't morally stomach. And so free mocap and the free mocap foundation is kind of the answer to that question. To answer your question about the clinical term, it is, free mocap is absolutely part of that. It's sort of like, I feel like I have to have something that I'm working towards and building towards, and I have to feel like I'm somehow pulling in the right direction in order to feel sane. And so free mocap is sort of like the embodiment of that effort and specifically sort of utilizing the specific skill sets and understandings that I have to try to feel like I'm, you know, to get back to that feeling I had of being attached to a good vehicle and pulling the world in a good direction. I think in terms of, you know, who I'm trying to reach, more recently, I think a lot of my focus has really shifted towards like the potential of technology as a liberatory tool. And specifically how I think if enough people in the world sort of gained the kind of technical competency necessary to like really engage in the open source community and like, then we can literally seize back the means of production and, you know, free people of having to live their entire lives attached and managed by technology. You know, free people of having to live their entire lives attached and managed by, you know, corporations. And now at this point in history, governments, because we are entering the, you know, into the AI era. And I think all the kinds of fears and dangers that I was talking about those years ago and those older recordings have just become far sharper and far more poignant. And so a lot of the focus now is starting to shift more in the direction of like, you know, free mocap specifically and then free open source software in general, as a way of like teaching people the skill sets that they would need in order to be able to claw back our culture and our society and our technology from these, you know, corporate entities that, you know, have shown themselves not to be worthy of the power that they have, you know, captured and gathered. And so I think, you know, winning is a hard question. I think in the, right now in the transition, we're moving from, you know, for the personal period of my life, you know, where I was, you know, a professor, still a professor at Northeastern, but this is my last year. So I'm transitioning into the world of trying to figure out how it is possible to support yourself, oneself, a lab scale of, you know, group of people, while building free software that does not, you know, charge subscription fees and, you know, like how do you survive in the world when you're, if you're not going to be attached to like a big research institution or a big corporation. And so that's kind of the question we're trying to answer right now in the very small scale of just like me and two or three other people in my lab. And that would be winning if we can get a stable position there, more broadly and more long term. You know, I would really, the dream is that if I can figure out that one case study of how to support myself as a researcher and educator and mentor and software developer and all the things that I am, by building a tool like FreeMoCap, if I can figure out how to do that, then that could potentially serve as a model that other people start to do. And like FreeMoCap could kind of serve as like a conduit and umbrella and like a support system for people who are trying to achieve their own sense of independence and escape and, you know, whatever it means for them to pull in the right direction. And I think that, you know, that's a long term dream, but the hope would be that like, you know, even though, you know, motion capture is not the, like, in the long list of inequities and wrongs in the universe, like inaccessibility of, you know, marketless motion capture is pretty low on the list, but it is on the list. And if we can sort of solve that tiny inequity, then it might serve as a model of how to solve bigger ones. And if the tool itself can serve as an inroad to like teaching a wider set of people the skill sets necessary to sort of work in the same direction, then it might be a, you know, catalytical project that could, you know, sort of start as a seed and wind up extending and touching things beyond its particular scope and particular domain.",
   
\normalsize


\chapter{Compiled Learnings: The Soul of FreeMoCap}
\label{src:CL}

\begin{framed}
\small
\textbf{Source marker:} CL \\
\textbf{Date:} February 2026 \\
\textbf{Source:} Analytical compilation from all source transcripts
\end{framed}

\small
\# Compiled Learnings: The Soul of FreeMoCap

This document captures what I've learned from the transcripts and interview that will form the basis of the revised strategic guide. \textbf{Note: This is compiled from available sources—claims marked with [?] indicate areas where I'm inferring or where follow-up questions would help.}

\bigskip\noindent\rule{\textwidth}{0.4pt}\bigskip

\#\# Part 1: The Conversion Story

\#\#\# The Timeline

- \textbf{Pre-2020}: Jon describes himself as "totally all in" on academia. Describes having had a "meteoric rise" as a research scientist. Fully believed scientific research was "important and noble and was helping at some fundamental level."

- \textbf{January 2020}: Just gotten tenure-track faculty job at Northeastern. Lab built, ready to start. "A very funny time to do planning."

- \textbf{March 2020}: Covid hits. Everyone sent home.

- \textbf{Summer 2020}: The betrayals begin:
  - Faculty meetings where a dean says they should "encourage students to get master's because master's students are the best income ratio - they pay full tuition but don't require any resources." Jon: "That kind of stuff is super gross."
  - Strong push for in-person classes pre-vaccine while "people are still actively dying all over the place" - clearly prioritizing money over safety.
  - Jon realizes: "You start to see different conversations" once you're on the professor side of the fence.

- \textbf{The Breaking Point}: Jon receives letter saying he must teach in-person, with a rider saying he can opt out if uncomfortable. He opts out. Then gets letter from department chair: policy changed. If you opt out of in-person teaching, "you cannot come onto campus for any reason at all, including going into your lab."

\begin{quote}\itshape
"As a first year research professor, that's basically saying, if you don't come in and teach in person against your will, we're going to tank your career."
\end{quote}

- \textbf{The Response}: "Okay, well, fuck you. I'll build my own lab in my house with webcams."

- \textbf{The \$3 Million Moment}: Teaching remotely, looking at webcam, counting Zoom rectangles. Realizes he's "speaking to a webcam that was worth \$3 million a year in student debt." But worse:

\begin{quote}\itshape
"The reason why they were going into that amount of debt was because of the promise of being able to learn from somebody like me... the fact that I was there doing the thing that I did was part of the recruitment that they used to bring these kids in and to convince them that it's worth it, which it fundamentally isn't and cannot be."
\end{quote}

\begin{quote}\itshape
"More than anything else, that moment was the time when I was realizing finally... I truly cannot stay here. I cannot continue to be in this position."
\end{quote}

\#\#\# The Emotional Reality

- \textbf{Grief}: "Flailing around in the grief of losing my faith in this institution that I kind of based my life on"

- \textbf{Trauma}: "Deeply traumatizing, I think"

- \textbf{The Stakes}: "I was 35 years old. I had just gotten the faculty job that I had arguably been working towards my entire life. And then all of a sudden while I'm there, I'm like, oh wait, this place is completely morally bankrupt and diametrically opposed to my worldview. And I can't stay here."

- \textbf{Betrayal}: "Realizing that actually what I was doing had been kind of harnessed and captured by this institution that I frankly despised and was morally repugnant to me"

\#\#\# The Simultaneous Discovery

As faith in academia collapsed, faith in FOSS emerged:

\begin{quote}\itshape
"In those moments of losing faith in the academic system... seeing something like Blender and starting to understand the business model and the community model... that galvanization happened together. As I was losing faith in the one, I started to gain respect for the other."
\end{quote}

\begin{quote}\itshape
"In those moments of losing faith in one institution, you kind of go back to try to figure out what your own morality was and what your own values are... recognizing [Blender] as another organism, a different institution and a different communal structure that felt like it was more aligned."
\end{quote}

\bigskip\noindent\rule{\textwidth}{0.4pt}\bigskip

\#\# Part 2: The Core Belief

\#\#\# "The open source community is what the scientific community pretends to be"

This phrase appears in:
- 2022 VFX Futures podcast
- 2022 Biomechanics podcast  
- 2026 State of the Skelly address

It seems to be a central thesis. The claim appears to be that science \textit{says} it is:
- Collaborative
- Building on shared knowledge
- Open
- For the common good

But Jon's experience suggests it's actually:
- Siloed
- Competitive
- "Fighting for survival"
- Captured by institutions that prioritize revenue over mission

\textbf{[?] Follow-up needed}: Does Jon see FOSS communities as genuinely achieving these ideals, or as \textit{closer} to them? What are the limitations he sees in FOSS?

\bigskip\noindent\rule{\textwidth}{0.4pt}\bigskip

\#\# Part 3: What FreeMoCap Is For

\#\#\# The Equality Statement

\begin{quote}\itshape
"We want indie game designers and animators to use the same tool to add motion capture assets to their zero budget art project that I am using for my federally funded scientific research program."
\end{quote}

Same tool. Zero budget = federal funding. This appears to be a core commitment to equality of access.

\#\#\# The Educational Trojan Horse

\begin{quote}\itshape
"If accidentally, through the execution of their own desires, they learn a little bit about computer vision and AI and technology and science. That's probably a good thing."
\end{quote}

\begin{quote}\itshape
"The goal for me as a scientist and as an educator is to make a tool that allows you to use these new technologies to do the thing you want to do with as little effort as possible... but also expose to you as much of the underlying technology as I can so that you... now have a little bit of familiarity with this really important and emerging technology just by way of your own desire to do the things you wanted to do."
\end{quote}

\#\#\# Concerns About Surveillance Capitalism

Jon has expressed concern about corporate versions of this technology:

\begin{quote}\itshape
"Markerless motion capture is going to happen somehow, somewhere. We have the ability to track people accurately through cameras and you better believe we're going to do it as a culture... I am absolutely concerned about what free mocap would look like if it was Facebook mocap... All I can really hope is that making a free open source version of this thing, which is absolutely going to be a technology that is part of our future, hopefully that will provide some sort of protection."
\end{quote}

\textbf{[?] Follow-up needed}: Is FreeMoCap explicitly conceived as a defensive/preemptive project against surveillance capitalism? Or is this more of an ancillary concern? How central is this to the mission?

\#\#\# A Sense of Urgency (2021)

\begin{quote}\itshape
"If you're sitting here in 2021 looking at the future and you're not feel like you're staring down the barrel of a gun, you should probably maybe pay better attention."
\end{quote}

\textbf{[?] Follow-up needed}: What specifically was Jon referring to here? Has this feeling intensified or shifted since 2021?

\#\#\# The Target Audience

\begin{quote}\itshape
"I'm pitching to people at or below me on the social hierarchy, whatever that means. And often what that means is young people."
\end{quote}

\begin{quote}\itshape
"That's all I got sitting here as a white man in power in 2021. That's what I got."
\end{quote}

\textbf{[?] Follow-up needed}: How does Jon think about his positionality now? What responsibility does he feel comes with his position?

\bigskip\noindent\rule{\textwidth}{0.4pt}\bigskip

\#\# Part 4: The Blender Inspiration

\#\#\# What Blender Represents

\begin{quote}\itshape
"I think the open source community is what the scientific community pretends to be. And I think that we should really adopt more of that model."
\end{quote}

\begin{quote}\itshape
"One of the beauties of the open source community is that if you are working on an open source project that's related to another open source project, you just combine forces."
\end{quote}

\begin{quote}\itshape
"A large scale, free, open source software that has its own legs and is growing in really interesting ways. I look at that and I think of that sort of large scale collaborative endeavor with an actual end goal project that is usable and useful for general population. And I see that as like, that's a better model for science. That is how we should be doing science."
\end{quote}

\#\#\# Why Blender Specifically (Technical)

\begin{quote}\itshape
"If you look at an animation... what you're looking at is artists who have used a GUI to have an incredibly fine grained control over minute aspects of 3D space... Having the tools that are available to animators, available to scientists, will make scientists better scientists and will also connect that artistic side of technology and 3D animation to the sciences and biomechanics."
\end{quote}

\textbf{[?] Follow-up needed}: What specifically has FreeMoCap borrowed from Blender's model? Where has it diverged? What aspects of Blender's approach don't fit FreeMoCap's situation?

\bigskip\noindent\rule{\textwidth}{0.4pt}\bigskip

\#\# Part 5: The Community

\#\#\# Named Contributors (from 2026 State of the Skelly)
- \textbf{Philip}: "Special shout out... for doing a lot of the plumbing necessary to keep the dang thing up and running and fixing the bugs and making it workable and also just basically dealing with all of my code slop"
- \textbf{AJC}: "Inimitable... work in the FreeMoCap community space... made a lot of the work in the Blender add-on"
- \textbf{Neon Xdeath}: Wrote the song "Meowmaline" used in early demo video
- \textbf{Taylor Davies}: Designed the logo
- \textbf{Aaron}: Working with RTM pose and Vit Pose
- \textbf{Brother Paul}: Helped with rough cut tutorial

\textbf{[?] Follow-up needed}: Who are these people? How did they find FreeMoCap? Why did they stay? What's the difference between people who download and leave vs. people who become community?

\#\#\# The Dots on the Map

\begin{quote}\itshape
"I see you. I see you there in the Faroe Islands. I see you doing your thing... Look at you in Oslo and Stockholm... Look at you in India. Who's this guy? Who are you? That's what I want to know."
\end{quote}

\begin{quote}\itshape
"Thanks for not unchecking the box, guys. I like to see the dots. It really, truly, truly keeps me going to just kind of like, every so often, log and be like, holy shit. You guys are using it."
\end{quote}

\begin{quote}\itshape
"It felt weird to do any data collection, but boy, I don't... it's hard to imagine."
\end{quote}

\textasciitilde{}10K users globally as of 2026.

\#\#\# The De-Centering Goal

\begin{quote}\itshape
"A big part of the transitional phase... for FreeMoCap over the next five years is one that really continues to build the community around it and kind of de-centers me as a specific person."
\end{quote}

\textbf{[?] Follow-up needed}: What would de-centering actually look like? What's currently in the way? Is there anything Jon is holding onto that he shouldn't be?

\bigskip\noindent\rule{\textwidth}{0.4pt}\bigskip

\#\# Part 6: The Path (Philosophy → Neuroscience → FOSS)

\#\#\# Origin

- Bachelor's in philosophy (philosophy of mind, philosophy of science)
- Applied to philosophy grad programs, rejected from all
- "By grace of God was not accepted"
- Worked at autism research facility: "Oh wow, data is cool"
- Applied to cognitive science programs, planned to study language
- Accidentally ended up with Brett Fajin studying visual control of locomotion
- "I haven't hit the ground yet. So here's hoping."

\#\#\# The Markerless MoCap Seed

- 2017: Saw OpenPose demo, "That's the future of motion capture"
- Been daydreaming about it since
- Working on proto-markerless mocap in lab pre-Covid
- Covid pivot made it the main project
- "Learned Python during the pandemic. I was a Matlab developer before that and a philosophy major before that."

\textbf{[?] Follow-up needed}: How does Jon's philosophy background shape how he thinks about FreeMoCap? Does he still feel like he's "falling" or has he landed somewhere?

\bigskip\noindent\rule{\textwidth}{0.4pt}\bigskip

\#\# Part 7: Tensions and Contradictions

\#\#\# The Institutional Contradiction

Jon appears to be (or was?) a professor who doesn't trust the ivory tower. Building free tools while potentially employed by institution charging high tuition.

\textbf{[?] Follow-up needed}: What's Jon's current relationship to academia? Did he leave Northeastern? How does he navigate the contradiction of being inside an institution he critiques?

\#\#\# The Telemetry Tension

\begin{quote}\itshape
"It felt weird to do any data collection, but boy..."
\end{quote}

Seems to value privacy and autonomy, but the ping data showing global usage appears emotionally sustaining.

\textbf{[?] Follow-up needed}: How does Jon think about this tension? What's the ethical line for data collection in his view?

\#\#\# The Proprietary Dependency

Currently depends on MediaPipe (Google) for core functionality. Jon is aware of the irony:

\begin{quote}\itshape
"I can't say trust and Google in the same word. That's not a good idea."
\end{quote}

\textbf{[?] Follow-up needed}: How does Jon think about depending on Google tools while building a project partly motivated by concerns about corporate tech?

\#\#\# The Scale vs. Sustainability Tension

\textasciitilde{}10K users, small team, "low available effort." 

\textbf{[?] Follow-up needed}: What's the actual resource situation? How many hours, how much help, how much money? What would sustainable look like?

\bigskip\noindent\rule{\textwidth}{0.4pt}\bigskip

\#\# Part 8: Key Phrases and Quotes

\textbf{These appear to be important and could be featured in the document:}

1. "The open source community is what the scientific community pretends to be."

2. "Okay, well, fuck you. I'll build my own lab in my house with webcams."

3. "Speaking to a webcam that was worth \$3 million a year in student debt."

4. "We want indie game designers and animators to use the same tool to add motion capture assets to their zero budget art project that I am using for my federally funded scientific research program."

5. "If you're sitting here in 2021 looking at the future and you're not feel like you're staring down the barrel of a gun, you should probably maybe pay better attention."

6. "Markerless motion capture is going to happen somehow, somewhere... I am absolutely concerned about what free mocap would look like if it was Facebook mocap."

7. "That's all I got sitting here as a white man in power in 2021."

8. "I see you there in the Faroe Islands... Thanks for not unchecking the box, guys. I like to see the dots. It really, truly, truly keeps me going."

9. "This place is completely morally bankrupt and diametrically opposed to my worldview. And I can't stay here."

10. "As I was losing faith in the one, I started to gain respect for the other."

\bigskip\noindent\rule{\textwidth}{0.4pt}\bigskip

\#\# Part 9: What the Current Document May Be Missing

Based on this material, the current "Theory to Victory" guide might benefit from:

1. \textbf{The visceral origin story} - the blackmail, the \$3M webcam moment, the grief
2. \textbf{The personal stakes} - 35 years old, life's work, institutional betrayal
3. \textbf{The emotional register} - trauma, grief, rage, hope
4. \textbf{The specific enemy} - not abstract "hegemony" but concrete institutional experiences
5. \textbf{The simultaneity} - losing faith in one institution while discovering another
6. \textbf{The position} - awareness of being a "white man in power" trying to give tools to those below

\textbf{[?] Needs verification}: Does the theoretical framing (Gramsci, Illich, Freire, McAlevey) resonate with how Jon actually thinks? Does he use that language? Where does it fit and where does it miss?

\bigskip\noindent\rule{\textwidth}{0.4pt}\bigskip

\#\# Part 10: Questions Still Unanswered

\#\#\# About the Origin and Motivation
1. Did Jon leave Northeastern? What's his current relationship to academia?
2. Is FreeMoCap conceived as a defensive/preemptive project against surveillance capitalism, or is that ancillary?
3. What was the "barrel of a gun" referring to specifically? Has that feeling changed?
4. How does the philosophy background shape the project?

\#\#\# About the Community
5. Who are the key community members and what are their stories?
6. What's the difference between people who leave vs. people who stay?
7. What would "de-centering" Jon actually look like?

\#\#\# About Sustainability
8. What's the actual funding/resource situation?
9. What's the governance structure now vs. planned?
10. What does "winning" look like concretely?

\#\#\# About the Frameworks
11. Do the theoretical frameworks (Gramsci, etc.) resonate with Jon's actual thinking?
12. Is there a simpler way to describe what FreeMoCap is about?
13. What do you want people to FEEL when they run FreeMoCap successfully?

\#\#\# The Hard Questions
14. What's the most likely outcome for FreeMoCap in 10 years, honestly?
15. What are you afraid of?
16. What's the question I should have asked but didn't?

\bigskip\noindent\rule{\textwidth}{0.4pt}\bigskip

\textit{This document will be updated as more information becomes available from the ongoing interview.}
\normalsize


\chapter{The Soul of FreeMoCap --- Initial Compilation}
\label{src:SF1}

\begin{framed}
\small
\textbf{Source marker:} SF1 \\
\textbf{Date:} February 2026 \\
\textbf{Source:} Narrative compilation
\end{framed}

\small
\# The Soul of FreeMoCap

\#\# A Document of Origin, Mission, and Vision

\textbf{Compiled from interviews, lectures, and public statements by Jonathan Samir Matthis, Founder of the FreeMoCap Foundation}

\textit{February 2026}

\bigskip\noindent\rule{\textwidth}{0.4pt}\bigskip

\#\# Epigraph

\begin{quote}\itshape
"The world that is is not the world that has to be. And even though the path is forever, the work is there. The handles are there. The shovels are there. We know how to pull. We know what the right direction is. We know how to dig the holes that need to be dug. And even if the distance looks insurmountable, there is a way to start."
\end{quote}

\bigskip\noindent\rule{\textwidth}{0.4pt}\bigskip

\# Part I: The Origin Story

\#\# The Before

Jonathan Samir Matthis's path to FreeMoCap was not direct. A philosophy major who applied to philosophy graduate programs and was rejected from all of them—"by grace of God," he now says—he wound up working at an autism research facility, discovered that "data is cool," and eventually found his way to studying the visual control of human locomotion.

By 2019, after years of graduate work and a postdoc, he had achieved what he'd been working toward his entire adult life: a tenure-track faculty position at Northeastern University in Boston. He arrived in summer 2019. The lab was built. The research program was ready to launch.

"I had about a year of like, 'Boy, I sure did make it. And this is probably fine,'" he recalls.

Then came 2020.

\#\# The Breaking

March 2020: COVID hits. Everyone is sent home. The lab sits empty.

But it wasn't the pandemic itself that broke something. It was what the pandemic revealed.

In faculty meetings, Jon watched administrators make calculations that turned his stomach:

\begin{quote}\itshape
"A dean or some similar type of person said something to the effect that we should encourage our students to get master's degrees because master's students are the best income ratio—they pay full tuition but they don't require any resources. That kind of stuff is super gross."
\end{quote}

There was pressure to return to in-person teaching before vaccines existed, "while people are still actively dying all over the place." The reason was obvious: the university needed the money. Safety was secondary.

Then came the moment that crystallized everything.

Jon received a letter saying he could opt out of in-person teaching if he felt uncomfortable. He opted out. Then he received a second letter: the policy had changed. If he opted out, he couldn't come to campus for any reason—including his lab.

\begin{quote}\itshape
"As a first year research professor, that's basically saying, if you don't come in and teach in person against your will, we're going to tank your career."
\end{quote}

His response:

\begin{quote}\itshape
"Okay, well, fuck you. I'll build my own lab in my house with webcams."
\end{quote}

\#\# The \$3 Million Webcam

But there was another moment—quieter, more devastating.

Teaching remotely, looking at his webcam, Jon counted the rectangles in the Zoom call. He did the math.

\begin{quote}\itshape
"I was speaking to a webcam that was worth \$3 million a year in student debt."
\end{quote}

Bad enough. But then the deeper cut:

\begin{quote}\itshape
"The reason why they were going into that amount of debt was because of the promise of being able to learn from somebody like me... the fact that I was there doing the thing that I did was part of the recruitment that they used to bring these kids in and to convince them that it's worth it, which it fundamentally isn't and cannot be."
\end{quote}

\begin{quote}\itshape
"More than anything else, that moment was the time when I was realizing finally... I truly cannot stay here. I cannot continue to be in this position."
\end{quote}

\#\# The Grief

What followed was not just anger. It was grief.

\begin{quote}\itshape
"I was 35 years old. I had just gotten the faculty job that I had arguably been working towards my entire life. And then all of a sudden while I'm there, I'm like, oh wait, this place is completely morally bankrupt and diametrically opposed to my worldview. And I can't stay here."
\end{quote}

\begin{quote}\itshape
"Realizing that actually what I was doing had been kind of harnessed and captured by this institution that I frankly despised and was morally repugnant to me—was really, really damaging, really hurtful. Deeply traumatizing, I think."
\end{quote}

Jon names this experience with clinical precision: \textbf{moral harm}.

\begin{quote}\itshape
"Moral harm is the specific kind of psychological trauma that occurs when you are forced to participate in an institution or a system that is morally repugnant to you in some way."
\end{quote}

\#\# The Discovery

But as faith in one institution collapsed, something else emerged.

During those months of "flailing around in the grief," Jon found himself spending more time with software like Blender—a free, open-source 3D animation tool built over decades by thousands of contributors worldwide and given away freely.

\begin{quote}\itshape
"In those moments of losing faith in the academic system... seeing something like Blender and starting to understand the business model and the community model... that galvanization happened together. As I was losing faith in the one, I started to gain respect for the other."
\end{quote}

The realization crystallized into a sentence he's repeated for years since:

\begin{quote}\itshape
"The open source community is what the scientific community pretends to be."
\end{quote}

Science \textit{claims} to be collaborative, open, building on shared knowledge for the common good. But it's actually siloed, competitive, captured by institutions that prioritize revenue over mission.

The open source community actually \textit{is} what science \textit{pretends} to be.

\bigskip\noindent\rule{\textwidth}{0.4pt}\bigskip

\# Part II: The Response

\#\# Not a Huff

The temptation was to storm out.

\begin{quote}\itshape
"There was a part of me that kind of wanted—when you want to stomp out the door and be in a huff or whatever—but that's not actually a strategic move because you're leaving a lot of resources on the table and you don't have anything to step into."
\end{quote}

Instead, Jon took a strategic approach. He had signed a contract. Tenure is a two-way street—and he had already decided this institution wasn't worthy of his long-term commitment. But he could use the time and resources to build something better.

\begin{quote}\itshape
"It was sort of like a prefigurative kind of approach of trying to—what if this institution is not what I think it should be, what is the thing that should be? What would allow me to continue to be the kind of scientist and researcher and mentor and teacher and all that I wanted to be without having to be attached to an institution that I couldn't morally stomach."
\end{quote}

\begin{quote}\itshape
"FreeMoCap and the FreeMoCap Foundation is kind of the answer to that question."
\end{quote}

\#\# The Pivot: Up vs. Down

The critical insight was about direction.

\begin{quote}\itshape
"A lot of the effort that I was putting in, almost a very big percentage of the effort I was putting into the world, was made in the service of the people who were above me on the hierarchy and trying to do the things that they wanted me to do so that I could outperform my peers and be allowed to stay in the warm, happy, shining tower."
\end{quote}

The shift:

\begin{quote}\itshape
"One of the big brain shifts that happened was deciding to just stop doing that and to just stop caring or considering what I was being asked to do by people above me in the hierarchy and start asking instead what I could do to support people who were at or below me in that same hierarchy."
\end{quote}

This wasn't abandoning ambition. It was redirecting it. Instead of publishing papers to please gatekeepers, he would build tools that serve people who need them.

\#\# What FreeMoCap Actually Is

FreeMoCap is a free, open-source markerless motion capture system. It uses standard webcams and computer vision to track human movement in 3D—the same technology that typically costs tens of thousands of dollars, made accessible to anyone with a laptop and some cheap cameras.

But that's just the surface.

\begin{quote}\itshape
"We want indie game designers and animators to use the same tool to add motion capture assets to their zero budget art project that I am using for my federally funded scientific research program."
\end{quote}

Same tool. Zero budget artist = federally funded researcher. Radical equality of access.

The deeper purpose:

\begin{quote}\itshape
"I don't want to tell you what you should care about. I'd rather just give you useful tools and let you do what you want to do with it."
\end{quote}

And the educational dimension:

\begin{quote}\itshape
"FreeMoCap wants to be designed kind of as a tool that teaches you things about how it works. And every individual person is going to be pulled in a different direction and drawn to different parts of the thing."
\end{quote}

\bigskip\noindent\rule{\textwidth}{0.4pt}\bigskip

\# Part III: The Larger Vision

\#\# The Critique

Jon has developed a sharp critique of what he calls the "techno capitalist corpo hegemony":

\begin{quote}\itshape
"Anytime there's a new technology that comes around, the techno capitalists or just the corporate capitalists basically try to rush to the new domain and start building moats around it so that they can sequester it and then use their access to that thing as a way to exploit the world at large. That is the true core of capitalism—to exploit disparities between yourself and others to extract resources from them."
\end{quote}

He sees this pattern repeating with AI:

\begin{quote}\itshape
"It has come from the techno capitalist corpo hegemony that is the source of so much of the problems and harm in our society."
\end{quote}

But he also sees AI differently than previous technologies:

\begin{quote}\itshape
"They're trying to build moats around the Moat Blaster 9000. Like, it is a tool that will teach you how it works and it will show you everything there is to know about itself. And they're trying to corral it and control it... and it's kind of just not working in the way that they would expect."
\end{quote}

\begin{quote}\itshape
"AI used properly is something that has the ability to be a transformatively liberative tool and one that has the potential to make every single human person that learns how to use it and learns how to wield it a more powerful and capable version of yourself."
\end{quote}

\#\# The Stakes

The fears Jon expressed in 2021 have only intensified:

\begin{quote}\itshape
"All the kinds of fears and dangers that I was talking about those years ago in those older recordings have just become far sharper and far more poignant."
\end{quote}

In 2021, he put it starkly:

\begin{quote}\itshape
"If you're sitting here in 2021 looking at the future and you're not feeling like you're staring down the barrel of a gun, you should probably maybe pay better attention."
\end{quote}

\#\# Seizing the Means

The focus has sharpened:

\begin{quote}\itshape
"A lot of my focus has really shifted towards the potential of technology as a liberatory tool. And specifically how I think if enough people in the world gained the kind of technical competency necessary to really engage in the open source community, then we can literally seize back the means of production."
\end{quote}

\begin{quote}\itshape
"Free people of having to live their entire lives attached and managed by corporations. And now at this point in history, governments, because we are entering into the AI era."
\end{quote}

FreeMoCap is explicitly positioned as a gateway:

\begin{quote}\itshape
"FreeMoCap specifically and then free open source software in general, as a way of teaching people the skill sets that they would need in order to be able to claw back our culture and our society and our technology from these corporate entities that have shown themselves not to be worthy of the power that they have captured and gathered."
\end{quote}

\bigskip\noindent\rule{\textwidth}{0.4pt}\bigskip

\# Part IV: The Transition

\#\# This Is the Last Year

As of 2026, Jon is in his final year at Northeastern. The prefigurative period is ending. The alternative must now sustain itself.

\begin{quote}\itshape
"Right now in the transition, we're moving from—this is my last year [at Northeastern]. So I'm transitioning into the world of trying to figure out how it is possible to support yourself, a lab scale group of people, while building free software that does not charge subscription fees... how do you survive in the world when you're not going to be attached to a big research institution or a big corporation."
\end{quote}

\#\# What Winning Looks Like

\textbf{Immediate/Survival:}
\begin{quote}\itshape
"That would be winning if we can get a stable position there."
\end{quote}

A small team—"me and two or three other people"—sustaining themselves while building free software. Not getting rich. Just surviving while doing the work.

\textbf{Medium-Term/Case Study:}
\begin{quote}\itshape
"The dream is that if I can figure out that one case study of how to support myself as a researcher and educator and mentor and software developer by building a tool like FreeMoCap, then that could potentially serve as a model that other people start to do."
\end{quote}

\begin{quote}\itshape
"FreeMoCap could serve as a conduit and umbrella and support system for people who are trying to achieve their own sense of independence and escape."
\end{quote}

\textbf{Long-Term/Catalytic:}
\begin{quote}\itshape
"Motion capture is not—in the long list of inequities and wrongs in the universe, inaccessibility of markerless motion capture is pretty low on the list, but it is on the list. And if we can sort of solve that tiny inequity, then it might serve as a model of how to solve bigger ones."
\end{quote}

\begin{quote}\itshape
"If the tool itself can serve as an inroad to teaching a wider set of people the skill sets necessary to work in the same direction, then it might be a catalytical project that could start as a seed and wind up extending and touching things beyond its particular scope and domain."
\end{quote}

\bigskip\noindent\rule{\textwidth}{0.4pt}\bigskip

\# Part V: The Soul

\#\# What Someone Should Feel

When asked what he wants someone to feel when they successfully capture their first skeleton:

\textbf{For users—especially young people or those who feel like technological outsiders:}

\begin{quote}\itshape
"I want them to kind of think, 'Oh, I can do this. I can make this happen. I can set it up. And even if I don't understand everything that I'm seeing, there's little glimmers of things that I can start to understand. I can feel the understandings beginning. And I can feel the opportunities to pull threads.'"
\end{quote}

\textbf{For people looking at the project from outside:}

\begin{quote}\itshape
"I want people to see FreeMoCap and the FreeMoCap Foundation as an example of—look. Look at what we can make. Look at what you can build if you just step away from the standard path. Look at what we have the capacity to do."
\end{quote}

\begin{quote}\itshape
"We can build tools that perform in the ways that they should, that just work and don't try to take anything from you and just give to you. It is possible to build and give gifts freely given to the world without extracting resources as a result."
\end{quote}

\#\# The Deepest Message

\begin{quote}\itshape
"I want to be able to show that that's a world you can live in, that that's a thing that you can do, and that we collectively as a group have the options of building things the way that they should be."
\end{quote}

\begin{quote}\itshape
"The world that is is not the world that has to be."
\end{quote}

\#\# Why It Keeps Him Going

\textasciitilde{}10,000 users worldwide. 115+ countries. Dots on a map.

\begin{quote}\itshape
"I see you. I see you there in the Faroe Islands. I see you doing your thing... Look at you in Oslo and Stockholm... Look at you in India. Who's this guy? Who are you? That's what I want to know."
\end{quote}

\begin{quote}\itshape
"Thanks for not unchecking the box, guys. I like to see the dots. It really, truly, truly keeps me going to just kind of like, every so often, log and be like, holy shit. You guys are using it."
\end{quote}

\#\# The Gatekeeping Instinct He Rejects

One anecdote captures the opposition:

\begin{quote}\itshape
"There is a reaction that I have had more than once when I talked about the software to people, which is, 'You should be careful not to make it too easy to use, otherwise the students won't understand what's going on in there.' And it's like insane to me to think that."
\end{quote}

\#\# Staying Sane

FreeMoCap isn't just a project. It's how Jon manages moral harm:

\begin{quote}\itshape
"FreeMoCap is absolutely part of that [managing moral harm]. It's sort of like, I have to have something that I'm working towards and building towards, and I have to feel like I'm somehow pulling in the right direction in order to feel sane. And so FreeMoCap is sort of like the embodiment of that effort."
\end{quote}

\begin{quote}\itshape
"Specifically utilizing the specific skill sets and understandings that I have to try to feel like I'm, you know, to get back to that feeling I had of being attached to a good vehicle and pulling the world in a good direction."
\end{quote}

\bigskip\noindent\rule{\textwidth}{0.4pt}\bigskip

\# Part VI: The Community

\#\# Named Contributors

From the 2026 State of the Skelly address, Jon names several key contributors:

- \textbf{Philip}: "Special shout out... for doing a lot of the plumbing necessary to keep the dang thing up and running and fixing the bugs and making it workable and also just basically dealing with all of my code slop"
- \textbf{AJC}: "Inimitable... work in the FreeMoCap community space... made a lot of the work in the Blender add-on"
- \textbf{Neon Xdeath}: Wrote the song "Meowmaline" used in early demo video
- \textbf{Taylor Davies}: Designed the logo
- \textbf{Aaron}: Working with RTM pose and Vit Pose
- \textbf{Brother Paul}: Helped with rough cut tutorial

The Discord server has grown to \textasciitilde{}2,500 members.

\#\# The De-Centering Goal

\begin{quote}\itshape
"A big part of the transitional phase... for FreeMoCap over the next five years is one that really continues to build the community around it and kind of de-centers me as a specific person."
\end{quote}

\bigskip\noindent\rule{\textwidth}{0.4pt}\bigskip

\# Part VII: Key Quotes Collection

\#\# On Academia

\begin{quote}\itshape
"None of this is okay. None of the general structure of academia education is morally defensible at almost any level."
\end{quote}

\begin{quote}\itshape
"Most of the research that gets done really shouldn't be done. It's not good research. It's just smart research. It's strategically the right move to get the publication, to get the grants, to get the job, to get the tenure."
\end{quote}

\begin{quote}\itshape
"You're supposed to produce more than two people who are also trying to get tenure at other schools, which is sort of the definition of a Ponzi scheme."
\end{quote}

\begin{quote}\itshape
"The open source community is what the scientific community pretends to be."
\end{quote}

\#\# On the Response

\begin{quote}\itshape
"Okay, well, fuck you. I'll build my own lab in my house with webcams."
\end{quote}

\begin{quote}\itshape
"Stop caring or considering what I was being asked to do by people above me in the hierarchy and start asking instead what I could do to support people who were at or below me in that same hierarchy."
\end{quote}

\begin{quote}\itshape
"It was sort of like a prefigurative kind of approach."
\end{quote}

\#\# On Technology and Capitalism

\begin{quote}\itshape
"The techno capitalist corpo hegemony that is the source of so much of the problems and harm in our society."
\end{quote}

\begin{quote}\itshape
"The true core of capitalism is to exploit disparities between yourself and others to extract resources from them."
\end{quote}

\begin{quote}\itshape
"They're trying to build moats around the Moat Blaster 9000."
\end{quote}

\#\# On Liberation

\begin{quote}\itshape
"We can literally seize back the means of production."
\end{quote}

\begin{quote}\itshape
"Claw back our culture and our society and our technology from these corporate entities that have shown themselves not to be worthy of the power that they have captured and gathered."
\end{quote}

\begin{quote}\itshape
"AI used properly is something that has the ability to be a transformatively liberative tool."
\end{quote}

\#\# On FreeMoCap

\begin{quote}\itshape
"We want indie game designers and animators to use the same tool to add motion capture assets to their zero budget art project that I am using for my federally funded scientific research program."
\end{quote}

\begin{quote}\itshape
"I don't want to tell you what you should care about. I'd rather just give you useful tools and let you do what you want to do with it."
\end{quote}

\begin{quote}\itshape
"FreeMoCap wants to be designed kind of as a tool that teaches you things about how it works."
\end{quote}

\#\# On What It's For

\begin{quote}\itshape
"I can do this. I can make this happen... I can feel the understandings beginning."
\end{quote}

\begin{quote}\itshape
"Look at what we can make. Look at what you can build if you just step away from the standard path."
\end{quote}

\begin{quote}\itshape
"It is possible to build and give gifts freely given to the world without extracting resources."
\end{quote}

\begin{quote}\itshape
"The world that is is not the world that has to be."
\end{quote}

\#\# On Staying Sane

\begin{quote}\itshape
"I have to have something that I'm working towards and building towards, and I have to feel like I'm somehow pulling in the right direction in order to feel sane."
\end{quote}

\begin{quote}\itshape
"I see you there in the Faroe Islands... Thanks for not unchecking the box, guys. I like to see the dots. It really, truly, truly keeps me going."
\end{quote}

\bigskip\noindent\rule{\textwidth}{0.4pt}\bigskip

\# Part VIII: Timeline

\begin{quote}\small\ttfamily
| Date | Event |\\
\hrule
| 2017 | Jon sees OpenPose demo, realizes markerless mocap is possible |\\
| Summer 2019 | Arrives at Northeastern as tenure-track professor |\\
| March 2020 | COVID hits, everyone sent home |\\
| Summer 2020 | "Lost all faith in the scientific, educational and academic process" |\\
| Summer 2020 | Begins building FreeMoCap in earnest ("fuck you, I'll build my own lab") |\\
| January 2021 | First public FreeMoCap post |\\
| July 2021 | "Meowmaline" demo video goes viral |\\
| 2021 | Epic Megagrant application video ("Purple Monkey") |\\
| September 2023 | Version 1.0 released |\\
| 2024 | \textasciitilde{}5,000 users, 115 countries |\\
| Fall 2024 | Teaching with explicit acknowledgment of moral harm |\\
| 2026 | \textasciitilde{}10,000 users; Jon's final year at Northeastern |\\
| 2026+ | Transition to independent operation of FreeMoCap Foundation |\\
\end{quote}

\bigskip\noindent\rule{\textwidth}{0.4pt}\bigskip

\# Part IX: Mapping Theory to Jon's Language

The theoretical frameworks from political organizing literature map closely to Jon's natural language:

\begin{quote}\small\ttfamily
| Theoretical Concept | Jon's Natural Language |\\
\hrule
| Gramsci's hegemony | "techno capitalist corpo hegemony" |\\
| Counter-hegemony / class consciousness | "seize back the means of production" |\\
| Freire's conscientização (critical consciousness) | "I can feel the understandings beginning" |\\
| Illich's convivial tools | "tools that just work and don't try to take anything from you" |\\
| Prefigurative politics | "prefigurative approach" (uses this exact term) |\\
| Moral injury / institutional betrayal | "moral harm" (uses this clinical term) |\\
| Banking model of education | critique of "outcompete your peers" model |\\
| War of position | strategic use of institutional resources while building alternative |\\
\end{quote}

\textbf{Recommendation for future documents}: Lead with Jon's language. Use theory to contextualize and deepen, not to frame or impose.

\bigskip\noindent\rule{\textwidth}{0.4pt}\bigskip

\# Part X: Open Questions

The following questions remain for future exploration:

\#\# About the Community
1. Who are Philip, AJC, and the other named contributors? What are their stories?
2. What's the difference between people who download and leave vs. people who become community?
3. What would "de-centering" Jon look like practically? What governance structures are planned?

\#\# About the Transition
4. What's the actual financial plan for post-Northeastern survival?
5. Who are the "two or three other people" in the lab transitioning with you?

\#\# About the Philosophy
6. How does Jon's background in philosophy of mind shape how he thinks about bodies, movement, technology?
7. Is there a simpler "elevator pitch" for FreeMoCap that doesn't require theory?

\#\# The Hard Questions
8. What is Jon most afraid of for the project?
9. What's the question that should have been asked but wasn't?

\bigskip\noindent\rule{\textwidth}{0.4pt}\bigskip

\# Closing

\begin{quote}\itshape
"Even though the path is forever, the work is there. The handles are there. The shovels are there. We know how to pull. We know what the right direction is. We know how to dig the holes that need to be dug. And even if the distance looks insurmountable, there is a way to start."
\end{quote}

\bigskip\noindent\rule{\textwidth}{0.4pt}\bigskip

\textit{This document was compiled from:}
- \textit{2022 VFX Futures podcast interview}
- \textit{2022 Biomechanics on our Minds podcast interview}
- \textit{2022 "This is FreeMoCap" video}
- \textit{2024 Fall semester introductory lecture}
- \textit{2026 State of the Skelly address}
- \textit{2026 Interview conducted for this compilation}

\textit{Prepared for the FreeMoCap Foundation, February 2026}
\normalsize


\chapter{The Soul of FreeMoCap --- Expanded Edition}
\label{src:SF2}

\begin{framed}
\small
\textbf{Source marker:} SF2 \\
\textbf{Date:} February 2026 \\
\textbf{Source:} Comprehensive narrative from all sources
\end{framed}

\small
\# The Soul of FreeMoCap

\#\# A Document of Origin, Mission, and Vision

\textbf{Compiled from interviews, lectures, and public statements by Jonathan Samir Matthis, Founder of the FreeMoCap Foundation}

\textit{February 2026}

\bigskip\noindent\rule{\textwidth}{0.4pt}\bigskip

\#\# Epigraph

\begin{quote}\itshape
"The world that is is not the world that has to be. And even though the path is forever, the work is there. The handles are there. The shovels are there. We know how to pull. We know what the right direction is. We know how to dig the holes that need to be dug. And even if the distance looks insurmountable, there is a way to start."
\end{quote}

\bigskip\noindent\rule{\textwidth}{0.4pt}\bigskip

\# Part I: The Origin Story

\#\# Chapter 1: The Path to the Tower

Jonathan Samir Matthis's path to FreeMoCap was not direct. It began in philosophy.

As an undergraduate, Jon studied philosophy of mind and philosophy of science—the deep questions about what consciousness is, what knowledge is, how we come to understand the world. He applied to philosophy graduate programs. He was rejected from all of them.

"By grace of God," he now says.

That rejection redirected him. He worked at an autism research facility and discovered that "data is cool"—that there was something deeply satisfying about the empirical investigation of reality. He applied to cognitive science programs, planning to study language. But the program he got into didn't have a language researcher. Looking through the faculty list, he found someone named Brett Fajin who studied "visual control of locomotion."

It sounded interesting enough. He said yes.

That accidental choice shaped the next two decades. Jon found himself studying how humans use vision to control movement—how we look at the ground to place our feet, how we navigate complex terrain, how the brain integrates visual information with motor commands. He got his PhD. He did a postdoc at University of Texas at Austin, studying people walking outdoors over rocks while wearing eye trackers and motion capture suits.

He built skills. He learned to hack together systems, to do basic computer vision, to work at the intersection of neuroscience, biomechanics, psychology, and engineering. His research got attention. He was described as having had a "meteoric rise."

By 2019, after nearly twenty years of education and training, he had achieved what he'd been working toward his entire adult life: a tenure-track faculty position at Northeastern University in Boston.

He arrived in summer 2019. The lab was built. The equipment was ordered. The research program was ready to launch.

"I had about a year of like, 'Boy, I sure did make it. And this is probably fine,'" he recalls.

Then came 2020.

\#\# Chapter 2: The Breaking

March 2020: COVID-19 hits. Everyone is sent home. The lab sits empty.

The pandemic was disorienting for everyone. But for Jon, it became something more. It wasn't the virus that broke something. It was what the pandemic revealed about the institution he had spent his life trying to join.

\#\#\# What He Saw

In faculty meetings, Jon watched administrators make calculations that turned his stomach.

\begin{quote}\itshape
"A dean or some similar type of person said something to the effect that we should encourage our students to get master's degrees because master's students are the best income ratio—they pay full tuition but they don't require any resources. That kind of stuff is super gross."
\end{quote}

The logic was naked: students as revenue sources, education as extraction. And it wasn't hidden. It was said out loud, in meetings, as if it were simply good business sense.

Then came the pressure to return to in-person teaching—before vaccines existed, "while people are still actively dying all over the place." The reason was obvious: the university needed the money. The physical campus was the product they were selling. If students could learn remotely, what were they paying for?

Safety was secondary. Revenue was primary.

Jon had been on the student side of academia his entire adult life—undergraduate, graduate student, postdoc. Those positions are temporary. You pass through. You don't see the other side of the conversations.

\begin{quote}\itshape
"When you become a professor, all of a sudden you're on the other side of the fence. And so you start to see different conversations about it."
\end{quote}

What he saw was an institution that talked about education and knowledge and human flourishing, but operated according to entirely different principles.

\#\#\# The Blackmail

Then came the moment that crystallized everything.

Jon received a letter informing him that he was scheduled to teach in person. But there was a rider on the letter—an apparent concession to faculty concerns. It said that if he didn't feel comfortable teaching in person, he could opt out and teach remotely instead.

He opted out. Obviously. People were dying. Vaccines didn't exist yet. He would teach from home.

Then he received a second letter from his department chair. The policy had changed. If he opted out of in-person teaching, he couldn't come to campus for any reason—including to access his laboratory.

Think about what this means. Jon was a first-year research professor. His career depended on producing research. His research required his lab. And the university was telling him: teach in person during a pandemic, or lose access to everything you need to do your job.

\begin{quote}\itshape
"As a first year research professor, that's basically saying, if you don't come in and teach in person against your will, we're going to tank your career."
\end{quote}

It was blackmail. Politely worded, institutionally sanctioned blackmail.

His response was immediate and visceral:

\begin{quote}\itshape
"Okay, well, fuck you. I'll build my own lab in my house with webcams."
\end{quote}

\#\#\# The \$3 Million Webcam

But there was another moment—quieter, more devastating, more lasting.

Eventually Jon did teach remotely. He sat in his apartment, looking at his webcam, lecturing to a grid of rectangles on a screen. And he found himself counting.

How many students in this Zoom call? What's the tuition? What's the room and board?

He did the math.

\begin{quote}\itshape
"I was speaking to a webcam that was worth \$3 million a year in student debt."
\end{quote}

Three million dollars. For a webcam lecture from a guy in his apartment.

But that wasn't the worst part. The worst part was understanding \textit{why} those students had taken on that debt.

\begin{quote}\itshape
"The reason why they were going into that amount of debt was because of the promise of being able to learn from somebody like me... the fact that I was there doing the thing that I did was part of the recruitment that they used to bring these kids in and to convince them that it's worth it, which it fundamentally isn't and cannot be."
\end{quote}

Jon was the product. His presence, his credentials, his research—these were the things the university used to justify charging \$50,000 a year. Students went into life-altering debt because they believed that proximity to people like him was worth it.

And now he was looking at them through a webcam, knowing that it wasn't worth it, knowing that he was complicit in the extraction, knowing that his participation in the system was part of what made the system work.

\begin{quote}\itshape
"More than anything else, that moment was the time when I was realizing finally... I truly cannot stay here. I cannot continue to be in this position."
\end{quote}

\#\# Chapter 3: The Grief

What followed was not just anger. Anger would have been easier. Anger has energy, direction, a target.

What Jon experienced was grief.

\begin{quote}\itshape
"I was 35 years old. I had just gotten the faculty job that I had arguably been working towards my entire life. And then all of a sudden while I'm there, I'm like, oh wait, this place is completely morally bankrupt and diametrically opposed to my worldview. And I can't stay here."
\end{quote}

Think about what that means. Nearly twenty years of work. A bachelor's degree, a PhD, a postdoc, the brutal competition of the academic job market. The sacrifices, the moves, the delayed life decisions, the relationships strained by the demands of the career. All of it aimed at one goal: a tenure-track position at a research university.

He got it. And then he realized he couldn't keep it.

\begin{quote}\itshape
"Realizing that actually what I was doing had been kind of harnessed and captured by this institution that I frankly despised and was morally repugnant to me—was really, really damaging, really hurtful. Deeply traumatizing, I think."
\end{quote}

This wasn't just disappointment. This was the collapse of a worldview, an identity, a life plan. Everything Jon thought he was working toward turned out to be something else entirely.

Jon has a clinical term for this experience: \textbf{moral harm}.

\begin{quote}\itshape
"Moral harm is the specific kind of psychological trauma that occurs when you are forced to participate in an institution or a system that is morally repugnant to you in some way."
\end{quote}

He uses this language deliberately. It's not just discomfort. It's not just disagreement. It's a form of psychological damage that occurs when you're trapped inside something you find morally intolerable—and when your participation makes you complicit.

This is the experience of the soldier who realizes the war is unjust. The employee who discovers the company is poisoning people. The true believer who learns the church is corrupt.

And it was the experience of a professor who realized that the university was not what it claimed to be—and that he had spent his life trying to join it.

\#\# Chapter 4: The Discovery

But grief, if you survive it, can open space for something new.

During those months of "flailing around in the grief," Jon found himself spending more time with a piece of software called Blender.

Blender is a 3D animation and modeling program. It's the kind of software used to make animated films, video games, visual effects. Programs like it—Maya, 3ds Max, Cinema 4D—typically cost thousands of dollars per year in licensing fees.

Blender costs nothing. It's completely free. And it's not free because it's inferior. Blender is used in professional film production. It has won awards. It competes with software that costs orders of magnitude more.

How is this possible?

Blender is free and open source software. It's built by a community of thousands of contributors around the world. The source code is public. Anyone can see how it works, modify it, improve it. The Blender Foundation coordinates development, but the software belongs to everyone.

Jon had known about Blender before. But in those months of crisis, he started to really understand what it represented.

\begin{quote}\itshape
"In those moments of losing faith in the academic system... seeing something like Blender and starting to understand the business model and the community model... that galvanization happened together. As I was losing faith in the one, I started to gain respect for the other."
\end{quote}

He started to see a pattern. Here was a community that actually worked the way science claimed to work. Knowledge shared freely. Collaboration across boundaries. Building on each other's contributions. No gatekeepers. No paywalls. No artificial scarcity.

\begin{quote}\itshape
"In those moments of losing faith in one institution, you kind of go back to try to figure out what your own morality was and what your own values are... recognizing [Blender] as another organism, a different institution and a different communal structure that felt like it was more aligned."
\end{quote}

The realization crystallized into a sentence he's repeated for years since:

\begin{quote}\itshape
"The open source community is what the scientific community pretends to be."
\end{quote}

Science \textit{claims} to be about collaboration, openness, shared knowledge, building on previous work for the common good. Scientists \textit{talk} about standing on the shoulders of giants.

But in practice? Publications locked behind paywalls. Data hoarded for competitive advantage. Research programs siloed by funding structures and institutional rivalries. Careers built on outcompeting peers rather than collaborating with them.

\begin{quote}\itshape
"We pretend as scientists that we are doing that, that we're working in this big sort of global collaborative endeavor, but it's not. Practically speaking, you're just not really able to do it."
\end{quote}

The open source community, by contrast, actually operates according to the principles that science only claims. The code is public. The development is transparent. The work builds on itself. Anyone can contribute. Anyone can benefit.

This wasn't just an alternative career path. It was an alternative model for how knowledge work could be organized.

\bigskip\noindent\rule{\textwidth}{0.4pt}\bigskip

\# Part II: The Response

\#\# Chapter 5: Lift Where You Stand

Around the same time as the institutional betrayals, another catalyst emerged.

May 2020: George Floyd is murdered by Minneapolis police. Protests erupt across the country and around the world. A moment of national reckoning about race, power, and justice.

Jon was watching. And he was asking himself questions.

\begin{quote}\itshape
"Also happened to coincide with the start of the George Floyd protests, at which point I think we all kind of started to ask these kind of questions about whether or not we were really adequately using our positions of power to do the most good for the most people."
\end{quote}

He had just realized that the institution he'd devoted his life to was morally bankrupt. Now he was watching a society-wide conversation about power, complicity, and responsibility. The question wasn't just "what's wrong with this system?" It was "what am I going to do about it?"

In that moment, Jon encountered a concept that became central to his approach:

\begin{quote}\itshape
"I encountered a very powerful concept of \textbf{lift where you stand}. The idea is that when you encounter a problem or inequity that causes you some moral harm, you don't just turn around and drop everything that you've done in the hopes of going off to join Greenpeace or something like that."
\end{quote}

This is a crucial insight. When people experience moral awakening—when they realize that something about their world is deeply wrong—there's a temptation toward dramatic gesture. Quit everything. Burn it down. Start over in a completely different domain.

But that impulse, however understandable, often leads nowhere. You leave behind all the skills, knowledge, and connections you've built. You arrive in the new domain as a novice. And the original problem remains unsolved.

Jon had spent twenty years learning how to study human movement, how to build experimental systems, how to work at the intersection of neuroscience and technology. That expertise was real. It had value. Throwing it away wouldn't help anyone.

\begin{quote}\itshape
"I don't know shit about the climate. I know about how to make laser skeletons."
\end{quote}

"Laser skeletons"—his shorthand for the kind of motion capture research he does. Tracking human bodies in 3D space. Understanding how people move.

It's a specific skill set. A narrow expertise. Not obviously connected to the great injustices of the world.

But the concept of "lift where you stand" offers a different approach:

\begin{quote}\itshape
"So the concept of lift where you stand is: when you're having a problem, what you do is you look around where you're standing, you find something that looks like a handle, you grab it and you pull as hard as you can in what feels like the right direction."
\end{quote}

Don't abandon your position. Don't pretend you're someone else. Look at where you actually are, what you actually know, what you can actually do. Find a point of leverage. And pull.

This crystallized into a key realization:

\begin{quote}\itshape
"On the list of global inequities, maybe motion capture is not particularly high, but what IS high is inaccessibility of science and technology."
\end{quote}

Motion capture systems typically cost tens of thousands of dollars. The cheapest professional options run around \$10,000. High-end systems used in film production cost much more. This means that motion capture—the ability to precisely record and analyze human movement—is available only to institutions with substantial resources.

Is that the biggest injustice in the world? Obviously not. But it's connected to something bigger: the way scientific tools and technical knowledge are locked behind barriers of money and institutional access.

And Jon knew how to build motion capture systems. That was his handle. That was where he could pull.

\#\# Chapter 6: The Deliberate Flip to Cheap

Before the crisis, Jon had been working on markerless motion capture as a side project. The idea: use computer vision and machine learning to track human bodies without the reflective markers that traditional systems require. Just cameras and software.

The original plan was to develop the system using GoPro cameras—moderately expensive, high quality—and then scale up to professional research-grade cameras. The standard trajectory: get it working with decent equipment, then make it better.

But around the time of the George Floyd protests, Jon made a strategic pivot that would define the project:

\begin{quote}\itshape
"What if we just pushed in the other direction and what if instead of pushing it towards research grade stuff, I push it towards the most garbage cameras you can possibly get on the assumption that if it works for the cheap cameras, it will also work for the good cameras, but not necessarily the other way around."
\end{quote}

This is a crucial decision. It's not just a technical choice. It's a statement about who the technology is for.

If you build for expensive cameras first, you create something that works for people who can afford expensive cameras. Maybe you eventually make a cheaper version. Maybe you don't.

If you build for garbage cameras first, you create something that works for everyone. The physics doesn't care how much your webcam cost. If the algorithms can handle the noise and distortion of a \$10 camera, they'll work even better with a \$500 camera.

The result: a motion capture system that costs \$30. Three garbage webcams. Ten dollars each.

\begin{quote}\itshape
"That is a \$30 motion capture system recorded in 11 minutes or less."
\end{quote}

This wasn't making do with less. This was deliberately choosing the constraints that would force the technology to be maximally accessible.

\#\# Chapter 7: Not a Huff

Given everything Jon had realized about the university, the natural impulse might be to quit. Storm out. Make a statement.

\begin{quote}\itshape
"There was a part of me that kind of wanted—when you want to stomp out the door and be in a huff or whatever—but that's not actually a strategic move because you're leaving a lot of resources on the table and you don't have anything to step into."
\end{quote}

This is the discipline of long-term thinking. Moral clarity doesn't require immediate action. In fact, immediate dramatic action often undermines long-term goals.

Jon had signed a contract to be at Northeastern for several years. He had access to resources, salary, equipment, students. He had time—time that was paid for, time that he could direct toward whatever he chose to work on.

Tenure is a two-way street. The university evaluates you, but you also evaluate the university. Jon had already decided this institution wasn't worthy of his long-term commitment. He wouldn't pursue tenure. He wouldn't play the game of publishing papers to please committees.

But he also wouldn't leave prematurely. He would use what the position offered to build something better.

\begin{quote}\itshape
"It was sort of like a prefigurative kind of approach of trying to—what if this institution is not what I think it should be, what is the thing that should be? What would allow me to continue to be the kind of scientist and researcher and mentor and teacher and all that I wanted to be without having to be attached to an institution that I couldn't morally stomach."
\end{quote}

The term "prefigurative" comes from political theory. It means building the future you want within the shell of the present. Not waiting for revolution. Not demanding that the world change first. Creating, now, the structures and practices and relationships that embody your values—even while still embedded in systems that violate them.

Jon would remain a professor. He would teach classes. He would draw a salary. But his real work would be building the alternative.

\begin{quote}\itshape
"FreeMoCap and the FreeMoCap Foundation is kind of the answer to that question."
\end{quote}

\#\# Chapter 8: The Pivot—Up vs. Down

The critical insight was about direction.

Academic careers are structured as ladders. You climb. Each rung is achieved by pleasing the people above you. Undergraduates please professors to get recommendations. Graduate students please advisors to get degrees. Postdocs please PIs to get jobs. Junior faculty please senior faculty to get tenure.

At every stage, the incentive is the same: serve up. Make the people above you happy. Outperform the people beside you. Climb.

\begin{quote}\itshape
"A lot of the effort that I was putting in, almost a very big percentage of the effort I was putting into the world, was made in the service of the people who were above me on the hierarchy and trying to do the things that they wanted me to do so that I could outperform my peers and be allowed to stay in the warm, happy, shining tower."
\end{quote}

Jon had spent twenty years in this mode. Working to please advisors, committees, reviewers, editors, deans. Trying to outcompete peers for positions, grants, publications. All directed upward in the hierarchy.

And where had it led? To a position in an institution he despised, doing work that served to extract money from students who couldn't afford it.

The shift was simple to state and difficult to execute:

\begin{quote}\itshape
"One of the big brain shifts that happened was deciding to just stop doing that and to just stop caring or considering what I was being asked to do by people above me in the hierarchy and start asking instead what I could do to support people who were at or below me in that same hierarchy."
\end{quote}

Stop serving up. Start serving down.

This wasn't abandoning ambition. It was redirecting it. Instead of asking "what do I need to do to get tenure?" the question became "what could I build that would actually help people?"

The people above Jon in the hierarchy wanted publications, grants, prestige, metrics. The people below—students, aspiring researchers, independent creators without institutional resources—wanted tools that worked and knowledge they could actually use.

Those wants pointed in different directions. Jon chose a direction.

\#\# Chapter 9: What Academia Actually Is

To understand the choice Jon made, you have to understand his analysis of what academia has become.

The \textit{myth} of academia is familiar. Universities as communities of scholars, seeking truth, advancing knowledge, educating the next generation. Professors as mentors, researchers as explorers, students as apprentices in the life of the mind.

The \textit{reality}, as Jon experienced it, is different.

\begin{quote}\itshape
"Most of the research that gets done really shouldn't be done. It's not good research. It's just smart research. It's strategically the right move to get the publication, to get the grants, to get the job, to get the tenure."
\end{quote}

Note the distinction: not \textit{good} research, but \textit{smart} research. Research that advances careers. Research that can be published. Research that grants will fund. Whether it actually advances human knowledge or benefits anyone is secondary.

\begin{quote}\itshape
"The nature of the institutions that sort of house it and the realities of the hyper competitive structures of society that we've sort of grown for ourselves are that you're not really able to do it in a way that really benefits anybody outside of the club."
\end{quote}

The structure produces the behavior. When resources are scarce, competition is fierce, and careers depend on metrics, people optimize for metrics. They do research that generates publications, whether or not that research matters. They teach in ways that generate good evaluations, whether or not students learn. They build empires of grants and graduate students, whether or not those empires produce anything of value.

And the whole structure is predicated on a mathematical impossibility:

\begin{quote}\itshape
"You're supposed to produce like, you know, more than two people who are also trying to get tenure at other schools, which is sort of the definition of a Ponzi scheme."
\end{quote}

Each professor trains multiple PhD students who are trying to become professors. But the number of professor positions isn't growing at the same rate. The math doesn't work. Most of those PhD students will not become professors, no matter how good they are, no matter how hard they work.

The system requires this failure. It runs on the labor of people who will be discarded. And everyone inside the system knows this, even as they continue to recruit new entrants.

Even "outreach"—the supposedly altruistic effort to share science with the public—is corrupted by the logic of the system:

\begin{quote}\itshape
"When you see scientists doing outreach, it's often this kind of just like side thought, like, I'm going to go to a high school, show them how cool it is to be super highly educated and then just leave and then hope that they're inspired enough to follow in your path, out compete their peers, and then maybe they someday can get into that shining tower at the top of the hill."
\end{quote}

The message isn't "here's knowledge that will help you." It's "isn't this cool? Maybe you can be like me someday—if you're lucky enough to win the competition."

This is the system Jon had spent twenty years climbing into. And this is the system he decided he couldn't serve.

\bigskip\noindent\rule{\textwidth}{0.4pt}\bigskip

\# Part III: The Philosophy

\#\# Chapter 10: What FreeMoCap Actually Is

FreeMoCap is a free, open-source markerless motion capture system.

In technical terms: it uses computer vision and machine learning to track human bodies in 3D space without requiring special markers or suits. You point some webcams at a person, record them moving, and the software produces a 3D skeleton—a digital representation of their body position over time.

This technology has existed in expensive, proprietary forms for years. Professional motion capture systems cost tens of thousands of dollars. They're used in film production (for CGI characters), in video game development (for character animation), in biomechanics research (for studying how people move), in clinical settings (for rehabilitation and assessment).

FreeMoCap does similar things with \$10 webcams and free software.

But that technical description misses the point. FreeMoCap isn't just a motion capture system. It's an embodiment of a philosophy.

\begin{quote}\itshape
"We want indie game designers and animators to use the same tool to add motion capture assets to their zero budget art project that I am using for my federally funded scientific research program."
\end{quote}

Same tool. The indie animator in their bedroom gets the same technology as the professor with grant funding. The high school student gets the same technology as the professional studio. The researcher in Nigeria gets the same technology as the researcher at MIT.

This is not charity. It's not "providing access to the less fortunate." It's the assertion that tools should just work, for everyone, without gatekeepers:

\begin{quote}\itshape
"I don't want to tell you what you should care about. I'd rather just give you useful tools and let you do what you want to do with it."
\end{quote}

Jon doesn't want to direct people's curiosity. He wants to remove the barriers that prevent them from following their own curiosity wherever it leads.

And there's an educational dimension—but not education in the traditional sense:

\begin{quote}\itshape
"FreeMoCap wants to be designed kind of as a tool that teaches you things about how it works. And every individual person is going to be pulled in a different direction and drawn to different parts of the thing."
\end{quote}

The tool doesn't lecture. It doesn't give lessons and quizzes. But its design makes its own workings visible. Use it long enough, and you start to understand computer vision, 3D geometry, coordinate systems, calibration. Not because anyone is teaching you, but because the understanding emerges from engagement.

\#\# Chapter 11: The Gatekeeping Instinct

One anecdote captures the opposition Jon faces—not from obvious enemies, but from well-meaning people within his own world:

\begin{quote}\itshape
"There is a reaction that I have had more than once when I talked about the software to people, which is, 'You should be careful not to make it too easy to use, otherwise the students won't understand what's going on in there.' And it's like insane to me to think that."
\end{quote}

Read that again. People are warning Jon that his software might be \textit{too accessible}. That making it easy to use might somehow harm students by preventing them from understanding it.

This is the gatekeeping instinct dressed up as pedagogical concern. The implicit assumption: understanding is scarce, precious, earned through suffering. If we make things too easy, people might get access without "deserving" it.

Jon rejects this completely. Making a tool easy to use doesn't prevent understanding—it enables it. When you're not wrestling with installation problems and obscure error messages, you can actually explore what the tool does. When the barrier to entry is low, more people enter, and some of them go deep.

The gatekeeping instinct is pervasive in academia. It manifests as:
- Journals locked behind paywalls
- Software requiring expensive licenses
- Knowledge expressed in unnecessarily obscure jargon
- Curricula designed as obstacle courses
- Credentials required for access to further credentials

Every barrier serves a dual purpose: it restricts access, and it makes those who pass feel special. The scarcity is manufactured, but the prestige is real.

FreeMoCap is designed to route around this. No credentials required. No payment required. No institutional affiliation required. Just download it and use it.

\#\# Chapter 12: The Techno-Capitalist Critique

Jon's critique extends beyond academia to the broader technology landscape.

\begin{quote}\itshape
"It has come from the techno capitalist corpo hegemony that is the source of so much of the problems and harm in our society."
\end{quote}

"Techno capitalist corpo hegemony" is a phrase worth unpacking. It points to the interlocking system of technology companies, venture capital, platform monopolies, and the ideology that serves them—the assumption that technological progress is identical to corporate profit, that disruption serves human welfare, that the market allocates technological benefits fairly.

\begin{quote}\itshape
"Anytime there's a new technology that comes around, the techno capitalists or just the corporate capitalists basically try to rush to the new domain and start building moats around it so that they can sequester it and then use their access to that thing as a way to exploit the world at large."
\end{quote}

The pattern is consistent. A new technology emerges. Companies race to capture it. They build moats—patents, proprietary standards, network effects, regulatory capture. Once the moat is built, they extract rent. Access to the technology becomes access to a toll booth.

This has happened with operating systems, with social networks, with cloud computing, with search. And it's happening now with artificial intelligence.

\begin{quote}\itshape
"That is the true core of capitalism—to exploit disparities between yourself and others to extract resources from them."
\end{quote}

This is a stark framing. Not "capitalism creates value through voluntary exchange." Not "markets coordinate information efficiently." The core function, in Jon's view, is exploitation of disparity. If you have something others need and can't get elsewhere, you can extract from them. Capitalism is the systematic organization of this extraction.

\#\# Chapter 13: AI as Potential Liberation

But Jon's view of technology is not simply critical. He sees potential for liberation—specifically in artificial intelligence.

\begin{quote}\itshape
"AI used properly is something that has the ability to be a transformatively liberative tool and one that has the potential to make every single human person that learns how to use it and learns how to wield it a more powerful and capable version of yourself."
\end{quote}

This might seem to contradict the critique of techno-capitalism. After all, the current AI boom is driven by exactly the corporations Jon criticizes. OpenAI, Google, Anthropic—these are well-funded entities with commercial interests.

But Jon draws a distinction between the technology and its current institutional capture:

\begin{quote}\itshape
"They're trying to build moats around the Moat Blaster 9000. Like, it is a tool that will teach you how it works and it will show you everything there is to know about itself. And they're trying to corral it and control it... and it's kind of just not working in the way that they would expect."
\end{quote}

The metaphor is vivid: a "Moat Blaster 9000" that destroys moats. AI is uniquely resistant to capture because of what it is. A language model can explain how language models work. A coding assistant can help you build your own coding assistant. The technology undermines the barriers that would restrict it.

This doesn't mean AI can't be used for harm, or that corporate AI isn't dangerous. It means that AI has unusual properties that make it potentially democratizing in ways that previous technologies weren't.

\#\# Chapter 14: Seizing the Means

The vision comes into focus:

\begin{quote}\itshape
"A lot of my focus has really shifted towards the potential of technology as a liberatory tool. And specifically how I think if enough people in the world gained the kind of technical competency necessary to really engage in the open source community, then we can literally seize back the means of production."
\end{quote}

"Seize the means of production" is of course a Marxist phrase. But Jon isn't talking about factories or farms. He's talking about the means of \textit{technological} production—the ability to create software, to build systems, to shape the digital infrastructure that increasingly shapes our lives.

\begin{quote}\itshape
"Free people of having to live their entire lives attached and managed by corporations. And now at this point in history, governments, because we are entering into the AI era."
\end{quote}

The stakes have risen. In 2020, the concern was corporations using technology to extract and control. Now, in 2026, governments are entering the game. AI systems will be used for surveillance, for administration, for enforcement. The question of who controls these systems becomes a question of who controls society.

\begin{quote}\itshape
"FreeMoCap specifically and then free open source software in general, as a way of teaching people the skill sets that they would need in order to be able to claw back our culture and our society and our technology from these corporate entities that have shown themselves not to be worthy of the power that they have captured and gathered."
\end{quote}

FreeMoCap is not just a motion capture system. It's a gateway—a point of entry into technical literacy, into the open source community, into the skills and culture of building technology rather than merely consuming it.

\bigskip\noindent\rule{\textwidth}{0.4pt}\bigskip

\# Part IV: The Transition

\#\# Chapter 15: The Last Year

As of 2026, Jon is in his final year at Northeastern University.

The prefigurative period is ending. For years, he used the resources of the institution—salary, time, equipment, legitimacy—to build the alternative. Now the alternative must stand on its own.

\begin{quote}\itshape
"Right now in the transition, we're moving from—this is my last year [at Northeastern]. So I'm transitioning into the world of trying to figure out how it is possible to support yourself, a lab scale group of people, while building free software that does not charge subscription fees... how do you survive in the world when you're not going to be attached to a big research institution or a big corporation."
\end{quote}

This is not a rhetorical question. It's an existential one. Jon is trying to solve the fundamental problem of open source sustainability: how do people who build things that are given away freely support themselves?

The problem is structural. In a market economy, you exchange goods or services for money. But free software, by definition, is given away. It's not scarce. It can't command a market price in the traditional sense.

There are models: donations, sponsorships, grants, consulting, paid support. But none of them are easy. Many open source projects have collapsed when their maintainers burned out or couldn't pay their bills. The history of free software is littered with abandoned projects and exhausted developers.

Jon is trying to find a path—not just for himself, but as a proof of concept.

\#\# Chapter 16: What Winning Looks Like

Jon describes winning at three levels:

\textbf{Immediate / Survival:}
\begin{quote}\itshape
"That would be winning if we can get a stable position there."
\end{quote}

A small team—"me and two or three other people"—sustaining themselves while building free software. Not getting rich. Not achieving fame. Just surviving while doing the work.

This is the minimum viable victory: demonstrate that it's possible to live this way.

\textbf{Medium-Term / Case Study:}
\begin{quote}\itshape
"The dream is that if I can figure out that one case study of how to support myself as a researcher and educator and mentor and software developer by building a tool like FreeMoCap, then that could potentially serve as a model that other people start to do."
\end{quote}

If Jon succeeds, his success becomes a template. Other researchers trapped in institutions they despise can see that there's a way out. Other developers building tools that should be free can see that it's possible to sustain that work.

\begin{quote}\itshape
"FreeMoCap could serve as a conduit and umbrella and support system for people who are trying to achieve their own sense of independence and escape."
\end{quote}

The FreeMoCap Foundation could become not just a project, but an infrastructure—a way for others to do similar work without having to solve all the structural problems from scratch.

\textbf{Long-Term / Catalytic:}
\begin{quote}\itshape
"Motion capture is not—in the long list of inequities and wrongs in the universe, inaccessibility of markerless motion capture is pretty low on the list, but it is on the list. And if we can sort of solve that tiny inequity, then it might serve as a model of how to solve bigger ones."
\end{quote}

Jon is clear-eyed about the relative importance of motion capture. It's not hunger or war or climate change. But that's the point of "lift where you stand"—you work with what you have, where you are.

\begin{quote}\itshape
"If the tool itself can serve as an inroad to teaching a wider set of people the skill sets necessary to work in the same direction, then it might be a catalytical project that could start as a seed and wind up extending and touching things beyond its particular scope and domain."
\end{quote}

The vision: FreeMoCap as one instance of a pattern. A proof that another way is possible. A gateway drug to technical literacy and open source participation. A seed that grows into something larger.

\bigskip\noindent\rule{\textwidth}{0.4pt}\bigskip

\# Part V: The Soul

\#\# Chapter 17: What Someone Should Feel

When asked what he wants someone to feel when they successfully capture their first skeleton, Jon gives an answer that reveals the emotional core of the project.

\textbf{For users—especially young people or those who feel like technological outsiders:}

\begin{quote}\itshape
"I want them to kind of think, 'Oh, I can do this. I can make this happen. I can set it up. And even if I don't understand everything that I'm seeing, there's little glimmers of things that I can start to understand. I can feel the understandings beginning. And I can feel the opportunities to pull threads.'"
\end{quote}

"I can do this." That's the feeling. Not "this is being done to me." Not "this is happening through some incomprehensible magic." But: I made this happen. I set this up. I did it.

And then: "I can feel the understandings beginning." The tool is designed to be not just usable but comprehensible. There are threads to pull. There are doors to open. The first success is not an endpoint but an invitation.

\textbf{For people looking at the project from outside:}

\begin{quote}\itshape
"I want people to see FreeMoCap and the FreeMoCap Foundation as an example of—look. Look at what we can make. Look at what you can build if you just step away from the standard path. Look at what we have the capacity to do."
\end{quote}

This is for the observers, the skeptics, the people who assume that serious technology requires serious resources. Look what's possible when you refuse those assumptions.

\begin{quote}\itshape
"We can build tools that perform in the ways that they should, that just work and don't try to take anything from you and just give to you. It is possible to build and give gifts freely given to the world without extracting resources as a result."
\end{quote}

Tools that "just work and don't try to take anything from you." This is a vision of technology without extraction. Without dark patterns. Without surveillance. Without monetization of attention. Just: here is a thing that does what it claims to do. Use it. It's yours.

\textbf{The deepest message:}

\begin{quote}\itshape
"I want to be able to show that that's a world you can live in, that that's a thing that you can do, and that we collectively as a group have the options of building things the way that they should be."
\end{quote}

It's possible. That's the message. This isn't utopian fantasy. It's an existence proof. The world can be different. We can build differently. And this is what it looks like.

\begin{quote}\itshape
"The world that is is not the world that has to be."
\end{quote}

\#\# Chapter 18: The Dots on the Map

There's a map. A world map with dots. Each dot is someone who has used FreeMoCap.

As of 2026, there are approximately 10,000 users across more than 115 countries.

Jon talks about this map with undisguised emotion:

\begin{quote}\itshape
"I see you. I see you there in the Faroe Islands. I see you doing your thing... Look at you in Oslo and Stockholm... Look at you in India. Who's this guy? Who are you? That's what I want to know."
\end{quote}

\begin{quote}\itshape
"Thanks for not unchecking the box, guys. I like to see the dots. It really, truly, truly keeps me going to just kind of like, every so often, log and be like, holy shit. You guys are using it."
\end{quote}

This is not vanity metrics. This is not growth hacking. This is: someone in the Faroe Islands is building skeletons with this software I made. Someone in Nigeria. Someone in a high school in Ohio. Someone in their garage.

The dots are evidence that it works. Not just technically—that the software runs—but socially. That the thing Jon is trying to do is actually happening. People are using the tools. People are learning. The gift is being received.

\#\# Chapter 19: Staying Sane

Finally, there's the personal dimension.

\begin{quote}\itshape
"FreeMoCap is absolutely part of that [managing moral harm]. It's sort of like, I have to have something that I'm working towards and building towards, and I have to feel like I'm somehow pulling in the right direction in order to feel sane. And so FreeMoCap is sort of like the embodiment of that effort."
\end{quote}

Remember: Jon experienced moral harm. The trauma of realizing that the institution he'd devoted his life to was morally bankrupt. That his work had been harnessed in service of something he despised.

FreeMoCap is not just a project. It's how he stays functional. It's how he transforms grief into action.

\begin{quote}\itshape
"Specifically utilizing the specific skill sets and understandings that I have to try to feel like I'm, you know, to get back to that feeling I had of being attached to a good vehicle and pulling the world in a good direction."
\end{quote}

"Attached to a good vehicle." There's something here about meaning—about the human need to feel that your work matters, that your efforts contribute to something worthwhile, that the thing you're part of is pulling in the right direction.

Academia, for Jon, turned out not to be a good vehicle. FreeMoCap is his attempt to build one.

\bigskip\noindent\rule{\textwidth}{0.4pt}\bigskip

\# Part VI: Theoretical Mapping

\#\# Chapter 20: Jon's Language and Political Theory

An interesting feature of Jon's speech is how closely it maps to formal political and social theory—often without using the academic terminology.

\begin{quote}\small\ttfamily
| Theoretical Concept | Jon's Natural Language |\\
\hrule
| Gramsci's hegemony | "techno capitalist corpo hegemony" |\\
| Counter-hegemony / class consciousness | "seize back the means of production" |\\
| Freire's conscientização (critical consciousness) | "I can feel the understandings beginning" |\\
| Illich's convivial tools | "tools that just work and don't try to take anything from you" |\\
| Prefigurative politics | "prefigurative approach" (uses this exact term) |\\
| Moral injury / institutional betrayal | "moral harm" (uses this clinical term) |\\
| Banking model of education | critique of "outcompete your peers" model |\\
| War of position | strategic use of institutional resources while building alternative |\\
| Praxis / situated action | "lift where you stand" - find a handle, pull in the right direction |\\
\end{quote}

This suggests that the theoretical frameworks are not imposed from outside—they map onto something real in Jon's thinking and experience. The theory can deepen and contextualize the analysis, but the analysis stands on its own.

For future documents that integrate this material with formal theory: lead with Jon's language. Let the theory serve as annotation, not foundation.

\bigskip\noindent\rule{\textwidth}{0.4pt}\bigskip

\# Part VII: Timeline

\begin{quote}\small\ttfamily
| Date | Event |\\
\hrule
| 2017 | Jon sees OpenPose demo, realizes markerless mocap is possible |\\
| Summer 2019 | Arrives at Northeastern as tenure-track professor |\\
| January 2020 | Lab construction complete, ready to launch research program |\\
| March 2020 | COVID-19 pandemic; everyone sent home |\\
| May-June 2020 | George Floyd protests; Jon questions "whether we were really adequately using our positions of power" |\\
| Summer 2020 | Institutional betrayals revealed; "lost all faith in the scientific, educational and academic process" |\\
| Summer 2020 | Strategic pivot to cheap cameras; discovers "lift where you stand" |\\
| Summer 2020 | Begins building FreeMoCap in earnest ("fuck you, I'll build my own lab") |\\
| 2020 | Encounters Blender, begins understanding FOSS model |\\
| January 2021 | First public FreeMoCap post |\\
| July 2021 | "Meowmaline" demo video goes viral |\\
| 2021 | Epic Megagrant application video ("Purple Monkey"); receives grant |\\
| 2021 | FreeMoCap Foundation established as 501(c)(3) |\\
| June 2022 | Dynamic Walking conference presentation; \textasciitilde{}800 Discord members |\\
| September 2023 | Version 1.0 released |\\
| 2024 | \textasciitilde{}5,000 users, 115 countries |\\
| Fall 2024 | Teaching with explicit acknowledgment of moral harm to students |\\
| 2026 | \textasciitilde{}10,000 users; \textasciitilde{}2,500 Discord members |\\
| 2026 | Jon's final year at Northeastern University |\\
| 2026+ | Transition to independent operation of FreeMoCap Foundation |\\
\end{quote}

\bigskip\noindent\rule{\textwidth}{0.4pt}\bigskip

\# Part VIII: Key Quotations

\#\# On Academia

\begin{quote}\itshape
"None of this is okay. None of the general structure of academia education is morally defensible at almost any level."
\end{quote}

\begin{quote}\itshape
"Most of the research that gets done really shouldn't be done. It's not good research. It's just smart research."
\end{quote}

\begin{quote}\itshape
"You're supposed to produce more than two people who are also trying to get tenure at other schools, which is sort of the definition of a Ponzi scheme."
\end{quote}

\begin{quote}\itshape
"The open source community is what the scientific community pretends to be."
\end{quote}

\begin{quote}\itshape
"Speaking to a webcam that was worth \$3 million a year in student debt."
\end{quote}

\begin{quote}\itshape
"You should be careful not to make it too easy to use, otherwise the students won't understand what's going on in there."—advice Jon considers "insane"
\end{quote}

\#\# On the Response

\begin{quote}\itshape
"Okay, well, fuck you. I'll build my own lab in my house with webcams."
\end{quote}

\begin{quote}\itshape
"Lift where you stand: when you're having a problem, what you do is you look around where you're standing, you find something that looks like a handle, you grab it and you pull as hard as you can in what feels like the right direction."
\end{quote}

\begin{quote}\itshape
"I don't know shit about the climate. I know about how to make laser skeletons."
\end{quote}

\begin{quote}\itshape
"On the list of global inequities, maybe motion capture is not particularly high, but what IS high is inaccessibility of science and technology."
\end{quote}

\begin{quote}\itshape
"What if we just pushed in the other direction... towards the most garbage cameras you can possibly get."
\end{quote}

\begin{quote}\itshape
"Stop caring or considering what I was being asked to do by people above me in the hierarchy and start asking instead what I could do to support people who were at or below me in that same hierarchy."
\end{quote}

\begin{quote}\itshape
"It was sort of like a prefigurative kind of approach."
\end{quote}

\#\# On Technology and Capitalism

\begin{quote}\itshape
"The techno capitalist corpo hegemony that is the source of so much of the problems and harm in our society."
\end{quote}

\begin{quote}\itshape
"The true core of capitalism is to exploit disparities between yourself and others to extract resources from them."
\end{quote}

\begin{quote}\itshape
"They're trying to build moats around the Moat Blaster 9000."
\end{quote}

\#\# On Liberation

\begin{quote}\itshape
"We can literally seize back the means of production."
\end{quote}

\begin{quote}\itshape
"Claw back our culture and our society and our technology from these corporate entities that have shown themselves not to be worthy of the power that they have captured and gathered."
\end{quote}

\begin{quote}\itshape
"AI used properly is something that has the ability to be a transformatively liberative tool."
\end{quote}

\#\# On FreeMoCap

\begin{quote}\itshape
"We want indie game designers and animators to use the same tool to add motion capture assets to their zero budget art project that I am using for my federally funded scientific research program."
\end{quote}

\begin{quote}\itshape
"I don't want to tell you what you should care about. I'd rather just give you useful tools and let you do what you want to do with it."
\end{quote}

\begin{quote}\itshape
"FreeMoCap wants to be designed kind of as a tool that teaches you things about how it works."
\end{quote}

\#\# On What It's For

\begin{quote}\itshape
"I can do this. I can make this happen... I can feel the understandings beginning."
\end{quote}

\begin{quote}\itshape
"Look at what we can make. Look at what you can build if you just step away from the standard path."
\end{quote}

\begin{quote}\itshape
"It is possible to build and give gifts freely given to the world without extracting resources."
\end{quote}

\begin{quote}\itshape
"The world that is is not the world that has to be."
\end{quote}

\#\# On Staying Sane

\begin{quote}\itshape
"I have to have something that I'm working towards and building towards, and I have to feel like I'm somehow pulling in the right direction in order to feel sane."
\end{quote}

\begin{quote}\itshape
"I see you there in the Faroe Islands... Thanks for not unchecking the box, guys. I like to see the dots. It really, truly, truly keeps me going."
\end{quote}

\begin{quote}\itshape
"Even though the path is forever, the work is there. The handles are there. The shovels are there."
\end{quote}

\bigskip\noindent\rule{\textwidth}{0.4pt}\bigskip

\# Closing

\begin{quote}\itshape
"Even though the path is forever, the work is there. The handles are there. The shovels are there. We know how to pull. We know what the right direction is. We know how to dig the holes that need to be dug. And even if the distance looks insurmountable, there is a way to start."
\end{quote}

\bigskip\noindent\rule{\textwidth}{0.4pt}\bigskip

\textit{This document was compiled from:}
- \textit{2022 VFX Futures podcast interview}
- \textit{2022 Biomechanics on our Minds podcast interview}
- \textit{2022 "This is FreeMoCap" video}
- \textit{2022 Dynamic Walking conference presentation}
- \textit{2024 Fall semester introductory lecture}
- \textit{2026 State of the Skelly address}
- \textit{2026 Interview conducted for this compilation}

\textit{Prepared for the FreeMoCap Foundation, February 2026}
\normalsize


\chapter{Movement-Building Worksheet (Version 1)}
\label{src:WS1}

\begin{framed}
\small
\textbf{Source marker:} WS1 \\
\textbf{Date:} February 2026 \\
\textbf{Source:} Based on Smucker's Hegemony How-To
\end{framed}

\small
\# FreeMoCap Movement-Building Worksheet
\#\# Developing the Intellectual Infrastructure for Hegemonic Change

\textit{Based on Jonathan Smucker's "Hegemony How-To" and political organizing theory}

\bigskip\noindent\rule{\textwidth}{0.4pt}\bigskip

\# How to Use This Worksheet

This worksheet is designed for both individual reflection and group discussion. We recommend:

1. \textbf{Individual first pass:} Each core team member completes sections independently
2. \textbf{Compare and discuss:} Share responses and identify patterns, disagreements, tensions
3. \textbf{Synthesize:} Develop shared answers that represent collective understanding
4. \textbf{Revisit quarterly:} These questions aren't "solved" once—return to them as the project evolves

Set aside dedicated time. This isn't a 30-minute exercise—budget 2-3 hours for serious engagement with each major section.

\bigskip\noindent\rule{\textwidth}{0.4pt}\bigskip

\# Part 1: The Hegemonic Vision

\textit{"Hegemony" means making your values become the common sense of society. Before strategizing, get clear on what you're actually fighting for.}

\#\# 1.1 The Current Common Sense

What do people in your target fields currently believe about motion capture? Complete these sentences as someone who has NEVER heard of FreeMoCap would:

\textbf{Researchers believe:}
\begin{quote}\itshape
"To do serious motion capture research, you need \_\_\_\_\_\_\_\_\_\_\_\_\_\_\_\_\_\_\_\_\_\_\_\_\_\_\_\_\_\_\_\_"
\end{quote}

\textbf{Clinicians believe:}
\begin{quote}\itshape
"Motion capture in clinical practice is \_\_\_\_\_\_\_\_\_\_\_\_\_\_\_\_\_\_\_\_\_\_\_\_\_\_\_\_\_\_\_\_"
\end{quote}

\textbf{Animators/creators believe:}
\begin{quote}\itshape
"If I want motion capture for my project, I should \_\_\_\_\_\_\_\_\_\_\_\_\_\_\_\_\_\_\_\_\_\_\_\_\_\_\_\_\_\_\_\_"
\end{quote}

\textbf{Students/hobbyists believe:}
\begin{quote}\itshape
"Motion capture is something that \_\_\_\_\_\_\_\_\_\_\_\_\_\_\_\_\_\_\_\_\_\_\_\_\_\_\_\_\_\_\_\_"
\end{quote}

\textbf{Where did these beliefs come from?} (institutions, companies, professional training, etc.)

```
\_\_\_\_\_\_\_\_\_\_\_\_\_\_\_\_\_\_\_\_\_\_\_\_\_\_\_\_\_\_\_\_\_\_\_\_\_\_\_\_\_\_\_\_\_\_\_\_\_\_\_\_\_\_\_\_\_\_\_\_\_\_\_\_\_\_\_\_\_\_\_\_\_\_\_\_\_
\_\_\_\_\_\_\_\_\_\_\_\_\_\_\_\_\_\_\_\_\_\_\_\_\_\_\_\_\_\_\_\_\_\_\_\_\_\_\_\_\_\_\_\_\_\_\_\_\_\_\_\_\_\_\_\_\_\_\_\_\_\_\_\_\_\_\_\_\_\_\_\_\_\_\_\_\_
\_\_\_\_\_\_\_\_\_\_\_\_\_\_\_\_\_\_\_\_\_\_\_\_\_\_\_\_\_\_\_\_\_\_\_\_\_\_\_\_\_\_\_\_\_\_\_\_\_\_\_\_\_\_\_\_\_\_\_\_\_\_\_\_\_\_\_\_\_\_\_\_\_\_\_\_\_
```

\#\# 1.2 The New Common Sense

What do you want people to believe instead? Complete these as someone who has fully internalized FreeMoCap's vision:

\textbf{Researchers will believe:}
\begin{quote}\itshape
"Motion capture is \_\_\_\_\_\_\_\_\_\_\_\_\_\_\_\_\_\_\_\_\_\_\_\_\_\_\_\_\_\_\_\_"
\end{quote}

\textbf{Clinicians will believe:}
\begin{quote}\itshape
"In my practice, I can \_\_\_\_\_\_\_\_\_\_\_\_\_\_\_\_\_\_\_\_\_\_\_\_\_\_\_\_\_\_\_\_"
\end{quote}

\textbf{Animators/creators will believe:}
\begin{quote}\itshape
"When I need motion capture, I \_\_\_\_\_\_\_\_\_\_\_\_\_\_\_\_\_\_\_\_\_\_\_\_\_\_\_\_\_\_\_\_"
\end{quote}

\textbf{Students/hobbyists will believe:}
\begin{quote}\itshape
"I can \_\_\_\_\_\_\_\_\_\_\_\_\_\_\_\_\_\_\_\_\_\_\_\_\_\_\_\_\_\_\_\_"
\end{quote}

\#\# 1.3 The Gap

Looking at the distance between current common sense and your vision:

\textbf{What must change for people to adopt the new common sense?}

\begin{quote}\small\ttfamily
| Category | What must change? | Who/what currently reinforces the OLD belief? |\\
\hrule
| Beliefs about cost | | |\\
| Beliefs about expertise required | | |\\
| Beliefs about quality/accuracy | | |\\
| Beliefs about who "deserves" access | | |\\
| Beliefs about what's "professional" | | |\\
\end{quote}

\#\# 1.4 Your Theory of Change

Complete this paragraph:

\begin{quote}\itshape
FreeMoCap will shift common sense about motion capture by \_\_\_\_\_\_\_\_\_\_\_\_\_\_\_\_\_\_\_\_\_\_\_\_\_\_\_\_\_\_\_\_.
The key leverage points are \_\_\_\_\_\_\_\_\_\_\_\_\_\_\_\_\_\_\_\_\_\_\_\_\_\_\_\_\_\_\_\_.
We will know we're succeeding when \_\_\_\_\_\_\_\_\_\_\_\_\_\_\_\_\_\_\_\_\_\_\_\_\_\_\_\_\_\_\_\_.
\end{quote}

\bigskip\noindent\rule{\textwidth}{0.4pt}\bigskip

\# Part 2: Public Narrative Development

\textit{Marshall Ganz's framework: Story of Self → Story of Us → Story of Now}

\#\# 2.1 Story of Self (Founder's Story)

\textit{Why YOU, Jon? What in your life led you to this work? The goal is to connect your personal journey to universal values.}

\textbf{The Choice Point:} Describe a specific moment when you CHOSE this path. Not "I gradually became interested in..." but "I was standing in [place] when [thing happened] and I realized..."

```
\_\_\_\_\_\_\_\_\_\_\_\_\_\_\_\_\_\_\_\_\_\_\_\_\_\_\_\_\_\_\_\_\_\_\_\_\_\_\_\_\_\_\_\_\_\_\_\_\_\_\_\_\_\_\_\_\_\_\_\_\_\_\_\_\_\_\_\_\_\_\_\_\_\_\_\_\_
\_\_\_\_\_\_\_\_\_\_\_\_\_\_\_\_\_\_\_\_\_\_\_\_\_\_\_\_\_\_\_\_\_\_\_\_\_\_\_\_\_\_\_\_\_\_\_\_\_\_\_\_\_\_\_\_\_\_\_\_\_\_\_\_\_\_\_\_\_\_\_\_\_\_\_\_\_
\_\_\_\_\_\_\_\_\_\_\_\_\_\_\_\_\_\_\_\_\_\_\_\_\_\_\_\_\_\_\_\_\_\_\_\_\_\_\_\_\_\_\_\_\_\_\_\_\_\_\_\_\_\_\_\_\_\_\_\_\_\_\_\_\_\_\_\_\_\_\_\_\_\_\_\_\_
\_\_\_\_\_\_\_\_\_\_\_\_\_\_\_\_\_\_\_\_\_\_\_\_\_\_\_\_\_\_\_\_\_\_\_\_\_\_\_\_\_\_\_\_\_\_\_\_\_\_\_\_\_\_\_\_\_\_\_\_\_\_\_\_\_\_\_\_\_\_\_\_\_\_\_\_\_
```

\textbf{The Challenge:} What obstacle or injustice did you witness or experience that made the status quo unacceptable?

```
\_\_\_\_\_\_\_\_\_\_\_\_\_\_\_\_\_\_\_\_\_\_\_\_\_\_\_\_\_\_\_\_\_\_\_\_\_\_\_\_\_\_\_\_\_\_\_\_\_\_\_\_\_\_\_\_\_\_\_\_\_\_\_\_\_\_\_\_\_\_\_\_\_\_\_\_\_
\_\_\_\_\_\_\_\_\_\_\_\_\_\_\_\_\_\_\_\_\_\_\_\_\_\_\_\_\_\_\_\_\_\_\_\_\_\_\_\_\_\_\_\_\_\_\_\_\_\_\_\_\_\_\_\_\_\_\_\_\_\_\_\_\_\_\_\_\_\_\_\_\_\_\_\_\_
\_\_\_\_\_\_\_\_\_\_\_\_\_\_\_\_\_\_\_\_\_\_\_\_\_\_\_\_\_\_\_\_\_\_\_\_\_\_\_\_\_\_\_\_\_\_\_\_\_\_\_\_\_\_\_\_\_\_\_\_\_\_\_\_\_\_\_\_\_\_\_\_\_\_\_\_\_
```

\textbf{The Values:} What values drove your choice? (Not "I believe in open source" but deeper: fairness? democratization? the belief that knowledge belongs to everyone?)

```
\_\_\_\_\_\_\_\_\_\_\_\_\_\_\_\_\_\_\_\_\_\_\_\_\_\_\_\_\_\_\_\_\_\_\_\_\_\_\_\_\_\_\_\_\_\_\_\_\_\_\_\_\_\_\_\_\_\_\_\_\_\_\_\_\_\_\_\_\_\_\_\_\_\_\_\_\_
\_\_\_\_\_\_\_\_\_\_\_\_\_\_\_\_\_\_\_\_\_\_\_\_\_\_\_\_\_\_\_\_\_\_\_\_\_\_\_\_\_\_\_\_\_\_\_\_\_\_\_\_\_\_\_\_\_\_\_\_\_\_\_\_\_\_\_\_\_\_\_\_\_\_\_\_\_
\_\_\_\_\_\_\_\_\_\_\_\_\_\_\_\_\_\_\_\_\_\_\_\_\_\_\_\_\_\_\_\_\_\_\_\_\_\_\_\_\_\_\_\_\_\_\_\_\_\_\_\_\_\_\_\_\_\_\_\_\_\_\_\_\_\_\_\_\_\_\_\_\_\_\_\_\_
```

\textbf{Draft your Story of Self (3-4 sentences, conversational):}

```
\_\_\_\_\_\_\_\_\_\_\_\_\_\_\_\_\_\_\_\_\_\_\_\_\_\_\_\_\_\_\_\_\_\_\_\_\_\_\_\_\_\_\_\_\_\_\_\_\_\_\_\_\_\_\_\_\_\_\_\_\_\_\_\_\_\_\_\_\_\_\_\_\_\_\_\_\_
\_\_\_\_\_\_\_\_\_\_\_\_\_\_\_\_\_\_\_\_\_\_\_\_\_\_\_\_\_\_\_\_\_\_\_\_\_\_\_\_\_\_\_\_\_\_\_\_\_\_\_\_\_\_\_\_\_\_\_\_\_\_\_\_\_\_\_\_\_\_\_\_\_\_\_\_\_
\_\_\_\_\_\_\_\_\_\_\_\_\_\_\_\_\_\_\_\_\_\_\_\_\_\_\_\_\_\_\_\_\_\_\_\_\_\_\_\_\_\_\_\_\_\_\_\_\_\_\_\_\_\_\_\_\_\_\_\_\_\_\_\_\_\_\_\_\_\_\_\_\_\_\_\_\_
\_\_\_\_\_\_\_\_\_\_\_\_\_\_\_\_\_\_\_\_\_\_\_\_\_\_\_\_\_\_\_\_\_\_\_\_\_\_\_\_\_\_\_\_\_\_\_\_\_\_\_\_\_\_\_\_\_\_\_\_\_\_\_\_\_\_\_\_\_\_\_\_\_\_\_\_\_
```

\#\# 2.2 Story of Us (Community Identity)

\textit{Who is the "we" of FreeMoCap? What shared experiences and values bind you together?}

\textbf{List the different constituencies who are or could be part of "us":}

```
1. \_\_\_\_\_\_\_\_\_\_\_\_\_\_\_\_\_\_\_\_\_\_\_\_\_\_\_\_\_\_\_\_
2. \_\_\_\_\_\_\_\_\_\_\_\_\_\_\_\_\_\_\_\_\_\_\_\_\_\_\_\_\_\_\_\_
3. \_\_\_\_\_\_\_\_\_\_\_\_\_\_\_\_\_\_\_\_\_\_\_\_\_\_\_\_\_\_\_\_
4. \_\_\_\_\_\_\_\_\_\_\_\_\_\_\_\_\_\_\_\_\_\_\_\_\_\_\_\_\_\_\_\_
5. \_\_\_\_\_\_\_\_\_\_\_\_\_\_\_\_\_\_\_\_\_\_\_\_\_\_\_\_\_\_\_\_
```

\textbf{What do these groups have in common?} (Not demographics—shared experiences, values, frustrations)

```
\_\_\_\_\_\_\_\_\_\_\_\_\_\_\_\_\_\_\_\_\_\_\_\_\_\_\_\_\_\_\_\_\_\_\_\_\_\_\_\_\_\_\_\_\_\_\_\_\_\_\_\_\_\_\_\_\_\_\_\_\_\_\_\_\_\_\_\_\_\_\_\_\_\_\_\_\_
\_\_\_\_\_\_\_\_\_\_\_\_\_\_\_\_\_\_\_\_\_\_\_\_\_\_\_\_\_\_\_\_\_\_\_\_\_\_\_\_\_\_\_\_\_\_\_\_\_\_\_\_\_\_\_\_\_\_\_\_\_\_\_\_\_\_\_\_\_\_\_\_\_\_\_\_\_
\_\_\_\_\_\_\_\_\_\_\_\_\_\_\_\_\_\_\_\_\_\_\_\_\_\_\_\_\_\_\_\_\_\_\_\_\_\_\_\_\_\_\_\_\_\_\_\_\_\_\_\_\_\_\_\_\_\_\_\_\_\_\_\_\_\_\_\_\_\_\_\_\_\_\_\_\_
```

\textbf{Complete this sentence that any FreeMoCap community member could say:}

\begin{quote}\itshape
"We are people who \_\_\_\_\_\_\_\_\_\_\_\_\_\_\_\_\_\_\_\_\_\_\_\_\_\_\_\_\_\_\_\_"
\end{quote}

\textbf{What's OUR challenge?} (The shared obstacle we face together)

```
\_\_\_\_\_\_\_\_\_\_\_\_\_\_\_\_\_\_\_\_\_\_\_\_\_\_\_\_\_\_\_\_\_\_\_\_\_\_\_\_\_\_\_\_\_\_\_\_\_\_\_\_\_\_\_\_\_\_\_\_\_\_\_\_\_\_\_\_\_\_\_\_\_\_\_\_\_
\_\_\_\_\_\_\_\_\_\_\_\_\_\_\_\_\_\_\_\_\_\_\_\_\_\_\_\_\_\_\_\_\_\_\_\_\_\_\_\_\_\_\_\_\_\_\_\_\_\_\_\_\_\_\_\_\_\_\_\_\_\_\_\_\_\_\_\_\_\_\_\_\_\_\_\_\_
```

\textbf{What's OUR hope?} (The shared vision that motivates us)

```
\_\_\_\_\_\_\_\_\_\_\_\_\_\_\_\_\_\_\_\_\_\_\_\_\_\_\_\_\_\_\_\_\_\_\_\_\_\_\_\_\_\_\_\_\_\_\_\_\_\_\_\_\_\_\_\_\_\_\_\_\_\_\_\_\_\_\_\_\_\_\_\_\_\_\_\_\_
\_\_\_\_\_\_\_\_\_\_\_\_\_\_\_\_\_\_\_\_\_\_\_\_\_\_\_\_\_\_\_\_\_\_\_\_\_\_\_\_\_\_\_\_\_\_\_\_\_\_\_\_\_\_\_\_\_\_\_\_\_\_\_\_\_\_\_\_\_\_\_\_\_\_\_\_\_
```

\textbf{Draft your Story of Us (3-4 sentences):}

```
\_\_\_\_\_\_\_\_\_\_\_\_\_\_\_\_\_\_\_\_\_\_\_\_\_\_\_\_\_\_\_\_\_\_\_\_\_\_\_\_\_\_\_\_\_\_\_\_\_\_\_\_\_\_\_\_\_\_\_\_\_\_\_\_\_\_\_\_\_\_\_\_\_\_\_\_\_
\_\_\_\_\_\_\_\_\_\_\_\_\_\_\_\_\_\_\_\_\_\_\_\_\_\_\_\_\_\_\_\_\_\_\_\_\_\_\_\_\_\_\_\_\_\_\_\_\_\_\_\_\_\_\_\_\_\_\_\_\_\_\_\_\_\_\_\_\_\_\_\_\_\_\_\_\_
\_\_\_\_\_\_\_\_\_\_\_\_\_\_\_\_\_\_\_\_\_\_\_\_\_\_\_\_\_\_\_\_\_\_\_\_\_\_\_\_\_\_\_\_\_\_\_\_\_\_\_\_\_\_\_\_\_\_\_\_\_\_\_\_\_\_\_\_\_\_\_\_\_\_\_\_\_
\_\_\_\_\_\_\_\_\_\_\_\_\_\_\_\_\_\_\_\_\_\_\_\_\_\_\_\_\_\_\_\_\_\_\_\_\_\_\_\_\_\_\_\_\_\_\_\_\_\_\_\_\_\_\_\_\_\_\_\_\_\_\_\_\_\_\_\_\_\_\_\_\_\_\_\_\_
```

\#\# 2.3 Story of Now (The Urgent Call)

\textit{Why NOW? What makes this moment critical? What opportunity will close if we don't act?}

\textbf{The Urgent Challenge:} What's happening RIGHT NOW that makes action necessary?

```
\_\_\_\_\_\_\_\_\_\_\_\_\_\_\_\_\_\_\_\_\_\_\_\_\_\_\_\_\_\_\_\_\_\_\_\_\_\_\_\_\_\_\_\_\_\_\_\_\_\_\_\_\_\_\_\_\_\_\_\_\_\_\_\_\_\_\_\_\_\_\_\_\_\_\_\_\_
\_\_\_\_\_\_\_\_\_\_\_\_\_\_\_\_\_\_\_\_\_\_\_\_\_\_\_\_\_\_\_\_\_\_\_\_\_\_\_\_\_\_\_\_\_\_\_\_\_\_\_\_\_\_\_\_\_\_\_\_\_\_\_\_\_\_\_\_\_\_\_\_\_\_\_\_\_
```

\textbf{The Closing Window:} What opportunity will we lose if we don't act soon?

```
\_\_\_\_\_\_\_\_\_\_\_\_\_\_\_\_\_\_\_\_\_\_\_\_\_\_\_\_\_\_\_\_\_\_\_\_\_\_\_\_\_\_\_\_\_\_\_\_\_\_\_\_\_\_\_\_\_\_\_\_\_\_\_\_\_\_\_\_\_\_\_\_\_\_\_\_\_
\_\_\_\_\_\_\_\_\_\_\_\_\_\_\_\_\_\_\_\_\_\_\_\_\_\_\_\_\_\_\_\_\_\_\_\_\_\_\_\_\_\_\_\_\_\_\_\_\_\_\_\_\_\_\_\_\_\_\_\_\_\_\_\_\_\_\_\_\_\_\_\_\_\_\_\_\_
```

\textbf{The Achievable Goal:} What specific thing can we accomplish if we act together now?

```
\_\_\_\_\_\_\_\_\_\_\_\_\_\_\_\_\_\_\_\_\_\_\_\_\_\_\_\_\_\_\_\_\_\_\_\_\_\_\_\_\_\_\_\_\_\_\_\_\_\_\_\_\_\_\_\_\_\_\_\_\_\_\_\_\_\_\_\_\_\_\_\_\_\_\_\_\_
\_\_\_\_\_\_\_\_\_\_\_\_\_\_\_\_\_\_\_\_\_\_\_\_\_\_\_\_\_\_\_\_\_\_\_\_\_\_\_\_\_\_\_\_\_\_\_\_\_\_\_\_\_\_\_\_\_\_\_\_\_\_\_\_\_\_\_\_\_\_\_\_\_\_\_\_\_
```

\textbf{The Specific Ask:} What do you want the listener to DO after hearing this story?

\begin{quote}\small\ttfamily
| Audience | Specific Ask |\\
\hrule
| Potential user | |\\
| Potential contributor | |\\
| Potential funder | |\\
| Institutional partner | |\\
\end{quote}

\textbf{Draft your Story of Now (3-4 sentences):}

```
\_\_\_\_\_\_\_\_\_\_\_\_\_\_\_\_\_\_\_\_\_\_\_\_\_\_\_\_\_\_\_\_\_\_\_\_\_\_\_\_\_\_\_\_\_\_\_\_\_\_\_\_\_\_\_\_\_\_\_\_\_\_\_\_\_\_\_\_\_\_\_\_\_\_\_\_\_
\_\_\_\_\_\_\_\_\_\_\_\_\_\_\_\_\_\_\_\_\_\_\_\_\_\_\_\_\_\_\_\_\_\_\_\_\_\_\_\_\_\_\_\_\_\_\_\_\_\_\_\_\_\_\_\_\_\_\_\_\_\_\_\_\_\_\_\_\_\_\_\_\_\_\_\_\_
\_\_\_\_\_\_\_\_\_\_\_\_\_\_\_\_\_\_\_\_\_\_\_\_\_\_\_\_\_\_\_\_\_\_\_\_\_\_\_\_\_\_\_\_\_\_\_\_\_\_\_\_\_\_\_\_\_\_\_\_\_\_\_\_\_\_\_\_\_\_\_\_\_\_\_\_\_
\_\_\_\_\_\_\_\_\_\_\_\_\_\_\_\_\_\_\_\_\_\_\_\_\_\_\_\_\_\_\_\_\_\_\_\_\_\_\_\_\_\_\_\_\_\_\_\_\_\_\_\_\_\_\_\_\_\_\_\_\_\_\_\_\_\_\_\_\_\_\_\_\_\_\_\_\_
```

\#\# 2.4 The Full Narrative

Now weave them together. Practice telling this out loud—it should take 2-3 minutes:

```
[Story of Self]
\_\_\_\_\_\_\_\_\_\_\_\_\_\_\_\_\_\_\_\_\_\_\_\_\_\_\_\_\_\_\_\_\_\_\_\_\_\_\_\_\_\_\_\_\_\_\_\_\_\_\_\_\_\_\_\_\_\_\_\_\_\_\_\_\_\_\_\_\_\_\_\_\_\_\_\_\_
\_\_\_\_\_\_\_\_\_\_\_\_\_\_\_\_\_\_\_\_\_\_\_\_\_\_\_\_\_\_\_\_\_\_\_\_\_\_\_\_\_\_\_\_\_\_\_\_\_\_\_\_\_\_\_\_\_\_\_\_\_\_\_\_\_\_\_\_\_\_\_\_\_\_\_\_\_
\_\_\_\_\_\_\_\_\_\_\_\_\_\_\_\_\_\_\_\_\_\_\_\_\_\_\_\_\_\_\_\_\_\_\_\_\_\_\_\_\_\_\_\_\_\_\_\_\_\_\_\_\_\_\_\_\_\_\_\_\_\_\_\_\_\_\_\_\_\_\_\_\_\_\_\_\_

[Bridge to Us]: "And I discovered I wasn't alone..."

[Story of Us]
\_\_\_\_\_\_\_\_\_\_\_\_\_\_\_\_\_\_\_\_\_\_\_\_\_\_\_\_\_\_\_\_\_\_\_\_\_\_\_\_\_\_\_\_\_\_\_\_\_\_\_\_\_\_\_\_\_\_\_\_\_\_\_\_\_\_\_\_\_\_\_\_\_\_\_\_\_
\_\_\_\_\_\_\_\_\_\_\_\_\_\_\_\_\_\_\_\_\_\_\_\_\_\_\_\_\_\_\_\_\_\_\_\_\_\_\_\_\_\_\_\_\_\_\_\_\_\_\_\_\_\_\_\_\_\_\_\_\_\_\_\_\_\_\_\_\_\_\_\_\_\_\_\_\_
\_\_\_\_\_\_\_\_\_\_\_\_\_\_\_\_\_\_\_\_\_\_\_\_\_\_\_\_\_\_\_\_\_\_\_\_\_\_\_\_\_\_\_\_\_\_\_\_\_\_\_\_\_\_\_\_\_\_\_\_\_\_\_\_\_\_\_\_\_\_\_\_\_\_\_\_\_

[Bridge to Now]: "And now we face a critical moment..."

[Story of Now]
\_\_\_\_\_\_\_\_\_\_\_\_\_\_\_\_\_\_\_\_\_\_\_\_\_\_\_\_\_\_\_\_\_\_\_\_\_\_\_\_\_\_\_\_\_\_\_\_\_\_\_\_\_\_\_\_\_\_\_\_\_\_\_\_\_\_\_\_\_\_\_\_\_\_\_\_\_
\_\_\_\_\_\_\_\_\_\_\_\_\_\_\_\_\_\_\_\_\_\_\_\_\_\_\_\_\_\_\_\_\_\_\_\_\_\_\_\_\_\_\_\_\_\_\_\_\_\_\_\_\_\_\_\_\_\_\_\_\_\_\_\_\_\_\_\_\_\_\_\_\_\_\_\_\_
\_\_\_\_\_\_\_\_\_\_\_\_\_\_\_\_\_\_\_\_\_\_\_\_\_\_\_\_\_\_\_\_\_\_\_\_\_\_\_\_\_\_\_\_\_\_\_\_\_\_\_\_\_\_\_\_\_\_\_\_\_\_\_\_\_\_\_\_\_\_\_\_\_\_\_\_\_

[The Ask]: "That's why I'm asking you to..."
\_\_\_\_\_\_\_\_\_\_\_\_\_\_\_\_\_\_\_\_\_\_\_\_\_\_\_\_\_\_\_\_\_\_\_\_\_\_\_\_\_\_\_\_\_\_\_\_\_\_\_\_\_\_\_\_\_\_\_\_\_\_\_\_\_\_\_\_\_\_\_\_\_\_\_\_\_
```

\bigskip\noindent\rule{\textwidth}{0.4pt}\bigskip

\# Part 3: The Clubhouse Audit

\textit{Smucker's "political identity paradox": the cohesion that makes groups effective also isolates them. This section helps identify where FreeMoCap may be inadvertently exclusionary.}

\#\# 3.1 Jargon Inventory

List terms that FreeMoCap uses that an outsider might not understand:

\begin{quote}\small\ttfamily
| Term | What it means | Accessible alternative |\\
\hrule
| Markerless | | |\\
| Pose estimation | | |\\
| Charuco board | | |\\
| Skelly/SkellyTracker | | |\\
| | | |\\
| | | |\\
| | | |\\
| | | |\\
\end{quote}

\textbf{Review recent Discord/GitHub conversations. What terms appear frequently that might confuse newcomers?}

```
\_\_\_\_\_\_\_\_\_\_\_\_\_\_\_\_\_\_\_\_\_\_\_\_\_\_\_\_\_\_\_\_\_\_\_\_\_\_\_\_\_\_\_\_\_\_\_\_\_\_\_\_\_\_\_\_\_\_\_\_\_\_\_\_\_\_\_\_\_\_\_\_\_\_\_\_\_
\_\_\_\_\_\_\_\_\_\_\_\_\_\_\_\_\_\_\_\_\_\_\_\_\_\_\_\_\_\_\_\_\_\_\_\_\_\_\_\_\_\_\_\_\_\_\_\_\_\_\_\_\_\_\_\_\_\_\_\_\_\_\_\_\_\_\_\_\_\_\_\_\_\_\_\_\_
```

\#\# 3.2 Implicit Norms

What does someone need to KNOW to participate in FreeMoCap that isn't written down anywhere?

\begin{quote}\small\ttfamily
| Category | Implicit knowledge required | How could we make this explicit? |\\
\hrule
| Technical skills | | |\\
| Tools/platforms (Git, Discord, etc.) | | |\\
| Communication norms | | |\\
| Decision-making processes | | |\\
| Social dynamics (who to ask, etc.) | | |\\
\end{quote}

\#\# 3.3 The Newcomer Experience

\textbf{Imagine someone who:}
- Has heard of FreeMoCap for the first time
- Is interested but has no open source experience
- Has a potential use case but isn't sure if FreeMoCap is right for them

\textbf{Map their journey:}

\begin{quote}\small\ttfamily
| Step | What happens now? | Pain points? | How to improve? |\\
\hrule
| Discovers FreeMoCap (where/how?) | | | |\\
| Tries to understand what it does | | | |\\
| Tries to install/use it | | | |\\
| Encounters a problem | | | |\\
| Tries to get help | | | |\\
| Considers contributing | | | |\\
| Makes first contribution | | | |\\
\end{quote}

\#\# 3.4 Who's NOT Here?

\textbf{Look at your current community. Who is missing?}

\begin{quote}\small\ttfamily
| Group that SHOULD be interested | Why aren't they here? | What barrier might they face? |\\
\hrule
| | | |\\
| | | |\\
| | | |\\
| | | |\\
\end{quote}

\textbf{Uncomfortable question: Is there any way that current community culture might actively repel certain people?}

```
\_\_\_\_\_\_\_\_\_\_\_\_\_\_\_\_\_\_\_\_\_\_\_\_\_\_\_\_\_\_\_\_\_\_\_\_\_\_\_\_\_\_\_\_\_\_\_\_\_\_\_\_\_\_\_\_\_\_\_\_\_\_\_\_\_\_\_\_\_\_\_\_\_\_\_\_\_
\_\_\_\_\_\_\_\_\_\_\_\_\_\_\_\_\_\_\_\_\_\_\_\_\_\_\_\_\_\_\_\_\_\_\_\_\_\_\_\_\_\_\_\_\_\_\_\_\_\_\_\_\_\_\_\_\_\_\_\_\_\_\_\_\_\_\_\_\_\_\_\_\_\_\_\_\_
```

\#\# 3.5 The Insider Test

Ask a current core contributor to answer honestly:

\begin{quote}\itshape
"What do you wish someone had told you when you first encountered FreeMoCap?"
\end{quote}

```
\_\_\_\_\_\_\_\_\_\_\_\_\_\_\_\_\_\_\_\_\_\_\_\_\_\_\_\_\_\_\_\_\_\_\_\_\_\_\_\_\_\_\_\_\_\_\_\_\_\_\_\_\_\_\_\_\_\_\_\_\_\_\_\_\_\_\_\_\_\_\_\_\_\_\_\_\_
\_\_\_\_\_\_\_\_\_\_\_\_\_\_\_\_\_\_\_\_\_\_\_\_\_\_\_\_\_\_\_\_\_\_\_\_\_\_\_\_\_\_\_\_\_\_\_\_\_\_\_\_\_\_\_\_\_\_\_\_\_\_\_\_\_\_\_\_\_\_\_\_\_\_\_\_\_
\_\_\_\_\_\_\_\_\_\_\_\_\_\_\_\_\_\_\_\_\_\_\_\_\_\_\_\_\_\_\_\_\_\_\_\_\_\_\_\_\_\_\_\_\_\_\_\_\_\_\_\_\_\_\_\_\_\_\_\_\_\_\_\_\_\_\_\_\_\_\_\_\_\_\_\_\_
```

\begin{quote}\itshape
"What almost made you give up or leave?"
\end{quote}

```
\_\_\_\_\_\_\_\_\_\_\_\_\_\_\_\_\_\_\_\_\_\_\_\_\_\_\_\_\_\_\_\_\_\_\_\_\_\_\_\_\_\_\_\_\_\_\_\_\_\_\_\_\_\_\_\_\_\_\_\_\_\_\_\_\_\_\_\_\_\_\_\_\_\_\_\_\_
\_\_\_\_\_\_\_\_\_\_\_\_\_\_\_\_\_\_\_\_\_\_\_\_\_\_\_\_\_\_\_\_\_\_\_\_\_\_\_\_\_\_\_\_\_\_\_\_\_\_\_\_\_\_\_\_\_\_\_\_\_\_\_\_\_\_\_\_\_\_\_\_\_\_\_\_\_
\_\_\_\_\_\_\_\_\_\_\_\_\_\_\_\_\_\_\_\_\_\_\_\_\_\_\_\_\_\_\_\_\_\_\_\_\_\_\_\_\_\_\_\_\_\_\_\_\_\_\_\_\_\_\_\_\_\_\_\_\_\_\_\_\_\_\_\_\_\_\_\_\_\_\_\_\_
```

\bigskip\noindent\rule{\textwidth}{0.4pt}\bigskip

\# Part 4: Organizing vs. Mobilizing Assessment

\textit{Jane McAlevey's distinction: Mobilizing activates existing supporters; Organizing expands your base by bringing in people who don't yet see themselves as part of the movement.}

\#\# 4.1 Current State Assessment

\textbf{Who currently participates in FreeMoCap?} (Be specific about backgrounds, how they found you)

```
\_\_\_\_\_\_\_\_\_\_\_\_\_\_\_\_\_\_\_\_\_\_\_\_\_\_\_\_\_\_\_\_\_\_\_\_\_\_\_\_\_\_\_\_\_\_\_\_\_\_\_\_\_\_\_\_\_\_\_\_\_\_\_\_\_\_\_\_\_\_\_\_\_\_\_\_\_
\_\_\_\_\_\_\_\_\_\_\_\_\_\_\_\_\_\_\_\_\_\_\_\_\_\_\_\_\_\_\_\_\_\_\_\_\_\_\_\_\_\_\_\_\_\_\_\_\_\_\_\_\_\_\_\_\_\_\_\_\_\_\_\_\_\_\_\_\_\_\_\_\_\_\_\_\_
\_\_\_\_\_\_\_\_\_\_\_\_\_\_\_\_\_\_\_\_\_\_\_\_\_\_\_\_\_\_\_\_\_\_\_\_\_\_\_\_\_\_\_\_\_\_\_\_\_\_\_\_\_\_\_\_\_\_\_\_\_\_\_\_\_\_\_\_\_\_\_\_\_\_\_\_\_
```

\textbf{How did they find FreeMoCap?} (What channels, who referred them, what were they searching for?)

```
\_\_\_\_\_\_\_\_\_\_\_\_\_\_\_\_\_\_\_\_\_\_\_\_\_\_\_\_\_\_\_\_\_\_\_\_\_\_\_\_\_\_\_\_\_\_\_\_\_\_\_\_\_\_\_\_\_\_\_\_\_\_\_\_\_\_\_\_\_\_\_\_\_\_\_\_\_
\_\_\_\_\_\_\_\_\_\_\_\_\_\_\_\_\_\_\_\_\_\_\_\_\_\_\_\_\_\_\_\_\_\_\_\_\_\_\_\_\_\_\_\_\_\_\_\_\_\_\_\_\_\_\_\_\_\_\_\_\_\_\_\_\_\_\_\_\_\_\_\_\_\_\_\_\_
```

\textbf{What do current participants have in common?} (Background, values, skills, demographics)

```
\_\_\_\_\_\_\_\_\_\_\_\_\_\_\_\_\_\_\_\_\_\_\_\_\_\_\_\_\_\_\_\_\_\_\_\_\_\_\_\_\_\_\_\_\_\_\_\_\_\_\_\_\_\_\_\_\_\_\_\_\_\_\_\_\_\_\_\_\_\_\_\_\_\_\_\_\_
\_\_\_\_\_\_\_\_\_\_\_\_\_\_\_\_\_\_\_\_\_\_\_\_\_\_\_\_\_\_\_\_\_\_\_\_\_\_\_\_\_\_\_\_\_\_\_\_\_\_\_\_\_\_\_\_\_\_\_\_\_\_\_\_\_\_\_\_\_\_\_\_\_\_\_\_\_
```

\textbf{Honest assessment: Are you primarily reaching people who are ALREADY:}
- [ ] Interested in open source?
- [ ] Technically skilled in Python/programming?
- [ ] Connected to academic research?
- [ ] Part of maker/hacker culture?

\textit{If you checked most of these, you may be mobilizing rather than organizing.}

\#\# 4.2 The Organizing Target

\textbf{Who would benefit from FreeMoCap but doesn't know it exists or doesn't see it as "for them"?}

\begin{quote}\small\ttfamily
| Constituency | What would they gain? | Why don't they know about/use FreeMoCap now? |\\
\hrule
| | | |\\
| | | |\\
| | | |\\
| | | |\\
\end{quote}

\#\# 4.3 Organic Leaders

\textit{McAlevey's concept: "Organic leaders" are people deeply respected within their own communities who can bring others along. They're not already activists—they're people with social capital in target constituencies.}

\textbf{For each target constituency, who are the organic leaders?}

\begin{quote}\small\ttfamily
| Constituency | Who do people in this group listen to? | Do you have any connection to them? |\\
\hrule
| Biomechanics researchers | | |\\
| Physical therapists | | |\\
| Animation/game dev educators | | |\\
| High school STEM teachers | | |\\
| Dance/movement communities | | |\\
| | | |\\
\end{quote}

\textbf{Concrete question: Can you name 3 specific people (not types of people) who could open doors to new constituencies?}

```
1. \_\_\_\_\_\_\_\_\_\_\_\_\_\_\_\_\_\_\_\_\_\_\_\_\_\_\_\_\_\_\_\_ (constituency: \_\_\_\_\_\_\_\_\_\_\_\_\_\_\_)
2. \_\_\_\_\_\_\_\_\_\_\_\_\_\_\_\_\_\_\_\_\_\_\_\_\_\_\_\_\_\_\_\_ (constituency: \_\_\_\_\_\_\_\_\_\_\_\_\_\_\_)
3. \_\_\_\_\_\_\_\_\_\_\_\_\_\_\_\_\_\_\_\_\_\_\_\_\_\_\_\_\_\_\_\_ (constituency: \_\_\_\_\_\_\_\_\_\_\_\_\_\_\_)
```

\#\# 4.4 Bloc Recruitment Strategy

\textit{Rather than recruiting individuals one-by-one, can you recruit existing GROUPS?}

\textbf{What organized groups could become FreeMoCap adopters/partners?}

\begin{quote}\small\ttfamily
| Type of group | Specific examples | Who would you contact? | What would you offer them? |\\
\hrule
| University labs | | | |\\
| Professional associations | | | |\\
| Educational programs | | | |\\
| Maker spaces | | | |\\
| Clinics/practices | | | |\\
| Online communities | | | |\\
\end{quote}

\bigskip\noindent\rule{\textwidth}{0.4pt}\bigskip

\# Part 5: Core and Base Structure

\textit{Every movement needs both: a committed CORE that provides direction and capacity, and a broader BASE that participates at varying levels.}

\#\# 5.1 Mapping Your Current Structure

\textbf{Who is currently in the CORE?} (Regular contributors, decision-makers)

\begin{quote}\small\ttfamily
| Name | Role/contribution | Time commitment | How long involved? |\\
\hrule
| | | | |\\
| | | | |\\
| | | | |\\
| | | | |\\
\end{quote}

\textbf{Is this enough people for sustainability?} 

```
\_\_\_\_\_\_\_\_\_\_\_\_\_\_\_\_\_\_\_\_\_\_\_\_\_\_\_\_\_\_\_\_\_\_\_\_\_\_\_\_\_\_\_\_\_\_\_\_\_\_\_\_\_\_\_\_\_\_\_\_\_\_\_\_\_\_\_\_\_\_\_\_\_\_\_\_\_
```

\textbf{What functions are UNDERSTAFFED in the core?}

\begin{quote}\small\ttfamily
| Function | Currently handled by | Ideal: who should own this? |\\
\hrule
| Technical development | | |\\
| Documentation | | |\\
| Community management | | |\\
| User support | | |\\
| Outreach/partnerships | | |\\
| Fundraising | | |\\
| Education/tutorials | | |\\
\end{quote}

\#\# 5.2 Levels of Engagement

\textbf{Design a ladder of engagement—how do people move from casual user to core contributor?}

\begin{quote}\small\ttfamily
| Level | Description | What they do | How many do you have? | How many do you need? |\\
\hrule
| 1: Aware | Knows FreeMoCap exists | Follows on social media | | |\\
| 2: User | Uses the software | Downloads, runs | | |\\
| 3: Active User | Regular user, gives feedback | Files issues, asks questions | | |\\
| 4: Contributor | Occasional contribution | PRs, docs, support others | | |\\
| 5: Regular Contributor | Consistent participation | Multiple contributions/month | | |\\
| 6: Core | Owns a domain | Leads working group, decision-making | | |\\
\end{quote}

\textbf{What are the BARRIERS between each level?} (Why don't people move up?)

```
Level 1→2: \_\_\_\_\_\_\_\_\_\_\_\_\_\_\_\_\_\_\_\_\_\_\_\_\_\_\_\_\_\_\_\_\_\_\_\_\_\_\_\_\_\_\_\_\_\_\_\_\_\_\_\_\_\_\_\_\_\_\_\_\_\_\_\_\_
Level 2→3: \_\_\_\_\_\_\_\_\_\_\_\_\_\_\_\_\_\_\_\_\_\_\_\_\_\_\_\_\_\_\_\_\_\_\_\_\_\_\_\_\_\_\_\_\_\_\_\_\_\_\_\_\_\_\_\_\_\_\_\_\_\_\_\_\_
Level 3→4: \_\_\_\_\_\_\_\_\_\_\_\_\_\_\_\_\_\_\_\_\_\_\_\_\_\_\_\_\_\_\_\_\_\_\_\_\_\_\_\_\_\_\_\_\_\_\_\_\_\_\_\_\_\_\_\_\_\_\_\_\_\_\_\_\_
Level 4→5: \_\_\_\_\_\_\_\_\_\_\_\_\_\_\_\_\_\_\_\_\_\_\_\_\_\_\_\_\_\_\_\_\_\_\_\_\_\_\_\_\_\_\_\_\_\_\_\_\_\_\_\_\_\_\_\_\_\_\_\_\_\_\_\_\_
Level 5→6: \_\_\_\_\_\_\_\_\_\_\_\_\_\_\_\_\_\_\_\_\_\_\_\_\_\_\_\_\_\_\_\_\_\_\_\_\_\_\_\_\_\_\_\_\_\_\_\_\_\_\_\_\_\_\_\_\_\_\_\_\_\_\_\_\_
```

\#\# 5.3 Non-Code Contributions

\textbf{List all the ways someone could contribute WITHOUT writing code:}

```
1. \_\_\_\_\_\_\_\_\_\_\_\_\_\_\_\_\_\_\_\_\_\_\_\_\_\_\_\_\_\_\_\_
2. \_\_\_\_\_\_\_\_\_\_\_\_\_\_\_\_\_\_\_\_\_\_\_\_\_\_\_\_\_\_\_\_
3. \_\_\_\_\_\_\_\_\_\_\_\_\_\_\_\_\_\_\_\_\_\_\_\_\_\_\_\_\_\_\_\_
4. \_\_\_\_\_\_\_\_\_\_\_\_\_\_\_\_\_\_\_\_\_\_\_\_\_\_\_\_\_\_\_\_
5. \_\_\_\_\_\_\_\_\_\_\_\_\_\_\_\_\_\_\_\_\_\_\_\_\_\_\_\_\_\_\_\_
6. \_\_\_\_\_\_\_\_\_\_\_\_\_\_\_\_\_\_\_\_\_\_\_\_\_\_\_\_\_\_\_\_
7. \_\_\_\_\_\_\_\_\_\_\_\_\_\_\_\_\_\_\_\_\_\_\_\_\_\_\_\_\_\_\_\_
8. \_\_\_\_\_\_\_\_\_\_\_\_\_\_\_\_\_\_\_\_\_\_\_\_\_\_\_\_\_\_\_\_
```

\textbf{Are these pathways clearly documented and promoted?} [ ] Yes [ ] No [ ] Partially

\textbf{For each, is there a named person who could mentor/onboard new contributors?} [ ] Yes [ ] No [ ] Some

\bigskip\noindent\rule{\textwidth}{0.4pt}\bigskip

\# Part 6: Governance and Decision-Making

\textit{Explicit governance enables participation. Implicit governance excludes.}

\#\# 6.1 Current State (Be Honest)

\textbf{How are decisions currently made in FreeMoCap?}

\begin{quote}\small\ttfamily
| Decision type | Who decides? | What's the process? | Is this documented? |\\
\hrule
| Technical direction | | | |\\
| Feature priorities | | | |\\
| Release timing | | | |\\
| Community policies | | | |\\
| Partnerships | | | |\\
| Spending/resources | | | |\\
| Who becomes a maintainer | | | |\\
\end{quote}

\textbf{If someone wanted to propose a significant change, what would they do?}

```
\_\_\_\_\_\_\_\_\_\_\_\_\_\_\_\_\_\_\_\_\_\_\_\_\_\_\_\_\_\_\_\_\_\_\_\_\_\_\_\_\_\_\_\_\_\_\_\_\_\_\_\_\_\_\_\_\_\_\_\_\_\_\_\_\_\_\_\_\_\_\_\_\_\_\_\_\_
\_\_\_\_\_\_\_\_\_\_\_\_\_\_\_\_\_\_\_\_\_\_\_\_\_\_\_\_\_\_\_\_\_\_\_\_\_\_\_\_\_\_\_\_\_\_\_\_\_\_\_\_\_\_\_\_\_\_\_\_\_\_\_\_\_\_\_\_\_\_\_\_\_\_\_\_\_
```

\textbf{If there's a disagreement, how is it resolved?}

```
\_\_\_\_\_\_\_\_\_\_\_\_\_\_\_\_\_\_\_\_\_\_\_\_\_\_\_\_\_\_\_\_\_\_\_\_\_\_\_\_\_\_\_\_\_\_\_\_\_\_\_\_\_\_\_\_\_\_\_\_\_\_\_\_\_\_\_\_\_\_\_\_\_\_\_\_\_
\_\_\_\_\_\_\_\_\_\_\_\_\_\_\_\_\_\_\_\_\_\_\_\_\_\_\_\_\_\_\_\_\_\_\_\_\_\_\_\_\_\_\_\_\_\_\_\_\_\_\_\_\_\_\_\_\_\_\_\_\_\_\_\_\_\_\_\_\_\_\_\_\_\_\_\_\_
```

\#\# 6.2 Governance Design

\textbf{What governance structure would serve FreeMoCap's mission?}

Consider:
- How democratic should decision-making be?
- What decisions should be made by whom?
- How do new people gain decision-making power?
- How do you balance efficiency with inclusion?

\textbf{Draft a simple governance document:}

```
FREEMOCAP GOVERNANCE (DRAFT)

Mission: \_\_\_\_\_\_\_\_\_\_\_\_\_\_\_\_\_\_\_\_\_\_\_\_\_\_\_\_\_\_\_\_\_\_\_\_\_\_\_\_\_\_\_\_\_\_\_\_\_\_\_\_\_\_\_\_\_\_\_\_\_\_\_

Values: 
1. \_\_\_\_\_\_\_\_\_\_\_\_\_\_\_\_\_\_\_\_\_\_\_\_\_\_\_\_\_\_\_\_
2. \_\_\_\_\_\_\_\_\_\_\_\_\_\_\_\_\_\_\_\_\_\_\_\_\_\_\_\_\_\_\_\_
3. \_\_\_\_\_\_\_\_\_\_\_\_\_\_\_\_\_\_\_\_\_\_\_\_\_\_\_\_\_\_\_\_

Decision-Making:
- Day-to-day technical decisions are made by: \_\_\_\_\_\_\_\_\_\_\_\_\_\_\_\_\_\_\_\_\_\_\_\_\_\_\_\_
- Feature priorities are decided by: \_\_\_\_\_\_\_\_\_\_\_\_\_\_\_\_\_\_\_\_\_\_\_\_\_\_\_\_\_\_\_\_\_\_\_\_\_
- Community policies are decided by: \_\_\_\_\_\_\_\_\_\_\_\_\_\_\_\_\_\_\_\_\_\_\_\_\_\_\_\_\_\_\_\_\_\_\_\_
- Major strategic decisions are decided by: \_\_\_\_\_\_\_\_\_\_\_\_\_\_\_\_\_\_\_\_\_\_\_\_\_\_\_\_\_\_

Becoming a Maintainer:
\_\_\_\_\_\_\_\_\_\_\_\_\_\_\_\_\_\_\_\_\_\_\_\_\_\_\_\_\_\_\_\_\_\_\_\_\_\_\_\_\_\_\_\_\_\_\_\_\_\_\_\_\_\_\_\_\_\_\_\_\_\_\_\_\_\_\_\_\_\_\_\_\_
\_\_\_\_\_\_\_\_\_\_\_\_\_\_\_\_\_\_\_\_\_\_\_\_\_\_\_\_\_\_\_\_\_\_\_\_\_\_\_\_\_\_\_\_\_\_\_\_\_\_\_\_\_\_\_\_\_\_\_\_\_\_\_\_\_\_\_\_\_\_\_\_\_

Conflict Resolution:
\_\_\_\_\_\_\_\_\_\_\_\_\_\_\_\_\_\_\_\_\_\_\_\_\_\_\_\_\_\_\_\_\_\_\_\_\_\_\_\_\_\_\_\_\_\_\_\_\_\_\_\_\_\_\_\_\_\_\_\_\_\_\_\_\_\_\_\_\_\_\_\_\_
\_\_\_\_\_\_\_\_\_\_\_\_\_\_\_\_\_\_\_\_\_\_\_\_\_\_\_\_\_\_\_\_\_\_\_\_\_\_\_\_\_\_\_\_\_\_\_\_\_\_\_\_\_\_\_\_\_\_\_\_\_\_\_\_\_\_\_\_\_\_\_\_\_
```

\#\# 6.3 Working Groups

\textbf{What working groups would help distribute leadership?}

\begin{quote}\small\ttfamily
| Working Group | Purpose | Potential Lead | Current Status |\\
\hrule
| Core Development | | | |\\
| Documentation | | | |\\
| Community | | | |\\
| Outreach/Partnerships | | | |\\
| Education | | | |\\
| | | | |\\
\end{quote}

\bigskip\noindent\rule{\textwidth}{0.4pt}\bigskip

\# Part 7: Strategic Priorities

\textit{Time to synthesize. What are the MOST IMPORTANT actions based on this analysis?}

\#\# 7.1 Force-Ranked Priorities

Look back at everything above. What are the TOP 5 priorities for the next 6 months?

\begin{quote}\small\ttfamily
| Priority | Why this matters | First concrete step | Who owns this? |\\
\hrule
| 1. | | | |\\
| 2. | | | |\\
| 3. | | | |\\
| 4. | | | |\\
| 5. | | | |\\
\end{quote}

\#\# 7.2 The One Thing

If you could only do ONE thing from this worksheet, what would have the biggest impact?

```
\_\_\_\_\_\_\_\_\_\_\_\_\_\_\_\_\_\_\_\_\_\_\_\_\_\_\_\_\_\_\_\_\_\_\_\_\_\_\_\_\_\_\_\_\_\_\_\_\_\_\_\_\_\_\_\_\_\_\_\_\_\_\_\_\_\_\_\_\_\_\_\_\_\_\_\_\_
\_\_\_\_\_\_\_\_\_\_\_\_\_\_\_\_\_\_\_\_\_\_\_\_\_\_\_\_\_\_\_\_\_\_\_\_\_\_\_\_\_\_\_\_\_\_\_\_\_\_\_\_\_\_\_\_\_\_\_\_\_\_\_\_\_\_\_\_\_\_\_\_\_\_\_\_\_
```

\textbf{Why?}

```
\_\_\_\_\_\_\_\_\_\_\_\_\_\_\_\_\_\_\_\_\_\_\_\_\_\_\_\_\_\_\_\_\_\_\_\_\_\_\_\_\_\_\_\_\_\_\_\_\_\_\_\_\_\_\_\_\_\_\_\_\_\_\_\_\_\_\_\_\_\_\_\_\_\_\_\_\_
\_\_\_\_\_\_\_\_\_\_\_\_\_\_\_\_\_\_\_\_\_\_\_\_\_\_\_\_\_\_\_\_\_\_\_\_\_\_\_\_\_\_\_\_\_\_\_\_\_\_\_\_\_\_\_\_\_\_\_\_\_\_\_\_\_\_\_\_\_\_\_\_\_\_\_\_\_
```

\textbf{What would "success" look like in 6 months?}

```
\_\_\_\_\_\_\_\_\_\_\_\_\_\_\_\_\_\_\_\_\_\_\_\_\_\_\_\_\_\_\_\_\_\_\_\_\_\_\_\_\_\_\_\_\_\_\_\_\_\_\_\_\_\_\_\_\_\_\_\_\_\_\_\_\_\_\_\_\_\_\_\_\_\_\_\_\_
\_\_\_\_\_\_\_\_\_\_\_\_\_\_\_\_\_\_\_\_\_\_\_\_\_\_\_\_\_\_\_\_\_\_\_\_\_\_\_\_\_\_\_\_\_\_\_\_\_\_\_\_\_\_\_\_\_\_\_\_\_\_\_\_\_\_\_\_\_\_\_\_\_\_\_\_\_
```

\#\# 7.3 What You're NOT Going to Do

\textit{Strategy is as much about what you don't do as what you do.}

\textbf{What should FreeMoCap explicitly DEPRIORITIZE for now?}

```
1. \_\_\_\_\_\_\_\_\_\_\_\_\_\_\_\_\_\_\_\_\_\_\_\_\_\_\_\_\_\_\_\_ (Why: \_\_\_\_\_\_\_\_\_\_\_\_\_\_\_\_\_\_\_\_\_\_\_\_\_\_\_\_\_\_\_)
2. \_\_\_\_\_\_\_\_\_\_\_\_\_\_\_\_\_\_\_\_\_\_\_\_\_\_\_\_\_\_\_\_ (Why: \_\_\_\_\_\_\_\_\_\_\_\_\_\_\_\_\_\_\_\_\_\_\_\_\_\_\_\_\_\_\_)
3. \_\_\_\_\_\_\_\_\_\_\_\_\_\_\_\_\_\_\_\_\_\_\_\_\_\_\_\_\_\_\_\_ (Why: \_\_\_\_\_\_\_\_\_\_\_\_\_\_\_\_\_\_\_\_\_\_\_\_\_\_\_\_\_\_\_)
```

\bigskip\noindent\rule{\textwidth}{0.4pt}\bigskip

\# Part 8: Personal Reflection (For Core Team Members)

\textit{Smucker asks: "How many times had I favored a particular action because I really thought it was likely to change a decision-maker's position or win over key allies, as opposed to gravitating toward an action because it expressed my activist identity?"}

\#\# 8.1 Expressive vs. Strategic

\textbf{Be honest with yourself. In the last month, how much of your FreeMoCap work was:}

\begin{quote}\small\ttfamily
| Category | \% of time |\\
\hrule
| Work that directly advances the mission (growing users, improving accessibility, building power) | |\\
| Work that feels good but isn't strategically necessary (perfecting code, interesting technical problems, internal community stuff) | |\\
| Maintenance/necessary overhead | |\\
\end{quote}

\textbf{What's one thing you do because it expresses your identity as [developer/academic/open source person] rather than because it advances FreeMoCap's mission?}

```
\_\_\_\_\_\_\_\_\_\_\_\_\_\_\_\_\_\_\_\_\_\_\_\_\_\_\_\_\_\_\_\_\_\_\_\_\_\_\_\_\_\_\_\_\_\_\_\_\_\_\_\_\_\_\_\_\_\_\_\_\_\_\_\_\_\_\_\_\_\_\_\_\_\_\_\_\_
\_\_\_\_\_\_\_\_\_\_\_\_\_\_\_\_\_\_\_\_\_\_\_\_\_\_\_\_\_\_\_\_\_\_\_\_\_\_\_\_\_\_\_\_\_\_\_\_\_\_\_\_\_\_\_\_\_\_\_\_\_\_\_\_\_\_\_\_\_\_\_\_\_\_\_\_\_
```

\#\# 8.2 Your Role

\textbf{What is your unique contribution to FreeMoCap?} (What would be lost if you left?)

```
\_\_\_\_\_\_\_\_\_\_\_\_\_\_\_\_\_\_\_\_\_\_\_\_\_\_\_\_\_\_\_\_\_\_\_\_\_\_\_\_\_\_\_\_\_\_\_\_\_\_\_\_\_\_\_\_\_\_\_\_\_\_\_\_\_\_\_\_\_\_\_\_\_\_\_\_\_
\_\_\_\_\_\_\_\_\_\_\_\_\_\_\_\_\_\_\_\_\_\_\_\_\_\_\_\_\_\_\_\_\_\_\_\_\_\_\_\_\_\_\_\_\_\_\_\_\_\_\_\_\_\_\_\_\_\_\_\_\_\_\_\_\_\_\_\_\_\_\_\_\_\_\_\_\_
```

\textbf{What do you do that SOMEONE ELSE could/should be doing?} (What should you delegate to grow others?)

```
\_\_\_\_\_\_\_\_\_\_\_\_\_\_\_\_\_\_\_\_\_\_\_\_\_\_\_\_\_\_\_\_\_\_\_\_\_\_\_\_\_\_\_\_\_\_\_\_\_\_\_\_\_\_\_\_\_\_\_\_\_\_\_\_\_\_\_\_\_\_\_\_\_\_\_\_\_
\_\_\_\_\_\_\_\_\_\_\_\_\_\_\_\_\_\_\_\_\_\_\_\_\_\_\_\_\_\_\_\_\_\_\_\_\_\_\_\_\_\_\_\_\_\_\_\_\_\_\_\_\_\_\_\_\_\_\_\_\_\_\_\_\_\_\_\_\_\_\_\_\_\_\_\_\_
```

\textbf{What's burning you out?} (What's unsustainable about your current involvement?)

```
\_\_\_\_\_\_\_\_\_\_\_\_\_\_\_\_\_\_\_\_\_\_\_\_\_\_\_\_\_\_\_\_\_\_\_\_\_\_\_\_\_\_\_\_\_\_\_\_\_\_\_\_\_\_\_\_\_\_\_\_\_\_\_\_\_\_\_\_\_\_\_\_\_\_\_\_\_
\_\_\_\_\_\_\_\_\_\_\_\_\_\_\_\_\_\_\_\_\_\_\_\_\_\_\_\_\_\_\_\_\_\_\_\_\_\_\_\_\_\_\_\_\_\_\_\_\_\_\_\_\_\_\_\_\_\_\_\_\_\_\_\_\_\_\_\_\_\_\_\_\_\_\_\_\_
```

\#\# 8.3 Development

\textbf{What skill or capacity do you need to develop to be more effective?}

```
\_\_\_\_\_\_\_\_\_\_\_\_\_\_\_\_\_\_\_\_\_\_\_\_\_\_\_\_\_\_\_\_\_\_\_\_\_\_\_\_\_\_\_\_\_\_\_\_\_\_\_\_\_\_\_\_\_\_\_\_\_\_\_\_\_\_\_\_\_\_\_\_\_\_\_\_\_
```

\textbf{Who could help you develop this?}

```
\_\_\_\_\_\_\_\_\_\_\_\_\_\_\_\_\_\_\_\_\_\_\_\_\_\_\_\_\_\_\_\_\_\_\_\_\_\_\_\_\_\_\_\_\_\_\_\_\_\_\_\_\_\_\_\_\_\_\_\_\_\_\_\_\_\_\_\_\_\_\_\_\_\_\_\_\_
```

\bigskip\noindent\rule{\textwidth}{0.4pt}\bigskip

\# Part 9: Reading and Learning Plan

\#\# 9.1 Core Reading List

\begin{quote}\small\ttfamily
| Book | Key Concepts | Priority | Status |\\
\hrule
| Smucker, "Hegemony How-To" | Political identity paradox, strategic vs. expressive | ⭐⭐⭐ | ✅ |\\
| McAlevey, "No Shortcuts" | Organizing vs. mobilizing, organic leaders | ⭐⭐⭐ | |\\
| Eghbal, "Working in Public" | Open source sustainability, community dynamics | ⭐⭐⭐ | |\\
| brown, "Emergent Strategy" | Adaptive organizing, fractal structure | ⭐⭐ | |\\
| Ganz, "What Is Public Narrative" (article) | Story of Self/Us/Now | ⭐⭐ | |\\
| Ransby, "Ella Baker and the Black Freedom Movement" | Group-centered leadership | ⭐⭐ | |\\
| Freire, "Pedagogy of the Oppressed" | Critical consciousness, participatory methods | ⭐ | |\\
| Benkler, "The Wealth of Networks" | Commons-based peer production | ⭐ | |\\
\end{quote}

\#\# 9.2 Learning Questions

\textbf{As you read, consider:}

1. What assumptions am I making about how change happens that this challenges?
2. What would [author] say about how FreeMoCap is currently operating?
3. What's one concrete thing I could do differently based on this?

\textbf{Keep a learning log. After each book, write:}

```
Title: \_\_\_\_\_\_\_\_\_\_\_\_\_\_\_\_\_\_\_\_\_\_\_\_\_\_\_\_\_\_\_\_
Date completed: \_\_\_\_\_\_\_\_\_\_\_\_\_\_\_\_\_\_\_\_\_\_\_\_\_\_\_\_\_\_\_\_

Key insight for FreeMoCap:
\_\_\_\_\_\_\_\_\_\_\_\_\_\_\_\_\_\_\_\_\_\_\_\_\_\_\_\_\_\_\_\_\_\_\_\_\_\_\_\_\_\_\_\_\_\_\_\_\_\_\_\_\_\_\_\_\_\_\_\_\_\_\_\_\_\_\_\_\_\_\_\_\_\_\_\_\_
\_\_\_\_\_\_\_\_\_\_\_\_\_\_\_\_\_\_\_\_\_\_\_\_\_\_\_\_\_\_\_\_\_\_\_\_\_\_\_\_\_\_\_\_\_\_\_\_\_\_\_\_\_\_\_\_\_\_\_\_\_\_\_\_\_\_\_\_\_\_\_\_\_\_\_\_\_

One thing I will do differently:
\_\_\_\_\_\_\_\_\_\_\_\_\_\_\_\_\_\_\_\_\_\_\_\_\_\_\_\_\_\_\_\_\_\_\_\_\_\_\_\_\_\_\_\_\_\_\_\_\_\_\_\_\_\_\_\_\_\_\_\_\_\_\_\_\_\_\_\_\_\_\_\_\_\_\_\_\_
\_\_\_\_\_\_\_\_\_\_\_\_\_\_\_\_\_\_\_\_\_\_\_\_\_\_\_\_\_\_\_\_\_\_\_\_\_\_\_\_\_\_\_\_\_\_\_\_\_\_\_\_\_\_\_\_\_\_\_\_\_\_\_\_\_\_\_\_\_\_\_\_\_\_\_\_\_

Question this raised:
\_\_\_\_\_\_\_\_\_\_\_\_\_\_\_\_\_\_\_\_\_\_\_\_\_\_\_\_\_\_\_\_\_\_\_\_\_\_\_\_\_\_\_\_\_\_\_\_\_\_\_\_\_\_\_\_\_\_\_\_\_\_\_\_\_\_\_\_\_\_\_\_\_\_\_\_\_
\_\_\_\_\_\_\_\_\_\_\_\_\_\_\_\_\_\_\_\_\_\_\_\_\_\_\_\_\_\_\_\_\_\_\_\_\_\_\_\_\_\_\_\_\_\_\_\_\_\_\_\_\_\_\_\_\_\_\_\_\_\_\_\_\_\_\_\_\_\_\_\_\_\_\_\_\_
```

\bigskip\noindent\rule{\textwidth}{0.4pt}\bigskip

\# Part 10: Accountability and Review

\#\# 10.1 Commitments

\textbf{Based on this worksheet, I commit to:}

\begin{quote}\small\ttfamily
| Commitment | Deadline | How I'll know it's done |\\
\hrule
| | | |\\
| | | |\\
| | | |\\
\end{quote}

\#\# 10.2 Review Schedule

\textbf{Schedule these now:}

- [ ] 30-day check-in: \_\_\_\_\_\_\_\_\_\_\_\_\_ (date)
- [ ] 90-day deep review: \_\_\_\_\_\_\_\_\_\_\_\_\_ (date)
- [ ] 6-month strategic review: \_\_\_\_\_\_\_\_\_\_\_\_\_ (date)

\#\# 10.3 Who Will Hold You Accountable?

\textbf{Share this worksheet with:}

```
1. \_\_\_\_\_\_\_\_\_\_\_\_\_\_\_\_\_\_\_\_\_\_\_\_\_\_\_\_\_\_\_\_
2. \_\_\_\_\_\_\_\_\_\_\_\_\_\_\_\_\_\_\_\_\_\_\_\_\_\_\_\_\_\_\_\_
```

\textbf{Ask them to check in with you on:} \_\_\_\_\_\_\_\_\_\_\_\_\_ (date)

\bigskip\noindent\rule{\textwidth}{0.4pt}\bigskip

\# Appendix: Quick Reference Concepts

\#\# Smucker's Key Terms

\begin{quote}\small\ttfamily
| Term | Definition | Application to FreeMoCap |\\
\hrule
| \textbf{Hegemony} | Making your values become the "common sense" of society | FreeMoCap aims to make "motion capture should be free/accessible" into common sense |\\
| \textbf{Political Identity Paradox} | Strong group identity enables commitment but creates insularity | Balance community cohesion with outward focus |\\
| \textbf{Bonding vs. Bridging} | Internal cohesion vs. external connection | Need both: strong core AND connections to new constituencies |\\
| \textbf{Expressive vs. Strategic} | Acting to express identity vs. acting to achieve outcomes | Prioritize actions that advance mission over actions that feel good |\\
| \textbf{Core and Base} | Committed leaders vs. broader supporters | Build both layers intentionally |\\
\end{quote}

\#\# McAlevey's Organizing Model

\begin{quote}\small\ttfamily
| Concept | Definition | Application |\\
\hrule
| \textbf{Mobilizing} | Activating existing supporters, staff-directed | Getting current community to take action |\\
| \textbf{Organizing} | Expanding base by reaching non-activists | Reaching people who don't know FreeMoCap exists |\\
| \textbf{Organic Leaders} | Respected figures within target constituencies | Find respected people in biomechanics, PT, animation, etc. |\\
| \textbf{Structure Tests} | Actions that reveal organizing capacity | Can you get N people to do X by date Y? |\\
\end{quote}

\#\# Ganz's Public Narrative

\begin{quote}\small\ttfamily
| Element | Purpose | Key Question |\\
\hrule
| \textbf{Story of Self} | Establish credibility through values | Why YOU? What choice led you here? |\\
| \textbf{Story of Us} | Build collective identity | Who is WE? What binds us? |\\
| \textbf{Story of Now} | Create urgency for action | Why NOW? What's the ask? |\\
\end{quote}

\bigskip\noindent\rule{\textwidth}{0.4pt}\bigskip

\textit{This worksheet is a living document. Return to it. Revise your answers. Share it with your team. The work of building a movement is never finished—but it must be started.}
\normalsize


\chapter{Movement-Building Worksheet (Version 2)}
\label{src:WS2}

\begin{framed}
\small
\textbf{Source marker:} WS2 \\
\textbf{Date:} February 2026 \\
\textbf{Source:} Expanded worksheet
\end{framed}

\small
// ============================================================================
// FREEMOCAP COUNTER-HEGEMONIC PROJECT WORKSHEET
// A Strategic Framework for Liberating Knowledge, Education, and Science
// ============================================================================

\#set document(
  title: "FreeMoCap as Counter-Hegemonic Project",
  author: "FreeMoCap Foundation",
)

\#set page(
  paper: "us-letter",
  margin: (x: 0.75in, y: 0.75in),
  numbering: "1",
  number-align: center,
  header: context \{
    if counter(page).get().first() > 1 [
      \#set text(8pt, fill: rgb("\#666"))
      \#emph[FreeMoCap Movement-Building Worksheet]
      \#h(1fr)
      \#emph[Liberating Knowledge Production]
    ]
  \}
)

\#set text(
  font: "New Computer Modern",
  size: 9.5pt,
  hyphenate: true,
)

\#set par(
  justify: true,
  leading: 0.6em,
)

\#set heading(numbering: "1.1")

\#show heading.where(level: 1): it => \{
  pagebreak(weak: true)
  v(0.3em)
  block(
    stroke: (bottom: 2pt + rgb("\#e94560")),
    inset: (bottom: 8pt),
    width: 100\%,
    [
      \#text(fill: rgb("\#1a1a2e"), weight: "bold", size: 14pt)[\#it.body]
    ]
  )
  v(0.4em)
\}

\#show heading.where(level: 2): it => \{
  v(0.6em)
  block(
    stroke: (left: 3pt + rgb("\#e94560")),
    inset: (left: 10pt, y: 6pt),
    [
      \#text(fill: rgb("\#1a1a2e"), weight: "bold", size: 11pt)[\#it.body]
    ]
  )
  v(0.3em)
\}

\#show heading.where(level: 3): it => \{
  v(0.4em)
  text(weight: "bold", size: 10pt, fill: rgb("\#0f3460"))[\#it.body]
  v(0.2em)
\}

// Custom components
\#let epigraph(quote, author) = \{
  block(
    width: 100\%,
    inset: (x: 1.5em, y: 0.6em),
    [
      \#set text(style: "italic", size: 9pt, fill: rgb("\#555"))
      \#quote
      \#h(1fr)
      — \#author
    ]
  )
\}

\#let callout(title, body, accent: rgb("\#e94560")) = \{
  block(
    width: 100\%,
    fill: accent.lighten(92\%),
    stroke: (left: 3pt + accent),
    inset: 10pt,
    radius: (right: 3pt),
    [
      \#text(weight: "bold", size: 9pt, fill: accent.darken(20\%))[\#title]
      \#if body != [] [
        \#v(0.2em)
        \#text(size: 9pt)[\#body]
      ]
    ]
  )
\}

\#let worksheet-field(prompt, lines: 3) = \{
  block(
    width: 100\%,
    inset: (y: 0.4em),
    [
      \#text(weight: "medium", size: 9pt, fill: rgb("\#333"))[\#prompt]
      \#v(0.2em)
      \#block(
        width: 100\%,
        height: lines \textit{ 1.2em,
        fill: rgb("\#fafafa"),
        stroke: 0.75pt + rgb("\#ddd"),
        radius: 3pt,
        inset: 6pt,
        []
      )
    ]
  )
\}

\#let completion-prompt(start) = \{
  block(
    width: 100\%,
    inset: (y: 0.25em),
    [
      \#text(size: 9pt)[
        \#box(
          fill: rgb("\#f0f0f0"),
          inset: (x: 6pt, y: 3pt),
          radius: 2pt,
          [\#text(style: "italic")[\#start]]
        )
        \#box(
          width: 1fr,
          stroke: (bottom: 0.75pt + rgb("\#bbb")),
          inset: (bottom: 2pt),
          []
        )
      ]
    ]
  )
\}

\#let resource-link(title, url, description) = \{
  block(
    width: 100\%,
    inset: (y: 0.2em),
    [
      \#text(size: 8.5pt)[
        \#link(url)[\#text(fill: rgb("\#0066cc"), weight: "medium")[\#title]]
        \#h(0.3em)
        \#text(fill: rgb("\#777"))[— \#description]
      ]
    ]
  )
\}

// ============================================================================
// TITLE PAGE
// ============================================================================

\#align(center)[
  \#v(1.5fr)
  
  \#block(
    width: 85\%,
    [
      \#text(size: 24pt, weight: "bold", fill: rgb("\#1a1a2e"))[
        FreeMoCap as Counter-Hegemonic Project
      ]
      
      \#v(0.6em)
      
      \#line(length: 50\%, stroke: 1.5pt + rgb("\#e94560"))
      
      \#v(0.6em)
      
      \#text(size: 12pt, fill: rgb("\#0f3460"))[
        A Strategic Worksheet for Liberating \textbackslash{} Knowledge, Education, and Science
      ]
    ]
  )
  
  \#v(1.5em)
  
  \#block(
    width: 75\%,
    fill: rgb("\#f8f8f8"),
    stroke: 0.75pt + rgb("\#ddd"),
    inset: 16pt,
    radius: 6pt,
    [
      \#set text(size: 9pt, style: "italic")
      \#set par(leading: 0.7em)
      
      "The great revolution in the history of man, past, present and future, is the revolution of those determined to be free."
      \#align(right)[— }Toussaint Louverture\textit{]
      
      \#v(0.8em)
      
      "Information is power. But like all power, there are those who want to keep it for themselves."
      \#align(right)[— }Aaron Swartz\textit{]
      
      \#v(0.8em)
      
      \#set text(style: "normal", size: 8.5pt)
      }On tools:\textit{ The Haitian revolutionaries got their muskets from the French. They got their military tactics from the French. They got the language of universal liberty from the French Revolution — then held France accountable to ideals France never intended for them. Tools are not ideologically fixed to their origins. The code doesn't know who's running it. }Appropriate everything.\textit{
    ]
  )
  
  \#v(2fr)
  
  \#text(size: 8.5pt, fill: rgb("\#666"))[
    Based on Jonathan Smucker's \_Hegemony How-To\_ and political organizing theory \textbackslash{}
    Prepared for the FreeMoCap Foundation
  ]
  
  \#v(1fr)
]

// ============================================================================
// TABLE OF CONTENTS
// ============================================================================

\#pagebreak()

\#align(center)[
  \#text(size: 14pt, weight: "bold")[Contents]
]

\#v(0.8em)

\#outline(
  title: none,
  indent: 1.2em,
  depth: 2,
)

\#v(1.5em)

\#callout(
  "How to Use This Worksheet",
  [
    + }Individual first pass:\textit{ Each core team member completes sections independently
    + }Compare and discuss:\textit{ Share responses and identify patterns, disagreements, tensions  
    + }Synthesize:\textit{ Develop shared answers that represent collective understanding
    + }Revisit quarterly:\textit{ These questions aren't "solved" once — return as the project evolves
    
    Budget 2-3 hours for serious engagement with each major section.
  ],
  accent: rgb("\#0f3460")
)

\#v(1em)

\#block(
  fill: rgb("\#fff8e6"),
  stroke: (left: 3pt + rgb("\#e6a700")),
  inset: 10pt,
  radius: (right: 3pt),
  [
    \#text(weight: "bold", size: 9pt, fill: rgb("\#996600"))[Framing: What Are We Actually Doing?]
    \#v(0.3em)
    \#text(size: 8.5pt)[
      FreeMoCap the motion capture software is }Stage 1\textit{ — a concrete demonstration that research-grade scientific tools can be free and accessible, that you don't need a university affiliation or \textbackslash{}\$200,000 budget to produce knowledge, and that the barriers are artificial.
      
      \#v(0.3em)
      
      The }actual project\textit{ is much larger: dismantling the current hegemony of knowledge production — the interlocking system of universities, journals, proprietary software, credentialism, and gatekeeping that determines who gets to participate in science, education, and knowledge creation.
    ]
  ]
)

// ============================================================================
// PART 1: MAPPING THE ENEMY
// ============================================================================

= Mapping the Enemy — The Current Hegemony

\#epigraph(
  "Before you can overturn a hegemony, you need to understand it clearly. What are the structures you're fighting? How do they maintain power? What ideology justifies them?",
  "Strategic Principle"
)

== The Institutional Landscape

Map the interlocking institutions that control knowledge production:

\#v(0.3em)

\#table(
  columns: (1.1fr, 1.1fr, 1.4fr, 1.4fr),
  inset: 6pt,
  align: (left, left, left, left),
  stroke: 0.5pt + rgb("\#ccc"),
  fill: (\_, row) => if row == 0 \{ rgb("\#f0f0f0") \} else \{ white \},
  
  [\#text(weight: "bold", size: 8pt)[Institution]], 
  [\#text(weight: "bold", size: 8pt)[What it Controls]], 
  [\#text(weight: "bold", size: 8pt)[How it Maintains Control]], 
  [\#text(weight: "bold", size: 8pt)[Justifying Ideology]],
  
  text(size: 8pt)[}Universities\textit{], 
  text(size: 8pt)[Access to education, research resources, credentials], 
  text(size: 8pt)[Accreditation, degree requirements, tenure system], 
  text(size: 8pt, style: "italic")["Expertise requires formal training"],
  
  text(size: 8pt)[}Academic Journals\textit{], 
  text(size: 8pt)[What counts as "real" knowledge], 
  text(size: 8pt)[Peer review gatekeeping, impact factors, paywalls], 
  text(size: 8pt, style: "italic")["Peer review ensures quality"],
  
  text(size: 8pt)[}Proprietary Software\textit{], 
  text(size: 8pt)[Tools for knowledge production], 
  text(size: 8pt)[Licensing, prices, closed formats], 
  text(size: 8pt, style: "italic")["Professional tools cost money"],
  
  text(size: 8pt)[}Professional Associations\textit{], 
  text(size: 8pt)[Who counts as "qualified"], 
  text(size: 8pt)[Certifications, ethical codes as gatekeeping], 
  text(size: 8pt, style: "italic")["Credentials protect the public"],
  
  text(size: 8pt)[}Funding Agencies\textit{], 
  text(size: 8pt)[What research gets done], 
  text(size: 8pt)[Grant systems, overhead rates], 
  text(size: 8pt, style: "italic")["Competitive funding ensures quality"],
)

\#v(0.5em)

}Add others you identify:\textit{

\#table(
  columns: (1fr, 1fr, 1fr, 1fr),
  inset: 6pt,
  stroke: 0.5pt + rgb("\#ccc"),
  rows: (auto, 1.8em, 1.8em),
  fill: (\_, row) => if row == 0 \{ rgb("\#f5f5f5") \} else \{ rgb("\#fafafa") \},
  
  text(size: 8pt)[Institution], text(size: 8pt)[Controls], text(size: 8pt)[Maintains Control], text(size: 8pt)[Ideology],
  [], [], [], [],
  [], [], [], [],
)

== The Hegemonic Common Sense

\#callout(
  "Key Insight",
  [Hegemony works by making contingent arrangements seem natural and inevitable. What do people currently believe that serves this system?],
  accent: rgb("\#e94560")
)

}Complete these sentences as a "normal person" embedded in current structures would:\textit{

\#completion-prompt("To be a real scientist, you need")
\#completion-prompt("Research isn't valid unless")
\#completion-prompt("Education requires")
\#completion-prompt("If software/journals/education were free,")
\#completion-prompt("People outside institutions can't do real research because")

\#v(0.5em)

\#worksheet-field("Where did these beliefs come from? Who benefits from people believing them?", lines: 3)

== Cracks in the Edifice

No hegemony is total. Where is the current system failing?

\#table(
  columns: (1.2fr, 1.2fr, 1.5fr),
  inset: 6pt,
  stroke: 0.5pt + rgb("\#ccc"),
  fill: (\_, row) => if row == 0 \{ rgb("\#f0f0f0") \} else \{ white \},
  
  text(size: 8pt, weight: "bold")[System Failure], 
  text(size: 8pt, weight: "bold")[Who Experiences This?], 
  text(size: 8pt, weight: "bold")[What Do They Currently Blame?],
  
  text(size: 8pt)[Reproducibility crisis], text(size: 8pt)[Researchers, public], text(size: 8pt, style: "italic")["Bad actors," not systemic incentives],
  text(size: 8pt)[Student debt crisis], text(size: 8pt)[Students, families], text(size: 8pt, style: "italic")["Expensive schools," not credentialism itself],
  text(size: 8pt)[Journal paywalls], text(size: 8pt)[Everyone outside elite institutions], text(size: 8pt, style: "italic")["Greedy publishers," not the publication model],
  text(size: 8pt)[Adjunctification], text(size: 8pt)[Early-career academics], text(size: 8pt, style: "italic")["Budget cuts," not the tenure system],
  text(size: 8pt)[Proprietary lock-in], text(size: 8pt)[Researchers, educators], text(size: 8pt, style: "italic")[Specific vendors, not proprietary model],
)

\#v(0.3em)

\#worksheet-field("How could you help people see the SYSTEMIC cause rather than blaming individuals?", lines: 3)

== Who Benefits, Who Loses

\#columns(2, gutter: 1em)[
  \#block(
    fill: rgb("\#f0fff0"),
    stroke: 0.5pt + rgb("\#9c9"),
    inset: 8pt,
    radius: 3pt,
    width: 100\%,
    [
      \#text(size: 8.5pt, weight: "bold")[Who benefits?]
      \#v(0.2em)
      \#text(size: 8pt)[
        - Tenured faculty at elite institutions
        - Academic publishers
        - Proprietary software companies
        - \#box(width: 1fr, stroke: (bottom: 0.5pt + rgb("\#ccc")), [])
      ]
    ]
  )
  
  \#colbreak()
  
  \#block(
    fill: rgb("\#fff0f0"),
    stroke: 0.5pt + rgb("\#c99"),
    inset: 8pt,
    radius: 3pt,
    width: 100\%,
    [
      \#text(size: 8.5pt, weight: "bold")[Who loses?]
      \#v(0.2em)
      \#text(size: 8pt)[
        - Students (debt, limited access)
        - Adjuncts and contingent faculty
        - Global South academics
        - Independent researchers
        - The public
      ]
    ]
  )
]

\#v(0.5em)

\#callout(
  "Critical Question: The people who lose vastly outnumber those who benefit. Why hasn't the system already changed?",
  [],
  accent: rgb("\#e94560")
)

\#worksheet-field("Your analysis:", lines: 3)

// ============================================================================
// PART 2: THE COUNTER-HEGEMONIC VISION
// ============================================================================

= The Counter-Hegemonic Vision

\#epigraph(
  "What new 'common sense' are you trying to create? What does the liberated landscape look like?",
  "Strategic Question"
)

== The World We're Building

}Complete these sentences as someone living in the world FreeMoCap is building:\textit{

\#completion-prompt("Anyone can do real science because")
\#completion-prompt("I learned [skill/knowledge] by")
\#completion-prompt("When I need research tools, I")
\#completion-prompt("Knowledge is validated by")
\#completion-prompt("Education happens")

== Core Principles of the New Hegemony

\#block(
  fill: rgb("\#f8f8f8"),
  stroke: 0.5pt + rgb("\#ddd"),
  inset: 12pt,
  radius: 4pt,
  width: 100\%,
  [
    \#text(size: 9pt, weight: "bold")[Draft 5-7 core principles of the world you're building:]
    
    \#v(0.4em)
    
    \#for i in range(1, 8) \{
      text(size: 8.5pt)[\#i. \#box(width: 1fr, stroke: (bottom: 0.5pt + rgb("\#ccc")), inset: (bottom: 3pt), [])]
      v(0.7em)
    \}
  ]
)

== Prefigurative Politics

\#callout(
  "Definition",
  [}Prefigurative politics\textit{ means embodying the future you want to create in how you organize now. FreeMoCap should BE the proof that another way is possible.],
  accent: rgb("\#0f3460")
)

\#worksheet-field("How does FreeMoCap currently embody these principles?", lines: 2)

\#worksheet-field("Where does FreeMoCap fall short of its own principles? (Be honest)", lines: 2)

// ============================================================================
// PART 3: THEORY OF CHANGE
// ============================================================================

= Theory of Change — How Hegemonies Fall

\#epigraph(
  "Smucker is clear: being right isn't enough. You need a plausible theory of how the current system gets replaced by yours.",
  "Strategic Principle"
)

== Historical Models

\#table(
  columns: (1fr, 1.2fr, 1.5fr, 1.3fr),
  inset: 5pt,
  stroke: 0.5pt + rgb("\#ccc"),
  fill: (\_, row) => if row == 0 \{ rgb("\#f0f0f0") \} else \{ white \},
  
  text(size: 8pt, weight: "bold")[Movement], 
  text(size: 8pt, weight: "bold")[Hegemony Challenged], 
  text(size: 8pt, weight: "bold")[How They Won], 
  text(size: 8pt, weight: "bold")[Lessons],
  
  text(size: 8pt)[Free Software], 
  text(size: 8pt)[Proprietary software], 
  text(size: 8pt)[Built alternatives, licensing infrastructure, changed culture],
  [],
  
  text(size: 8pt)[Open Access], 
  text(size: 8pt)[Journal paywalls], 
  text(size: 8pt)[Mandates, repositories, Sci-Hub],
  [],
  
  text(size: 8pt)[Wikipedia], 
  text(size: 8pt)[Expert-only encyclopedias], 
  text(size: 8pt)[Just built it, proved it worked],
  [],
  
  text(size: 8pt)[Linux], 
  text(size: 8pt)[Microsoft hegemony], 
  text(size: 8pt)[Server dominance first, Android],
  [],
)

\#worksheet-field("What do successful counter-hegemonic projects have in common?", lines: 2)

== Your Theory of Change

\#block(
  fill: rgb("\#f5f5f5"),
  stroke: 0.5pt + rgb("\#ddd"),
  inset: 12pt,
  radius: 4pt,
  [
    \#text(size: 9pt)[Complete this theory of change:]
    
    \#v(0.4em)
    \#text(size: 8.5pt)[
      }If\textit{ FreeMoCap \#box(width: 1fr, stroke: (bottom: 0.5pt + rgb("\#aaa")), [])
      
      \#v(0.6em)
      
      }Then\textit{ [constituency] will \#box(width: 1fr, stroke: (bottom: 0.5pt + rgb("\#aaa")), [])
      
      \#v(0.6em)
      
      }Which will cause\textit{ \#box(width: 1fr, stroke: (bottom: 0.5pt + rgb("\#aaa")), [])
      
      \#v(0.6em)
      
      }Leading to\textit{ \#box(width: 1fr, stroke: (bottom: 0.5pt + rgb("\#aaa")), [])
      
      \#v(0.6em)
      
      }Until eventually\textit{ \#box(width: 1fr, stroke: (bottom: 0.5pt + rgb("\#aaa")), [])
    ]
  ]
)

== The Stages

If FreeMoCap motion capture is Stage 1, what are the subsequent stages?

\#table(
  columns: (0.4fr, 1fr, 1fr, 1.2fr, 1.2fr),
  inset: 5pt,
  stroke: 0.5pt + rgb("\#ccc"),
  fill: (\_, row) => if row == 0 \{ rgb("\#f0f0f0") \} else if row == 1 \{ rgb("\#f8f8f8") \} else \{ white \},
  
  text(size: 8pt, weight: "bold")[Stage], 
  text(size: 8pt, weight: "bold")[Focus], 
  text(size: 8pt, weight: "bold")[Goal], 
  text(size: 8pt, weight: "bold")[Success Looks Like], 
  text(size: 8pt, weight: "bold")[Enables Next Stage By...],
  
  text(size: 8pt)[}1\textit{], text(size: 8pt)[Motion capture], text(size: 8pt)[Prove FOSS can match proprietary], text(size: 8pt)[Adoption, citations], text(size: 8pt)[Builds credibility],
  text(size: 8pt)[}2\textit{], [], [], [], [],
  text(size: 8pt)[}3\textit{], [], [], [], [],
  text(size: 8pt)[}4\textit{], [], [], [], [],
  text(size: 8pt)[}5\textit{], [], [], [], [],
)

== Multiple Fronts

\#table(
  columns: (0.9fr, 1.2fr, 1.2fr, 0.9fr),
  inset: 5pt,
  stroke: 0.5pt + rgb("\#ccc"),
  fill: (\_, row) => if row == 0 \{ rgb("\#f0f0f0") \} else \{ white \},
  
  text(size: 8pt, weight: "bold")[Front], 
  text(size: 8pt, weight: "bold")[Victory Looks Like], 
  text(size: 8pt, weight: "bold")[Who's Fighting Here?], 
  text(size: 8pt, weight: "bold")[FreeMoCap's Role],
  
  text(size: 8pt)[}Tools\textit{ (FOSS)], text(size: 8pt)[FOSS default], text(size: 8pt)[NumPy, R, Jupyter], text(size: 8pt)[Active combatant],
  text(size: 8pt)[}Publishing\textit{], text(size: 8pt)[Paywalls gone], text(size: 8pt)[PLOS, arXiv, Sci-Hub], [],
  text(size: 8pt)[}Education\textit{], text(size: 8pt)[Degrees not required], text(size: 8pt)[MOOCs, bootcamps], [],
  text(size: 8pt)[}Funding\textit{], text(size: 8pt)[Alt funding models], text(size: 8pt)[Crowdfunding, citizen science], [],
)

\#worksheet-field("Where should FreeMoCap focus? Where build alliances?", lines: 2)

// ============================================================================
// PART 4: PUBLIC NARRATIVE
// ============================================================================

= The Public Narrative

\#epigraph(
  "Your narrative must serve the LARGER project, not just the motion capture software.",
  "Strategic Principle"
)

\#callout(
  "Marshall Ganz's Framework",
  [
    }Story of Self\textit{ → Why YOU? What choice led you here? \textbackslash{}
    }Story of Us\textit{ → Who is WE? What binds us together? \textbackslash{}
    }Story of Now\textit{ → Why NOW? What's the urgent ask?
  ],
  accent: rgb("\#0f3460")
)

== Story of Self

}Why are YOU, Jon, fighting to liberate knowledge production?\textit{

\#worksheet-field("The Encounter: When did you first experience the injustice of the current system?", lines: 3)

\#worksheet-field("The Choice: What moment committed you to this fight?", lines: 3)

\#worksheet-field("The Values: What deep values drive this? (Not 'open source is good' — the values underneath)", lines: 2)

\#block(
  fill: rgb("\#fffde7"),
  stroke: (left: 3pt + rgb("\#fbc02d")),
  inset: 10pt,
  radius: (right: 3pt),
  [
    \#text(weight: "bold", size: 8.5pt, fill: rgb("\#996600"))[Draft your Story of Self (3-4 sentences):]
    \#v(0.3em)
    \#block(height: 3em, width: 100\%, fill: white, stroke: 0.5pt + rgb("\#ddd"), radius: 3pt, [])
  ]
)

== Story of Us

}Who is the "we" fighting for liberated knowledge? Much bigger than "FreeMoCap users."\textit{

\#block(
  fill: rgb("\#f8f8f8"),
  stroke: 0.5pt + rgb("\#ddd"),
  inset: 10pt,
  radius: 3pt,
  [
    \#text(size: 8.5pt, weight: "bold")[The broadest possible "we" — who has been harmed/excluded?]
    \#text(size: 8pt)[
      - Students crushed by debt for credentials that gatekeep knowledge
      - Researchers at under-resourced institutions  
      - Global South academics locked out by paywalls
      - Independent researchers dismissed as "not real scientists"
      - Curious people told "you're not qualified"
      - \#box(width: 1fr, stroke: (bottom: 0.5pt + rgb("\#ccc")), [])
    ]
  ]
)

\#completion-prompt("The shared experience — what do all these people have in common?")
\#completion-prompt("The shared enemy (the SYSTEM, not 'universities'):")
\#completion-prompt("The shared vision — what world are 'we' building?")

== Story of Now

}Why is THIS MOMENT critical?\textit{

\#worksheet-field("What's happening NOW that creates urgency? (AI, reproducibility crisis, debt crisis, climate...)", lines: 3)

}The ask for different audiences:\textit{

\#table(
  columns: (1fr, 2fr),
  inset: 6pt,
  stroke: 0.5pt + rgb("\#ccc"),
  fill: (\_, row) => if row == 0 \{ rgb("\#f0f0f0") \} else \{ rgb("\#fafafa") \},
  
  text(size: 8pt, weight: "bold")[Audience], 
  text(size: 8pt, weight: "bold")[What Do You Want Them to DO?],
  
  text(size: 8pt)[Someone experiencing system failures], [],
  text(size: 8pt)[A potential contributor], [],
  text(size: 8pt)[A potential ally organization], [],
  text(size: 8pt)[Someone with resources], [],
)

== The Integrated Narrative

\#block(
  fill: rgb("\#f0f8f0"),
  stroke: (left: 3pt + rgb("\#4caf50")),
  inset: 12pt,
  radius: (right: 3pt),
  [
    \#text(size: 9pt, weight: "bold")[Weave them together. 3-5 minutes to tell. Practice out loud.]
    
    \#v(0.4em)
    
    \#text(size: 8.5pt)[}[Story of Self]\textit{]
    \#block(height: 2em, width: 100\%, fill: white, stroke: 0.5pt + rgb("\#ddd"), radius: 3pt, [])
    
    \#v(0.3em)
    \#align(center)[\#text(size: 8pt, style: "italic")["And I discovered I wasn't alone..."]]
    \#v(0.3em)
    
    \#text(size: 8.5pt)[}[Story of Us]\textit{]
    \#block(height: 2em, width: 100\%, fill: white, stroke: 0.5pt + rgb("\#ddd"), radius: 3pt, [])
    
    \#v(0.3em)
    \#align(center)[\#text(size: 8pt, style: "italic")["And right now, we face a critical moment..."]]
    \#v(0.3em)
    
    \#text(size: 8.5pt)[}[Story of Now + The Ask]\textit{]
    \#block(height: 2em, width: 100\%, fill: rgb("\#fffde7"), stroke: 0.5pt + rgb("\#fbc02d"), radius: 3pt, [])
  ]
)

// ============================================================================
// PART 5: ALLIANCE MAPPING
// ============================================================================

= Alliance Mapping

\#epigraph(
  "You're not fighting alone. Many movements share pieces of this vision.",
  "Strategic Principle"
)

== Allied Movements

\#table(
  columns: (0.9fr, 1.2fr, 1.2fr, 1.3fr),
  inset: 5pt,
  stroke: 0.5pt + rgb("\#ccc"),
  fill: (\_, row) => if row == 0 \{ rgb("\#f0f0f0") \} else \{ white \},
  
  text(size: 8pt, weight: "bold")[Movement], 
  text(size: 8pt, weight: "bold")[What They Fight For], 
  text(size: 8pt, weight: "bold")[Overlap], 
  text(size: 8pt, weight: "bold")[Key Organizations],
  
  text(size: 8pt)[Free Software], text(size: 8pt)[Software freedom], text(size: 8pt)[FOSS as default], text(size: 8pt)[FSF, SFLC],
  text(size: 8pt)[Open Access], text(size: 8pt)[Free research access], text(size: 8pt)[Against paywalls], text(size: 8pt)[SPARC, PLOS],
  text(size: 8pt)[Open Science], text(size: 8pt)[Reproducibility], text(size: 8pt)[Open tools/data], text(size: 8pt)[COS, FOSTER],
  text(size: 8pt)[Right to Repair], text(size: 8pt)[Tech ownership], text(size: 8pt)[Anti lock-in], text(size: 8pt)[iFixit],
  text(size: 8pt)[Citizen Science], text(size: 8pt)[Public participation], text(size: 8pt)[Outside institutions], text(size: 8pt)[SciStarter],
  text(size: 8pt)[Hacker Culture], text(size: 8pt)[Information freedom], text(size: 8pt)[Tool-building], text(size: 8pt)[Hackerspaces, CCC],
  text(size: 8pt)[Library Movement], text(size: 8pt)[Public access], text(size: 8pt)[Against enclosure], text(size: 8pt)[Internet Archive],
)

== Intellectual Lineages

\#table(
  columns: (1fr, 2fr, 1.5fr),
  inset: 5pt,
  stroke: 0.5pt + rgb("\#ccc"),
  fill: (\_, row) => if row == 0 \{ rgb("\#f0f0f0") \} else \{ white \},
  
  text(size: 8pt, weight: "bold")[Thinker], 
  text(size: 8pt, weight: "bold")[Key Ideas], 
  text(size: 8pt, weight: "bold")[Application to FreeMoCap],
  
  text(size: 8pt)[Ivan Illich], text(size: 8pt)[Deschooling, convivial tools], [],
  text(size: 8pt)[Aaron Swartz], text(size: 8pt)[Information liberation, guerrilla open access], [],
  text(size: 8pt)[Paulo Freire], text(size: 8pt)[Critical pedagogy, conscientização], [],
  text(size: 8pt)[Yochai Benkler], text(size: 8pt)[Commons-based peer production], [],
  text(size: 8pt)[Elinor Ostrom], text(size: 8pt)[Governance of commons], [],
)

== Building Alliances

}For top 5 potential allies:\textit{

\#table(
  columns: (0.6fr, 1.2fr, 1.2fr, 1.4fr),
  inset: 5pt,
  stroke: 0.5pt + rgb("\#ccc"),
  fill: (\_, row) => if row == 0 \{ rgb("\#f0f0f0") \} else \{ rgb("\#fafafa") \},
  
  text(size: 8pt, weight: "bold")[Ally], 
  text(size: 8pt, weight: "bold")[What You Offer], 
  text(size: 8pt, weight: "bold")[What You Need], 
  text(size: 8pt, weight: "bold")[First Step],
  
  text(size: 8pt)[1.], [], [], [],
  text(size: 8pt)[2.], [], [], [],
  text(size: 8pt)[3.], [], [], [],
  text(size: 8pt)[4.], [], [], [],
  text(size: 8pt)[5.], [], [], [],
)

// ============================================================================
// PART 6: ORGANIZING FOR THE LONG HAUL
// ============================================================================

= Organizing for the Long Haul

\#epigraph(
  "Counter-hegemonic projects take decades. How do you build an organization that can sustain this?",
  "Strategic Question"
)

== The FreeMoCap Foundation as Vehicle

\#callout(
  "Critical Question",
  [Is the FreeMoCap Foundation the right organizational vehicle for the larger vision? Should there be a broader entity that FreeMoCap is one project }within\textit{?],
  accent: rgb("\#e94560")
)

\#worksheet-field("Your thinking:", lines: 3)

== Developing Leadership

\#block(
  fill: rgb("\#f5f5f5"),
  stroke: 0.5pt + rgb("\#ddd"),
  inset: 10pt,
  radius: 3pt,
  [
    \#text(size: 8.5pt, style: "italic")["Strong people don't need a strong leader." — Ella Baker]
  ]
)

\#v(0.3em)

\#worksheet-field("What leadership roles will you need that don't exist yet?", lines: 2)

\#worksheet-field("How do people BECOME leaders in your organization? (Make this explicit)", lines: 2)

== Organizational Culture

\#columns(2, gutter: 1em)[
  \#text(size: 8.5pt, weight: "bold")[Values to cultivate:]
  \#text(size: 8pt)[
    - Radical accessibility
    - Distributed leadership
    - Long-term thinking
    - Solidarity with allies
    - Prefigurative practice
  ]
  
  \#colbreak()
  
  \#text(size: 8.5pt, weight: "bold")[Patterns to AVOID:]
  \#text(size: 8pt)[
    - Founder dependency
    - Insider/outsider dynamics
    - Technical elitism
    - Burnout/martyrdom culture
    - Purity politics
  ]
]

\#worksheet-field("Which risks are you most vulnerable to?", lines: 2)

// ============================================================================
// PART 7: STRATEGIC QUESTIONS
// ============================================================================

= Strategic Questions to Keep Asking

\#epigraph(
  "These aren't one-time questions. Return to them regularly.",
  "Ongoing Practice"
)

== The Smucker Questions (Ask Monthly)

\#block(
  fill: rgb("\#fff5f5"),
  stroke: (left: 3pt + rgb("\#e94560")),
  inset: 10pt,
  radius: (right: 3pt),
  [
    \#text(size: 8.5pt)[
      }"Is what we're doing likely to advance our mission, or does it primarily express our identity?"\textit{
      \#block(height: 1.8em, width: 100\%, fill: white, stroke: 0.5pt + rgb("\#ddd"), radius: 3pt, [])
      
      \#v(0.5em)
      
      }"Are we building power, or performing righteousness?"\textit{
      \#block(height: 1.8em, width: 100\%, fill: white, stroke: 0.5pt + rgb("\#ddd"), radius: 3pt, [])
      
      \#v(0.5em)
      
      }"Who are we bringing IN who wasn't here before?"\textit{
      \#block(height: 1.8em, width: 100\%, fill: white, stroke: 0.5pt + rgb("\#ddd"), radius: 3pt, [])
      
      \#v(0.5em)
      
      }"What have we WON recently?"\textit{
      \#block(height: 1.8em, width: 100\%, fill: white, stroke: 0.5pt + rgb("\#ddd"), radius: 3pt, [])
    ]
  ]
)

== The Sustainability Questions

\#block(
  fill: rgb("\#fffde7"),
  stroke: (left: 3pt + rgb("\#fbc02d")),
  inset: 10pt,
  radius: (right: 3pt),
  [
    \#text(size: 8.5pt)[
      }"Can this continue without Jon?"\textit{
      \#block(height: 1.5em, width: 100\%, fill: white, stroke: 0.5pt + rgb("\#ddd"), radius: 3pt, [])
      
      \#v(0.4em)
      
      }"What happens if a key contributor burns out?"\textit{
      \#block(height: 1.5em, width: 100\%, fill: white, stroke: 0.5pt + rgb("\#ddd"), radius: 3pt, [])
      
      \#v(0.4em)
      
      }"Are we moving at a sustainable pace?"\textit{
      \#block(height: 1.5em, width: 100\%, fill: white, stroke: 0.5pt + rgb("\#ddd"), radius: 3pt, [])
    ]
  ]
)

// ============================================================================
// PART 8: IMMEDIATE ACTIONS
// ============================================================================

= Immediate Actions

\#epigraph(
  "Strategy without action is fantasy. What are you doing THIS MONTH?",
  "Call to Action"
)

== Top Priorities for Next 90 Days

\#table(
  columns: (0.4fr, 1.5fr, 1.5fr, 0.7fr, 0.7fr),
  inset: 5pt,
  stroke: 0.5pt + rgb("\#ccc"),
  fill: (\_, row) => if row == 0 \{ rgb("\#f0f0f0") \} else \{ rgb("\#fafafa") \},
  
  text(size: 8pt, weight: "bold")[\textbackslash{}\#], 
  text(size: 8pt, weight: "bold")[Priority], 
  text(size: 8pt, weight: "bold")[First Concrete Step], 
  text(size: 8pt, weight: "bold")[Deadline], 
  text(size: 8pt, weight: "bold")[Owner],
  
  text(size: 8pt)[1], [], [], [], [],
  text(size: 8pt)[2], [], [], [], [],
  text(size: 8pt)[3], [], [], [], [],
)

== What You're NOT Doing

\#callout("Strategy means choosing. What are you explicitly deprioritizing?", [], accent: rgb("\#6c757d"))

\#text(size: 8.5pt)[
  1. \#box(width: 0.5fr, stroke: (bottom: 0.5pt + rgb("\#ccc")), []) \_(Why:\_ \#box(width: 0.35fr, stroke: (bottom: 0.5pt + rgb("\#ccc")), [])\_)\_
  
  2. \#box(width: 0.5fr, stroke: (bottom: 0.5pt + rgb("\#ccc")), []) \_(Why:\_ \#box(width: 0.35fr, stroke: (bottom: 0.5pt + rgb("\#ccc")), [])\_)\_
  
  3. \#box(width: 0.5fr, stroke: (bottom: 0.5pt + rgb("\#ccc")), []) \_(Why:\_ \#box(width: 0.35fr, stroke: (bottom: 0.5pt + rgb("\#ccc")), [])\_)\_
]

== Accountability

\#columns(2, gutter: 1em)[
  }Share this with:\textit{
  \#block(height: 1.5em, width: 100\%, fill: rgb("\#fafafa"), stroke: 0.5pt + rgb("\#ddd"), radius: 3pt, [])
  
  \#colbreak()
  
  }Review dates:\textit{
  \#text(size: 8pt)[
    30-day: \#box(width: 1fr, stroke: (bottom: 0.5pt + rgb("\#ccc")), []) \textbackslash{}
    90-day: \#box(width: 1fr, stroke: (bottom: 0.5pt + rgb("\#ccc")), [])
  ]
]

// ============================================================================
// PART 9: READING \& RESOURCES
// ============================================================================

= Reading \& Resources

== Core Reading

\#block(
  fill: rgb("\#f8f8f8"),
  stroke: 0.5pt + rgb("\#ddd"),
  inset: 12pt,
  radius: 4pt,
  [
    \#text(size: 9pt, weight: "bold")[Tier 1: Essential]
    
    \#resource-link(
      "Hegemony How-To: A Roadmap for Radicals",
      "https://www.akpress.org/hegemony-how-to.html",
      "Jonathan Smucker — Core strategic framework"
    )
    
    \#resource-link(
      "Deschooling Society",
      "https://www.arvindguptatoys.com/arvindgupta/DESCHOOLING.pdf",
      "Ivan Illich — Foundational institutional critique"
    )
    
    \#resource-link(
      "Tools for Conviviality",
      "https://www.davidtinapple.com/illich/1973\_tools\_for\_conviviality.pdf",
      "Ivan Illich — Liberatory technology"
    )
    
    \#resource-link(
      "No Shortcuts: Organizing for Power",
      "https://www.janemcalevey.com/no-shortcuts",
      "Jane McAlevey — Organizing vs. mobilizing"
    )
    
    \#resource-link(
      "Working in Public",
      "https://press.stripe.com/working-in-public",
      "Nadia Eghbal — Open source sustainability"
    )
    
    \#v(0.8em)
    
    \#text(size: 9pt, weight: "bold")[Tier 2: Deepening]
    
    \#resource-link(
      "Pedagogy of the Oppressed",
      "https://envs.ucsc.edu/internships/internship-readings/freire-pedagogy-of-the-oppressed.pdf",
      "Paulo Freire — Education as liberation"
    )
    
    \#resource-link(
      "Pedagogy of Hope",
      "https://archive.org/details/pedagogy-of-hope-reliving-pedagogy-of-the-oppressed",
      "Paulo Freire — Hope as ontological necessity"
    )
    
    \#resource-link(
      "Emergent Strategy",
      "https://www.akpress.org/emergentstrategy.html",
      "adrienne maree brown — Adaptive organizing"
    )
    
    \#resource-link(
      "Guerrilla Open Access Manifesto",
      "https://archive.org/details/GuesrillaOpenAccessManifesto",
      "Aaron Swartz — Information liberation ethics"
    )
    
    \#resource-link(
      "The Wealth of Networks",
      "https://www.benkler.org/Benkler\_Wealth\_Of\_Networks.pdf",
      "Yochai Benkler — Commons-based peer production"
    )
  ]
)

== Key Concepts Reference

\#table(
  columns: (1fr, 2.5fr),
  inset: 6pt,
  stroke: 0.5pt + rgb("\#ccc"),
  fill: (\_, row) => if row == 0 \{ rgb("\#f0f0f0") \} else \{ white \},
  
  text(size: 8pt, weight: "bold")[Term], 
  text(size: 8pt, weight: "bold")[Definition \& Application],
  
  text(size: 8pt)[}Hegemony\textit{], 
  text(size: 8pt)[Making your values become "common sense." Goal: make "mocap should be free" into common sense.],
  
  text(size: 8pt)[}Political Identity Paradox\textit{], 
  text(size: 8pt)[Strong identity enables commitment but creates insularity. Balance cohesion with outward focus.],
  
  text(size: 8pt)[}Expressive vs. Strategic\textit{], 
  text(size: 8pt)[Acting to express identity vs. to achieve outcomes. Prioritize mission over feeling good.],
  
  text(size: 8pt)[}Organizing vs. Mobilizing\textit{], 
  text(size: 8pt)[Mobilizing activates existing supporters; organizing expands base to non-activists.],
  
  text(size: 8pt)[}Conscientização\textit{], 
  text(size: 8pt)[Freire: developing critical consciousness — fatalism → systemic understanding → action.],
  
  text(size: 8pt)[}Prefigurative Politics*], 
  text(size: 8pt)[Embody the future in how you organize now. FreeMoCap should BE proof another way works.],
)

// ============================================================================
// PART 10: VISION DOCUMENT
// ============================================================================

= The Vision Document

\#epigraph(
  "Draft a 1-page document articulating the full scope of what FreeMoCap is building toward. This becomes your north star.",
  "Final Synthesis"
)

\#block(
  stroke: 1.5pt + rgb("\#1a1a2e"),
  inset: 16pt,
  radius: 6pt,
  [
    \#align(center)[
      \#text(size: 12pt, weight: "bold")[FREEMOCAP VISION DOCUMENT]
      \#text(size: 8pt, fill: rgb("\#666"))[ — DRAFT]
    ]
    
    \#v(0.8em)
    
    \#text(size: 9pt, weight: "bold")[THE PROBLEM]
    \#block(height: 3em, width: 100\%, fill: rgb("\#fafafa"), stroke: 0.5pt + rgb("\#ddd"), radius: 3pt, inset: 6pt, [])
    
    \#v(0.6em)
    
    \#text(size: 9pt, weight: "bold")[THE VISION]
    \#block(height: 3em, width: 100\%, fill: rgb("\#fafafa"), stroke: 0.5pt + rgb("\#ddd"), radius: 3pt, inset: 6pt, [])
    
    \#v(0.6em)
    
    \#text(size: 9pt, weight: "bold")[THE THEORY OF CHANGE]
    \#block(height: 3em, width: 100\%, fill: rgb("\#fafafa"), stroke: 0.5pt + rgb("\#ddd"), radius: 3pt, inset: 6pt, [])
    
    \#v(0.6em)
    
    \#text(size: 9pt, weight: "bold")[THE CURRENT STAGE]
    \#block(height: 2.5em, width: 100\%, fill: rgb("\#fafafa"), stroke: 0.5pt + rgb("\#ddd"), radius: 3pt, inset: 6pt, [])
    
    \#v(0.6em)
    
    \#text(size: 9pt, weight: "bold")[THE CALL]
    \#block(height: 2.5em, width: 100\%, fill: rgb("\#fffde7"), stroke: 0.5pt + rgb("\#fbc02d"), radius: 3pt, inset: 6pt, [])
  ]
)

// ============================================================================
// CLOSING
// ============================================================================

\#v(2em)

\#align(center)[
  \#block(
    width: 80\%,
    fill: rgb("\#f5f5f5"),
    stroke: (left: 4pt + rgb("\#e94560")),
    inset: 16pt,
    radius: (right: 4pt),
    [
      \#text(size: 10pt, weight: "bold")[
        This is a living document.
      ]
      
      \#v(0.5em)
      
      \#text(size: 9pt)[
        The work of building counter-hegemonic power is never complete. Return to these questions. Revise your answers. Share them with co-conspirators. Build the infrastructure for a long fight.
      ]
      
      \#v(0.5em)
      
      \#text(size: 9pt)[
        The current system wasn't built in a day. It won't fall in a day.
      ]
      
      \#v(0.3em)
      
      \#text(size: 11pt, weight: "bold")[
        But it WILL fall.
      ]
    ]
  )
  
  \#v(1.5em)
  
  \#line(length: 30\%, stroke: 0.5pt + rgb("\#ddd"))
  
  \#v(0.8em)
  
  \#text(size: 8pt, fill: rgb("\#666"))[
    FreeMoCap Foundation \textbackslash{}
    \#link("https://freemocap.org")[freemocap.org] · \#link("https://github.com/freemocap")[github.com/freemocap]
    
    \#v(0.3em)
    
    Based on frameworks from Smucker, Freire, McAlevey, Ganz, Illich, brown, and others.
  ]
]
\normalsize


\chapter{Liberation Movement Worksheet}
\label{src:WS3}

\begin{framed}
\small
\textbf{Source marker:} WS3 \\
\textbf{Date:} February 2026 \\
\textbf{Source:} Integrating liberation theory with organizing strategy
\end{framed}

\small
\# FreeMoCap as Counter-Hegemonic Project
\#\# A Strategic Worksheet for Liberating Knowledge, Education, and Science

\textit{"The master's tools will never dismantle the master's house." — Audre Lorde}

\textit{"Information is power. But like all power, there are those who want to keep it for themselves." — Aaron Swartz}

\bigskip\noindent\rule{\textwidth}{0.4pt}\bigskip

\# Framing: What Are We Actually Doing?

FreeMoCap the motion capture software is \textbf{Stage 1} — a concrete demonstration that:

- Research-grade scientific tools can be free, open, and accessible to everyone
- You don't need a university affiliation to do real science
- You don't need a \$200,000 budget to produce knowledge
- The barriers are artificial, maintained by institutions that benefit from scarcity

The \textbf{actual project} is much larger: dismantling the current hegemony of knowledge production — the interlocking system of universities, academic journals, proprietary software, professional credentialism, and institutional gatekeeping that determines who gets to participate in science, education, and the creation of knowledge.

This worksheet helps you develop the intellectual infrastructure for that larger fight.

\bigskip\noindent\rule{\textwidth}{0.4pt}\bigskip

\# Part 1: Mapping the Enemy — The Current Hegemony

\textit{Before you can overturn a hegemony, you need to understand it clearly. What are the structures you're fighting? How do they maintain power? What ideology justifies them?}

\#\# 1.1 The Institutional Landscape

\textbf{Map the interlocking institutions that control knowledge production:}

\begin{quote}\small\ttfamily
| Institution | What it controls | How it maintains control | Ideology that justifies it |\\
\hrule
| \textbf{Universities} | Access to education, research resources, credentials | Accreditation, degree requirements, tenure system, facility access | "Expertise requires formal training," "Quality requires institutional oversight" |\\
| \textbf{Academic Journals} | What counts as "real" knowledge, career advancement | Peer review as gatekeeping, impact factors, paywalls | "Peer review ensures quality," "Publishing validates research" |\\
| \textbf{Proprietary Software} | Tools for knowledge production | Licensing, prices, closed formats, vendor lock-in | "Professional tools cost money," "You get what you pay for" |\\
| \textbf{Professional Associations} | Who counts as "qualified" | Certifications, continuing ed requirements, ethical codes as gatekeeping | "Credentials protect the public," "Standards ensure quality" |\\
| \textbf{Funding Agencies} | What research gets done | Grant systems, institutional requirements, overhead rates | "Competitive funding ensures quality," "Accountability requires institutions" |\\
\end{quote}

\textbf{Add others you see:}

\begin{quote}\small\ttfamily
| Institution | What it controls | How it maintains control | Ideology that justifies it |\\
\hrule
| | | | |\\
| | | | |\\
| | | | |\\
\end{quote}

\#\# 1.2 The Hegemonic Common Sense

\textit{Hegemony works by making contingent arrangements seem natural and inevitable. What do people currently believe that serves this system?}

\textbf{Complete these sentences as a "normal person" embedded in current structures would:}

\begin{quote}\itshape
"To be a real scientist, you need \_\_\_\_\_\_\_\_\_\_\_\_\_\_\_\_\_\_\_\_\_\_\_\_\_\_\_\_\_\_\_\_\_\_\_\_\_\_\_\_\_\_\_\_\_\_\_"
\end{quote}

\begin{quote}\itshape
"Research isn't valid unless \_\_\_\_\_\_\_\_\_\_\_\_\_\_\_\_\_\_\_\_\_\_\_\_\_\_\_\_\_\_\_\_\_\_\_\_\_\_\_\_\_\_\_\_\_\_\_"
\end{quote}

\begin{quote}\itshape
"Education requires \_\_\_\_\_\_\_\_\_\_\_\_\_\_\_\_\_\_\_\_\_\_\_\_\_\_\_\_\_\_\_\_\_\_\_\_\_\_\_\_\_\_\_\_\_\_\_"
\end{quote}

\begin{quote}\itshape
"If software/journals/education were free, \_\_\_\_\_\_\_\_\_\_\_\_\_\_\_\_\_\_\_\_\_\_\_\_\_\_\_\_\_\_\_\_\_\_\_\_\_\_\_\_\_\_\_\_\_\_\_"
\end{quote}

\begin{quote}\itshape
"People outside institutions can't do real research because \_\_\_\_\_\_\_\_\_\_\_\_\_\_\_\_\_\_\_\_\_\_\_\_\_\_\_\_\_\_\_\_\_\_\_\_\_\_\_\_\_\_\_\_\_\_\_"
\end{quote}

\begin{quote}\itshape
"We need universities because \_\_\_\_\_\_\_\_\_\_\_\_\_\_\_\_\_\_\_\_\_\_\_\_\_\_\_\_\_\_\_\_\_\_\_\_\_\_\_\_\_\_\_\_\_\_\_"
\end{quote}

\begin{quote}\itshape
"Credentials matter because \_\_\_\_\_\_\_\_\_\_\_\_\_\_\_\_\_\_\_\_\_\_\_\_\_\_\_\_\_\_\_\_\_\_\_\_\_\_\_\_\_\_\_\_\_\_\_"
\end{quote}

\textbf{Where did these beliefs come from?} Who benefits from people believing them?

```
\_\_\_\_\_\_\_\_\_\_\_\_\_\_\_\_\_\_\_\_\_\_\_\_\_\_\_\_\_\_\_\_\_\_\_\_\_\_\_\_\_\_\_\_\_\_\_\_\_\_\_\_\_\_\_\_\_\_\_\_\_\_\_\_\_\_\_\_\_\_\_\_\_\_\_\_\_
\_\_\_\_\_\_\_\_\_\_\_\_\_\_\_\_\_\_\_\_\_\_\_\_\_\_\_\_\_\_\_\_\_\_\_\_\_\_\_\_\_\_\_\_\_\_\_\_\_\_\_\_\_\_\_\_\_\_\_\_\_\_\_\_\_\_\_\_\_\_\_\_\_\_\_\_\_
\_\_\_\_\_\_\_\_\_\_\_\_\_\_\_\_\_\_\_\_\_\_\_\_\_\_\_\_\_\_\_\_\_\_\_\_\_\_\_\_\_\_\_\_\_\_\_\_\_\_\_\_\_\_\_\_\_\_\_\_\_\_\_\_\_\_\_\_\_\_\_\_\_\_\_\_\_
\_\_\_\_\_\_\_\_\_\_\_\_\_\_\_\_\_\_\_\_\_\_\_\_\_\_\_\_\_\_\_\_\_\_\_\_\_\_\_\_\_\_\_\_\_\_\_\_\_\_\_\_\_\_\_\_\_\_\_\_\_\_\_\_\_\_\_\_\_\_\_\_\_\_\_\_\_
```

\#\# 1.3 Cracks in the Edifice

\textit{No hegemony is total. Where is the current system failing? Where are people already dissatisfied?}

\begin{quote}\small\ttfamily
| System Failure | Who experiences this? | What do they currently blame? |\\
\hrule
| Reproducibility crisis | Researchers, public | "Bad actors," not systemic incentives |\\
| Student debt crisis | Students, families | "Expensive schools," not credentialism itself |\\
| Journal paywalls | Everyone outside elite institutions | "Greedy publishers," not the publication model |\\
| Adjunctification | Early-career academics | "Budget cuts," not the tenure system |\\
| Research irrelevance | Practitioners, public | "Ivory tower academics," not funding structures |\\
| Proprietary lock-in | Researchers, educators | Specific vendors, not proprietary model |\\
\end{quote}

\textbf{Add others:}

\begin{quote}\small\ttfamily
| System Failure | Who experiences this? | What do they currently blame? |\\
\hrule
| | | |\\
| | | |\\
| | | |\\
\end{quote}

\textbf{Key question:} These people are experiencing real problems with the system. How could you help them see the \textit{systemic} cause rather than blaming individuals or specific bad actors?

```
\_\_\_\_\_\_\_\_\_\_\_\_\_\_\_\_\_\_\_\_\_\_\_\_\_\_\_\_\_\_\_\_\_\_\_\_\_\_\_\_\_\_\_\_\_\_\_\_\_\_\_\_\_\_\_\_\_\_\_\_\_\_\_\_\_\_\_\_\_\_\_\_\_\_\_\_\_
\_\_\_\_\_\_\_\_\_\_\_\_\_\_\_\_\_\_\_\_\_\_\_\_\_\_\_\_\_\_\_\_\_\_\_\_\_\_\_\_\_\_\_\_\_\_\_\_\_\_\_\_\_\_\_\_\_\_\_\_\_\_\_\_\_\_\_\_\_\_\_\_\_\_\_\_\_
\_\_\_\_\_\_\_\_\_\_\_\_\_\_\_\_\_\_\_\_\_\_\_\_\_\_\_\_\_\_\_\_\_\_\_\_\_\_\_\_\_\_\_\_\_\_\_\_\_\_\_\_\_\_\_\_\_\_\_\_\_\_\_\_\_\_\_\_\_\_\_\_\_\_\_\_\_
```

\#\# 1.4 Who Benefits, Who Loses

\textbf{Under the current system:}

\begin{quote}\small\ttfamily
| Who benefits? | How? |\\
\hrule
| Tenured faculty at elite institutions | |\\
| Academic publishers | |\\
| Proprietary software companies | |\\
| | |\\
| | |\\
\end{quote}

\begin{quote}\small\ttfamily
| Who loses? | How? |\\
\hrule
| Students (debt, limited access) | |\\
| Adjuncts and contingent faculty | |\\
| Researchers at under-resourced institutions | |\\
| Global South academics | |\\
| Independent researchers | |\\
| Practitioners who could use research | |\\
| The public | |\\
| | |\\
| | |\\
\end{quote}

\textbf{The people who lose vastly outnumber those who benefit. Why hasn't the system already changed?}

```
\_\_\_\_\_\_\_\_\_\_\_\_\_\_\_\_\_\_\_\_\_\_\_\_\_\_\_\_\_\_\_\_\_\_\_\_\_\_\_\_\_\_\_\_\_\_\_\_\_\_\_\_\_\_\_\_\_\_\_\_\_\_\_\_\_\_\_\_\_\_\_\_\_\_\_\_\_
\_\_\_\_\_\_\_\_\_\_\_\_\_\_\_\_\_\_\_\_\_\_\_\_\_\_\_\_\_\_\_\_\_\_\_\_\_\_\_\_\_\_\_\_\_\_\_\_\_\_\_\_\_\_\_\_\_\_\_\_\_\_\_\_\_\_\_\_\_\_\_\_\_\_\_\_\_
\_\_\_\_\_\_\_\_\_\_\_\_\_\_\_\_\_\_\_\_\_\_\_\_\_\_\_\_\_\_\_\_\_\_\_\_\_\_\_\_\_\_\_\_\_\_\_\_\_\_\_\_\_\_\_\_\_\_\_\_\_\_\_\_\_\_\_\_\_\_\_\_\_\_\_\_\_
```

\bigskip\noindent\rule{\textwidth}{0.4pt}\bigskip

\# Part 2: The Counter-Hegemonic Vision

\textit{What new "common sense" are you trying to create? What does the liberated landscape look like?}

\#\# 2.1 The World We're Building

\textbf{Complete these sentences as someone living in the world FreeMoCap is building:}

\begin{quote}\itshape
"Anyone can do real science because \_\_\_\_\_\_\_\_\_\_\_\_\_\_\_\_\_\_\_\_\_\_\_\_\_\_\_\_\_\_\_\_\_\_\_\_\_\_\_\_\_\_\_\_\_\_\_"
\end{quote}

\begin{quote}\itshape
"I learned [skill/knowledge] by \_\_\_\_\_\_\_\_\_\_\_\_\_\_\_\_\_\_\_\_\_\_\_\_\_\_\_\_\_\_\_\_\_\_\_\_\_\_\_\_\_\_\_\_\_\_\_"
\end{quote}

\begin{quote}\itshape
"When I need research tools, I \_\_\_\_\_\_\_\_\_\_\_\_\_\_\_\_\_\_\_\_\_\_\_\_\_\_\_\_\_\_\_\_\_\_\_\_\_\_\_\_\_\_\_\_\_\_\_"
\end{quote}

\begin{quote}\itshape
"Knowledge is validated by \_\_\_\_\_\_\_\_\_\_\_\_\_\_\_\_\_\_\_\_\_\_\_\_\_\_\_\_\_\_\_\_\_\_\_\_\_\_\_\_\_\_\_\_\_\_\_"
\end{quote}

\begin{quote}\itshape
"Education happens \_\_\_\_\_\_\_\_\_\_\_\_\_\_\_\_\_\_\_\_\_\_\_\_\_\_\_\_\_\_\_\_\_\_\_\_\_\_\_\_\_\_\_\_\_\_\_"
\end{quote}

\begin{quote}\itshape
"My work is valued because \_\_\_\_\_\_\_\_\_\_\_\_\_\_\_\_\_\_\_\_\_\_\_\_\_\_\_\_\_\_\_\_\_\_\_\_\_\_\_\_\_\_\_\_\_\_\_, not because I have credentials from \_\_\_\_\_\_\_\_\_\_\_\_\_\_\_\_\_\_\_\_\_\_\_\_\_\_\_\_\_\_\_\_\_\_\_\_\_\_\_\_\_\_\_\_\_\_\_"
\end{quote}

\#\# 2.2 Core Principles of the New Hegemony

\textbf{What are the foundational beliefs of the world you're building?}

\textit{Draft 5-7 core principles:}

```
1. \_\_\_\_\_\_\_\_\_\_\_\_\_\_\_\_\_\_\_\_\_\_\_\_\_\_\_\_\_\_\_\_\_\_\_\_\_\_\_\_\_\_\_\_\_\_\_\_\_\_\_\_\_\_\_\_\_\_\_\_\_\_\_\_\_\_\_\_\_\_\_\_\_\_\_\_\_

2. \_\_\_\_\_\_\_\_\_\_\_\_\_\_\_\_\_\_\_\_\_\_\_\_\_\_\_\_\_\_\_\_\_\_\_\_\_\_\_\_\_\_\_\_\_\_\_\_\_\_\_\_\_\_\_\_\_\_\_\_\_\_\_\_\_\_\_\_\_\_\_\_\_\_\_\_\_

3. \_\_\_\_\_\_\_\_\_\_\_\_\_\_\_\_\_\_\_\_\_\_\_\_\_\_\_\_\_\_\_\_\_\_\_\_\_\_\_\_\_\_\_\_\_\_\_\_\_\_\_\_\_\_\_\_\_\_\_\_\_\_\_\_\_\_\_\_\_\_\_\_\_\_\_\_\_

4. \_\_\_\_\_\_\_\_\_\_\_\_\_\_\_\_\_\_\_\_\_\_\_\_\_\_\_\_\_\_\_\_\_\_\_\_\_\_\_\_\_\_\_\_\_\_\_\_\_\_\_\_\_\_\_\_\_\_\_\_\_\_\_\_\_\_\_\_\_\_\_\_\_\_\_\_\_

5. \_\_\_\_\_\_\_\_\_\_\_\_\_\_\_\_\_\_\_\_\_\_\_\_\_\_\_\_\_\_\_\_\_\_\_\_\_\_\_\_\_\_\_\_\_\_\_\_\_\_\_\_\_\_\_\_\_\_\_\_\_\_\_\_\_\_\_\_\_\_\_\_\_\_\_\_\_

6. \_\_\_\_\_\_\_\_\_\_\_\_\_\_\_\_\_\_\_\_\_\_\_\_\_\_\_\_\_\_\_\_\_\_\_\_\_\_\_\_\_\_\_\_\_\_\_\_\_\_\_\_\_\_\_\_\_\_\_\_\_\_\_\_\_\_\_\_\_\_\_\_\_\_\_\_\_

7. \_\_\_\_\_\_\_\_\_\_\_\_\_\_\_\_\_\_\_\_\_\_\_\_\_\_\_\_\_\_\_\_\_\_\_\_\_\_\_\_\_\_\_\_\_\_\_\_\_\_\_\_\_\_\_\_\_\_\_\_\_\_\_\_\_\_\_\_\_\_\_\_\_\_\_\_\_
```

\textbf{For each principle, articulate:}
- Why it's true (the positive case)
- What current belief it displaces
- What changes if people accept it

\#\# 2.3 Prefigurative Politics

\textit{"Prefigurative politics" means embodying the future you want to create in how you organize now. FreeMoCap should BE the proof that another way is possible.}

\textbf{How does FreeMoCap currently embody the principles above?}

\begin{quote}\small\ttfamily
| Principle | How FreeMoCap demonstrates this NOW |\\
\hrule
| | |\\
| | |\\
| | |\\
| | |\\
\end{quote}

\textbf{Where does FreeMoCap fall short of its own principles?} (Be honest)

```
\_\_\_\_\_\_\_\_\_\_\_\_\_\_\_\_\_\_\_\_\_\_\_\_\_\_\_\_\_\_\_\_\_\_\_\_\_\_\_\_\_\_\_\_\_\_\_\_\_\_\_\_\_\_\_\_\_\_\_\_\_\_\_\_\_\_\_\_\_\_\_\_\_\_\_\_\_
\_\_\_\_\_\_\_\_\_\_\_\_\_\_\_\_\_\_\_\_\_\_\_\_\_\_\_\_\_\_\_\_\_\_\_\_\_\_\_\_\_\_\_\_\_\_\_\_\_\_\_\_\_\_\_\_\_\_\_\_\_\_\_\_\_\_\_\_\_\_\_\_\_\_\_\_\_
\_\_\_\_\_\_\_\_\_\_\_\_\_\_\_\_\_\_\_\_\_\_\_\_\_\_\_\_\_\_\_\_\_\_\_\_\_\_\_\_\_\_\_\_\_\_\_\_\_\_\_\_\_\_\_\_\_\_\_\_\_\_\_\_\_\_\_\_\_\_\_\_\_\_\_\_\_
```

\textbf{What would it look like to more fully embody these principles?}

```
\_\_\_\_\_\_\_\_\_\_\_\_\_\_\_\_\_\_\_\_\_\_\_\_\_\_\_\_\_\_\_\_\_\_\_\_\_\_\_\_\_\_\_\_\_\_\_\_\_\_\_\_\_\_\_\_\_\_\_\_\_\_\_\_\_\_\_\_\_\_\_\_\_\_\_\_\_
\_\_\_\_\_\_\_\_\_\_\_\_\_\_\_\_\_\_\_\_\_\_\_\_\_\_\_\_\_\_\_\_\_\_\_\_\_\_\_\_\_\_\_\_\_\_\_\_\_\_\_\_\_\_\_\_\_\_\_\_\_\_\_\_\_\_\_\_\_\_\_\_\_\_\_\_\_
\_\_\_\_\_\_\_\_\_\_\_\_\_\_\_\_\_\_\_\_\_\_\_\_\_\_\_\_\_\_\_\_\_\_\_\_\_\_\_\_\_\_\_\_\_\_\_\_\_\_\_\_\_\_\_\_\_\_\_\_\_\_\_\_\_\_\_\_\_\_\_\_\_\_\_\_\_
```

\bigskip\noindent\rule{\textwidth}{0.4pt}\bigskip

\# Part 3: Theory of Change — How Hegemonies Fall

\textit{Smucker is clear: being right isn't enough. You need a plausible theory of how the current system gets replaced by yours.}

\#\# 3.1 Historical Models

\textbf{How have previous counter-hegemonic projects succeeded?}

\begin{quote}\small\ttfamily
| Movement/Project | What hegemony did they challenge? | How did they win? | What can FreeMoCap learn? |\\
\hrule
| Free Software Movement | Proprietary software as default | Built viable alternatives, created licensing infrastructure, changed developer culture | |\\
| Open Access Movement | Journal paywalls | Mandates, repositories, demonstrated demand, Sci-Hub as civil disobedience | |\\
| Wikipedia | Expert-only encyclopedias | Just built it, proved it worked, became too useful to ignore | |\\
| Linux | Microsoft/proprietary OS hegemony | Server dominance first, enterprise adoption, Android | |\\
| | | | |\\
\end{quote}

\textbf{What do successful counter-hegemonic projects have in common?}

```
\_\_\_\_\_\_\_\_\_\_\_\_\_\_\_\_\_\_\_\_\_\_\_\_\_\_\_\_\_\_\_\_\_\_\_\_\_\_\_\_\_\_\_\_\_\_\_\_\_\_\_\_\_\_\_\_\_\_\_\_\_\_\_\_\_\_\_\_\_\_\_\_\_\_\_\_\_
\_\_\_\_\_\_\_\_\_\_\_\_\_\_\_\_\_\_\_\_\_\_\_\_\_\_\_\_\_\_\_\_\_\_\_\_\_\_\_\_\_\_\_\_\_\_\_\_\_\_\_\_\_\_\_\_\_\_\_\_\_\_\_\_\_\_\_\_\_\_\_\_\_\_\_\_\_
\_\_\_\_\_\_\_\_\_\_\_\_\_\_\_\_\_\_\_\_\_\_\_\_\_\_\_\_\_\_\_\_\_\_\_\_\_\_\_\_\_\_\_\_\_\_\_\_\_\_\_\_\_\_\_\_\_\_\_\_\_\_\_\_\_\_\_\_\_\_\_\_\_\_\_\_\_
```

\#\# 3.2 Your Theory of Change

\textbf{Complete this theory of change for the broader FreeMoCap project:}

\begin{quote}\itshape
\textbf{If} FreeMoCap [does X] \_\_\_\_\_\_\_\_\_\_\_\_\_\_\_\_\_\_\_\_\_\_\_\_\_\_\_\_\_\_\_\_\_\_\_\_\_\_\_\_\_\_\_\_\_\_\_

\textbf{Then} [Y constituency] will \_\_\_\_\_\_\_\_\_\_\_\_\_\_\_\_\_\_\_\_\_\_\_\_\_\_\_\_\_\_\_\_\_\_\_\_\_\_\_\_\_\_\_\_\_\_\_

\textbf{Which will cause} [Z institutional change] \_\_\_\_\_\_\_\_\_\_\_\_\_\_\_\_\_\_\_\_\_\_\_\_\_\_\_\_\_\_\_\_\_\_\_\_\_\_\_\_\_\_\_\_\_\_\_

\textbf{Leading to} [broader shift] \_\_\_\_\_\_\_\_\_\_\_\_\_\_\_\_\_\_\_\_\_\_\_\_\_\_\_\_\_\_\_\_\_\_\_\_\_\_\_\_\_\_\_\_\_\_\_

\textbf{Until eventually} [new hegemony] \_\_\_\_\_\_\_\_\_\_\_\_\_\_\_\_\_\_\_\_\_\_\_\_\_\_\_\_\_\_\_\_\_\_\_\_\_\_\_\_\_\_\_\_\_\_\_
\end{quote}

\textbf{What are the key assumptions in this theory?} What has to be true for it to work?

```
1. \_\_\_\_\_\_\_\_\_\_\_\_\_\_\_\_\_\_\_\_\_\_\_\_\_\_\_\_\_\_\_\_\_\_\_\_\_\_\_\_\_\_\_\_\_\_\_\_\_\_\_\_\_\_\_\_\_\_\_\_\_\_\_\_\_\_\_\_\_\_\_\_\_\_\_\_\_
2. \_\_\_\_\_\_\_\_\_\_\_\_\_\_\_\_\_\_\_\_\_\_\_\_\_\_\_\_\_\_\_\_\_\_\_\_\_\_\_\_\_\_\_\_\_\_\_\_\_\_\_\_\_\_\_\_\_\_\_\_\_\_\_\_\_\_\_\_\_\_\_\_\_\_\_\_\_
3. \_\_\_\_\_\_\_\_\_\_\_\_\_\_\_\_\_\_\_\_\_\_\_\_\_\_\_\_\_\_\_\_\_\_\_\_\_\_\_\_\_\_\_\_\_\_\_\_\_\_\_\_\_\_\_\_\_\_\_\_\_\_\_\_\_\_\_\_\_\_\_\_\_\_\_\_\_
4. \_\_\_\_\_\_\_\_\_\_\_\_\_\_\_\_\_\_\_\_\_\_\_\_\_\_\_\_\_\_\_\_\_\_\_\_\_\_\_\_\_\_\_\_\_\_\_\_\_\_\_\_\_\_\_\_\_\_\_\_\_\_\_\_\_\_\_\_\_\_\_\_\_\_\_\_\_
```

\textbf{How could you test these assumptions?}

```
\_\_\_\_\_\_\_\_\_\_\_\_\_\_\_\_\_\_\_\_\_\_\_\_\_\_\_\_\_\_\_\_\_\_\_\_\_\_\_\_\_\_\_\_\_\_\_\_\_\_\_\_\_\_\_\_\_\_\_\_\_\_\_\_\_\_\_\_\_\_\_\_\_\_\_\_\_
\_\_\_\_\_\_\_\_\_\_\_\_\_\_\_\_\_\_\_\_\_\_\_\_\_\_\_\_\_\_\_\_\_\_\_\_\_\_\_\_\_\_\_\_\_\_\_\_\_\_\_\_\_\_\_\_\_\_\_\_\_\_\_\_\_\_\_\_\_\_\_\_\_\_\_\_\_
```

\#\# 3.3 The Stages

\textbf{If FreeMoCap motion capture is Stage 1, what are the subsequent stages?}

\begin{quote}\small\ttfamily
| Stage | Focus | Goal | Success looks like | Enables next stage by... |\\
\hrule
| \textbf{1: FreeMoCap} | Motion capture | Prove FOSS can match/beat proprietary research tools | Adoption in labs, citations, recognized quality | Builds credibility, community, organizational capacity |\\
| \textbf{2:} | | | | |\\
| \textbf{3:} | | | | |\\
| \textbf{4:} | | | | |\\
| \textbf{5:} | | | | |\\
\end{quote}

\textbf{What capabilities/resources/credibility do you need to build at each stage?}

```
Stage 1 → 2 requires: \_\_\_\_\_\_\_\_\_\_\_\_\_\_\_\_\_\_\_\_\_\_\_\_\_\_\_\_\_\_\_\_\_\_\_\_\_\_\_\_\_\_\_\_\_\_\_\_\_\_\_\_\_\_\_\_\_\_\_
Stage 2 → 3 requires: \_\_\_\_\_\_\_\_\_\_\_\_\_\_\_\_\_\_\_\_\_\_\_\_\_\_\_\_\_\_\_\_\_\_\_\_\_\_\_\_\_\_\_\_\_\_\_\_\_\_\_\_\_\_\_\_\_\_\_
Stage 3 → 4 requires: \_\_\_\_\_\_\_\_\_\_\_\_\_\_\_\_\_\_\_\_\_\_\_\_\_\_\_\_\_\_\_\_\_\_\_\_\_\_\_\_\_\_\_\_\_\_\_\_\_\_\_\_\_\_\_\_\_\_\_
Stage 4 → 5 requires: \_\_\_\_\_\_\_\_\_\_\_\_\_\_\_\_\_\_\_\_\_\_\_\_\_\_\_\_\_\_\_\_\_\_\_\_\_\_\_\_\_\_\_\_\_\_\_\_\_\_\_\_\_\_\_\_\_\_\_
```

\#\# 3.4 Multiple Fronts

\textit{Counter-hegemonic projects usually require pressure on multiple fronts simultaneously.}

\textbf{What are the different "fronts" in this fight?}

\begin{quote}\small\ttfamily
| Front | What victory looks like | Who's already fighting here? | FreeMoCap's role |\\
\hrule
| \textbf{Tools} (FOSS research software) | FOSS default for research | NumPy, R, Jupyter, etc. | Active combatant |\\
| \textbf{Publishing} (Open Access) | Paywalls gone, new validation models | PLOS, arXiv, Sci-Hub, SPARC | Ally? Participant? |\\
| \textbf{Education} (alternative credentials) | Degrees not required, learning accessible | MOOCs, bootcamps, unschooling | ? |\\
| \textbf{Funding} (alternative models) | Research funded outside grant system | Crowdfunding, DAOs, citizen science | ? |\\
| \textbf{Community} (independent research) | Legitimate research outside academy | Indie researchers, community labs | ? |\\
| | | | |\\
\end{quote}

\textbf{Where should FreeMoCap focus? Where should it build alliances?}

```
\_\_\_\_\_\_\_\_\_\_\_\_\_\_\_\_\_\_\_\_\_\_\_\_\_\_\_\_\_\_\_\_\_\_\_\_\_\_\_\_\_\_\_\_\_\_\_\_\_\_\_\_\_\_\_\_\_\_\_\_\_\_\_\_\_\_\_\_\_\_\_\_\_\_\_\_\_
\_\_\_\_\_\_\_\_\_\_\_\_\_\_\_\_\_\_\_\_\_\_\_\_\_\_\_\_\_\_\_\_\_\_\_\_\_\_\_\_\_\_\_\_\_\_\_\_\_\_\_\_\_\_\_\_\_\_\_\_\_\_\_\_\_\_\_\_\_\_\_\_\_\_\_\_\_
\_\_\_\_\_\_\_\_\_\_\_\_\_\_\_\_\_\_\_\_\_\_\_\_\_\_\_\_\_\_\_\_\_\_\_\_\_\_\_\_\_\_\_\_\_\_\_\_\_\_\_\_\_\_\_\_\_\_\_\_\_\_\_\_\_\_\_\_\_\_\_\_\_\_\_\_\_
```

\bigskip\noindent\rule{\textwidth}{0.4pt}\bigskip

\# Part 4: The Public Narrative (Expanded)

\textit{Your narrative must serve the LARGER project, not just the motion capture software.}

\#\# 4.1 Story of Self

\textit{Why are YOU, Jon, fighting to liberate knowledge production? What personal experience made this your fight?}

\textbf{The Encounter:} When did you first experience the injustice of the current system? What specific moment showed you something was wrong?

```
\_\_\_\_\_\_\_\_\_\_\_\_\_\_\_\_\_\_\_\_\_\_\_\_\_\_\_\_\_\_\_\_\_\_\_\_\_\_\_\_\_\_\_\_\_\_\_\_\_\_\_\_\_\_\_\_\_\_\_\_\_\_\_\_\_\_\_\_\_\_\_\_\_\_\_\_\_
\_\_\_\_\_\_\_\_\_\_\_\_\_\_\_\_\_\_\_\_\_\_\_\_\_\_\_\_\_\_\_\_\_\_\_\_\_\_\_\_\_\_\_\_\_\_\_\_\_\_\_\_\_\_\_\_\_\_\_\_\_\_\_\_\_\_\_\_\_\_\_\_\_\_\_\_\_
\_\_\_\_\_\_\_\_\_\_\_\_\_\_\_\_\_\_\_\_\_\_\_\_\_\_\_\_\_\_\_\_\_\_\_\_\_\_\_\_\_\_\_\_\_\_\_\_\_\_\_\_\_\_\_\_\_\_\_\_\_\_\_\_\_\_\_\_\_\_\_\_\_\_\_\_\_
\_\_\_\_\_\_\_\_\_\_\_\_\_\_\_\_\_\_\_\_\_\_\_\_\_\_\_\_\_\_\_\_\_\_\_\_\_\_\_\_\_\_\_\_\_\_\_\_\_\_\_\_\_\_\_\_\_\_\_\_\_\_\_\_\_\_\_\_\_\_\_\_\_\_\_\_\_
```

\textbf{The Choice:} What moment of choice committed you to this fight? (Not a gradual drift — a moment when you could have done otherwise but chose this.)

```
\_\_\_\_\_\_\_\_\_\_\_\_\_\_\_\_\_\_\_\_\_\_\_\_\_\_\_\_\_\_\_\_\_\_\_\_\_\_\_\_\_\_\_\_\_\_\_\_\_\_\_\_\_\_\_\_\_\_\_\_\_\_\_\_\_\_\_\_\_\_\_\_\_\_\_\_\_
\_\_\_\_\_\_\_\_\_\_\_\_\_\_\_\_\_\_\_\_\_\_\_\_\_\_\_\_\_\_\_\_\_\_\_\_\_\_\_\_\_\_\_\_\_\_\_\_\_\_\_\_\_\_\_\_\_\_\_\_\_\_\_\_\_\_\_\_\_\_\_\_\_\_\_\_\_
\_\_\_\_\_\_\_\_\_\_\_\_\_\_\_\_\_\_\_\_\_\_\_\_\_\_\_\_\_\_\_\_\_\_\_\_\_\_\_\_\_\_\_\_\_\_\_\_\_\_\_\_\_\_\_\_\_\_\_\_\_\_\_\_\_\_\_\_\_\_\_\_\_\_\_\_\_
\_\_\_\_\_\_\_\_\_\_\_\_\_\_\_\_\_\_\_\_\_\_\_\_\_\_\_\_\_\_\_\_\_\_\_\_\_\_\_\_\_\_\_\_\_\_\_\_\_\_\_\_\_\_\_\_\_\_\_\_\_\_\_\_\_\_\_\_\_\_\_\_\_\_\_\_\_
```

\textbf{The Cost:} What have you given up or risked by making this choice? (Sacrifice demonstrates commitment and values.)

```
\_\_\_\_\_\_\_\_\_\_\_\_\_\_\_\_\_\_\_\_\_\_\_\_\_\_\_\_\_\_\_\_\_\_\_\_\_\_\_\_\_\_\_\_\_\_\_\_\_\_\_\_\_\_\_\_\_\_\_\_\_\_\_\_\_\_\_\_\_\_\_\_\_\_\_\_\_
\_\_\_\_\_\_\_\_\_\_\_\_\_\_\_\_\_\_\_\_\_\_\_\_\_\_\_\_\_\_\_\_\_\_\_\_\_\_\_\_\_\_\_\_\_\_\_\_\_\_\_\_\_\_\_\_\_\_\_\_\_\_\_\_\_\_\_\_\_\_\_\_\_\_\_\_\_
\_\_\_\_\_\_\_\_\_\_\_\_\_\_\_\_\_\_\_\_\_\_\_\_\_\_\_\_\_\_\_\_\_\_\_\_\_\_\_\_\_\_\_\_\_\_\_\_\_\_\_\_\_\_\_\_\_\_\_\_\_\_\_\_\_\_\_\_\_\_\_\_\_\_\_\_\_
```

\textbf{The Values:} What deep values drive this work? Not "open source is good" but the VALUES underneath — justice? freedom? democracy? human flourishing?

```
\_\_\_\_\_\_\_\_\_\_\_\_\_\_\_\_\_\_\_\_\_\_\_\_\_\_\_\_\_\_\_\_\_\_\_\_\_\_\_\_\_\_\_\_\_\_\_\_\_\_\_\_\_\_\_\_\_\_\_\_\_\_\_\_\_\_\_\_\_\_\_\_\_\_\_\_\_
\_\_\_\_\_\_\_\_\_\_\_\_\_\_\_\_\_\_\_\_\_\_\_\_\_\_\_\_\_\_\_\_\_\_\_\_\_\_\_\_\_\_\_\_\_\_\_\_\_\_\_\_\_\_\_\_\_\_\_\_\_\_\_\_\_\_\_\_\_\_\_\_\_\_\_\_\_
\_\_\_\_\_\_\_\_\_\_\_\_\_\_\_\_\_\_\_\_\_\_\_\_\_\_\_\_\_\_\_\_\_\_\_\_\_\_\_\_\_\_\_\_\_\_\_\_\_\_\_\_\_\_\_\_\_\_\_\_\_\_\_\_\_\_\_\_\_\_\_\_\_\_\_\_\_
```

\textbf{Draft Story of Self (the version that serves the LARGER project):}

```
\_\_\_\_\_\_\_\_\_\_\_\_\_\_\_\_\_\_\_\_\_\_\_\_\_\_\_\_\_\_\_\_\_\_\_\_\_\_\_\_\_\_\_\_\_\_\_\_\_\_\_\_\_\_\_\_\_\_\_\_\_\_\_\_\_\_\_\_\_\_\_\_\_\_\_\_\_
\_\_\_\_\_\_\_\_\_\_\_\_\_\_\_\_\_\_\_\_\_\_\_\_\_\_\_\_\_\_\_\_\_\_\_\_\_\_\_\_\_\_\_\_\_\_\_\_\_\_\_\_\_\_\_\_\_\_\_\_\_\_\_\_\_\_\_\_\_\_\_\_\_\_\_\_\_
\_\_\_\_\_\_\_\_\_\_\_\_\_\_\_\_\_\_\_\_\_\_\_\_\_\_\_\_\_\_\_\_\_\_\_\_\_\_\_\_\_\_\_\_\_\_\_\_\_\_\_\_\_\_\_\_\_\_\_\_\_\_\_\_\_\_\_\_\_\_\_\_\_\_\_\_\_
\_\_\_\_\_\_\_\_\_\_\_\_\_\_\_\_\_\_\_\_\_\_\_\_\_\_\_\_\_\_\_\_\_\_\_\_\_\_\_\_\_\_\_\_\_\_\_\_\_\_\_\_\_\_\_\_\_\_\_\_\_\_\_\_\_\_\_\_\_\_\_\_\_\_\_\_\_
\_\_\_\_\_\_\_\_\_\_\_\_\_\_\_\_\_\_\_\_\_\_\_\_\_\_\_\_\_\_\_\_\_\_\_\_\_\_\_\_\_\_\_\_\_\_\_\_\_\_\_\_\_\_\_\_\_\_\_\_\_\_\_\_\_\_\_\_\_\_\_\_\_\_\_\_\_
\_\_\_\_\_\_\_\_\_\_\_\_\_\_\_\_\_\_\_\_\_\_\_\_\_\_\_\_\_\_\_\_\_\_\_\_\_\_\_\_\_\_\_\_\_\_\_\_\_\_\_\_\_\_\_\_\_\_\_\_\_\_\_\_\_\_\_\_\_\_\_\_\_\_\_\_\_
```

\#\# 4.2 Story of Us

\textit{Who is the "we" fighting for liberated knowledge? This must be MUCH bigger than "FreeMoCap users."}

\textbf{The broadest possible "we":}

Who has been harmed by or excluded from the current knowledge system?

```
- Students crushed by debt for credentials that gatekeep knowledge
- Researchers at under-resourced institutions
- Global South academics locked out by paywalls and equipment costs
- Independent researchers dismissed as "not real scientists"
- Practitioners who can't access research relevant to their work
- Curious people told "you're not qualified to understand this"
- \_\_\_\_\_\_\_\_\_\_\_\_\_\_\_\_\_\_\_\_\_\_\_\_\_\_\_\_\_\_\_\_\_\_\_\_\_\_\_\_\_\_\_\_\_\_\_\_\_\_\_\_\_\_\_\_\_\_\_\_\_\_\_\_\_\_\_\_\_\_\_\_\_\_\_\_\_
- \_\_\_\_\_\_\_\_\_\_\_\_\_\_\_\_\_\_\_\_\_\_\_\_\_\_\_\_\_\_\_\_\_\_\_\_\_\_\_\_\_\_\_\_\_\_\_\_\_\_\_\_\_\_\_\_\_\_\_\_\_\_\_\_\_\_\_\_\_\_\_\_\_\_\_\_\_
- \_\_\_\_\_\_\_\_\_\_\_\_\_\_\_\_\_\_\_\_\_\_\_\_\_\_\_\_\_\_\_\_\_\_\_\_\_\_\_\_\_\_\_\_\_\_\_\_\_\_\_\_\_\_\_\_\_\_\_\_\_\_\_\_\_\_\_\_\_\_\_\_\_\_\_\_\_
```

\textbf{The shared experience:} What do all these people have in common?

```
\_\_\_\_\_\_\_\_\_\_\_\_\_\_\_\_\_\_\_\_\_\_\_\_\_\_\_\_\_\_\_\_\_\_\_\_\_\_\_\_\_\_\_\_\_\_\_\_\_\_\_\_\_\_\_\_\_\_\_\_\_\_\_\_\_\_\_\_\_\_\_\_\_\_\_\_\_
\_\_\_\_\_\_\_\_\_\_\_\_\_\_\_\_\_\_\_\_\_\_\_\_\_\_\_\_\_\_\_\_\_\_\_\_\_\_\_\_\_\_\_\_\_\_\_\_\_\_\_\_\_\_\_\_\_\_\_\_\_\_\_\_\_\_\_\_\_\_\_\_\_\_\_\_\_
```

\textbf{The shared enemy:} (Not "universities" in general — the SYSTEM, the ideology, the gatekeeping)

```
\_\_\_\_\_\_\_\_\_\_\_\_\_\_\_\_\_\_\_\_\_\_\_\_\_\_\_\_\_\_\_\_\_\_\_\_\_\_\_\_\_\_\_\_\_\_\_\_\_\_\_\_\_\_\_\_\_\_\_\_\_\_\_\_\_\_\_\_\_\_\_\_\_\_\_\_\_
\_\_\_\_\_\_\_\_\_\_\_\_\_\_\_\_\_\_\_\_\_\_\_\_\_\_\_\_\_\_\_\_\_\_\_\_\_\_\_\_\_\_\_\_\_\_\_\_\_\_\_\_\_\_\_\_\_\_\_\_\_\_\_\_\_\_\_\_\_\_\_\_\_\_\_\_\_
```

\textbf{The shared vision:} What world are "we" building together?

```
\_\_\_\_\_\_\_\_\_\_\_\_\_\_\_\_\_\_\_\_\_\_\_\_\_\_\_\_\_\_\_\_\_\_\_\_\_\_\_\_\_\_\_\_\_\_\_\_\_\_\_\_\_\_\_\_\_\_\_\_\_\_\_\_\_\_\_\_\_\_\_\_\_\_\_\_\_
\_\_\_\_\_\_\_\_\_\_\_\_\_\_\_\_\_\_\_\_\_\_\_\_\_\_\_\_\_\_\_\_\_\_\_\_\_\_\_\_\_\_\_\_\_\_\_\_\_\_\_\_\_\_\_\_\_\_\_\_\_\_\_\_\_\_\_\_\_\_\_\_\_\_\_\_\_
```

\textbf{Draft Story of Us:}

```
\_\_\_\_\_\_\_\_\_\_\_\_\_\_\_\_\_\_\_\_\_\_\_\_\_\_\_\_\_\_\_\_\_\_\_\_\_\_\_\_\_\_\_\_\_\_\_\_\_\_\_\_\_\_\_\_\_\_\_\_\_\_\_\_\_\_\_\_\_\_\_\_\_\_\_\_\_
\_\_\_\_\_\_\_\_\_\_\_\_\_\_\_\_\_\_\_\_\_\_\_\_\_\_\_\_\_\_\_\_\_\_\_\_\_\_\_\_\_\_\_\_\_\_\_\_\_\_\_\_\_\_\_\_\_\_\_\_\_\_\_\_\_\_\_\_\_\_\_\_\_\_\_\_\_
\_\_\_\_\_\_\_\_\_\_\_\_\_\_\_\_\_\_\_\_\_\_\_\_\_\_\_\_\_\_\_\_\_\_\_\_\_\_\_\_\_\_\_\_\_\_\_\_\_\_\_\_\_\_\_\_\_\_\_\_\_\_\_\_\_\_\_\_\_\_\_\_\_\_\_\_\_
\_\_\_\_\_\_\_\_\_\_\_\_\_\_\_\_\_\_\_\_\_\_\_\_\_\_\_\_\_\_\_\_\_\_\_\_\_\_\_\_\_\_\_\_\_\_\_\_\_\_\_\_\_\_\_\_\_\_\_\_\_\_\_\_\_\_\_\_\_\_\_\_\_\_\_\_\_
\_\_\_\_\_\_\_\_\_\_\_\_\_\_\_\_\_\_\_\_\_\_\_\_\_\_\_\_\_\_\_\_\_\_\_\_\_\_\_\_\_\_\_\_\_\_\_\_\_\_\_\_\_\_\_\_\_\_\_\_\_\_\_\_\_\_\_\_\_\_\_\_\_\_\_\_\_
```

\#\# 4.3 Story of Now

\textit{Why is THIS MOMENT critical? What's the urgency?}

\textbf{What's happening now that creates opportunity or urgency?}

Consider:
- AI/ML democratizing some capabilities while concentrating others
- Reproducibility crisis delegitimizing current institutions
- Student debt crisis making people question credentialism
- Pandemic demonstrating need for open science
- Growing distrust of institutions generally
- Climate crisis requiring faster/more distributed knowledge production

```
The urgent opportunity: \_\_\_\_\_\_\_\_\_\_\_\_\_\_\_\_\_\_\_\_\_\_\_\_\_\_\_\_\_\_\_\_\_\_\_\_\_\_\_\_\_\_\_\_\_\_\_\_\_\_\_\_\_\_\_
\_\_\_\_\_\_\_\_\_\_\_\_\_\_\_\_\_\_\_\_\_\_\_\_\_\_\_\_\_\_\_\_\_\_\_\_\_\_\_\_\_\_\_\_\_\_\_\_\_\_\_\_\_\_\_\_\_\_\_\_\_\_\_\_\_\_\_\_\_\_\_\_\_\_\_\_\_
\_\_\_\_\_\_\_\_\_\_\_\_\_\_\_\_\_\_\_\_\_\_\_\_\_\_\_\_\_\_\_\_\_\_\_\_\_\_\_\_\_\_\_\_\_\_\_\_\_\_\_\_\_\_\_\_\_\_\_\_\_\_\_\_\_\_\_\_\_\_\_\_\_\_\_\_\_
```

\textbf{What window is closing?}

```
If we don't act now: \_\_\_\_\_\_\_\_\_\_\_\_\_\_\_\_\_\_\_\_\_\_\_\_\_\_\_\_\_\_\_\_\_\_\_\_\_\_\_\_\_\_\_\_\_\_\_\_\_\_\_\_\_\_\_\_\_\_
\_\_\_\_\_\_\_\_\_\_\_\_\_\_\_\_\_\_\_\_\_\_\_\_\_\_\_\_\_\_\_\_\_\_\_\_\_\_\_\_\_\_\_\_\_\_\_\_\_\_\_\_\_\_\_\_\_\_\_\_\_\_\_\_\_\_\_\_\_\_\_\_\_\_\_\_\_
\_\_\_\_\_\_\_\_\_\_\_\_\_\_\_\_\_\_\_\_\_\_\_\_\_\_\_\_\_\_\_\_\_\_\_\_\_\_\_\_\_\_\_\_\_\_\_\_\_\_\_\_\_\_\_\_\_\_\_\_\_\_\_\_\_\_\_\_\_\_\_\_\_\_\_\_\_
```

\textbf{The specific ask (for different audiences):}

\begin{quote}\small\ttfamily
| Audience | What do you want them to DO? |\\
\hrule
| Someone experiencing the system's failures | |\\
| A potential contributor | |\\
| A potential ally organization | |\\
| Someone with resources | |\\
| Someone with platform/influence | |\\
\end{quote}

\textbf{Draft Story of Now:}

```
\_\_\_\_\_\_\_\_\_\_\_\_\_\_\_\_\_\_\_\_\_\_\_\_\_\_\_\_\_\_\_\_\_\_\_\_\_\_\_\_\_\_\_\_\_\_\_\_\_\_\_\_\_\_\_\_\_\_\_\_\_\_\_\_\_\_\_\_\_\_\_\_\_\_\_\_\_
\_\_\_\_\_\_\_\_\_\_\_\_\_\_\_\_\_\_\_\_\_\_\_\_\_\_\_\_\_\_\_\_\_\_\_\_\_\_\_\_\_\_\_\_\_\_\_\_\_\_\_\_\_\_\_\_\_\_\_\_\_\_\_\_\_\_\_\_\_\_\_\_\_\_\_\_\_
\_\_\_\_\_\_\_\_\_\_\_\_\_\_\_\_\_\_\_\_\_\_\_\_\_\_\_\_\_\_\_\_\_\_\_\_\_\_\_\_\_\_\_\_\_\_\_\_\_\_\_\_\_\_\_\_\_\_\_\_\_\_\_\_\_\_\_\_\_\_\_\_\_\_\_\_\_
\_\_\_\_\_\_\_\_\_\_\_\_\_\_\_\_\_\_\_\_\_\_\_\_\_\_\_\_\_\_\_\_\_\_\_\_\_\_\_\_\_\_\_\_\_\_\_\_\_\_\_\_\_\_\_\_\_\_\_\_\_\_\_\_\_\_\_\_\_\_\_\_\_\_\_\_\_
\_\_\_\_\_\_\_\_\_\_\_\_\_\_\_\_\_\_\_\_\_\_\_\_\_\_\_\_\_\_\_\_\_\_\_\_\_\_\_\_\_\_\_\_\_\_\_\_\_\_\_\_\_\_\_\_\_\_\_\_\_\_\_\_\_\_\_\_\_\_\_\_\_\_\_\_\_
```

\#\# 4.4 The Integrated Narrative

\textbf{Weave these together. This should take 3-5 minutes to tell. Practice out loud.}

```
[Story of Self — your encounter with injustice, your choice, your values]
\_\_\_\_\_\_\_\_\_\_\_\_\_\_\_\_\_\_\_\_\_\_\_\_\_\_\_\_\_\_\_\_\_\_\_\_\_\_\_\_\_\_\_\_\_\_\_\_\_\_\_\_\_\_\_\_\_\_\_\_\_\_\_\_\_\_\_\_\_\_\_\_\_\_\_\_\_
\_\_\_\_\_\_\_\_\_\_\_\_\_\_\_\_\_\_\_\_\_\_\_\_\_\_\_\_\_\_\_\_\_\_\_\_\_\_\_\_\_\_\_\_\_\_\_\_\_\_\_\_\_\_\_\_\_\_\_\_\_\_\_\_\_\_\_\_\_\_\_\_\_\_\_\_\_
\_\_\_\_\_\_\_\_\_\_\_\_\_\_\_\_\_\_\_\_\_\_\_\_\_\_\_\_\_\_\_\_\_\_\_\_\_\_\_\_\_\_\_\_\_\_\_\_\_\_\_\_\_\_\_\_\_\_\_\_\_\_\_\_\_\_\_\_\_\_\_\_\_\_\_\_\_
\_\_\_\_\_\_\_\_\_\_\_\_\_\_\_\_\_\_\_\_\_\_\_\_\_\_\_\_\_\_\_\_\_\_\_\_\_\_\_\_\_\_\_\_\_\_\_\_\_\_\_\_\_\_\_\_\_\_\_\_\_\_\_\_\_\_\_\_\_\_\_\_\_\_\_\_\_

[Bridge]: "And I discovered I wasn't alone..."

[Story of Us — the shared experience, the shared enemy, the shared vision]
\_\_\_\_\_\_\_\_\_\_\_\_\_\_\_\_\_\_\_\_\_\_\_\_\_\_\_\_\_\_\_\_\_\_\_\_\_\_\_\_\_\_\_\_\_\_\_\_\_\_\_\_\_\_\_\_\_\_\_\_\_\_\_\_\_\_\_\_\_\_\_\_\_\_\_\_\_
\_\_\_\_\_\_\_\_\_\_\_\_\_\_\_\_\_\_\_\_\_\_\_\_\_\_\_\_\_\_\_\_\_\_\_\_\_\_\_\_\_\_\_\_\_\_\_\_\_\_\_\_\_\_\_\_\_\_\_\_\_\_\_\_\_\_\_\_\_\_\_\_\_\_\_\_\_
\_\_\_\_\_\_\_\_\_\_\_\_\_\_\_\_\_\_\_\_\_\_\_\_\_\_\_\_\_\_\_\_\_\_\_\_\_\_\_\_\_\_\_\_\_\_\_\_\_\_\_\_\_\_\_\_\_\_\_\_\_\_\_\_\_\_\_\_\_\_\_\_\_\_\_\_\_
\_\_\_\_\_\_\_\_\_\_\_\_\_\_\_\_\_\_\_\_\_\_\_\_\_\_\_\_\_\_\_\_\_\_\_\_\_\_\_\_\_\_\_\_\_\_\_\_\_\_\_\_\_\_\_\_\_\_\_\_\_\_\_\_\_\_\_\_\_\_\_\_\_\_\_\_\_

[Bridge]: "And right now, we face a critical moment..."

[Story of Now — the urgency, the opportunity, the ask]
\_\_\_\_\_\_\_\_\_\_\_\_\_\_\_\_\_\_\_\_\_\_\_\_\_\_\_\_\_\_\_\_\_\_\_\_\_\_\_\_\_\_\_\_\_\_\_\_\_\_\_\_\_\_\_\_\_\_\_\_\_\_\_\_\_\_\_\_\_\_\_\_\_\_\_\_\_
\_\_\_\_\_\_\_\_\_\_\_\_\_\_\_\_\_\_\_\_\_\_\_\_\_\_\_\_\_\_\_\_\_\_\_\_\_\_\_\_\_\_\_\_\_\_\_\_\_\_\_\_\_\_\_\_\_\_\_\_\_\_\_\_\_\_\_\_\_\_\_\_\_\_\_\_\_
\_\_\_\_\_\_\_\_\_\_\_\_\_\_\_\_\_\_\_\_\_\_\_\_\_\_\_\_\_\_\_\_\_\_\_\_\_\_\_\_\_\_\_\_\_\_\_\_\_\_\_\_\_\_\_\_\_\_\_\_\_\_\_\_\_\_\_\_\_\_\_\_\_\_\_\_\_
\_\_\_\_\_\_\_\_\_\_\_\_\_\_\_\_\_\_\_\_\_\_\_\_\_\_\_\_\_\_\_\_\_\_\_\_\_\_\_\_\_\_\_\_\_\_\_\_\_\_\_\_\_\_\_\_\_\_\_\_\_\_\_\_\_\_\_\_\_\_\_\_\_\_\_\_\_

[The Call]: "That's why I'm asking you to..."
\_\_\_\_\_\_\_\_\_\_\_\_\_\_\_\_\_\_\_\_\_\_\_\_\_\_\_\_\_\_\_\_\_\_\_\_\_\_\_\_\_\_\_\_\_\_\_\_\_\_\_\_\_\_\_\_\_\_\_\_\_\_\_\_\_\_\_\_\_\_\_\_\_\_\_\_\_
\_\_\_\_\_\_\_\_\_\_\_\_\_\_\_\_\_\_\_\_\_\_\_\_\_\_\_\_\_\_\_\_\_\_\_\_\_\_\_\_\_\_\_\_\_\_\_\_\_\_\_\_\_\_\_\_\_\_\_\_\_\_\_\_\_\_\_\_\_\_\_\_\_\_\_\_\_
```

\bigskip\noindent\rule{\textwidth}{0.4pt}\bigskip

\# Part 5: Alliance Mapping

\textit{You're not fighting alone. Many movements share pieces of this vision. Who are your allies?}

\#\# 5.1 The Allied Movements

\textbf{Map movements/communities that share parts of your vision:}

\begin{quote}\small\ttfamily
| Movement/Community | What they fight for | Where interests overlap | Where they might diverge | Key organizations/figures |\\
\hrule
| Free Software Movement | Software freedom | FOSS as default, anti-proprietary | May focus on licenses over access | FSF, SFLC, Stallman's writings |\\
| Open Access | Free access to research | Against paywalls, for open publishing | Often works within academic system | SPARC, PLOS, Suber |\\
| Open Science | Reproducibility, transparency | Open tools, open data, open methods | May not challenge institutions themselves | COS, FOSTER |\\
| Right to Repair | Ownership of technology | Against proprietary lock-in | More consumer-focused | iFixit, Repair.org |\\
| Citizen Science | Public participation in research | Research outside institutions | May defer to academic legitimacy | SciStarter, Zooniverse |\\
| Unschooling/Deschooling | Education without institutions | Against credentialism | May be individualist | Illich's legacy, various |\\
| Hacker Culture | Information wants to be free | Tool-building, sharing knowledge | Can be apolitical | Hackerspaces, CCC |\\
| Library Movement | Public access to knowledge | Against information enclosure | May work within systems | ALA, Internet Archive |\\
| Indie/Community Research | Research outside academy | Legitimacy of non-institutional work | Often isolated | Various indie researchers |\\
| | | | | |\\
\end{quote}

\#\# 5.2 Potential Organizational Allies

\textbf{List specific organizations that might be allies:}

\begin{quote}\small\ttfamily
| Organization | What they do | Potential for alliance | Who do you know there? |\\
\hrule
| Software Freedom Conservancy | | | |\\
| Internet Archive | | | |\\
| Electronic Frontier Foundation | | | |\\
| Mozilla Foundation | | | |\\
| Wikimedia Foundation | | | |\\
| Public Library of Science | | | |\\
| Center for Open Science | | | |\\
| NumFOCUS | | | |\\
| | | | |\\
| | | | |\\
| | | | |\\
\end{quote}

\#\# 5.3 Intellectual Lineages

\textbf{What thinkers/traditions inform this work?}

\begin{quote}\small\ttfamily
| Thinker/Tradition | Key Ideas | How it informs FreeMoCap |\\
\hrule
| Ivan Illich | Deschooling, convivial tools, institutional critique | |\\
| Aaron Swartz | Information liberation, guerrilla open access | |\\
| Richard Stallman | Free software as freedom, not just price | |\\
| Paulo Freire | Critical pedagogy, education as liberation | |\\
| Donna Haraway | Situated knowledge, critique of objectivity | |\\
| Yochai Benkler | Commons-based peer production | |\\
| Elinor Ostrom | Governance of commons | |\\
| | | |\\
| | | |\\
\end{quote}

\textbf{Which of these have you read? Which should you prioritize?}

\begin{quote}\small\ttfamily
| Must Read Soon | Why |\\
\hrule
| | |\\
| | |\\
| | |\\
\end{quote}

\#\# 5.4 Building Alliances

\textbf{For your top 5 potential allies, develop an approach:}

\begin{quote}\small\ttfamily
| Ally | What you could offer them | What you need from them | First step to build relationship |\\
\hrule
| 1. | | | |\\
| 2. | | | |\\
| 3. | | | |\\
| 4. | | | |\\
| 5. | | | |\\
\end{quote}

\bigskip\noindent\rule{\textwidth}{0.4pt}\bigskip

\# Part 6: Organizing for the Long Haul

\textit{Counter-hegemonic projects take decades. How do you build an organization that can sustain this?}

\#\# 6.1 The FreeMoCap Foundation as Vehicle

\textbf{Is the FreeMoCap Foundation the right organizational vehicle for the larger vision?}

\begin{quote}\small\ttfamily
| Consideration | Current State | What's needed for larger vision |\\
\hrule
| Legal structure | 501(c)(3) | Appropriate? Limiting? |\\
| Name/brand | "FreeMoCap" = motion capture | Does this scale to larger mission? |\\
| Mission statement | | Does it articulate the larger vision? |\\
| Governance | | Appropriate for movement-building? |\\
\end{quote}

\textbf{Should there be a broader organizational entity that FreeMoCap is one project of?}

```
\_\_\_\_\_\_\_\_\_\_\_\_\_\_\_\_\_\_\_\_\_\_\_\_\_\_\_\_\_\_\_\_\_\_\_\_\_\_\_\_\_\_\_\_\_\_\_\_\_\_\_\_\_\_\_\_\_\_\_\_\_\_\_\_\_\_\_\_\_\_\_\_\_\_\_\_\_
\_\_\_\_\_\_\_\_\_\_\_\_\_\_\_\_\_\_\_\_\_\_\_\_\_\_\_\_\_\_\_\_\_\_\_\_\_\_\_\_\_\_\_\_\_\_\_\_\_\_\_\_\_\_\_\_\_\_\_\_\_\_\_\_\_\_\_\_\_\_\_\_\_\_\_\_\_
\_\_\_\_\_\_\_\_\_\_\_\_\_\_\_\_\_\_\_\_\_\_\_\_\_\_\_\_\_\_\_\_\_\_\_\_\_\_\_\_\_\_\_\_\_\_\_\_\_\_\_\_\_\_\_\_\_\_\_\_\_\_\_\_\_\_\_\_\_\_\_\_\_\_\_\_\_
```

\#\# 6.2 Developing Leadership

\textit{Ella Baker: "Strong people don't need a strong leader."}

\textbf{How do you develop leaders who can carry this forward without you?}

\textbf{Current leadership capacity:}

\begin{quote}\small\ttfamily
| Person | Current role | Potential | What development do they need? |\\
\hrule
| | | | |\\
| | | | |\\
| | | | |\\
| | | | |\\
\end{quote}

\textbf{What leadership roles will you need that don't exist yet?}

```
1. \_\_\_\_\_\_\_\_\_\_\_\_\_\_\_\_\_\_\_\_\_\_\_\_\_\_\_\_\_\_\_\_\_\_\_\_\_\_\_\_\_\_\_\_\_\_\_\_\_\_\_\_\_\_\_\_\_\_\_\_\_\_\_\_\_\_\_\_\_\_\_\_\_\_\_\_\_
2. \_\_\_\_\_\_\_\_\_\_\_\_\_\_\_\_\_\_\_\_\_\_\_\_\_\_\_\_\_\_\_\_\_\_\_\_\_\_\_\_\_\_\_\_\_\_\_\_\_\_\_\_\_\_\_\_\_\_\_\_\_\_\_\_\_\_\_\_\_\_\_\_\_\_\_\_\_
3. \_\_\_\_\_\_\_\_\_\_\_\_\_\_\_\_\_\_\_\_\_\_\_\_\_\_\_\_\_\_\_\_\_\_\_\_\_\_\_\_\_\_\_\_\_\_\_\_\_\_\_\_\_\_\_\_\_\_\_\_\_\_\_\_\_\_\_\_\_\_\_\_\_\_\_\_\_
```

\textbf{How do people BECOME leaders in your organization?} (Make this explicit)

```
\_\_\_\_\_\_\_\_\_\_\_\_\_\_\_\_\_\_\_\_\_\_\_\_\_\_\_\_\_\_\_\_\_\_\_\_\_\_\_\_\_\_\_\_\_\_\_\_\_\_\_\_\_\_\_\_\_\_\_\_\_\_\_\_\_\_\_\_\_\_\_\_\_\_\_\_\_
\_\_\_\_\_\_\_\_\_\_\_\_\_\_\_\_\_\_\_\_\_\_\_\_\_\_\_\_\_\_\_\_\_\_\_\_\_\_\_\_\_\_\_\_\_\_\_\_\_\_\_\_\_\_\_\_\_\_\_\_\_\_\_\_\_\_\_\_\_\_\_\_\_\_\_\_\_
\_\_\_\_\_\_\_\_\_\_\_\_\_\_\_\_\_\_\_\_\_\_\_\_\_\_\_\_\_\_\_\_\_\_\_\_\_\_\_\_\_\_\_\_\_\_\_\_\_\_\_\_\_\_\_\_\_\_\_\_\_\_\_\_\_\_\_\_\_\_\_\_\_\_\_\_\_
```

\#\# 6.3 Organizational Culture

\textbf{What culture do you want to create?}

\begin{quote}\small\ttfamily
| Value | What it looks like in practice | Current state (1-10) |\\
\hrule
| Radical accessibility | | |\\
| Distributed leadership | | |\\
| Long-term thinking | | |\\
| Solidarity with allied movements | | |\\
| Prefigurative practice | | |\\
| | | |\\
| | | |\\
\end{quote}

\textbf{What cultural patterns do you want to AVOID?}

Consider:
- Founder dependency
- Insider/outsider dynamics
- Technical elitism
- Burnout/martyrdom culture
- Purity politics
- Short-term thinking

```
\_\_\_\_\_\_\_\_\_\_\_\_\_\_\_\_\_\_\_\_\_\_\_\_\_\_\_\_\_\_\_\_\_\_\_\_\_\_\_\_\_\_\_\_\_\_\_\_\_\_\_\_\_\_\_\_\_\_\_\_\_\_\_\_\_\_\_\_\_\_\_\_\_\_\_\_\_
\_\_\_\_\_\_\_\_\_\_\_\_\_\_\_\_\_\_\_\_\_\_\_\_\_\_\_\_\_\_\_\_\_\_\_\_\_\_\_\_\_\_\_\_\_\_\_\_\_\_\_\_\_\_\_\_\_\_\_\_\_\_\_\_\_\_\_\_\_\_\_\_\_\_\_\_\_
\_\_\_\_\_\_\_\_\_\_\_\_\_\_\_\_\_\_\_\_\_\_\_\_\_\_\_\_\_\_\_\_\_\_\_\_\_\_\_\_\_\_\_\_\_\_\_\_\_\_\_\_\_\_\_\_\_\_\_\_\_\_\_\_\_\_\_\_\_\_\_\_\_\_\_\_\_
```

\#\# 6.4 Sustainability

\textbf{How does this work get resourced for the long haul?}

\begin{quote}\small\ttfamily
| Resource | Current source | Vulnerability | Alternative sources |\\
\hrule
| Money | | | |\\
| Core contributor time | | | |\\
| Infrastructure | | | |\\
| Legitimacy/credibility | | | |\\
\end{quote}

\textbf{What's the path to sustainable funding that doesn't compromise the mission?}

```
\_\_\_\_\_\_\_\_\_\_\_\_\_\_\_\_\_\_\_\_\_\_\_\_\_\_\_\_\_\_\_\_\_\_\_\_\_\_\_\_\_\_\_\_\_\_\_\_\_\_\_\_\_\_\_\_\_\_\_\_\_\_\_\_\_\_\_\_\_\_\_\_\_\_\_\_\_
\_\_\_\_\_\_\_\_\_\_\_\_\_\_\_\_\_\_\_\_\_\_\_\_\_\_\_\_\_\_\_\_\_\_\_\_\_\_\_\_\_\_\_\_\_\_\_\_\_\_\_\_\_\_\_\_\_\_\_\_\_\_\_\_\_\_\_\_\_\_\_\_\_\_\_\_\_
\_\_\_\_\_\_\_\_\_\_\_\_\_\_\_\_\_\_\_\_\_\_\_\_\_\_\_\_\_\_\_\_\_\_\_\_\_\_\_\_\_\_\_\_\_\_\_\_\_\_\_\_\_\_\_\_\_\_\_\_\_\_\_\_\_\_\_\_\_\_\_\_\_\_\_\_\_
```

\bigskip\noindent\rule{\textwidth}{0.4pt}\bigskip

\# Part 7: Strategic Questions to Keep Asking

\textit{These aren't one-time questions. Return to them regularly.}

\#\# 7.1 The Smucker Questions

\textbf{Ask yourself monthly:}

\begin{quote}\itshape
"Is what we're doing likely to advance our mission, or does it primarily express our identity as [open source developers / academics / hackers / activists]?"
\end{quote}

Current answer:
```
\_\_\_\_\_\_\_\_\_\_\_\_\_\_\_\_\_\_\_\_\_\_\_\_\_\_\_\_\_\_\_\_\_\_\_\_\_\_\_\_\_\_\_\_\_\_\_\_\_\_\_\_\_\_\_\_\_\_\_\_\_\_\_\_\_\_\_\_\_\_\_\_\_\_\_\_\_
\_\_\_\_\_\_\_\_\_\_\_\_\_\_\_\_\_\_\_\_\_\_\_\_\_\_\_\_\_\_\_\_\_\_\_\_\_\_\_\_\_\_\_\_\_\_\_\_\_\_\_\_\_\_\_\_\_\_\_\_\_\_\_\_\_\_\_\_\_\_\_\_\_\_\_\_\_
```

\begin{quote}\itshape
"Are we building power, or are we performing righteousness?"
\end{quote}

Current answer:
```
\_\_\_\_\_\_\_\_\_\_\_\_\_\_\_\_\_\_\_\_\_\_\_\_\_\_\_\_\_\_\_\_\_\_\_\_\_\_\_\_\_\_\_\_\_\_\_\_\_\_\_\_\_\_\_\_\_\_\_\_\_\_\_\_\_\_\_\_\_\_\_\_\_\_\_\_\_
\_\_\_\_\_\_\_\_\_\_\_\_\_\_\_\_\_\_\_\_\_\_\_\_\_\_\_\_\_\_\_\_\_\_\_\_\_\_\_\_\_\_\_\_\_\_\_\_\_\_\_\_\_\_\_\_\_\_\_\_\_\_\_\_\_\_\_\_\_\_\_\_\_\_\_\_\_
```

\begin{quote}\itshape
"Who are we bringing IN to this work who wasn't here before? Or are we just activating the already-convinced?"
\end{quote}

Current answer:
```
\_\_\_\_\_\_\_\_\_\_\_\_\_\_\_\_\_\_\_\_\_\_\_\_\_\_\_\_\_\_\_\_\_\_\_\_\_\_\_\_\_\_\_\_\_\_\_\_\_\_\_\_\_\_\_\_\_\_\_\_\_\_\_\_\_\_\_\_\_\_\_\_\_\_\_\_\_
\_\_\_\_\_\_\_\_\_\_\_\_\_\_\_\_\_\_\_\_\_\_\_\_\_\_\_\_\_\_\_\_\_\_\_\_\_\_\_\_\_\_\_\_\_\_\_\_\_\_\_\_\_\_\_\_\_\_\_\_\_\_\_\_\_\_\_\_\_\_\_\_\_\_\_\_\_
```

\begin{quote}\itshape
"What have we WON recently? What concrete victory can we point to?"
\end{quote}

Current answer:
```
\_\_\_\_\_\_\_\_\_\_\_\_\_\_\_\_\_\_\_\_\_\_\_\_\_\_\_\_\_\_\_\_\_\_\_\_\_\_\_\_\_\_\_\_\_\_\_\_\_\_\_\_\_\_\_\_\_\_\_\_\_\_\_\_\_\_\_\_\_\_\_\_\_\_\_\_\_
\_\_\_\_\_\_\_\_\_\_\_\_\_\_\_\_\_\_\_\_\_\_\_\_\_\_\_\_\_\_\_\_\_\_\_\_\_\_\_\_\_\_\_\_\_\_\_\_\_\_\_\_\_\_\_\_\_\_\_\_\_\_\_\_\_\_\_\_\_\_\_\_\_\_\_\_\_
```

\#\# 7.2 The Scale Questions

\begin{quote}\itshape
"Is what we're doing now building toward the larger vision, or are we getting stuck at Stage 1?"
\end{quote}

```
\_\_\_\_\_\_\_\_\_\_\_\_\_\_\_\_\_\_\_\_\_\_\_\_\_\_\_\_\_\_\_\_\_\_\_\_\_\_\_\_\_\_\_\_\_\_\_\_\_\_\_\_\_\_\_\_\_\_\_\_\_\_\_\_\_\_\_\_\_\_\_\_\_\_\_\_\_
\_\_\_\_\_\_\_\_\_\_\_\_\_\_\_\_\_\_\_\_\_\_\_\_\_\_\_\_\_\_\_\_\_\_\_\_\_\_\_\_\_\_\_\_\_\_\_\_\_\_\_\_\_\_\_\_\_\_\_\_\_\_\_\_\_\_\_\_\_\_\_\_\_\_\_\_\_
```

\begin{quote}\itshape
"Are we building the organizational capacity to do MORE than motion capture software?"
\end{quote}

```
\_\_\_\_\_\_\_\_\_\_\_\_\_\_\_\_\_\_\_\_\_\_\_\_\_\_\_\_\_\_\_\_\_\_\_\_\_\_\_\_\_\_\_\_\_\_\_\_\_\_\_\_\_\_\_\_\_\_\_\_\_\_\_\_\_\_\_\_\_\_\_\_\_\_\_\_\_
\_\_\_\_\_\_\_\_\_\_\_\_\_\_\_\_\_\_\_\_\_\_\_\_\_\_\_\_\_\_\_\_\_\_\_\_\_\_\_\_\_\_\_\_\_\_\_\_\_\_\_\_\_\_\_\_\_\_\_\_\_\_\_\_\_\_\_\_\_\_\_\_\_\_\_\_\_
```

\begin{quote}\itshape
"Are we connecting with allied movements, or operating in isolation?"
\end{quote}

```
\_\_\_\_\_\_\_\_\_\_\_\_\_\_\_\_\_\_\_\_\_\_\_\_\_\_\_\_\_\_\_\_\_\_\_\_\_\_\_\_\_\_\_\_\_\_\_\_\_\_\_\_\_\_\_\_\_\_\_\_\_\_\_\_\_\_\_\_\_\_\_\_\_\_\_\_\_
\_\_\_\_\_\_\_\_\_\_\_\_\_\_\_\_\_\_\_\_\_\_\_\_\_\_\_\_\_\_\_\_\_\_\_\_\_\_\_\_\_\_\_\_\_\_\_\_\_\_\_\_\_\_\_\_\_\_\_\_\_\_\_\_\_\_\_\_\_\_\_\_\_\_\_\_\_
```

\#\# 7.3 The Sustainability Questions

\begin{quote}\itshape
"Can this continue without Jon?"
\end{quote}

```
\_\_\_\_\_\_\_\_\_\_\_\_\_\_\_\_\_\_\_\_\_\_\_\_\_\_\_\_\_\_\_\_\_\_\_\_\_\_\_\_\_\_\_\_\_\_\_\_\_\_\_\_\_\_\_\_\_\_\_\_\_\_\_\_\_\_\_\_\_\_\_\_\_\_\_\_\_
\_\_\_\_\_\_\_\_\_\_\_\_\_\_\_\_\_\_\_\_\_\_\_\_\_\_\_\_\_\_\_\_\_\_\_\_\_\_\_\_\_\_\_\_\_\_\_\_\_\_\_\_\_\_\_\_\_\_\_\_\_\_\_\_\_\_\_\_\_\_\_\_\_\_\_\_\_
```

\begin{quote}\itshape
"What happens if a key contributor burns out or leaves?"
\end{quote}

```
\_\_\_\_\_\_\_\_\_\_\_\_\_\_\_\_\_\_\_\_\_\_\_\_\_\_\_\_\_\_\_\_\_\_\_\_\_\_\_\_\_\_\_\_\_\_\_\_\_\_\_\_\_\_\_\_\_\_\_\_\_\_\_\_\_\_\_\_\_\_\_\_\_\_\_\_\_
\_\_\_\_\_\_\_\_\_\_\_\_\_\_\_\_\_\_\_\_\_\_\_\_\_\_\_\_\_\_\_\_\_\_\_\_\_\_\_\_\_\_\_\_\_\_\_\_\_\_\_\_\_\_\_\_\_\_\_\_\_\_\_\_\_\_\_\_\_\_\_\_\_\_\_\_\_
```

\begin{quote}\itshape
"Are we moving at a sustainable pace?"
\end{quote}

```
\_\_\_\_\_\_\_\_\_\_\_\_\_\_\_\_\_\_\_\_\_\_\_\_\_\_\_\_\_\_\_\_\_\_\_\_\_\_\_\_\_\_\_\_\_\_\_\_\_\_\_\_\_\_\_\_\_\_\_\_\_\_\_\_\_\_\_\_\_\_\_\_\_\_\_\_\_
\_\_\_\_\_\_\_\_\_\_\_\_\_\_\_\_\_\_\_\_\_\_\_\_\_\_\_\_\_\_\_\_\_\_\_\_\_\_\_\_\_\_\_\_\_\_\_\_\_\_\_\_\_\_\_\_\_\_\_\_\_\_\_\_\_\_\_\_\_\_\_\_\_\_\_\_\_
```

\bigskip\noindent\rule{\textwidth}{0.4pt}\bigskip

\# Part 8: Immediate Actions

\textit{Strategy without action is fantasy. What are you doing THIS MONTH?}

\#\# 8.1 Top Priorities for Next 90 Days

Based on this worksheet, what are the highest-leverage actions?

\begin{quote}\small\ttfamily
| Priority | Why this matters | Concrete first step | Deadline | Owner |\\
\hrule
| 1. | | | | |\\
| 2. | | | | |\\
| 3. | | | | |\\
\end{quote}

\#\# 8.2 What You're NOT Doing

\textit{Strategy means choosing. What are you explicitly deprioritizing?}

```
1. \_\_\_\_\_\_\_\_\_\_\_\_\_\_\_\_\_\_\_\_\_\_\_\_\_\_\_\_\_\_\_\_\_ (Why: \_\_\_\_\_\_\_\_\_\_\_\_\_\_\_\_\_\_\_\_\_\_\_\_\_\_\_\_\_\_\_\_)
2. \_\_\_\_\_\_\_\_\_\_\_\_\_\_\_\_\_\_\_\_\_\_\_\_\_\_\_\_\_\_\_\_\_ (Why: \_\_\_\_\_\_\_\_\_\_\_\_\_\_\_\_\_\_\_\_\_\_\_\_\_\_\_\_\_\_\_\_)
3. \_\_\_\_\_\_\_\_\_\_\_\_\_\_\_\_\_\_\_\_\_\_\_\_\_\_\_\_\_\_\_\_\_ (Why: \_\_\_\_\_\_\_\_\_\_\_\_\_\_\_\_\_\_\_\_\_\_\_\_\_\_\_\_\_\_\_\_)
```

\#\# 8.3 Accountability

\textbf{Who will you share this with?}

```
\_\_\_\_\_\_\_\_\_\_\_\_\_\_\_\_\_\_\_\_\_\_\_\_\_\_\_\_\_\_\_\_\_\_\_\_\_\_\_\_\_\_\_\_\_\_\_\_\_\_\_\_\_\_\_\_\_\_\_\_\_\_\_\_\_\_\_\_\_\_\_\_\_\_\_\_\_
```

\textbf{When will you review progress?}

- 30-day check-in: \_\_\_\_\_\_\_\_\_\_\_\_\_\_\_\_
- 90-day review: \_\_\_\_\_\_\_\_\_\_\_\_\_\_\_\_
- 6-month strategic review: \_\_\_\_\_\_\_\_\_\_\_\_\_\_\_\_

\bigskip\noindent\rule{\textwidth}{0.4pt}\bigskip

\# Part 9: Reading List for the Larger Fight

\#\# Tier 1: Core Strategic Reading

\begin{quote}\small\ttfamily
| Book | Why | Status |\\
\hrule
| Smucker, \textit{Hegemony How-To} | Core strategic framework for counter-hegemonic organizing | ✅ |\\
| Illich, \textit{Deschooling Society} | The foundational critique of institutional education | |\\
| Illich, \textit{Tools for Conviviality} | What technology looks like when it liberates rather than dominates | |\\
| McAlevey, \textit{No Shortcuts} | Organizing vs. mobilizing — how to actually build power | |\\
| Eghbal, \textit{Working in Public} | The realities of open source sustainability and community | |\\
\end{quote}

\#\# Tier 2: Deepening the Analysis

\begin{quote}\small\ttfamily
| Book | Why |\\
\hrule
| Freire, \textit{Pedagogy of the Oppressed} | Education as liberation practice, critical consciousness |\\
| Benkler, \textit{The Wealth of Networks} | The economics and possibility of commons-based peer production |\\
| brown, \textit{Emergent Strategy} | Adaptive organizing, building movements that can evolve |\\
| Swartz, \textit{The Boy Who Could Change the World} (collected writings) | The ethics and tactics of information liberation |\\
| Ransby, \textit{Ella Baker and the Black Freedom Movement} | Group-centered leadership, developing others |\\
\end{quote}

\#\# Tier 3: Historical and Theoretical Context

\begin{quote}\small\ttfamily
| Book | Why |\\
\hrule
| Gramsci, \textit{Prison Notebooks} (selections) | The original theory of hegemony |\\
| Ostrom, \textit{Governing the Commons} | How commons can be managed without privatization or state control |\\
| Levy, \textit{Hackers: Heroes of the Computer Revolution} | The hacker ethic and its political implications |\\
| Coleman, \textit{Coding Freedom} | Anthropology of free software movement |\\
| Kelty, \textit{Two Bits} | The cultural significance of free software |\\
\end{quote}

\#\# Key Articles/Shorter Readings

- Aaron Swartz, "Guerrilla Open Access Manifesto"
- Ganz, "What Is Public Narrative" (Harvard Kennedy School)
- Jo Freeman, "The Tyranny of Structurelessness" (on making governance explicit)
- Stallman, "Why Software Should Be Free"

\bigskip\noindent\rule{\textwidth}{0.4pt}\bigskip

\# Part 10: The Vision Document

\textit{After working through this worksheet, draft a 1-page vision document that articulates the full scope of what FreeMoCap is building toward. This becomes your north star.}

\#\# Draft Vision Document

\textbf{The Problem:}

```
\_\_\_\_\_\_\_\_\_\_\_\_\_\_\_\_\_\_\_\_\_\_\_\_\_\_\_\_\_\_\_\_\_\_\_\_\_\_\_\_\_\_\_\_\_\_\_\_\_\_\_\_\_\_\_\_\_\_\_\_\_\_\_\_\_\_\_\_\_\_\_\_\_\_\_\_\_
\_\_\_\_\_\_\_\_\_\_\_\_\_\_\_\_\_\_\_\_\_\_\_\_\_\_\_\_\_\_\_\_\_\_\_\_\_\_\_\_\_\_\_\_\_\_\_\_\_\_\_\_\_\_\_\_\_\_\_\_\_\_\_\_\_\_\_\_\_\_\_\_\_\_\_\_\_
\_\_\_\_\_\_\_\_\_\_\_\_\_\_\_\_\_\_\_\_\_\_\_\_\_\_\_\_\_\_\_\_\_\_\_\_\_\_\_\_\_\_\_\_\_\_\_\_\_\_\_\_\_\_\_\_\_\_\_\_\_\_\_\_\_\_\_\_\_\_\_\_\_\_\_\_\_
\_\_\_\_\_\_\_\_\_\_\_\_\_\_\_\_\_\_\_\_\_\_\_\_\_\_\_\_\_\_\_\_\_\_\_\_\_\_\_\_\_\_\_\_\_\_\_\_\_\_\_\_\_\_\_\_\_\_\_\_\_\_\_\_\_\_\_\_\_\_\_\_\_\_\_\_\_
\_\_\_\_\_\_\_\_\_\_\_\_\_\_\_\_\_\_\_\_\_\_\_\_\_\_\_\_\_\_\_\_\_\_\_\_\_\_\_\_\_\_\_\_\_\_\_\_\_\_\_\_\_\_\_\_\_\_\_\_\_\_\_\_\_\_\_\_\_\_\_\_\_\_\_\_\_
```

\textbf{The Vision:}

```
\_\_\_\_\_\_\_\_\_\_\_\_\_\_\_\_\_\_\_\_\_\_\_\_\_\_\_\_\_\_\_\_\_\_\_\_\_\_\_\_\_\_\_\_\_\_\_\_\_\_\_\_\_\_\_\_\_\_\_\_\_\_\_\_\_\_\_\_\_\_\_\_\_\_\_\_\_
\_\_\_\_\_\_\_\_\_\_\_\_\_\_\_\_\_\_\_\_\_\_\_\_\_\_\_\_\_\_\_\_\_\_\_\_\_\_\_\_\_\_\_\_\_\_\_\_\_\_\_\_\_\_\_\_\_\_\_\_\_\_\_\_\_\_\_\_\_\_\_\_\_\_\_\_\_
\_\_\_\_\_\_\_\_\_\_\_\_\_\_\_\_\_\_\_\_\_\_\_\_\_\_\_\_\_\_\_\_\_\_\_\_\_\_\_\_\_\_\_\_\_\_\_\_\_\_\_\_\_\_\_\_\_\_\_\_\_\_\_\_\_\_\_\_\_\_\_\_\_\_\_\_\_
\_\_\_\_\_\_\_\_\_\_\_\_\_\_\_\_\_\_\_\_\_\_\_\_\_\_\_\_\_\_\_\_\_\_\_\_\_\_\_\_\_\_\_\_\_\_\_\_\_\_\_\_\_\_\_\_\_\_\_\_\_\_\_\_\_\_\_\_\_\_\_\_\_\_\_\_\_
\_\_\_\_\_\_\_\_\_\_\_\_\_\_\_\_\_\_\_\_\_\_\_\_\_\_\_\_\_\_\_\_\_\_\_\_\_\_\_\_\_\_\_\_\_\_\_\_\_\_\_\_\_\_\_\_\_\_\_\_\_\_\_\_\_\_\_\_\_\_\_\_\_\_\_\_\_
```

\textbf{The Theory of Change:}

```
\_\_\_\_\_\_\_\_\_\_\_\_\_\_\_\_\_\_\_\_\_\_\_\_\_\_\_\_\_\_\_\_\_\_\_\_\_\_\_\_\_\_\_\_\_\_\_\_\_\_\_\_\_\_\_\_\_\_\_\_\_\_\_\_\_\_\_\_\_\_\_\_\_\_\_\_\_
\_\_\_\_\_\_\_\_\_\_\_\_\_\_\_\_\_\_\_\_\_\_\_\_\_\_\_\_\_\_\_\_\_\_\_\_\_\_\_\_\_\_\_\_\_\_\_\_\_\_\_\_\_\_\_\_\_\_\_\_\_\_\_\_\_\_\_\_\_\_\_\_\_\_\_\_\_
\_\_\_\_\_\_\_\_\_\_\_\_\_\_\_\_\_\_\_\_\_\_\_\_\_\_\_\_\_\_\_\_\_\_\_\_\_\_\_\_\_\_\_\_\_\_\_\_\_\_\_\_\_\_\_\_\_\_\_\_\_\_\_\_\_\_\_\_\_\_\_\_\_\_\_\_\_
\_\_\_\_\_\_\_\_\_\_\_\_\_\_\_\_\_\_\_\_\_\_\_\_\_\_\_\_\_\_\_\_\_\_\_\_\_\_\_\_\_\_\_\_\_\_\_\_\_\_\_\_\_\_\_\_\_\_\_\_\_\_\_\_\_\_\_\_\_\_\_\_\_\_\_\_\_
\_\_\_\_\_\_\_\_\_\_\_\_\_\_\_\_\_\_\_\_\_\_\_\_\_\_\_\_\_\_\_\_\_\_\_\_\_\_\_\_\_\_\_\_\_\_\_\_\_\_\_\_\_\_\_\_\_\_\_\_\_\_\_\_\_\_\_\_\_\_\_\_\_\_\_\_\_
```

\textbf{The Current Stage:}

```
\_\_\_\_\_\_\_\_\_\_\_\_\_\_\_\_\_\_\_\_\_\_\_\_\_\_\_\_\_\_\_\_\_\_\_\_\_\_\_\_\_\_\_\_\_\_\_\_\_\_\_\_\_\_\_\_\_\_\_\_\_\_\_\_\_\_\_\_\_\_\_\_\_\_\_\_\_
\_\_\_\_\_\_\_\_\_\_\_\_\_\_\_\_\_\_\_\_\_\_\_\_\_\_\_\_\_\_\_\_\_\_\_\_\_\_\_\_\_\_\_\_\_\_\_\_\_\_\_\_\_\_\_\_\_\_\_\_\_\_\_\_\_\_\_\_\_\_\_\_\_\_\_\_\_
\_\_\_\_\_\_\_\_\_\_\_\_\_\_\_\_\_\_\_\_\_\_\_\_\_\_\_\_\_\_\_\_\_\_\_\_\_\_\_\_\_\_\_\_\_\_\_\_\_\_\_\_\_\_\_\_\_\_\_\_\_\_\_\_\_\_\_\_\_\_\_\_\_\_\_\_\_
\_\_\_\_\_\_\_\_\_\_\_\_\_\_\_\_\_\_\_\_\_\_\_\_\_\_\_\_\_\_\_\_\_\_\_\_\_\_\_\_\_\_\_\_\_\_\_\_\_\_\_\_\_\_\_\_\_\_\_\_\_\_\_\_\_\_\_\_\_\_\_\_\_\_\_\_\_
```

\textbf{The Call:}

```
\_\_\_\_\_\_\_\_\_\_\_\_\_\_\_\_\_\_\_\_\_\_\_\_\_\_\_\_\_\_\_\_\_\_\_\_\_\_\_\_\_\_\_\_\_\_\_\_\_\_\_\_\_\_\_\_\_\_\_\_\_\_\_\_\_\_\_\_\_\_\_\_\_\_\_\_\_
\_\_\_\_\_\_\_\_\_\_\_\_\_\_\_\_\_\_\_\_\_\_\_\_\_\_\_\_\_\_\_\_\_\_\_\_\_\_\_\_\_\_\_\_\_\_\_\_\_\_\_\_\_\_\_\_\_\_\_\_\_\_\_\_\_\_\_\_\_\_\_\_\_\_\_\_\_
\_\_\_\_\_\_\_\_\_\_\_\_\_\_\_\_\_\_\_\_\_\_\_\_\_\_\_\_\_\_\_\_\_\_\_\_\_\_\_\_\_\_\_\_\_\_\_\_\_\_\_\_\_\_\_\_\_\_\_\_\_\_\_\_\_\_\_\_\_\_\_\_\_\_\_\_\_
```

\bigskip\noindent\rule{\textwidth}{0.4pt}\bigskip

\textit{This is a living document. The work of building counter-hegemonic power is never complete. Return to these questions. Revise your answers. Share them with co-conspirators. Build the infrastructure for a long fight.}

\textit{The current system wasn't built in a day. It won't fall in a day. But it WILL fall.}
\normalsize


\chapter{Liberation Movement Compendium}
\label{src:LC}

\begin{framed}
\small
\textbf{Source marker:} LC \\
\textbf{Date:} February 2026 \\
\textbf{Source:} Complete reference compendium
\end{framed}

\small
\textbackslash{}documentclass[11pt,letterpaper,twoside]\{book\}

\% ============================================================================
\% PACKAGES
\% ============================================================================
\textbackslash{}usepackage[utf8]\{inputenc\}
\textbackslash{}usepackage[T1]\{fontenc\}
\textbackslash{}usepackage\{lmodern\}
\textbackslash{}usepackage[margin=1in]\{geometry\}
\textbackslash{}usepackage\{graphicx\}
\textbackslash{}usepackage\{xcolor\}
\textbackslash{}usepackage\{hyperref\}
\textbackslash{}usepackage\{enumitem\}
\textbackslash{}usepackage\{titlesec\}
\textbackslash{}usepackage\{fancyhdr\}
\textbackslash{}usepackage\{booktabs\}
\textbackslash{}usepackage\{longtable\}
\textbackslash{}usepackage\{array\}
\textbackslash{}usepackage\{tabularx\}
\textbackslash{}usepackage\{multirow\}
\textbackslash{}usepackage\{caption\}
\textbackslash{}usepackage\{epigraph\}
\textbackslash{}usepackage\{tcolorbox\}
\textbackslash{}usepackage\{framed\}
\textbackslash{}usepackage\{parskip\}
\textbackslash{}usepackage\{setspace\}
\textbackslash{}usepackage\{microtype\}
\textbackslash{}usepackage[style=authoryear-comp,backend=biber,maxcitenames=2]\{biblatex\}

\% ============================================================================
\% COLORS
\% ============================================================================
\textbackslash{}definecolor\{navy\}\{RGB\}\{26,26,46\}
\textbackslash{}definecolor\{accent\}\{RGB\}\{233,69,96\}
\textbackslash{}definecolor\{blue\}\{RGB\}\{15,52,96\}
\textbackslash{}definecolor\{lightgray\}\{RGB\}\{248,248,248\}
\textbackslash{}definecolor\{medgray\}\{RGB\}\{200,200,200\}

\% ============================================================================
\% HYPERREF SETUP
\% ============================================================================
\textbackslash{}hypersetup\{
    colorlinks=true,
    linkcolor=navy,
    citecolor=blue,
    urlcolor=accent,
    pdftitle=\{FreeMoCap as Counter-Hegemonic Project\},
    pdfauthor=\{FreeMoCap Foundation\},
\}

\% ============================================================================
\% HEADER/FOOTER
\% ============================================================================
\textbackslash{}pagestyle\{fancy\}
\textbackslash{}fancyhf\{\}
\textbackslash{}fancyhead[LE,RO]\{\textbackslash{}thepage\}
\textbackslash{}fancyhead[RE]\{\textbackslash{}leftmark\}
\textbackslash{}fancyhead[LO]\{\textbackslash{}rightmark\}
\textbackslash{}renewcommand\{\textbackslash{}headrulewidth\}\{0.4pt\}

\% ============================================================================
\% TITLE FORMATTING
\% ============================================================================
\textbackslash{}titleformat\{\textbackslash{}chapter\}[display]
  \{\textbackslash{}normalfont\textbackslash{}huge\textbackslash{}bfseries\textbackslash{}color\{navy\}\}
  \{\textbackslash{}chaptertitlename\textbackslash{} \textbackslash{}thechapter\}\{20pt\}\{\textbackslash{}Huge\}
\textbackslash{}titleformat\{\textbackslash{}section\}
  \{\textbackslash{}normalfont\textbackslash{}Large\textbackslash{}bfseries\textbackslash{}color\{navy\}\}
  \{\textbackslash{}thesection\}\{1em\}\{\}
\textbackslash{}titleformat\{\textbackslash{}subsection\}
  \{\textbackslash{}normalfont\textbackslash{}large\textbackslash{}bfseries\textbackslash{}color\{blue\}\}
  \{\textbackslash{}thesubsection\}\{1em\}\{\}

\% ============================================================================
\% CUSTOM ENVIRONMENTS
\% ============================================================================
\textbackslash{}tcbuselibrary\{skins,breakable\}

\textbackslash{}newtcolorbox\{keyinsight\}\{
  colback=accent!5,
  colframe=accent,
  fonttitle=\textbackslash{}bfseries,
  title=Key Insight,
  breakable
\}

\textbackslash{}newtcolorbox\{definition\}\{
  colback=blue!5,
  colframe=blue,
  fonttitle=\textbackslash{}bfseries,
  title=Definition,
  breakable
\}

\textbackslash{}newtcolorbox\{freemocapapp\}\{
  colback=lightgray,
  colframe=navy,
  fonttitle=\textbackslash{}bfseries,
  title=FreeMoCap Application,
  breakable
\}

\textbackslash{}newtcolorbox\{cautionary\}\{
  colback=accent!10,
  colframe=accent!70!black,
  fonttitle=\textbackslash{}bfseries,
  title=Cautionary Note,
  breakable
\}

\textbackslash{}newenvironment\{epigraphbox\}[2]\{\%
  \textbackslash{}begin\{quote\}
  \textbackslash{}itshape \#1
  \textbackslash{}begin\{flushright\}
  --- \textbackslash{}textup\{\#2\}
  \textbackslash{}end\{flushright\}
  \textbackslash{}end\{quote\}
\}\{\}

\% ============================================================================
\% DOCUMENT
\% ============================================================================
\textbackslash{}begin\{document\}

\% ============================================================================
\% TITLE PAGE
\% ============================================================================
\textbackslash{}begin\{titlepage\}
\textbackslash{}centering
\textbackslash{}vspace\textit{\{2cm\}

\{\textbackslash{}Huge\textbackslash{}bfseries\textbackslash{}color\{navy\} FreeMoCap as\textbackslash{}\textbackslash{}[0.3em] Counter-Hegemonic Project\textbackslash{}par\}

\textbackslash{}vspace\{1cm\}

\{\textbackslash{}Large\textbackslash{}color\{accent\}\textbackslash{}rule\{0.5\textbackslash{}textwidth\}\{2pt\}\textbackslash{}par\}

\textbackslash{}vspace\{1cm\}

\{\textbackslash{}Large A Strategic Framework for Liberating\textbackslash{}\textbackslash{}[0.3em] Knowledge, Education, and Science\textbackslash{}par\}

\textbackslash{}vspace\{2cm\}

\textbackslash{}begin\{tcolorbox\}[colback=lightgray,colframe=medgray,width=0.8\textbackslash{}textwidth]
\textbackslash{}small\textbackslash{}itshape
``The great revolution in the history of man, past, present and future, is the revolution of those determined to be free.''
\textbackslash{}begin\{flushright\}
--- \textbackslash{}textup\{Toussaint Louverture\}
\textbackslash{}end\{flushright\}

\textbackslash{}vspace\{0.5cm\}

``Information is power. But like all power, there are those who want to keep it for themselves.''
\textbackslash{}begin\{flushright\}
--- \textbackslash{}textup\{Aaron Swartz\}
\textbackslash{}end\{flushright\}

\textbackslash{}vspace\{0.5cm\}

\textbackslash{}normalfont\textbackslash{}small
\textbackslash{}textbf\{On tools:\} The Haitian revolutionaries got their muskets from the French. They got their military tactics from the French. They got the language of universal liberty from the French Revolution---then held France accountable to ideals France never intended for them. Tools are not ideologically fixed to their origins. The code doesn't know who's running it. \textbackslash{}textbf\{Appropriate everything.\}
\textbackslash{}end\{tcolorbox\}

\textbackslash{}vfill

\{\textbackslash{}large Based on the works of Smucker, Freire, Illich, McAlevey, Gramsci,\textbackslash{}\textbackslash{}
Benkler, Ostrom, and brown\textbackslash{}par\}

\textbackslash{}vspace\{1cm\}

\{\textbackslash{}large Prepared for the FreeMoCap Foundation\textbackslash{}par\}

\textbackslash{}vspace\{0.5cm\}

\{\textbackslash{}normalsize \textbackslash{}today\textbackslash{}par\}

\textbackslash{}end\{titlepage\}

\% ============================================================================
\% FRONT MATTER
\% ============================================================================
\textbackslash{}frontmatter

\textbackslash{}tableofcontents

\textbackslash{}chapter}\{Preface\}
\textbackslash{}addcontentsline\{toc\}\{chapter\}\{Preface\}

This document compiles a comprehensive strategic planning session applying political organizing theory to the FreeMoCap Foundation's mission of liberating knowledge production. It draws on Jonathan Smucker's \textbackslash{}textit\{Hegemony How-To\}, Paulo Freire's critical pedagogy, Ivan Illich's institutional critique, Jane McAlevey's organizing framework, Antonio Gramsci's theory of hegemony, Yochai Benkler's analysis of commons-based peer production, Elinor Ostrom's governance principles, and adrienne maree brown's emergent strategy.

The document includes:
\textbackslash{}begin\{itemize\}
\textbackslash{}item Strategic analysis of FreeMoCap as a counter-hegemonic project
\textbackslash{}item A comprehensive movement-building worksheet
\textbackslash{}item Deep dives into foundational theoretical texts
\textbackslash{}item Case studies of open source projects (Blender, GIMP, Krita, Inkscape)
\textbackslash{}item Actionable frameworks for building sustainable alternatives to institutional knowledge gatekeeping
\textbackslash{}end\{itemize\}

The central reframing: FreeMoCap is not merely ``open source motion capture software.'' It is \textbackslash{}textbf\{Stage 1 of a larger counter-hegemonic project\} to dismantle the interlocking systems of universities, journals, proprietary software, credentialism, and gatekeeping that determine who gets to participate in science, education, and knowledge creation.

\% ============================================================================
\% MAIN MATTER
\% ============================================================================
\textbackslash{}mainmatter

\% ============================================================================
\% PART I: STRATEGIC FRAMEWORK
\% ============================================================================
\textbackslash{}part\{Strategic Framework\}

\textbackslash{}chapter\{FreeMoCap and the Current Hegemony\}

\textbackslash{}section\{The Organization\}

The FreeMoCap Foundation is a 501(c)(3) nonprofit organization founded in 2021. Its flagship project, FreeMoCap, provides free, open-source markerless motion capture software that enables research-grade biomechanical analysis using standard webcams and computer vision techniques.

\textbackslash{}subsection\{Mission and Vision\}

The stated mission is to provide ``free motion capture for everyone''---democratizing access to technology that has traditionally required expensive proprietary systems costing tens to hundreds of thousands of dollars.

\textbackslash{}subsection\{Technical Achievement\}

FreeMoCap achieves markerless motion capture using:
\textbackslash{}begin\{itemize\}
\textbackslash{}item MediaPipe for pose estimation
\textbackslash{}item Anipose for 3D triangulation
\textbackslash{}item Standard webcams (no specialized hardware required)
\textbackslash{}item Python-based open architecture
\textbackslash{}end\{itemize\}

This technical stack represents a proof-of-concept that research-grade scientific tools can be free and accessible, that institutional affiliation and large budgets are not prerequisites for producing knowledge, and that the barriers to scientific participation are artificial.

\textbackslash{}section\{Mapping the Enemy: The Current Hegemony\}

\textbackslash{}begin\{definition\}
\textbackslash{}textbf\{Hegemony\} (Gramsci): The process by which ruling groups maintain power not primarily through force but by making their worldview seem like ``common sense''---natural, inevitable, and unquestionable.
\textbackslash{}end\{definition\}

FreeMoCap challenges an interlocking system of institutions that control knowledge production:

\textbackslash{}begin\{table\}[h]
\textbackslash{}centering
\textbackslash{}caption\{The Institutional Landscape of Knowledge Control\}
\textbackslash{}begin\{tabularx\}\{\textbackslash{}textwidth\}\{>\{\textbackslash{}bfseries\}l X X\}
\textbackslash{}toprule
Institution \& What It Controls \& Justifying Ideology \textbackslash{}\textbackslash{}
\textbackslash{}midrule
Universities \& Access to education, research resources, credentials \& ``Expertise requires formal training'' \textbackslash{}\textbackslash{}
Academic Journals \& What counts as ``real'' knowledge \& ``Peer review ensures quality'' \textbackslash{}\textbackslash{}
Proprietary Software \& Tools for knowledge production \& ``Professional tools cost money'' \textbackslash{}\textbackslash{}
Professional Associations \& Who counts as ``qualified'' \& ``Credentials protect the public'' \textbackslash{}\textbackslash{}
Funding Agencies \& What research gets done \& ``Competitive funding ensures quality'' \textbackslash{}\textbackslash{}
\textbackslash{}bottomrule
\textbackslash{}end\{tabularx\}
\textbackslash{}end\{table\}

\textbackslash{}subsection\{The Hegemonic Common Sense\}

Hegemony works by making contingent arrangements seem natural and inevitable. The current ``common sense'' includes beliefs such as:

\textbackslash{}begin\{itemize\}
\textbackslash{}item To be a real scientist, you need institutional affiliation
\textbackslash{}item Research isn't valid unless published in peer-reviewed journals
\textbackslash{}item Education requires accredited institutions
\textbackslash{}item If software/journals/education were free, quality would suffer
\textbackslash{}item People outside institutions can't do real research because they lack training and oversight
\textbackslash{}end\{itemize\}

These beliefs serve the interests of those who benefit from the current system while appearing to be neutral descriptions of reality.

\textbackslash{}section\{Counter-Hegemonic Vision\}

The counter-hegemonic vision inverts these assumptions:

\textbackslash{}begin\{itemize\}
\textbackslash{}item Anyone can do real science when given access to tools and knowledge
\textbackslash{}item Learning happens through practice, collaboration, and curiosity---not credential accumulation
\textbackslash{}item Knowledge is validated by reproducibility, utility, and community review
\textbackslash{}item Research tools should be free because knowledge is a public good
\textbackslash{}item The barriers are artificial constructions that benefit gatekeepers
\textbackslash{}end\{itemize\}

\textbackslash{}begin\{keyinsight\}
FreeMoCap motion capture software is \textbackslash{}textbf\{Stage 1\}---a concrete demonstration that research-grade scientific tools can be free and accessible. The \textbackslash{}textbf\{actual project\} is much larger: dismantling the hegemony of knowledge production itself.
\textbackslash{}end\{keyinsight\}


\textbackslash{}chapter\{Applied Strategic Frameworks\}

\textbackslash{}section\{Jonathan Smucker: Hegemony How-To\}

Smucker's \textbackslash{}textit\{Hegemony How-To: A Roadmap for Radicals\} provides the core strategic framework. Key concepts include:

\textbackslash{}subsection\{The Political Identity Paradox\}

\textbackslash{}begin\{definition\}
\textbackslash{}textbf\{Political Identity Paradox\}: Strong group identity is necessary for sustained commitment, but the same identity can create insularity that prevents building broader coalitions. Groups must balance internal cohesion (``bonding'') with external connection (``bridging'').
\textbackslash{}end\{definition\}

\textbackslash{}textbf\{Application to FreeMoCap\}: The open source / hacker / FOSS identity provides motivation and community, but can create barriers for researchers, educators, or practitioners who don't identify with that culture.

\textbackslash{}subsection\{Expressive vs. Strategic Activism\}

\textbackslash{}begin\{definition\}
\textbackslash{}textbf\{Expressive activism\}: Actions that express group identity and values, making participants feel good about themselves.\textbackslash{}\textbackslash{}
\textbackslash{}textbf\{Strategic activism\}: Actions designed to build power and achieve concrete goals, even when they don't feel as satisfying.
\textbackslash{}end\{definition\}

The key question: ``Is what we're doing likely to advance our mission, or does it primarily express our identity?''

\textbackslash{}subsection\{Core and Base Structure\}

Movements need both:
\textbackslash{}begin\{itemize\}
\textbackslash{}item \textbackslash{}textbf\{Core\}: Dedicated contributors who do sustained work
\textbackslash{}item \textbackslash{}textbf\{Base\}: Broader supporters who engage at lower intensity
\textbackslash{}end\{itemize\}

The core-base relationship requires explicit onramps---pathways from casual interest to deeper involvement.

\textbackslash{}section\{Jane McAlevey: Organizing vs. Mobilizing\}

McAlevey's \textbackslash{}textit\{No Shortcuts\} distinguishes three approaches:

\textbackslash{}begin\{enumerate\}
\textbackslash{}item \textbackslash{}textbf\{Advocacy\}: Professionals speak on behalf of a constituency (least powerful)
\textbackslash{}item \textbackslash{}textbf\{Mobilizing\}: Activating people who already agree with you
\textbackslash{}item \textbackslash{}textbf\{Organizing\}: Expanding your base by reaching people who don't yet identify with your cause (most powerful)
\textbackslash{}end\{enumerate\}

\textbackslash{}begin\{keyinsight\}
The critical question: Are you \textbackslash{}textbf\{organizing\} (expanding who participates in knowledge production) or just \textbackslash{}textbf\{mobilizing\} (activating people already convinced FOSS is good)?
\textbackslash{}end\{keyinsight\}

McAlevey emphasizes \textbackslash{}textbf\{organic leaders\}---respected figures within target constituencies who can bring others along. These are not self-selected activists but people with existing credibility in their communities.

\textbackslash{}section\{Marshall Ganz: Public Narrative\}

Ganz's framework structures persuasive storytelling:

\textbackslash{}begin\{enumerate\}
\textbackslash{}item \textbackslash{}textbf\{Story of Self\}: Why YOU are called to this work (personal encounter with injustice, choice to act, values at stake)
\textbackslash{}item \textbackslash{}textbf\{Story of Us\}: Who is WE? What shared experiences and values bind the community?
\textbackslash{}item \textbackslash{}textbf\{Story of Now\}: Why is THIS MOMENT critical? What urgent challenge demands action?
\textbackslash{}end\{enumerate\}

The three stories weave together: personal commitment connects to collective identity connects to urgent action.

\textbackslash{}section\{Paulo Freire: Critical Consciousness\}

Freire's \textbackslash{}textit\{Pedagogy of the Oppressed\} provides the educational philosophy:

\textbackslash{}subsection\{The Banking Model Critique\}

Traditional education treats students as empty vessels for knowledge deposits. Experts possess knowledge; students passively receive it. This model reproduces oppressive structures by training people to accept rather than question.

\textbackslash{}subsection\{Problem-Posing Education\}

The alternative: dialogic learning where teachers and students learn together, confronting real problems. Education becomes ``the practice of freedom'' rather than domination.

\textbackslash{}subsection\{Conscientização (Critical Consciousness)\}

Development of awareness moves through stages:
\textbackslash{}begin\{enumerate\}
\textbackslash{}item \textbackslash{}textbf\{Magical consciousness\}: Fatalistic acceptance---``that's just how things are''
\textbackslash{}item \textbackslash{}textbf\{Naive consciousness\}: Recognizes problems but blames individuals, not systems
\textbackslash{}item \textbackslash{}textbf\{Critical consciousness\}: Sees structural causes, understands situation historically/politically, recognizes capacity for transformative action
\textbackslash{}end\{enumerate\}

\textbackslash{}begin\{freemocapapp\}
FreeMoCap can be a site for developing critical consciousness about knowledge production itself. Moving people from ``motion capture is expensive, that's just how it is'' (magical) through ``some companies benefit from scarcity'' (naive) to ``the entire structure of proprietary tools, academic gatekeeping, and credentialism is a transformable system'' (critical).
\textbackslash{}end\{freemocapapp\}

\textbackslash{}subsection\{Hope as Ontological Necessity\}

In \textbackslash{}textit\{Pedagogy of Hope\}, Freire argues that hope is not naive optimism but a foundational commitment that transformation is possible. Against neoliberal ``there is no alternative'' fatalism, Freire insists that another world is possible---and that believing this is a prerequisite for building it.


\textbackslash{}chapter\{Movement-Building Worksheet\}

This chapter presents a strategic worksheet for developing FreeMoCap as a counter-hegemonic project. It is designed for individual completion followed by collective discussion.

\textbackslash{}section\{Part 1: Mapping the Enemy\}

\textbackslash{}subsection\{Exercise: Hegemonic Common Sense\}

Complete these sentences as a ``normal person'' embedded in current structures would:

\textbackslash{}begin\{enumerate\}
\textbackslash{}item To be a real scientist, you need \textbackslash{}rule\{8cm\}\{0.4pt\}
\textbackslash{}item Research isn't valid unless \textbackslash{}rule\{8cm\}\{0.4pt\}
\textbackslash{}item Education requires \textbackslash{}rule\{8cm\}\{0.4pt\}
\textbackslash{}item If software/journals/education were free, \textbackslash{}rule\{8cm\}\{0.4pt\}
\textbackslash{}item People outside institutions can't do real research because \textbackslash{}rule\{8cm\}\{0.4pt\}
\textbackslash{}end\{enumerate\}

\textbackslash{}textbf\{Reflection\}: Where did these beliefs come from? Who benefits from people believing them?

\textbackslash{}subsection\{Cracks in the Edifice\}

No hegemony is total. Identify where the current system is failing:

\textbackslash{}begin\{table\}[h]
\textbackslash{}centering
\textbackslash{}begin\{tabularx\}\{\textbackslash{}textwidth\}\{l X X\}
\textbackslash{}toprule
\textbackslash{}textbf\{System Failure\} \& \textbackslash{}textbf\{Who Experiences This?\} \& \textbackslash{}textbf\{What Do They Blame?\} \textbackslash{}\textbackslash{}
\textbackslash{}midrule
Reproducibility crisis \& Researchers, public \& ``Bad actors'' \textbackslash{}\textbackslash{}
Student debt crisis \& Students, families \& ``Expensive schools'' \textbackslash{}\textbackslash{}
Journal paywalls \& Global South academics \& ``Greedy publishers'' \textbackslash{}\textbackslash{}
Adjunctification \& Early-career academics \& ``Budget cuts'' \textbackslash{}\textbackslash{}
Proprietary lock-in \& Researchers, educators \& Specific vendors \textbackslash{}\textbackslash{}
\textbackslash{}bottomrule
\textbackslash{}end\{tabularx\}
\textbackslash{}end\{table\}

\textbackslash{}textbf\{Key question\}: How could you help people see the SYSTEMIC cause rather than blaming individuals?

\textbackslash{}section\{Part 2: Theory of Change\}

Complete this theory of change:

\textbackslash{}begin\{tcolorbox\}[colback=lightgray,colframe=medgray]
\textbackslash{}textbf\{If\} FreeMoCap \textbackslash{}rule\{6cm\}\{0.4pt\}

\textbackslash{}textbf\{Then\} [constituency] will \textbackslash{}rule\{6cm\}\{0.4pt\}

\textbackslash{}textbf\{Which will cause\} \textbackslash{}rule\{6cm\}\{0.4pt\}

\textbackslash{}textbf\{Leading to\} \textbackslash{}rule\{6cm\}\{0.4pt\}

\textbackslash{}textbf\{Until eventually\} \textbackslash{}rule\{6cm\}\{0.4pt\}
\textbackslash{}end\{tcolorbox\}

\textbackslash{}subsection\{The Stages\}

If FreeMoCap motion capture is Stage 1, what are subsequent stages?

\textbackslash{}begin\{table\}[h]
\textbackslash{}centering
\textbackslash{}begin\{tabularx\}\{\textbackslash{}textwidth\}\{c X X X\}
\textbackslash{}toprule
\textbackslash{}textbf\{Stage\} \& \textbackslash{}textbf\{Focus\} \& \textbackslash{}textbf\{Goal\} \& \textbackslash{}textbf\{Enables Next By...\} \textbackslash{}\textbackslash{}
\textbackslash{}midrule
1 \& Motion capture \& Prove FOSS can match proprietary \& Builds credibility \textbackslash{}\textbackslash{}
2 \& \& \& \textbackslash{}\textbackslash{}
3 \& \& \& \textbackslash{}\textbackslash{}
4 \& \& \& \textbackslash{}\textbackslash{}
5 \& \& \& \textbackslash{}\textbackslash{}
\textbackslash{}bottomrule
\textbackslash{}end\{tabularx\}
\textbackslash{}end\{table\}

\textbackslash{}section\{Part 3: Public Narrative Development\}

\textbackslash{}subsection\{Story of Self\}

\textbackslash{}begin\{enumerate\}
\textbackslash{}item \textbackslash{}textbf\{The Encounter\}: When did you first experience the injustice of the current system?
\textbackslash{}item \textbackslash{}textbf\{The Choice\}: What moment committed you to this fight?
\textbackslash{}item \textbackslash{}textbf\{The Values\}: What deep values drive this? (Not ``open source is good''---the values \textbackslash{}textit\{underneath\})
\textbackslash{}end\{enumerate\}

\textbackslash{}subsection\{Story of Us\}

Who is the ``we'' fighting for liberated knowledge? Much bigger than ``FreeMoCap users.''

The broadest possible ``we''---who has been harmed/excluded:
\textbackslash{}begin\{itemize\}
\textbackslash{}item Students crushed by debt for credentials that gatekeep knowledge
\textbackslash{}item Researchers at under-resourced institutions
\textbackslash{}item Global South academics locked out by paywalls
\textbackslash{}item Independent researchers dismissed as ``not real scientists''
\textbackslash{}item Curious people told ``you're not qualified''
\textbackslash{}end\{itemize\}

\textbackslash{}subsection\{Story of Now\}

What's happening NOW that creates urgency? (AI democratization/threat, reproducibility crisis, student debt crisis, climate emergency requiring distributed research...)

\textbackslash{}section\{Part 4: Strategic Questions\}

Return to these monthly:

\textbackslash{}begin\{tcolorbox\}[colback=accent!5,colframe=accent]
\textbackslash{}begin\{enumerate\}
\textbackslash{}item ``Is what we're doing likely to advance our mission, or does it primarily express our identity?''
\textbackslash{}item ``Are we building power, or performing righteousness?''
\textbackslash{}item ``Who are we bringing IN who wasn't here before?''
\textbackslash{}item ``What have we WON recently?''
\textbackslash{}end\{enumerate\}
\textbackslash{}end\{tcolorbox\}

\textbackslash{}subsection\{Sustainability Questions\}

\textbackslash{}begin\{enumerate\}
\textbackslash{}item ``Can this continue without Jon?''
\textbackslash{}item ``What happens if a key contributor burns out?''
\textbackslash{}item ``Are we moving at a sustainable pace?''
\textbackslash{}end\{enumerate\}


\% ============================================================================
\% PART II: THEORETICAL FOUNDATIONS
\% ============================================================================
\textbackslash{}part\{Theoretical Foundations\}

\textbackslash{}chapter\{Ivan Illich: Institutional Critique\}

\textbackslash{}section\{Deschooling Society (1971)\}

Ivan Illich was an Austrian priest who spent most of his life working in Latin America and became one of the most radical critics of modern institutions.

\textbackslash{}subsection\{Core Argument\}

Schools teach students to confuse process with substance---to equate teaching with learning, grade advancement with education, a diploma with competence. Universal education through schooling is not feasible, and wouldn't be even if attempted through alternative institutions built on the style of present schools.

\textbackslash{}subsection\{Learning Webs\}

Instead of schools, Illich proposes ``educational webs which heighten the opportunity for each one to transform each moment of his living into one of learning, sharing, and caring.''

He proposed four Learning Networks that anticipated the internet:
\textbackslash{}begin\{enumerate\}
\textbackslash{}item \textbackslash{}textbf\{Reference services to educational objects\}: Like a search engine for learning resources
\textbackslash{}item \textbackslash{}textbf\{Skills exchanges\}: Databases of people willing to share skills
\textbackslash{}item \textbackslash{}textbf\{Peer-matching networks\}: Connecting learners with similar interests
\textbackslash{}item \textbackslash{}textbf\{Professional educators available on demand\}
\textbackslash{}end\{enumerate\}

\textbackslash{}begin\{freemocapapp\}
Illich's critique of how institutions create artificial scarcity and dependency maps directly onto the academic/proprietary software complex. His vision of peer-to-peer learning networks is essentially what open source communities can be.
\textbackslash{}end\{freemocapapp\}

\textbackslash{}section\{Tools for Conviviality (1973)\}

This book extends Illich's institutional critique to technology itself.

\textbackslash{}subsection\{The Two Watersheds\}

Illich's central insight is that modern tools and institutions pass through two critical thresholds:

\textbackslash{}textbf\{First watershed\}: A new approach (medicine, education, transportation) genuinely improves things. Anaesthesia, antibiotics, basic literacy---real benefits.

\textbackslash{}textbf\{Second watershed\}: The same approach, pushed further, becomes \textbackslash{}textit\{counterproductive\}. Medicine creates iatrogenic (doctor-caused) illness. Schools create ignorance and dependency. Cars destroy walkable cities.

\textbackslash{}begin\{definition\}
\textbackslash{}textbf\{Counterproductivity\}: When institutions of modern industrial society impede their purported aims. The medical system produces illness; the educational system produces ignorance; the judicial system perpetuates injustice.
\textbackslash{}end\{definition\}

\textbackslash{}subsection\{Radical Monopoly\}

\textbackslash{}begin\{definition\}
\textbackslash{}textbf\{Radical monopoly\}: When one \textbackslash{}textit\{type\} of solution (not one brand) becomes compulsory, excluding alternatives. ``That motor traffic curtains the right to walk, not that more people drive Chevies than Ford's, constitutes radical monopoly.''
\textbackslash{}end\{definition\}

Cars didn't just compete with walking---they reshaped cities so that walking became impossible. The ``choice'' to drive became mandatory.

Illich calculated: In 1970s America, if you add time working to afford a car, time driving (including traffic), time in healthcare from accidents, time in oil industry to fuel cars... the average American traveled about 6 km/hour. Roughly walking speed.

\textbackslash{}subsection\{Convivial vs. Manipulative Tools\}

\textbackslash{}begin\{definition\}
\textbackslash{}textbf\{Conviviality\}: ``Autonomous and creative intercourse among persons, and the intercourse of persons with their environment---in contrast with the conditioned response of persons to the demands made upon them by others, and by a man-made environment.''
\textbackslash{}end\{definition\}

\textbackslash{}textbf\{Convivial tools\}:
\textbackslash{}begin\{itemize\}
\textbackslash{}item Can be used by anyone without special training
\textbackslash{}item Don't create dependency on experts
\textbackslash{}item Enhance human autonomy and creativity
\textbackslash{}item Allow multiple uses
\textbackslash{}item Don't require institutional permission
\textbackslash{}end\{itemize\}

The telephone was Illich's example of a high-tech convivial tool---anyone could use it to say anything to anyone. Most hand tools are convivial unless artificially restricted through institutional arrangements (licensing requirements, proprietary formats, etc.).

\textbackslash{}begin\{keyinsight\}
The distinction between convivial and manipulatory tools is \textbackslash{}textbf\{independent of the level of technology\}. High technology can be convivial; simple tools can be manipulative.
\textbackslash{}end\{keyinsight\}

\textbackslash{}subsection\{Influence on Personal Computing\}

The book's vision of tools developed and maintained by a community of users significantly influenced the first developers of the personal computer, notably Lee Felsenstein. The personal computer was explicitly conceived as a convivial tool---something anyone could use without permission from institutions.

\textbackslash{}begin\{freemocapapp\}
The entire proprietary motion capture industry represents a \textbackslash{}textbf\{radical monopoly\}: not just expensive brands, but a TYPE of solution (institutionally-controlled, expert-mediated, credential-requiring) that excludes amateur, independent, or community-based alternatives. FreeMoCap represents a convivial alternative---research-grade capability without institutional gatekeeping.

Key questions from Illich:
\textbackslash{}begin\{itemize\}
\textbackslash{}item Is FreeMoCap approaching its second watershed? (When does community support burden exceed benefits?)
\textbackslash{}item Does FreeMoCap create new dependencies even as it removes old ones?
\textbackslash{}item What institutional arrangements might restrict FreeMoCap's convivial potential?
\textbackslash{}end\{itemize\}
\textbackslash{}end\{freemocapapp\}


\textbackslash{}chapter\{Paulo Freire: Critical Pedagogy\}

\textbackslash{}section\{Biography\}

Paulo Freire (1921--1997) was born in Recife, Brazil. His family was destroyed by the 1929 Depression---he experienced hunger while doing homework, and later wrote that ``words became pieces of food.'' He started literacy projects in 1947 and was imprisoned for 70 days after the 1964 US-backed coup, then exiled for 16 years.

He wrote \textbackslash{}textit\{Pedagogy of the Oppressed\} in 1967--68 in Chile, based on his work with peasants. He returned to Brazil in 1980 and became São Paulo's Education Minister in 1988.

\textbackslash{}section\{Pedagogy of the Oppressed (1968)\}

\textbackslash{}subsection\{The Banking Model\}

Traditional education treats students as empty vessels into which teachers deposit knowledge. Students memorize and regurgitate; teachers possess expertise and authority. This model reproduces oppressive structures by training people to accept rather than question.

\textbackslash{}subsection\{Problem-Posing Education\}

The alternative is dialogic learning where teachers and students learn together, confronting real problems. The teacher is no longer merely ``the-one-who-teaches'' but one who is taught in dialogue with the students.

\textbackslash{}subsection\{Conscientização\}

Development of critical consciousness moves through stages:

\textbackslash{}begin\{enumerate\}
\textbackslash{}item \textbackslash{}textbf\{Magical consciousness\}: Fatalistic acceptance. Suffering is natural/inevitable. ``That's just how things are.''
\textbackslash{}item \textbackslash{}textbf\{Naive consciousness\}: Recognizes problems but blames individuals or bad actors, not systems. ``Some companies are greedy'' rather than ``the system incentivizes this.''
\textbackslash{}item \textbackslash{}textbf\{Critical consciousness\}: Sees structural causes, understands situation historically and politically, recognizes capacity for transformative action. ``This system was built by particular people for particular purposes, and it can be transformed.''
\textbackslash{}end\{enumerate\}

\textbackslash{}subsection\{Praxis\}

\textbackslash{}begin\{definition\}
\textbackslash{}textbf\{Praxis\}: The unity of reflection and action. Neither alone is sufficient for liberation. Action without reflection is mere activism; reflection without action is mere verbalism.
\textbackslash{}end\{definition\}

\textbackslash{}subsection\{Dialogue\}

True dialogue requires:
\textbackslash{}begin\{itemize\}
\textbackslash{}item \textbackslash{}textbf\{Love\}: Commitment to the other's flourishing
\textbackslash{}item \textbackslash{}textbf\{Humility\}: Recognition that no one knows everything
\textbackslash{}item \textbackslash{}textbf\{Faith in people\}: Belief in human capacity for transformation
\textbackslash{}item \textbackslash{}textbf\{Hope\}: Conviction that change is possible
\textbackslash{}item \textbackslash{}textbf\{Critical thinking\}: Willingness to question assumptions
\textbackslash{}end\{itemize\}

Dialogue is not monologue, not hostile argument, not depositing ideas---it is an encounter between persons mediated by the world to ``name the world.''

\textbackslash{}subsection\{Fear of Freedom\}

Freire noted that the oppressed often fear liberation---the responsibility, uncertainty, and change it brings. People accustomed to oppressive structures may resist alternatives precisely because freedom is uncomfortable.

\textbackslash{}section\{Pedagogy of Hope (1992)\}

Written 25 years after \textbackslash{}textit\{Pedagogy of the Oppressed\}, this book revisits and deepens Freire's themes.

\textbackslash{}subsection\{Hope as Ontological Necessity\}

Hope is not naive optimism but a foundational commitment that transformation is possible. Against neoliberal ``there is no alternative'' ideology, Freire insists that another world is possible---and that believing this is prerequisite for building it.

\textbackslash{}begin\{keyinsight\}
Hope is an ``ontological need.'' Without it, we cannot even begin the work of transformation. Despair is not realism; it is surrender.
\textbackslash{}end\{keyinsight\}

\textbackslash{}subsection\{Response to Critics\}

Freire addresses criticism about the sexist language in \textbackslash{}textit\{Pedagogy of the Oppressed\} and changes his writing accordingly---modeling the capacity for growth and self-critique he advocates.

\textbackslash{}begin\{freemocapapp\}
\textbackslash{}begin\{enumerate\}
\textbackslash{}item \textbackslash{}textbf\{Banking model applies to technical education\}: Experts deposit knowledge, students memorize/regurgitate. FreeMoCap can be a site for problem-posing education where people learn by doing, in dialogue, confronting real problems.
\textbackslash{}item \textbackslash{}textbf\{Conscientização applies to knowledge system itself\}: Moving people from ``motion capture is expensive, that's just how it is'' (magical) to ``entire structure of proprietary tools + academic gatekeeping + credentialism is transformable system'' (critical).
\textbackslash{}item \textbackslash{}textbf\{Hope as ontological need\}: Neoliberal ``there is no alternative'' ideology is what FreeMoCap challenges. The project embodies conviction another arrangement is possible.
\textbackslash{}item \textbackslash{}textbf\{Praxis\}: FreeMoCap is action and reflection together---building software while developing critical understanding of why it matters.
\textbackslash{}item \textbackslash{}textbf\{Fear of freedom\}: People accustomed to proprietary tools may resist free alternatives precisely because freedom is uncomfortable.
\textbackslash{}end\{enumerate\}
\textbackslash{}end\{freemocapapp\}


\textbackslash{}chapter\{Jane McAlevey: Organizing for Power\}

\textbackslash{}section\{No Shortcuts (2016)\}

Jane McAlevey was a labor organizer who became an academic. \textbackslash{}textit\{No Shortcuts: Organizing for Power in the New Gilded Age\} distills decades of on-the-ground experience into a sharp strategic framework.

\textbackslash{}subsection\{Three Approaches to Change\}

\textbackslash{}begin\{enumerate\}
\textbackslash{}item \textbackslash{}textbf\{Advocacy\}: Professionals speak on behalf of a constituency. (Least powerful)
\textbackslash{}item \textbackslash{}textbf\{Mobilizing\}: Activating people who already agree with you.
\textbackslash{}item \textbackslash{}textbf\{Organizing\}: Expanding your base by reaching people who don't yet identify with your cause. (Most powerful)
\textbackslash{}end\{enumerate\}

\textbackslash{}begin\{keyinsight\}
The great social movements of previous eras gained their power from mass organizing, a strategy today's progressives have mostly abandoned in favor of shallow mobilization or advocacy.
\textbackslash{}end\{keyinsight\}

\textbackslash{}subsection\{Organic Leaders\}

\textbackslash{}begin\{definition\}
\textbackslash{}textbf\{Organic leaders\}: Respected figures within target constituencies who can bring others along. These are not self-selected activists but people with existing credibility in their communities.
\textbackslash{}end\{definition\}

Leadership development without previous leadership identification is ``a bicycle without wheels''---it severely limits how far a movement can go.

\textbackslash{}subsection\{Structure Tests\}

McAlevey emphasizes measurable tests of organizational power:
\textbackslash{}begin\{itemize\}
\textbackslash{}item Can you turn out members at scale?
\textbackslash{}item Can you win elections?
\textbackslash{}item Can you sustain a strike?
\textbackslash{}end\{itemize\}

\textbackslash{}subsection\{Whole-Worker Organizing\}

Engage workers as full human beings embedded in communities, not just as employees. Their concerns extend beyond the workplace; effective organizing connects to those broader concerns.

\textbackslash{}subsection\{Critique of Alinsky\}

McAlevey argues that Alinsky shifted the CIO organizing model to a mobilizing model by:
\textbackslash{}begin\{itemize\}
\textbackslash{}item Giving up on altering the power structure itself, in favor of winning campaigns within it
\textbackslash{}item Separating unskilled workers from natural allies
\textbackslash{}item Adopting top-down control of organizers
\textbackslash{}end\{itemize\}

\textbackslash{}begin\{freemocapapp\}
Key questions from McAlevey:
\textbackslash{}begin\{itemize\}
\textbackslash{}item Are you \textbackslash{}textbf\{organizing\} (expanding who participates in knowledge production) or just \textbackslash{}textbf\{mobilizing\} (activating people already convinced FOSS is good)?
\textbackslash{}item Who are the \textbackslash{}textbf\{organic leaders\} in your target communities (research labs, educational institutions, clinical settings)?
\textbackslash{}item What are your \textbackslash{}textbf\{structure tests\}---measurable indicators that you're building real power?
\textbackslash{}end\{itemize\}
\textbackslash{}end\{freemocapapp\}


\textbackslash{}chapter\{adrienne maree brown: Emergent Strategy\}

\textbackslash{}section\{Emergent Strategy (2017)\}

adrienne maree brown (lowercase intentional) draws on complexity science, Octavia Butler's science fiction, and her mentor Grace Lee Boggs to develop an adaptive, nature-inspired approach to social change.

\textbackslash{}subsection\{Core Definition\}

\textbackslash{}begin\{definition\}
\textbackslash{}textbf\{Emergent strategy\}: ``A strategy for building complex patterns and systems of change through relatively small interactions.''
\textbackslash{}end\{definition\}

\textbackslash{}subsection\{The Fractal Principle\}

\textbackslash{}begin\{keyinsight\}
``How we are at the small scale is how we are at the large scale. What we practice at the small scale can reverberate to the largest scale.''
\textbackslash{}end\{keyinsight\}

If your small group operates hierarchically, your movement will tend toward hierarchy. If your core team practices radical transparency, that will ripple outward. The patterns you establish in small interactions set the template for larger systems.

\textbackslash{}subsection\{Core Principles\}

\textbackslash{}begin\{itemize\}
\textbackslash{}item ``Small is good, small is all. (The large is a reflection of the small.)''
\textbackslash{}item ``Change is constant. (Be like water.)''
\textbackslash{}item ``There is always enough time for the right work.''
\textbackslash{}item ``Trust the People. (If you trust the people, they become trustworthy.)''
\textbackslash{}item ``Move at the speed of trust.''
\textbackslash{}item ``Focus on critical connections more than critical mass---build the resilience by building the relationships.''
\textbackslash{}item ``Never a failure, always a lesson.''
\textbackslash{}item ``Less prep, more presence.''
\textbackslash{}item ``What you pay attention to grows.''
\textbackslash{}end\{itemize\}

\textbackslash{}subsection\{Elements\}

\textbackslash{}begin\{itemize\}
\textbackslash{}item \textbackslash{}textbf\{Fractal\}: Patterns repeat at different scales
\textbackslash{}item \textbackslash{}textbf\{Adaptive\}: Flexibility in the face of change
\textbackslash{}item \textbackslash{}textbf\{Interdependent/Decentralized\}: Power distributed, connections emphasized
\textbackslash{}item \textbackslash{}textbf\{Nonlinear/Iterative\}: Progress isn't straight---cycles of growth, decomposition, regrowth
\textbackslash{}item \textbackslash{}textbf\{Resilient\}: Ability to withstand and recover from challenges
\textbackslash{}item \textbackslash{}textbf\{Transformative\}: Creating new possibilities
\textbackslash{}end\{itemize\}

\textbackslash{}subsection\{Relationship to Other Frameworks\}

Emergent Strategy is often seen as complement or corrective to more traditional organizing frameworks like McAlevey's. Where McAlevey emphasizes structure, discipline, and measurable outcomes, brown emphasizes relationship, adaptation, and trust.

\textbackslash{}begin\{cautionary\}
Some critics find brown's approach too vague---``almost everything that was either cogent or resonated with me in this book could be slotted in the `inspiration' category... there just isn't much in the way of strategy.''

But defenders argue this misses the point: emergent strategy isn't a playbook, it's an orientation. The ``strategy'' is trusting emergent processes rather than trying to control outcomes.
\textbackslash{}end\{cautionary\}

\textbackslash{}begin\{freemocapapp\}
brown's approach suggests:
\textbackslash{}begin\{itemize\}
\textbackslash{}item Pay attention to HOW the core team works together, not just what you produce
\textbackslash{}item ``Move at the speed of trust''---don't scale faster than relationships can sustain
\textbackslash{}item Accept nonlinear progress: failed features, abandoned approaches, pivots are not failures but compost
\textbackslash{}item ``What you pay attention to grows''---where is FreeMoCap's attention going?
\textbackslash{}end\{itemize\}

Potential tension with other frameworks: McAlevey would say ``structure tests'' and measurable organizing outcomes matter; brown would say ``critical connections more than critical mass.'' These aren't necessarily contradictory but require conscious balancing.
\textbackslash{}end\{freemocapapp\}


\textbackslash{}chapter\{Yochai Benkler: Commons-Based Peer Production\}

\textbackslash{}section\{The Wealth of Networks (2006)\}

Yochai Benkler is a Yale law professor who provided the theoretical foundation for understanding ``commons-based peer production''---the mode of production that creates Wikipedia, Linux, and open source generally.

\textbackslash{}subsection\{The Central Argument\}

Standard economic theory said there were two ways to organize production: \textbackslash{}textbf\{markets\} (price signals coordinate individual action) and \textbackslash{}textbf\{firms\} (hierarchical management coordinates action). Benkler identified a third:

\textbackslash{}begin\{definition\}
\textbackslash{}textbf\{Commons-based peer production\}: ``Radically decentralized, collaborative, and nonproprietary; based on sharing resources and outputs among widely distributed, loosely connected individuals who cooperate with each other without relying on either market signals or managerial commands.''
\textbackslash{}end\{definition\}

\textbackslash{}subsection\{Why It Works\}

Ronald Coase's theory of transaction costs explains why firms exist: the costs of coordinating through markets (finding suppliers, negotiating contracts, enforcing agreements) sometimes exceed the costs of hierarchical organization. Firms internalize these transactions.

Benkler's insight: When technology dropped the cost of communications, distributed the material capital necessary for knowledge work throughout a large population, and allowed individuals to share designs and incremental improvements with each other, a third option emerged. These individuals were able to pool their knowledge and resources, and coordinate action toward shared goals, without the mediation of firm hierarchies or markets.

\textbackslash{}begin\{keyinsight\}
``It is not intuitive that thousands of volunteers could beat the largest and best financed business enterprises in the world at their own game. Yet they do.''
\textbackslash{}end\{keyinsight\}

\textbackslash{}subsection\{What Enables Peer Production\}

\textbackslash{}begin\{itemize\}
\textbackslash{}item \textbackslash{}textbf\{Low capital requirements\}: You need a computer, not a factory
\textbackslash{}item \textbackslash{}textbf\{Modular tasks\}: Work can be broken into small, independent pieces
\textbackslash{}item \textbackslash{}textbf\{Low-cost integration\}: Digital tools make combining contributions cheap
\textbackslash{}item \textbackslash{}textbf\{Intrinsic motivation\}: People contribute for non-monetary reasons
\textbackslash{}end\{itemize\}

\textbackslash{}subsection\{Motivation Beyond Money\}

Commons-based projects are the results of individuals acting ``out of social and psychological motivations to do something interesting.'' The range includes pleasure, socially/psychologically rewarding experiences, reputation, learning, and economic calculation of possible adjacent benefits.

This matters because peer production isn't charity or altruism in the conventional sense---it's a different economic logic that can be more efficient than markets or firms for information goods.

\textbackslash{}subsection\{Cultural and Political Implications\}

The networked information economy ``boosts the potential of freedom and autonomy and shifts the balance of power between markets, the State, and civil society.''

The basic tools enabled by the Internet---cutting, pasting, rendering, annotating, commenting---make active utilization and conscious discussion of cultural symbols and artifacts easier to create, sustain, and read.

\textbackslash{}begin\{freemocapapp\}
Benkler provides the theoretical foundation for understanding WHY open source motion capture can exist at all---and why it can potentially outcompete proprietary solutions:
\textbackslash{}begin\{itemize\}
\textbackslash{}item \textbackslash{}textbf\{Low marginal cost\}: Once FreeMoCap exists, distributing it costs essentially nothing
\textbackslash{}item \textbackslash{}textbf\{Modularity\}: Different people can work on different components
\textbackslash{}item \textbackslash{}textbf\{Intrinsic motivation\}: Contributors may be motivated by interest, reputation, ideology, or adjacent benefits (learning, career)
\textbackslash{}item \textbackslash{}textbf\{Commons as resource\}: FreeMoCap benefits from and contributes to broader Python/computer vision commons
\textbackslash{}end\{itemize\}

Key question: What are the limits of peer production for FreeMoCap? Benkler's model works best for information goods---but FreeMoCap interfaces with physical hardware, institutional contexts, and sustained maintenance needs that may require hybrid models.
\textbackslash{}end\{freemocapapp\}


\textbackslash{}chapter\{Antonio Gramsci: Hegemony and Counter-Hegemony\}

\textbackslash{}section\{The Prison Notebooks (1929--1935)\}

Antonio Gramsci (1891--1937) was an Italian Marxist who led the Italian Communist Party before being imprisoned by Mussolini's fascist regime in 1926. He wrote over 30 notebooks and 3,000 pages in prison, under censorship, using coded language. He died shortly after release.

\textbackslash{}subsection\{The Central Concept: Hegemony\}

\textbackslash{}begin\{definition\}
\textbackslash{}textbf\{Hegemony\}: Capitalism maintains control not just through violence and political/economic coercion, but ideologically, through a hegemonic culture in which the values of the ruling class become the ``common sense'' values of all.
\textbackslash{}end\{definition\}

A consensus culture develops in which working-class people identify their own good with the good of the dominant class, helping maintain the status quo rather than revolting.

\textbackslash{}subsection\{Common Sense (Senso Comune)\}

``Common sense'' for Gramsci isn't good judgment---it's the accumulated, contradictory, often incoherent set of beliefs that ordinary people absorb from their culture. Much of this common sense serves the interests of dominant classes without people realizing it.

Counter-hegemonic work involves transforming common sense---making different arrangements seem natural and obvious.

\textbackslash{}subsection\{Civil Society vs. Political Society\}

\textbackslash{}begin\{itemize\}
\textbackslash{}item \textbackslash{}textbf\{Political society\}: Coercion (police, army, courts)
\textbackslash{}item \textbackslash{}textbf\{Civil society\}: Consent (schools, churches, media, cultural institutions)
\textbackslash{}end\{itemize\}

Hegemony operates primarily through civil society. This is why cultural struggle matters---and why simply seizing state power isn't enough.

\textbackslash{}subsection\{Organic Intellectuals\}

Gramsci distinguishes between \textbackslash{}textbf\{traditional intellectuals\}, who see themselves as a class apart from society, and \textbackslash{}textbf\{organic intellectuals\}, who emerge from and serve their class.

Organic intellectuals ``do not simply describe social life in accordance with scientific rules but instead articulate, through the language of culture, the feelings and experiences which the masses could not express for themselves.''

Counter-hegemonic movements need to develop their own organic intellectuals.

\textbackslash{}subsection\{War of Position vs. War of Maneuver\}

\textbackslash{}begin\{definition\}
\textbackslash{}textbf\{War of maneuver\}: Frontal assault on the state (revolution, insurrection)\textbackslash{}\textbackslash{}
\textbackslash{}textbf\{War of position\}: Long, slow work of building alternative institutions, developing organic intellectuals, and transforming common sense
\textbackslash{}end\{definition\}

In advanced societies with strong civil societies, war of maneuver doesn't work. Counter-hegemonic forces must engage in war of position.

\textbackslash{}subsection\{The Historic Bloc\}

A \textbackslash{}textbf\{historic bloc\} isn't just an alliance---it's a coherent worldview embodied in institutions that feels natural to those who live within it. It forms the basis of consent to a certain social order, producing and reproducing the hegemony of the dominant class.

\textbackslash{}begin\{freemocapapp\}
Gramsci provides the theoretical framework for understanding what FreeMoCap is actually fighting:

The ``common sense'' that FreeMoCap must transform:
\textbackslash{}begin\{itemize\}
\textbackslash{}item ``Real research requires institutional affiliation''
\textbackslash{}item ``Professional tools cost money''
\textbackslash{}item ``Credentials ensure quality''
\textbackslash{}item ``Science is done by experts in universities''
\textbackslash{}end\{itemize\}

The ``historic bloc'' FreeMoCap confronts: universities, journals, proprietary software companies, professional associations, funding agencies---not as a conspiracy but as a self-reinforcing system that feels natural.

The ``war of position'' FreeMoCap wages: building alternative tools, developing organic intellectuals (researchers who came up through FreeMoCap), transforming what seems possible.
\textbackslash{}end\{freemocapapp\}


\textbackslash{}chapter\{Elinor Ostrom: Governing the Commons\}

\textbackslash{}section\{Governing the Commons (1990)\}

Elinor Ostrom (1933--2012) won the Nobel Prize in Economics in 2009 for demonstrating that communities can successfully manage common resources without either privatization or state control.

\textbackslash{}subsection\{The Core Intervention\}

Before Ostrom, the dominant models were:
\textbackslash{}begin\{enumerate\}
\textbackslash{}item \textbackslash{}textbf\{Tragedy of the Commons\}: Shared resources will inevitably be destroyed by self-interested individuals
\textbackslash{}item \textbackslash{}textbf\{Two solutions\}: Either privatize (create property rights) or nationalize (state regulation)
\textbackslash{}end\{enumerate\}

Ostrom conducted fieldwork on actually-existing commons that had endured for centuries---Swiss alpine meadows, Japanese village forests, Spanish irrigation systems---and identified what they had in common.

\textbackslash{}begin\{keyinsight\}
Based on her fieldwork, Ostrom demonstrated that there are practical algorithms for the collective use of a limited common resource, which solve the many issues with both government/regulation driven solutions and market-based ones.
\textbackslash{}end\{keyinsight\}

\textbackslash{}subsection\{The Eight Design Principles\}

Ostrom identified eight principles present in sustainable commons institutions:

\textbackslash{}begin\{enumerate\}
\textbackslash{}item \textbackslash{}textbf\{Clearly defined boundaries\}: Who can use the resource, and what are its limits
\textbackslash{}item \textbackslash{}textbf\{Proportional equivalence between benefits and costs\}: Those who contribute more get more
\textbackslash{}item \textbackslash{}textbf\{Collective-choice arrangements\}: Those affected by rules participate in making them
\textbackslash{}item \textbackslash{}textbf\{Monitoring\}: Users monitor each other or hire monitors accountable to users
\textbackslash{}item \textbackslash{}textbf\{Graduated sanctions\}: Violations have consequences that escalate
\textbackslash{}item \textbackslash{}textbf\{Conflict-resolution mechanisms\}: Cheap, accessible ways to resolve disputes
\textbackslash{}item \textbackslash{}textbf\{Minimal recognition of rights to organize\}: Outside authorities recognize the community's right to self-govern
\textbackslash{}item \textbackslash{}textbf\{Nested enterprises\}: For larger systems, governance is organized in multiple nested layers
\textbackslash{}end\{enumerate\}

\textbackslash{}subsection\{The Polycentric Approach\}

Ostrom cautioned against single governmental units at global level to solve collective action problems. Her proposal was a \textbackslash{}textbf\{polycentric approach\}, where key management decisions should be made as close to the scene of events and the actors involved as possible.

Not one central authority, but multiple overlapping, semi-autonomous governance systems.

\textbackslash{}subsection\{Failures of Nationalization\}

In some countries, national agencies issued elaborate regulations concerning the use of forests but were unable to enforce them. The consequence was that nationalization created open-access resources where limited-access common-property resources had previously existed.

\textbackslash{}begin\{cautionary\}
Well-meaning state intervention can DESTROY functioning commons by undermining local governance mechanisms.
\textbackslash{}end\{cautionary\}

\textbackslash{}begin\{freemocapapp\}
Open source software is a digital commons. Ostrom's design principles translate directly:

\textbackslash{}begin\{enumerate\}
\textbackslash{}item \textbackslash{}textbf\{Boundaries\}: Who is a ``contributor''? Who can merge PRs? Clear governance.
\textbackslash{}item \textbackslash{}textbf\{Proportionality\}: Do active contributors have voice proportional to contribution?
\textbackslash{}item \textbackslash{}textbf\{Collective-choice\}: Do users/contributors participate in roadmap decisions?
\textbackslash{}item \textbackslash{}textbf\{Monitoring\}: Are there ways to track contributions, quality, compliance with norms?
\textbackslash{}item \textbackslash{}textbf\{Graduated sanctions\}: What happens when someone violates norms? (Rejected PRs → warnings → bans?)
\textbackslash{}item \textbackslash{}textbf\{Conflict resolution\}: How are disputes resolved? (GitHub issues? Discord? Maintainer decisions?)
\textbackslash{}item \textbackslash{}textbf\{External recognition\}: Does the broader ecosystem (Python community, academic institutions) recognize FreeMoCap's legitimacy?
\textbackslash{}item \textbackslash{}textbf\{Nested enterprises\}: As FreeMoCap grows, does governance scale appropriately?
\textbackslash{}end\{enumerate\}

Key insight: A commons isn't just ``shared''---it requires institutional arrangements. FreeMoCap's long-term success depends on developing governance structures, not just code.
\textbackslash{}end\{freemocapapp\}


\textbackslash{}chapter\{Nadia Eghbal: Open Source Sustainability\}

\textbackslash{}section\{Working in Public (2020)\}

Nadia Eghbal worked at GitHub studying open source developers. \textbackslash{}textit\{Working in Public: The Making and Maintenance of Open Source Software\} synthesizes her findings.

\textbackslash{}subsection\{Central Insight\}

Open source has shifted from providing an optimistic model for public collaboration to undergoing constant maintenance by the often unseen solo operators who write and publish code that millions of users rely on every day.

\textbackslash{}subsection\{The Maintainer's Trajectory\}

``The cycle looks something like this: Open source developers write and publish their code in public. They enjoy months, maybe years, in the spotlight. But, eventually, popularity offers diminishing returns... a developer who writes code in public must work with thousands of strangers.''

Or, as one observer memorably put it: ``Running a successful open source project is Good Will Hunting in reverse, where you start out as a respected genius and end up being a janitor who gets into fights.''

\textbackslash{}subsection\{The Attention Problem\}

It's not the excessive consumption of code but the excessive participation from users vying for a maintainer's attention that has made the work untenable for maintenance today.

A genuine conflict exists for maintainers who want to encourage newcomers to work on a project but become overwhelmed by the volume of typically low-quality work that emerges. The maintainer cannot personally mentor each new contributor.

\textbackslash{}subsection\{Sustainability Statistics\}

Industry data reveals alarming sustainability challenges:
\textbackslash{}begin\{itemize\}
\textbackslash{}item 60\textbackslash{}\% of open source maintainers have quit or considered quitting
\textbackslash{}item 44\textbackslash{}\% cite burnout as the primary reason
\textbackslash{}item Only 40\textbackslash{}\% of maintainers are paid at all
\textbackslash{}end\{itemize\}

\textbackslash{}begin\{freemocapapp\}
This is the sustainability bible for FreeMoCap. Key questions:
\textbackslash{}begin\{itemize\}
\textbackslash{}item How do you build something that doesn't burn out its maintainers?
\textbackslash{}item How do you handle the attention cost of a growing user base?
\textbackslash{}item What governance structures distribute the burden?
\textbackslash{}end\{itemize\}
\textbackslash{}end\{freemocapapp\}


\textbackslash{}section\{How the Frameworks Connect\}

These five thinkers form a coherent framework:

\textbackslash{}begin\{itemize\}
\textbackslash{}item \textbackslash{}textbf\{Gramsci\} explains WHAT you're fighting (hegemony---the system that makes current arrangements seem natural)
\textbackslash{}item \textbackslash{}textbf\{Illich\} explains what ALTERNATIVE you're building (convivial tools that enhance autonomy vs. radical monopolies that create dependency)
\textbackslash{}item \textbackslash{}textbf\{Benkler\} explains WHY your alternative can work economically (commons-based peer production as a viable mode of production)
\textbackslash{}item \textbackslash{}textbf\{Ostrom\} explains HOW to govern your alternative sustainably (design principles for successful commons)
\textbackslash{}item \textbackslash{}textbf\{brown\} explains HOW TO WORK as you build (emergent strategy principles for adaptive, relationship-centered organizing)
\textbackslash{}end\{itemize\}

For FreeMoCap specifically:
\textbackslash{}begin\{itemize\}
\textbackslash{}item You're fighting a hegemonic bloc (Gramsci) of proprietary tools, credentialism, institutional gatekeeping
\textbackslash{}item You're building convivial tools (Illich) that enhance research autonomy
\textbackslash{}item This works because peer production (Benkler) can outcompete proprietary production for information goods
\textbackslash{}item Long-term success requires governance (Ostrom) appropriate to commons management
\textbackslash{}item The process should be adaptive, trust-based, and fractal (brown)
\textbackslash{}end\{itemize\}


\% ============================================================================
\% PART III: CASE STUDIES
\% ============================================================================
\textbackslash{}part\{Case Studies: Open Source Success and Failure\}

\textbackslash{}chapter\{Comparative Analysis of Creative Software\}

Open source creative software has followed radically different trajectories over the past three decades. \textbackslash{}textbf\{Blender\} now produces Oscar-winning films and receives over €130,000 monthly from corporate giants. Meanwhile, \textbackslash{}textbf\{GIMP\} operates on volunteer time with effectively two paid developers and 6--8 year gaps between major releases.

The divergence reveals critical lessons about what makes open source projects succeed---or stagnate---in professional markets.

\textbackslash{}section\{The Core Differentiator\}

The core differentiator isn't technical capability but \textbackslash{}textbf\{organizational DNA\}: Blender began as commercial software built by professionals, transitioned through history's first crowdfunding campaign, and created a virtuous cycle through ``dogfooding'' via open movies. GIMP started as a student project, never established corporate relationships, and developed what critics call an ``insular'' culture resistant to professional input.


\textbackslash{}chapter\{Blender: The Success Story\}

\textbackslash{}section\{Professional Origins\}

Blender traces to January 2, 1994, when Dutch animator Ton Roosendaal began writing 3D software at NeoGeo, which quickly became the Netherlands' largest animation studio. The software was designed by 3D professionals for professional workflows---a crucial distinction from projects started in academic settings.

When NeoGeo's parent company collapsed in 2002 after burning through €4.5 million in venture capital, Roosendaal launched what he calls ``the first-ever crowdfunding campaign.'' The ``Free Blender'' campaign raised €100,000 in just seven weeks to purchase the source code from investors. On October 13, 2002, Blender became GPL-licensed open source.

\textbackslash{}section\{The 2.80 Transformation\}

The transformation that made Blender professionally competitive came with version 2.80 in July 2019---the result of four years of development including the ``Code Quest'' project that brought core contributors to Amsterdam.

The release introduced:
\textbackslash{}begin\{itemize\}
\textbackslash{}item Left-click selection (aligning with industry convention)
\textbackslash{}item EEVEE real-time renderer
\textbackslash{}item Completely redesigned interface
\textbackslash{}end\{itemize\}

Ubisoft Animation Studio called it a ``game-changer for the CGI industry.''

\textbackslash{}section\{Corporate Sponsorship\}

Corporate backing accelerated dramatically in 2019 when Epic Games donated \textbackslash{}\$1.2 million over three years. This convinced other tech giants to follow---NVIDIA, AMD, Unity, Meta, Apple, and Microsoft all became sponsors at €30,000--120,000+ annually.

The Blender Development Fund now receives contributions from over 7,500 individuals and 38 organizations, enabling employment of 26 full-time staff plus freelancers.

\textbackslash{}section\{The Dogfooding Strategy\}

Blender Foundation's most brilliant innovation was funding \textbackslash{}textbf\{open movies\} rather than funding developers directly. Starting with \textbackslash{}textit\{Elephants Dream\} (2006), these films serve triple duty:
\textbackslash{}begin\{enumerate\}
\textbackslash{}item Prove Blender can produce professional-quality animation
\textbackslash{}item Identify bugs and workflow issues during actual production
\textbackslash{}item Generate visibility that attracts users and sponsors
\textbackslash{}end\{enumerate\}

\textbackslash{}begin\{keyinsight\}
``Blender's dogfooding efforts were a powerful force in both proving that the software was good enough (to make movies) and demonstrating how it could be better (for the animators who made the movies). I'm not aware of any such efforts for GIMP.''
\textbackslash{}end\{keyinsight\}

This strategy culminated in 2025 when \textbackslash{}textit\{Flow\}, created entirely in Blender, won the Academy Award for Best Animated Feature---definitive proof of professional capability.

\textbackslash{}section\{Professional Studio Adoption\}

\textbackslash{}begin\{itemize\}
\textbackslash{}item Ubisoft Animation Studio replaced its in-house DCC tool with Blender in 2019
\textbackslash{}item Khara Inc. (Evangelion) adopted Blender as primary production tool
\textbackslash{}item Tangent Animation produced \textbackslash{}textit\{Next Gen\} (2018)---purchased by Netflix for \textbackslash{}\$30 million---using Blender almost exclusively
\textbackslash{}item Pixar revealed in 2015 that Blender was a supported internal tool
\textbackslash{}end\{itemize\}


\textbackslash{}chapter\{GIMP: Thirty Years of Missed Opportunities\}

\textbackslash{}section\{Academic Origins\}

Spencer Kimball and Peter Mattis started GIMP in 1995 at UC Berkeley's Experimental Computing Facility after ``Common LISP messily dumped core'' on a compilers class project. They decided to write ``something... ANYTHING... useful. Something in C.''

The software was developed by computer science students, not professional designers.

\textbackslash{}section\{GTK: The Bigger Legacy\}

GIMP's most significant contribution may be GTK (GIMP ToolKit), which Mattis created after becoming ``disenchanted with Motif.'' GTK became GNOME's foundation and influenced countless applications---arguably more impactful than GIMP itself.

\textbackslash{}section\{Glacial Development\}

Major version gaps tell the story:

\textbackslash{}begin\{table\}[h]
\textbackslash{}centering
\textbackslash{}begin\{tabular\}\{ll\}
\textbackslash{}toprule
\textbackslash{}textbf\{Versions\} \& \textbackslash{}textbf\{Gap\} \textbackslash{}\textbackslash{}
\textbackslash{}midrule
1.0 → 2.0 \& 6 years \textbackslash{}\textbackslash{}
2.8 → 2.10 \& 6 years \textbackslash{}\textbackslash{}
2.10 → 3.0 \& 7 years \textbackslash{}\textbackslash{}
\textbackslash{}bottomrule
\textbackslash{}end\{tabular\}
\textbackslash{}end\{table\}

GIMP 3.0 finally shipped in March 2025 with non-destructive editing---a feature Photoshop added around 2010. One user noted: ``We've been waiting for non-destructive editing for 23 years.''

\textbackslash{}section\{Governance Problems\}

Governance has drawn sharp criticism. Hacker News commenters describe GIMP maintainers as ``completely insular and barely tolerating'' outside contributions: ``Unless you're part of the inner clique, you won't get anything merged.''

When the CinePaint fork achieved commercial use with paid developers improving GIMP's capabilities, ``GIMP developers were too stubborn to acknowledge this effort and work with them.''

The Glimpse fork (2019--2021), created to address GIMP's controversial name, demonstrated the difficulty of community splits. Despite legitimate concerns, the fork failed to attract contributors and was discontinued within two years.

\textbackslash{}section\{Structural Funding Problems\}

GIMP donations through GNOME Foundation can ``only be used for community needs'' like conferences---not paid development. GIMP also has ``over \textbackslash{}\$1 million in Bitcoin'' it ``cannot currently access due to legal constraints.''


\textbackslash{}chapter\{Krita: The Strategic Pivot\}

\textbackslash{}section\{Origins and the Deventer Decision\}

Krita's origins as KImageShop (1999) within KDE seemed to doom it as another GIMP clone. But at a March 2010 sprint in Deventer, the team made a pivotal decision: abandon the generic image manipulation market and focus exclusively on \textbackslash{}textbf\{digital painting from scratch\}.

Creator Boudewijn Rempt explained: ``It turned out to be a great decision!'' Rather than competing with Photoshop and GIMP on photo editing, Krita targeted Corel Painter and SAI, serving concept artists, illustrators, and animators.

\textbackslash{}section\{Copying Blender's Model\}

The Krita Foundation (established 2012) was explicitly ``modeled on the Blender Foundation.'' Three Kickstarter campaigns raised approximately €85,000 (2014--2016), funding animation tools and supporting full-time developer Dmitry Kazakov.

\textbackslash{}section\{Revenue Diversification\}

Revenue diversification through Steam sales (\textbackslash{}textasciitilde\textbackslash{}\$10 per copy), plus presence on Microsoft Store, Epic Games Store, and Mac App Store, created sustainable income beyond donations.

By 2019, Krita could hire three new full-time developers and reported over 3 million downloads annually. Steam reviews are ``Overwhelmingly Positive'' (96\textbackslash{}\% of 2,800+ reviews).

\textbackslash{}section\{Professional Recognition\}

Krita won ImagineFX's ``Artist's Choice'' award in 2014 and was demonstrated at SIGGRAPH. Japanese animators adopted it for production work. The brush engine is frequently described as superior to competitors, validating the painting-focused strategy.


\textbackslash{}chapter\{Inkscape: Steady Progress, Limited Resources\}

\textbackslash{}section\{The Fork\}

Inkscape emerged in late 2003 as a fork of Sodipodi, led by Ted Gould, Bryce Harrington, Nathan Hurst, and MenTaLguY. The break happened because Sodipodi's maintainer ``had an active life outside open source''---months passed without contribution reviews.

\textbackslash{}section\{Governance\}

The founders established egalitarian governance where ``individual developers derive their authority from their ability and active involvement'' rather than top-down control. Since 2006, the Software Freedom Conservancy has provided fiscal sponsorship.

\textbackslash{}section\{Technical Limitations\}

Inkscape's SVG-native approach makes it ideal for web graphics and icon design but limits print workflow capabilities. Performance remains CPU-bound with no GPU acceleration---complex files with thousands of nodes cause visible lag.

PC Magazine rated it 3/5 stars, concluding it's ``not suitable for busy professionals.''

\textbackslash{}section\{Resource Constraints\}

The project operates on approximately \textbackslash{}\$15,000--40,000 annual budget, primarily covering conference attendance rather than developer salaries.


\textbackslash{}chapter\{Patterns and Lessons\}

\textbackslash{}section\{Why Corporate Money Flows to Blender\}

The corporate sponsorship gap reveals how professional adoption creates funding feedback loops. Companies sponsor Blender because they use it in production pipelines---GPU vendors want their technology well-supported, game engines benefit from the content creation ecosystem, and cloud providers use Blender for synthetic data generation.

GIMP never created this corporate interest. Without professional studio adoption, there's no pipeline integration requiring corporate support. Without corporate funding, development remains slow. Without faster development, professional adoption remains unlikely.

\textbackslash{}section\{Funding Comparison\}

\textbackslash{}begin\{table\}[h]
\textbackslash{}centering
\textbackslash{}begin\{tabularx\}\{\textbackslash{}textwidth\}\{l r r X\}
\textbackslash{}toprule
\textbackslash{}textbf\{Project\} \& \textbackslash{}textbf\{Monthly Funding\} \& \textbackslash{}textbf\{Paid Developers\} \& \textbackslash{}textbf\{Corporate Sponsors\} \textbackslash{}\textbackslash{}
\textbackslash{}midrule
Blender \& €130,000+ \& 26 FT + 12 contractors \& Epic, NVIDIA, AMD, Apple, Meta, Microsoft, Adobe \textbackslash{}\textbackslash{}
Krita \& \textbackslash{}textasciitilde€5,000 \& 2--3 FT \& Intel, Epic (grant) \textbackslash{}\textbackslash{}
Inkscape \& \textbackslash{}textasciitilde€1,500 \& 0 FT \& None major \textbackslash{}\textbackslash{}
GIMP \& \textbackslash{}textasciitilde€2,500 \& 0 FT (volunteer) \& None \textbackslash{}\textbackslash{}
\textbackslash{}bottomrule
\textbackslash{}end\{tabularx\}
\textbackslash{}end\{table\}

\textbackslash{}section\{Five Patterns of Success\}

Five patterns separate projects achieving industry legitimacy from those remaining perpetual alternatives:

\textbackslash{}begin\{enumerate\}
\textbackslash{}item \textbackslash{}textbf\{Professional DNA matters.\} Projects founded by professionals for professional use inherently understand workflows better than academic or hobbyist efforts.

\textbackslash{}item \textbackslash{}textbf\{Dogfooding proves capability.\} Blender's open movies provide undeniable evidence of production readiness.

\textbackslash{}item \textbackslash{}textbf\{Centralized vision beats distributed volunteering.\} Roosendaal's 30-year leadership provided consistent direction. GIMP's distributed governance created long development cycles and resistance to outside input.

\textbackslash{}item \textbackslash{}textbf\{Corporate sponsorship requires enterprise relevance.\} Companies fund tools they use or that support their products.

\textbackslash{}item \textbackslash{}textbf\{Foundation structure enables sustainability.\} Legal entities can employ developers, creating stable velocity versus volunteer availability.
\textbackslash{}end\{enumerate\}

\textbackslash{}section\{Maintainer Burnout\}

Industry data reveals alarming sustainability challenges: 60\textbackslash{}\% of open source maintainers have quit or considered quitting, with 44\textbackslash{}\% citing burnout as primary reason. Only 40\textbackslash{}\% of maintainers are paid at all.

GIMP development slowed in summer 2023 when ``the maintainer and some of the main developers fell ill after the Libre Graphics Meeting conference.''

The XZ Utils backdoor (2024) demonstrated how maintainer burnout creates security vulnerabilities---malicious actors offered to ``help'' an overwhelmed solo maintainer, then inserted backdoors.

Blender mitigated burnout risk through institutional structure: multiple paid developers, planned leadership succession, and diversified funding ensuring no single-point-of-failure dependencies.

\textbackslash{}begin\{freemocapapp\}
\textbackslash{}textbf\{Lessons for FreeMoCap:\}

\textbackslash{}begin\{enumerate\}
\textbackslash{}item \textbackslash{}textbf\{Foundation structure\}: Establish formal legal entity that can employ developers and accept corporate sponsorship
\textbackslash{}item \textbackslash{}textbf\{Dogfooding\}: Create demonstration projects that prove FreeMoCap's professional capability (published research, clinical applications, artistic projects)
\textbackslash{}item \textbackslash{}textbf\{Corporate relationships\}: Identify companies that benefit from accessible motion capture (game studios, AR/VR, healthcare tech) and cultivate sponsorship
\textbackslash{}item \textbackslash{}textbf\{Niche clarity\}: Like Krita, be clear about what you're for---don't try to compete with everything
\textbackslash{}item \textbackslash{}textbf\{Governance clarity\}: Establish explicit decision-making processes and contribution pathways
\textbackslash{}item \textbackslash{}textbf\{Succession planning\}: The project must be able to survive founder absence
\textbackslash{}end\{enumerate\}

\textbackslash{}textbf\{Warning signs to avoid:\}
\textbackslash{}begin\{itemize\}
\textbackslash{}item Insular culture resistant to outside input (GIMP pattern)
\textbackslash{}item Volunteer-only development creating glacial pace
\textbackslash{}item Lack of professional use cases creating funding chicken-and-egg
\textbackslash{}item Founder dependency without succession planning
\textbackslash{}end\{itemize\}
\textbackslash{}end\{freemocapapp\}


\% ============================================================================
\% PART IV: CONCLUSION
\% ============================================================================
\textbackslash{}part\{Conclusion\}

\textbackslash{}chapter\{Synthesis: Building the Counter-Hegemonic Project\}

\textbackslash{}section\{The Integrated Framework\}

The theoretical and practical frameworks presented in this document converge on a coherent approach to building FreeMoCap as a counter-hegemonic project:

\textbackslash{}begin\{enumerate\}
\textbackslash{}item \textbackslash{}textbf\{Understand the enemy\} (Gramsci): The system of universities, journals, proprietary software, and credentialism maintains power not through force but by making current arrangements seem natural. Counter-hegemonic work transforms this ``common sense.''

\textbackslash{}item \textbackslash{}textbf\{Build convivial alternatives\} (Illich): Tools that enhance autonomy rather than create dependency. FreeMoCap should enable research without requiring institutional permission.

\textbackslash{}item \textbackslash{}textbf\{Leverage peer production economics\} (Benkler): Commons-based peer production can outcompete proprietary production for information goods. The conditions are right for open scientific tools.

\textbackslash{}item \textbackslash{}textbf\{Govern the commons sustainably\} (Ostrom): Success requires institutional arrangements---clear boundaries, proportional voice, conflict resolution, nested governance as scale increases.

\textbackslash{}item \textbackslash{}textbf\{Organize, don't just mobilize\} (McAlevey): Expand the base by reaching people who don't yet identify with the cause. Identify organic leaders in target communities.

\textbackslash{}item \textbackslash{}textbf\{Work at the speed of trust\} (brown): What you practice at small scale sets patterns for the whole system. Build relationships, not just code.

\textbackslash{}item \textbackslash{}textbf\{Tell your story\} (Ganz): Weave together Story of Self, Story of Us, Story of Now into compelling narrative that moves people to action.

\textbackslash{}item \textbackslash{}textbf\{Develop critical consciousness\} (Freire): Help people move from fatalistic acceptance through blame of individuals to understanding systemic causes and capacity for transformation.
\textbackslash{}end\{enumerate\}

\textbackslash{}section\{The Road Ahead\}

FreeMoCap is Stage 1. The motion capture software proves that research-grade tools can be free and accessible. But the larger project---dismantling the hegemony of knowledge production---requires sustained effort across multiple fronts:

\textbackslash{}begin\{itemize\}
\textbackslash{}item \textbackslash{}textbf\{Tools\}: Building more convivial research tools (FreeMoCap and beyond)
\textbackslash{}item \textbackslash{}textbf\{Publishing\}: Supporting open access and alternative validation mechanisms
\textbackslash{}item \textbackslash{}textbf\{Education\}: Creating pathways to knowledge that don't require credentials
\textbackslash{}item \textbackslash{}textbf\{Funding\}: Developing alternative models for supporting research
\textbackslash{}item \textbackslash{}textbf\{Community\}: Building networks of independent researchers
\textbackslash{}end\{itemize\}

\textbackslash{}section\{Final Words\}

\textbackslash{}begin\{tcolorbox\}[colback=lightgray,colframe=accent,fonttitle=\textbackslash{}bfseries]
This is a living document. The work of building counter-hegemonic power is never complete. Return to these questions. Revise your answers. Share them with co-conspirators. Build the infrastructure for a long fight.

The current system wasn't built in a day. It won't fall in a day.

\textbackslash{}textbf\{But it WILL fall.\}
\textbackslash{}end\{tcolorbox\}


\% ============================================================================
\% BACK MATTER
\% ============================================================================
\textbackslash{}backmatter

\textbackslash{}chapter\textit{\{Reading List\}
\textbackslash{}addcontentsline\{toc\}\{chapter\}\{Reading List\}

\textbackslash{}section}\{Tier 1: Essential\}

\textbackslash{}begin\{itemize\}
\textbackslash{}item Smucker, Jonathan Matthew. \textbackslash{}textit\{Hegemony How-To: A Roadmap for Radicals\}. AK Press, 2017.
\textbackslash{}item Illich, Ivan. \textbackslash{}textit\{Deschooling Society\}. Harper \textbackslash{}\& Row, 1971. [PDF available online]
\textbackslash{}item Illich, Ivan. \textbackslash{}textit\{Tools for Conviviality\}. Harper \textbackslash{}\& Row, 1973. [PDF available online]
\textbackslash{}item McAlevey, Jane. \textbackslash{}textit\{No Shortcuts: Organizing for Power in the New Gilded Age\}. Oxford University Press, 2016.
\textbackslash{}item Eghbal, Nadia. \textbackslash{}textit\{Working in Public: The Making and Maintenance of Open Source Software\}. Stripe Press, 2020.
\textbackslash{}end\{itemize\}

\textbackslash{}section\textit{\{Tier 2: Deepening\}

\textbackslash{}begin\{itemize\}
\textbackslash{}item Freire, Paulo. \textbackslash{}textit\{Pedagogy of the Oppressed\}. Continuum, 1970.
\textbackslash{}item Freire, Paulo. \textbackslash{}textit\{Pedagogy of Hope: Reliving Pedagogy of the Oppressed\}. Continuum, 1992.
\textbackslash{}item brown, adrienne maree. \textbackslash{}textit\{Emergent Strategy: Shaping Change, Changing Worlds\}. AK Press, 2017.
\textbackslash{}item Benkler, Yochai. \textbackslash{}textit\{The Wealth of Networks: How Social Production Transforms Markets and Freedom\}. Yale University Press, 2006. [PDF available at benkler.org]
\textbackslash{}item Swartz, Aaron. ``Guerrilla Open Access Manifesto.'' 2008.
\textbackslash{}end\{itemize\}

\textbackslash{}section}\{Tier 3: Context\}

\textbackslash{}begin\{itemize\}
\textbackslash{}item Gramsci, Antonio. \textbackslash{}textit\{Selections from the Prison Notebooks\}. International Publishers, 1971.
\textbackslash{}item Ostrom, Elinor. \textbackslash{}textit\{Governing the Commons: The Evolution of Institutions for Collective Action\}. Cambridge University Press, 1990.
\textbackslash{}item Kelty, Christopher. \textbackslash{}textit\{Two Bits: The Cultural Significance of Free Software\}. Duke University Press, 2008.
\textbackslash{}item Ransby, Barbara. \textbackslash{}textit\{Ella Baker and the Black Freedom Movement: A Radical Democratic Vision\}. University of North Carolina Press, 2003.
\textbackslash{}end\{itemize\}


\textbackslash{}chapter*\{Key Concepts Glossary\}
\textbackslash{}addcontentsline\{toc\}\{chapter\}\{Key Concepts Glossary\}

\textbackslash{}begin\{description\}
\textbackslash{}item[Banking model] (Freire) Traditional education that treats students as empty vessels for knowledge deposits.

\textbackslash{}item[Commons-based peer production] (Benkler) Decentralized, collaborative production without reliance on markets or hierarchies.

\textbackslash{}item[Conscientização] (Freire) Development of critical consciousness---from fatalism to systemic understanding to transformative action.

\textbackslash{}item[Convivial tools] (Illich) Tools that enhance human autonomy and creativity without creating expert dependency.

\textbackslash{}item[Counter-hegemony] (Gramsci) The project of building alternative ``common sense'' to challenge dominant ideology.

\textbackslash{}item[Dogfooding] Using your own product in real production contexts to prove capability and identify improvements.

\textbackslash{}item[Emergent strategy] (brown) Building complex change through small interactions, trusting patterns to emerge.

\textbackslash{}item[Hegemony] (Gramsci) Maintaining power by making dominant values seem like universal common sense.

\textbackslash{}item[Organic intellectuals] (Gramsci) Intellectuals who emerge from and serve their class, articulating experiences the masses cannot express.

\textbackslash{}item[Organizing vs. mobilizing] (McAlevey) Expanding your base vs. activating existing supporters.

\textbackslash{}item[Political identity paradox] (Smucker) Strong identity enables commitment but can create insularity preventing coalition-building.

\textbackslash{}item[Praxis] (Freire) Unity of reflection and action---neither alone is sufficient for liberation.

\textbackslash{}item[Radical monopoly] (Illich) When one type of solution becomes compulsory, excluding alternatives.

\textbackslash{}item[Second watershed] (Illich) Point at which an approach becomes counterproductive---medicine creates illness, schools create ignorance.

\textbackslash{}item[War of position] (Gramsci) Long-term work of building alternative institutions and transforming common sense (vs. war of maneuver/frontal assault).
\textbackslash{}end\{description\}


\textbackslash{}end\{document\}
\normalsize


\end{document}
